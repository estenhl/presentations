\documentclass[final,11pt]{beamer}

% Set the font size and paper size
\usepackage[size=a0,orientation=portrait]{beamerposter}

% Load necessary packages
\usepackage{array}
\usepackage{caption}
\usepackage{enumitem}
\usepackage{graphicx}
\usepackage{lipsum}
\usepackage{makecell}
\usepackage{pgfplots}
\usepackage{ragged2e}
\usepackage{subcaption}
\usepackage{tikz}
\usepackage{transparent}

\usetikzlibrary{arrows.meta}
\usetikzlibrary{calc}

% Define the theme
\usetheme{metropolis}

% Colours
\definecolor{uiored}{HTML}{DD0000}
\definecolor{uiogrey}{HTML}{B2B3B7}
\definecolor{uioblack}{HTML}{000000}
\definecolor{uiowhite}{HTML}{FFFFFF}

% Define the colors
\definecolor{background}{HTML}{FAFAFA}
\setbeamercolor{block title}{bg=background,fg=red}
\setbeamercolor{block body}{bg=background,fg=black}
\setbeamercolor{title}{bg=uiored,fg=uiowhite}
\setbeamercolor{authors}{bg=uioblack,fg=uiowhite}

\setbeamertemplate{page number}{}
\setlength{\paperwidth}{\textwidth}
\setbeamersize{text margin left=0pt, text margin right=0pt}

% Title and author information
\title{Detecting structural brain aberrations in dementia patients\\[1cm]with explainable artificial intelligence}


\definecolor{cb-green}{HTML}{4dac93}
\definecolor{cb-blue}{HTML}{3594d6}

\def\verticalspace{0.61cm}

\captionsetup[figure]{labelformat=empty}

\begin{document}

% Start the poster
    \begin{frame}[t]
        \thispagestyle{empty}

        % Title and author block
        \begin{beamercolorbox}[sep=0em,wd=\textwidth]{title}
            \centering\\[6cm]
            \fontsize{78}{78}{\textbf{\inserttitle}}\\[6cm]

        \end{beamercolorbox}

        \newcommand{\authortext}[1]{
            \fontsize{32}{32}{\textbf{####1}}
        }
        \newcommand{\affiliationtext}[1]{
            \fontsize{28}{28}{\textit{####1}}
        }
        \vspace{-0.03cm}

        \begin{beamercolorbox}[sep=0em, wd=\textwidth]{authors}
            \def\columnwidth{0.135\textwidth}
            \vspace{1cm}\\
            \hspace{1cm}
            \begin{tabular}{m{\columnwidth}p{\columnwidth}p{\columnwidth}p{\columnwidth}p{\columnwidth}p{\columnwidth}p{\columnwidth}}
                \vspace{-0.5cm}\makecell{\authortext{Esten H. Leonardsen}\\[0.25cm]\affiliationtext{University of Oslo}}&
                \makecell{\authortext{Karin Persson}\\[0.25cm]\affiliationtext{National Centre for}\\\affiliationtext{Ageing and Health}}&
                \makecell{\authortext{Geir Selbæk}\\[0.25cm]\affiliationtext{National Centre for}\\\affiliationtext{Ageing and Health}}&
                \makecell{\authortext{Ole A. Andreassen}\\[0.25cm]\affiliationtext{University of Oslo}}&
                \makecell{\authortext{Lars T. Westlye}\\[0.25cm]\affiliationtext{University of Oslo}}&
                \makecell{\authortext{Thomas Wolfers}\\[0.25cm]\affiliationtext{University of Tübingen}}&
                \makecell{\authortext{Yunpeng Wang}\\[0.25cm]\affiliationtext{University of Oslo}}\\
            \end{tabular}
            \vspace{1cm}
        \end{beamercolorbox}


        \newcommand{\mriinput}[2]{
    \def\mridepth{3}
    \node[yslant=1, inner sep=0pt] (input) at (#1, #2) {
        \includegraphics[width=5cm, height=8.5cm]{data/mri.png}
    };
    \draw[fill=black] (input.north east) --
        ($ (input.north east) - (\mridepth, 0) $) --
        ($ (input.north west) - (\mridepth, 0) $) --
        (input.north west) -- cycle;
    \draw[fill=black] (input.north west) --
        ($ (input.north west) - (\mridepth, 0) $) --
        ($ (input.south west) - (\mridepth, 0) $) --
        (input.south west) -- cycle;
    \draw[] (input.north east) --
        ($ (input.north east) + (\mridepth, 0) $) --
        ($ (input.north west) + (\mridepth, 0) $) --
        (input.north west) -- cycle;
    \draw[]  ($ (input.north west) + (\mridepth, 0) $) --
        ($ (input.south west) + (\mridepth, 0) $) --
        (input.south west);
    \draw[] ($ (input.north east) + (\mridepth, 0) $) --
        ($ (input.south east) + (\mridepth, 0) $) --
        ($ (input.south west) + (\mridepth, 0) $);
    \node[
        anchor=south,
        align=center,
        font=\fontsize{32}{32}\linespread{0.8}\selectfont
    ] at ($ (input.north east) - (0, 0) $) {
        T1-weighted\\
        structural MRI
    };
}

\newcommand{\convside}[6]{
    \node[
        fill=#5,
        inner sep=0pt,
        outer sep=0pt,
        minimum width=#3,
        minimum height=#4,
        draw=black
    ] (#6) at (#1, #2) {};
}

\newcommand{\convtop}[4]{
    \draw[fill=#4,draw=black] #1 --
    ($ #1 + (#3, #3) $) --
    ($ #1 + (#3+#2, #3) $) --
    ($ #1 + (#2, 0) $);
}

\newcommand{\convfront}[3]{
    \draw[black, fill=#3] #1 --
                        ($ #1 + (1*#2, 1*#2) $) --
                        ($ #1 + (1*#2, 1*#2 - 2*#2) $) --
                        ($ #1 + (0, -2*#2) $);
}


\newcommand{\convchannel}[5]{
    \def\huemin{20}
    \def\huemax{80}
    \pgfmathsetmacro{\iterations}{#5-1}
    \foreach \i in {0,...,\iterations} {
        \pgfmathsetmacro{\hue}{int(random(\huemin, \huemax))}
        \convside{#1}{#2+\i*-0.75}{#3}{#4/#5}{black!\hue}{n\i0}
        %\convtop{($ (n00.north west) + (0.5*\i*#4/#5, 0.5*\i*#4/#5) $)}{#3}{0.5*#4/#5}{black!\hue}

        \foreach \j in {0,...,\iterations} {
            \pgfmathsetmacro{\innerhue}{int(random(\huemin, \huemax))}
            \ifnum\j=0
                \pgfmathsetmacro{\innerhue}{\hue}
            \fi
            \convfront{($ (n00.north east) + (0.5*\j*#4/#5, 0.5*\j*#4/#5 - \i*#4/#5) $)}{0.5*#4/#5}{black!\innerhue}

            \ifnum\i=0
                \convtop{($ (n\i0.north west) + (0.5*\j*#4/#5, 0.5*\j*#4/#5) $)}{#3}{0.5*#4/#5}{black!\innerhue}
            \fi
        }
    }
}

\newcommand{\convlayer}[6]{
    \pgfmathsetmacro{\iterations}{#6-1}
    \foreach \i in {0,...,\iterations}{
        \pgfmathsetmacro{\x}{#1 + \i * 0.33}
        \convchannel{\x}{#2}{#3}{#4}{#5}
    }
}

\newcommand{\cnnarrow}[2]{
    \begin{scope}[transparency group, opacity=1]%0.5]
        \draw[-stealth, line width=15pt,red] #1 -- #2;
    \end{scope}
}

\newcommand{\cnn}[2]{
    \convlayer{#1}{#2-0.25}{0.33cm}{9cm}{12}{3}
    \cnnarrow{(#1 + 3.07, #2 - 2.1)}{(#1+7.5, #2 - 2.1)}
    \convlayer{#1 + 6.4}{#2-1}{0.33cm}{6cm}{8}{5}
    \cnnarrow{(#1 + 9.37, #2 - 2.1)}{(#1+13, #2 - 2.1)}
    \convlayer{#1 + 11.9}{#2-1.75}{0.33cm}{3cm}{4}{7}
    \cnnarrow{(#1 + 14.8, #2 - 2.1)}{(#1+18, #2 - 2.1)}
    \convlayer{#1 + 16.5}{#2-2.1}{0.33cm}{1.5cm}{2}{9}
    \cnnarrow{(#1 + 19.67, #2 - 2.1)}{(#1+22.2, #2 - 2.1)}

    \draw[thick, dashed] (#1 - 0.7, #2 + 5.1) --
                        (#1 + 20.6, #2 + 5.1) --
                        (#1 + 20.6, #2 - 9.4) --
                        (#1 - 0.7, #2 - 9.4) -- cycle;
    \node[anchor=south, text depth=0, font=\fontsize{32}{32}] at (#1 + 9.95, #2 + 5.1) {
        \textbf{3D Convolutional Neural Network}
    };
}

\newcommand{\lrpchannel}[5]{
    \def\huemin{30}
    \def\huemax{210}
    \colorlet{bgcolor}{black!90}
    \pgfmathsetmacro{\iterations}{#5-1}
    \foreach \i in {0,...,\iterations} {
        \pgfmathsetmacro{\red}{int(random(-150, 100))}
        \colorlet{fillcolor}{bgcolor}

        \ifnum\red>0
            \colorlet{fillcolor}{red!\red!bgcolor}
        \fi

        \convside{#1}{#2+\i*-0.75}{#3}{#4/#5}{fillcolor}{n\i0}
        %\convtop{($ (n00.north west) + (0.5*\i*#4/#5, 0.5*\i*#4/#5) $)}{#3}{0.5*#4/#5}{fillcolor}

        \foreach \j in {0,...,\iterations} {
            \pgfmathsetmacro{\innerred}{int(random(-150, 100))}
            \colorlet{innerfillcolor}{bgcolor}

            \ifnum\innerred>0
                \colorlet{innerfillcolor}{red!\innerred!bgcolor}
            \fi

            \ifnum\j=0
                \colorlet{innerfillcolor}{fillcolor}
            \fi

            \convfront{($ (n00.north east) + (0.5*\j*#4/#5, 0.5*\j*#4/#5 - \i*#4/#5) $)}{0.5*#4/#5}{innerfillcolor}
            \ifnum\i=0
                \convtop{($ (n\i0.north west) + (0.5*\j*#4/#5, 0.5*\j*#4/#5) $)}{#3}{0.5*#4/#5}{innerfillcolor}
            \fi
        }
    }
}

\newcommand{\lrplayer}[6]{
    \pgfmathsetmacro{\iterations}{#6-1}
    \foreach \i in {0,...,\iterations}{
        \pgfmathsetmacro{\x}{#1 + \i * 0.33}
        \lrpchannel{\x}{#2}{#3}{#4}{#5}
    }
}

\newcommand{\lrp}[2]{

    \lrplayer{#1}{#2-1}{0.33cm}{1.5cm}{2}{9}
    \cnnarrow{(#1 + 3.18, #2 - 1)}{(#1+6, #2 - 1)}
    \lrplayer{#1+4.55}{#2-0.65}{0.33cm}{3cm}{4}{7}
    \cnnarrow{(#1 + 7.45, #2 - 1)}{(#1+13, #2 - 1)}
    \lrplayer{#1 + 9.25}{#2+0.1}{0.33cm}{6cm}{8}{5}
    \cnnarrow{(#1 + 12.23, #2 - 1)}{(#1+20, #2 - 1)}
    \lrplayer{#1 + 14.85}{#2 + 0.85}{0.33cm}{9cm}{12}{3}

    \draw[thick, dashed] (#1 - 0.7, #2 + 6.25) --
                        (#1 + 20.6, #2 + 6.25) --
                        (#1 + 20.6, #2 - 8.25) --
                        (#1 - 0.7, #2 - 8.25) -- cycle;
    \node[anchor=south, text depth=0, font=\fontsize{32}{32}] at (#1 + 9.95, #2 + 6.25) {
        \textbf{Layerwise relevance propagation}
    };
}

\newcommand{\heatmap}[2]{
    \node[yslant=1, inner sep=0pt] (map) at (#1, #2) {
        \includegraphics[width=5cm, height=8.5cm]{data/heatmap.png}
    };
    \draw[fill=black] (map.north east) --
        ($ (map.north east) - (\mridepth, 0) $) --
        ($ (map.north west) - (\mridepth, 0) $) --
        (map.north west) -- cycle;
    \draw[fill=black] (map.north west) --
        ($ (map.north west) - (\mridepth, 0) $) --
        ($ (map.south west) - (\mridepth, 0) $) --
        (map.south west) -- cycle;
    \draw[] (map.north east) --
        ($ (map.north east) + (\mridepth, 0) $) --
        ($ (map.north west) + (\mridepth, 0) $) --
        (map.north west) -- cycle;
    \draw[]  ($ (map.north west) + (\mridepth, 0) $) --
        ($ (map.south west) + (\mridepth, 0) $) --
        (map.south west);
    \draw[] ($ (map.north east) + (\mridepth, 0) $) --
        ($ (map.south east) + (\mridepth, 0) $) --
        ($ (map.south west) + (\mridepth, 0) $);
    \node[
        anchor=south,
        align=center,
        font=\fontsize{32}{32}\linespread{0.8}\selectfont
    ] at (map.north east) {
        Explanatory\\
        heatmaps
    };
}


\begin{figure}[h]
    \begin{tikzpicture}
        \mriinput{-1.2}{-1.15}

        \cnnarrow{(-1.2, -1.15)}{(7, -1.15)}

        \cnn{6}{1}

        \node[align=center,font=\fontsize{32}{32}] at (34, -1.15) {
            \textit{Predicted probability}\\\textit{of dementia}
        };
        \cnnarrow{(39.7, -1.15)}{(53, -1.15)}

        \lrp{41.5}{-0.15}
        \cnnarrow{(59.43, -1.15)}{(67, -1.15)}

        \heatmap{68.6}{-1.15}
    \end{tikzpicture}
\end{figure}


        \vfill

        \begin{beamercolorbox}[sep=0pt,wd=\textwidth]{title}
            \begin{tikzpicture}
                \node[] at (-40.25, 0) {};
                \node[] at (0, 0) {
                    \includegraphics[height=5cm]{data/uio.png}
                };
                \node[inner sep=0.5cm] at (40.25, 0) {
                    \includegraphics[height=5cm]{
                        data/qr.png
                    }
                };
            \end{tikzpicture}
        \end{beamercolorbox}


        % \begin{columns}[t]
        %     \begin{column}{.33\textwidth}
        %         \begin{block}{Introduction}
        %             \parbox{\linewidth}{\justify
        % With over 55 million people affected globally and a projected threefold increase in prevalence by 2050,
        % dementia presents a paramount public health challenge for the coming decades.
        % Deep learning applied to magnetic resonance imaging (MRI) scans have shown great promise for diagnosis and prognosis in dementia, but its clinical adoption is limited. This is partially attributed
        % to the opaqueness of deep neural networks (DNNs), causing insufficient understanding of what underlies their decisions.
        % Layerwise relevance propagation (LRP) is a technique for explaining the decision of DNNs via heatmaps highlighting regions of an
        % image contributing to the prediction, potentially ameliorating the distrust impeding their clinical use. Furthermore,
        % the explanations procured by LRP are highly individualized and could shed light on the specific manifestation of the disease in the brain,
        % information which could prove crucial for accurate diagnosis and treatment.
        %             }
        %         \end{block}
        %     \end{column}
        %     \begin{column}{.63\textwidth}
        %         \newcommand{\mriinput}[2]{
    \def\mridepth{3}
    \node[yslant=1, inner sep=0pt] (input) at (#1, #2) {
        \includegraphics[width=5cm, height=8.5cm]{data/mri.png}
    };
    \draw[fill=black] (input.north east) --
        ($ (input.north east) - (\mridepth, 0) $) --
        ($ (input.north west) - (\mridepth, 0) $) --
        (input.north west) -- cycle;
    \draw[fill=black] (input.north west) --
        ($ (input.north west) - (\mridepth, 0) $) --
        ($ (input.south west) - (\mridepth, 0) $) --
        (input.south west) -- cycle;
    \draw[] (input.north east) --
        ($ (input.north east) + (\mridepth, 0) $) --
        ($ (input.north west) + (\mridepth, 0) $) --
        (input.north west) -- cycle;
    \draw[]  ($ (input.north west) + (\mridepth, 0) $) --
        ($ (input.south west) + (\mridepth, 0) $) --
        (input.south west);
    \draw[] ($ (input.north east) + (\mridepth, 0) $) --
        ($ (input.south east) + (\mridepth, 0) $) --
        ($ (input.south west) + (\mridepth, 0) $);
    \node[
        anchor=south,
        align=center,
        font=\fontsize{32}{32}\linespread{0.8}\selectfont
    ] at ($ (input.north east) - (0, 0) $) {
        T1-weighted\\
        structural MRI
    };
}

\newcommand{\convside}[6]{
    \node[
        fill=#5,
        inner sep=0pt,
        outer sep=0pt,
        minimum width=#3,
        minimum height=#4,
        draw=black
    ] (#6) at (#1, #2) {};
}

\newcommand{\convtop}[4]{
    \draw[fill=#4,draw=black] #1 --
    ($ #1 + (#3, #3) $) --
    ($ #1 + (#3+#2, #3) $) --
    ($ #1 + (#2, 0) $);
}

\newcommand{\convfront}[3]{
    \draw[black, fill=#3] #1 --
                        ($ #1 + (1*#2, 1*#2) $) --
                        ($ #1 + (1*#2, 1*#2 - 2*#2) $) --
                        ($ #1 + (0, -2*#2) $);
}


\newcommand{\convchannel}[5]{
    \def\huemin{20}
    \def\huemax{80}
    \pgfmathsetmacro{\iterations}{#5-1}
    \foreach \i in {0,...,\iterations} {
        \pgfmathsetmacro{\hue}{int(random(\huemin, \huemax))}
        \convside{#1}{#2+\i*-0.75}{#3}{#4/#5}{black!\hue}{n\i0}
        %\convtop{($ (n00.north west) + (0.5*\i*#4/#5, 0.5*\i*#4/#5) $)}{#3}{0.5*#4/#5}{black!\hue}

        \foreach \j in {0,...,\iterations} {
            \pgfmathsetmacro{\innerhue}{int(random(\huemin, \huemax))}
            \ifnum\j=0
                \pgfmathsetmacro{\innerhue}{\hue}
            \fi
            \convfront{($ (n00.north east) + (0.5*\j*#4/#5, 0.5*\j*#4/#5 - \i*#4/#5) $)}{0.5*#4/#5}{black!\innerhue}

            \ifnum\i=0
                \convtop{($ (n\i0.north west) + (0.5*\j*#4/#5, 0.5*\j*#4/#5) $)}{#3}{0.5*#4/#5}{black!\innerhue}
            \fi
        }
    }
}

\newcommand{\convlayer}[6]{
    \pgfmathsetmacro{\iterations}{#6-1}
    \foreach \i in {0,...,\iterations}{
        \pgfmathsetmacro{\x}{#1 + \i * 0.33}
        \convchannel{\x}{#2}{#3}{#4}{#5}
    }
}

\newcommand{\cnnarrow}[2]{
    \begin{scope}[transparency group, opacity=1]%0.5]
        \draw[-stealth, line width=15pt,red] #1 -- #2;
    \end{scope}
}

\newcommand{\cnn}[2]{
    \convlayer{#1}{#2-0.25}{0.33cm}{9cm}{12}{3}
    \cnnarrow{(#1 + 3.07, #2 - 2.1)}{(#1+7.5, #2 - 2.1)}
    \convlayer{#1 + 6.4}{#2-1}{0.33cm}{6cm}{8}{5}
    \cnnarrow{(#1 + 9.37, #2 - 2.1)}{(#1+13, #2 - 2.1)}
    \convlayer{#1 + 11.9}{#2-1.75}{0.33cm}{3cm}{4}{7}
    \cnnarrow{(#1 + 14.8, #2 - 2.1)}{(#1+18, #2 - 2.1)}
    \convlayer{#1 + 16.5}{#2-2.1}{0.33cm}{1.5cm}{2}{9}
    \cnnarrow{(#1 + 19.67, #2 - 2.1)}{(#1+22.2, #2 - 2.1)}

    \draw[thick, dashed] (#1 - 0.7, #2 + 5.1) --
                        (#1 + 20.6, #2 + 5.1) --
                        (#1 + 20.6, #2 - 9.4) --
                        (#1 - 0.7, #2 - 9.4) -- cycle;
    \node[anchor=south, text depth=0, font=\fontsize{32}{32}] at (#1 + 9.95, #2 + 5.1) {
        \textbf{3D Convolutional Neural Network}
    };
}

\newcommand{\lrpchannel}[5]{
    \def\huemin{30}
    \def\huemax{210}
    \colorlet{bgcolor}{black!90}
    \pgfmathsetmacro{\iterations}{#5-1}
    \foreach \i in {0,...,\iterations} {
        \pgfmathsetmacro{\red}{int(random(-150, 100))}
        \colorlet{fillcolor}{bgcolor}

        \ifnum\red>0
            \colorlet{fillcolor}{red!\red!bgcolor}
        \fi

        \convside{#1}{#2+\i*-0.75}{#3}{#4/#5}{fillcolor}{n\i0}
        %\convtop{($ (n00.north west) + (0.5*\i*#4/#5, 0.5*\i*#4/#5) $)}{#3}{0.5*#4/#5}{fillcolor}

        \foreach \j in {0,...,\iterations} {
            \pgfmathsetmacro{\innerred}{int(random(-150, 100))}
            \colorlet{innerfillcolor}{bgcolor}

            \ifnum\innerred>0
                \colorlet{innerfillcolor}{red!\innerred!bgcolor}
            \fi

            \ifnum\j=0
                \colorlet{innerfillcolor}{fillcolor}
            \fi

            \convfront{($ (n00.north east) + (0.5*\j*#4/#5, 0.5*\j*#4/#5 - \i*#4/#5) $)}{0.5*#4/#5}{innerfillcolor}
            \ifnum\i=0
                \convtop{($ (n\i0.north west) + (0.5*\j*#4/#5, 0.5*\j*#4/#5) $)}{#3}{0.5*#4/#5}{innerfillcolor}
            \fi
        }
    }
}

\newcommand{\lrplayer}[6]{
    \pgfmathsetmacro{\iterations}{#6-1}
    \foreach \i in {0,...,\iterations}{
        \pgfmathsetmacro{\x}{#1 + \i * 0.33}
        \lrpchannel{\x}{#2}{#3}{#4}{#5}
    }
}

\newcommand{\lrp}[2]{

    \lrplayer{#1}{#2-1}{0.33cm}{1.5cm}{2}{9}
    \cnnarrow{(#1 + 3.18, #2 - 1)}{(#1+6, #2 - 1)}
    \lrplayer{#1+4.55}{#2-0.65}{0.33cm}{3cm}{4}{7}
    \cnnarrow{(#1 + 7.45, #2 - 1)}{(#1+13, #2 - 1)}
    \lrplayer{#1 + 9.25}{#2+0.1}{0.33cm}{6cm}{8}{5}
    \cnnarrow{(#1 + 12.23, #2 - 1)}{(#1+20, #2 - 1)}
    \lrplayer{#1 + 14.85}{#2 + 0.85}{0.33cm}{9cm}{12}{3}

    \draw[thick, dashed] (#1 - 0.7, #2 + 6.25) --
                        (#1 + 20.6, #2 + 6.25) --
                        (#1 + 20.6, #2 - 8.25) --
                        (#1 - 0.7, #2 - 8.25) -- cycle;
    \node[anchor=south, text depth=0, font=\fontsize{32}{32}] at (#1 + 9.95, #2 + 6.25) {
        \textbf{Layerwise relevance propagation}
    };
}

\newcommand{\heatmap}[2]{
    \node[yslant=1, inner sep=0pt] (map) at (#1, #2) {
        \includegraphics[width=5cm, height=8.5cm]{data/heatmap.png}
    };
    \draw[fill=black] (map.north east) --
        ($ (map.north east) - (\mridepth, 0) $) --
        ($ (map.north west) - (\mridepth, 0) $) --
        (map.north west) -- cycle;
    \draw[fill=black] (map.north west) --
        ($ (map.north west) - (\mridepth, 0) $) --
        ($ (map.south west) - (\mridepth, 0) $) --
        (map.south west) -- cycle;
    \draw[] (map.north east) --
        ($ (map.north east) + (\mridepth, 0) $) --
        ($ (map.north west) + (\mridepth, 0) $) --
        (map.north west) -- cycle;
    \draw[]  ($ (map.north west) + (\mridepth, 0) $) --
        ($ (map.south west) + (\mridepth, 0) $) --
        (map.south west);
    \draw[] ($ (map.north east) + (\mridepth, 0) $) --
        ($ (map.south east) + (\mridepth, 0) $) --
        ($ (map.south west) + (\mridepth, 0) $);
    \node[
        anchor=south,
        align=center,
        font=\fontsize{32}{32}\linespread{0.8}\selectfont
    ] at (map.north east) {
        Explanatory\\
        heatmaps
    };
}


\begin{figure}[h]
    \begin{tikzpicture}
        \mriinput{-1.2}{-1.15}

        \cnnarrow{(-1.2, -1.15)}{(7, -1.15)}

        \cnn{6}{1}

        \node[align=center,font=\fontsize{32}{32}] at (34, -1.15) {
            \textit{Predicted probability}\\\textit{of dementia}
        };
        \cnnarrow{(39.7, -1.15)}{(53, -1.15)}

        \lrp{41.5}{-0.15}
        \cnnarrow{(59.43, -1.15)}{(67, -1.15)}

        \heatmap{68.6}{-1.15}
    \end{tikzpicture}
\end{figure}

        %     \end{column}
        % \end{columns}

        % \vspace{\verticalspace}

        % % First column
        % \begin{columns}[t]
        % \begin{column}{.33\textwidth}

        %     \begin{block}{Methods}
        %         \parbox{\linewidth}{\justify
        % We compiled structural MRI scans from a balanced set of 1708 dementia patients and healthy controls, and fit
        % a simple fully convolutional network (SFCN) to differentiate between them. Next, we implemented LRP on top of the trained
        % model to generate explanations in the form of heatmaps, accompanying its predictions. We validated the heatmaps by comparing an
        % average map compiled from all true positives to a statistical reference map constructed with a GingerALE meta-analysis,
        % containing spatial locations with observed deviations in dementia from 124 relevant publications. Following the validation, we employed the explainable pipeline
        % in an exploratory analysis of 1256 patients with mild cognitive impairment (MCI). Here, we utilized its predictions and heatmaps
        % to predict progression to dementia in the 5 years following the scan, and to investigate associations between spatial variability in the heatmaps
        % and impairments in specific cognitive domains (Figure \ref{fig:outline}).
        %         }
        %     \end{block}

        % \end{column}

        % % Second column
        % \begin{column}{.63\textwidth}
        %     \begin{block}{Results}
        %         \begin{columns}[t]
        %             \begin{column}{.45\textwidth}
        %                 \parbox[t]{\textwidth}{\justify
        %                 \begin{itemize}[leftmargin=0em,labelindent=\parindent]

        %                     \item[\textbullet] The best performing classifier was able to differentiate dementia
        %                     patients from controls with an out-of-sample AUC of 0.9.
        %                     \vspace{1.7cm}
        %                     \begin{figure}
        %                         \begin{tikzpicture}
        %                             \def\mriwidth{6.25cm}
        %                             \node[inner sep=0pt, outer sep=0pt] (zeroth) at (0, 0) {
        %                                 \includegraphics[
        %                                     height=\mriwidth,
        %                                     clip=true,
        %                                     trim = 128mm 232mm 64mm 0mm
        %                                 ]{data/test_90.png}
        %                             };
        %                             \node[inner sep=0pt, outer sep=0pt, anchor=west] (first) at (zeroth.east) {
        %                                 \includegraphics[
        %                                     height=\mriwidth,
        %                                     clip=true,
        %                                     trim = 192mm 232mm 0mm 0mm
        %                                 ]{data/test_90.png}
        %                             };
        %                             \node[inner sep=0pt, outer sep=0pt, anchor=west] (second) at (first.east) {
        %                                 \includegraphics[
        %                                     height=\mriwidth,
        %                                     clip=true,
        %                                     trim = 0mm 157mm 192mm 77mm
        %                                 ]{data/test_90.png}
        %                             };
        %                             \draw[fill=black] (zeroth.north west) -- (second.north east) -- ($ (second.south east) - (0, 1) $) -- ($ (zeroth.south west) -(0, 1) $) -- cycle;
        %                             \node[inner sep=0pt, outer sep=0pt] (zeroth) at (0, 0) {
        %                                 \includegraphics[
        %                                     height=\mriwidth,
        %                                     clip=true,
        %                                     trim = 128mm 232mm 64mm 0mm
        %                                 ]{data/test_90.png}
        %                             };
        %                             \node[inner sep=0pt, outer sep=0pt, anchor=west] (first) at (zeroth.east) {
        %                                 \includegraphics[
        %                                     height=\mriwidth,
        %                                     clip=true,
        %                                     trim = 192mm 232mm 0mm 0mm
        %                                 ]{data/test_90.png}
        %                             };
        %                             \node[inner sep=0pt, outer sep=0pt, anchor=west] (second) at (first.east) {
        %                                 \includegraphics[
        %                                     height=\mriwidth,
        %                                     clip=true,
        %                                     trim = 0mm 157mm 192mm 77mm
        %                                 ]{data/test_90.png}
        %                             };

        %                             \node[text=white, text depth=0] at ($ (first.south) - (0, 0.5) $) (overlap-text) {Overlap};
        %                             \node[fill=yellow, anchor=east] at ($ (overlap-text.west) - (0.2, 0) $) (overlap) {};
        %                             \node[text=white, text depth=0, anchor=east] at ($ (overlap.west) - (0.4, 0) $) (lrp-text) {LRP};
        %                             \node[fill=green, anchor=east] at ($ (lrp-text.west) - (0.2, 0) $) (lrp) {};
        %                             \node[fill=red, anchor=west] at ($ (overlap-text.east) + (0.4, 0) $) (ale) {};
        %                             \node[text=white, text depth=0, anchor=west] at ($ (ale.east) + (0.2, 0) $) (ale-text) {ALE};
        %                         \end{tikzpicture}
        %                         \centering
        %                         \caption{%
        %                             \parbox{\textwidth}{\justify%
        %                                 \textbf{Figure~\thefigure:}~ Three axial brain slices, showing the concordance between the average heatmap from our pipeline and the statistical reference map from GingerALE.
        %                             }
        %                         }\label{fig:overlap}
        %                     \end{figure}
        %                     \vspace{0.8cm}
        %                     \item[\textbullet] The average heatmap for dementia patients highly resembled the statistical reference map (Figure \ref{fig:overlap}),
        %                     yielding a normalized cross-correlation of 0.64.
        %                 \end{itemize}
        %                 }
        %             \end{column}
        %             \begin{column}{.45\textwidth}
        %                 \parbox[t]{\textwidth}{\justify
        %                 \begin{itemize}[leftmargin=0em,labelindent=\parindent]
        %                     \item[\textbullet] In MCI patients, we predicted
        %                     progression within 5 years with an AUC of 0.9 (Figure \ref{fig:prognosis}).
        %                     \item[\textbullet] Inter-individual variation in the heatmaps were associated with distinct patterns of performance on
        %                     neuropsychological tests (Figure 3).
        %                     \vspace{0.3cm}
        %                     \begin{figure}
        %                         \definecolor{color0}{rgb}{0.62, 0.004, 0.259}
        %                         \definecolor{color1}{rgb}{0.755, 0.154, 0.291}
        %                         \definecolor{color2}{rgb}{0.866, 0.29, 0.298}
        %                         \definecolor{color3}{rgb}{0.943, 0.406, 0.268}
        %                         \definecolor{color4}{rgb}{0.975, 0.557, 0.323}
        %                         \definecolor{color5}{rgb}{0.993, 0.709, 0.403}
        %                         \definecolor{color6}{rgb}{0.995, 0.832, 0.506}
        %                         \definecolor{color7}{rgb}{0.998, 0.926, 0.625}
        %                         \definecolor{color8}{rgb}{0.998, 0.999, 0.746}
        %                         \definecolor{color9}{rgb}{0.937, 0.975, 0.65}
        %                         \definecolor{color10}{rgb}{0.838, 0.935, 0.609}
        %                         \definecolor{color11}{rgb}{0.693, 0.876, 0.639}
        %                         \definecolor{color12}{rgb}{0.527, 0.811, 0.645}
        %                         \definecolor{color13}{rgb}{0.368, 0.725, 0.662}
        %                         \definecolor{color14}{rgb}{0.24, 0.582, 0.721}
        %                         \definecolor{color15}{rgb}{0.267, 0.441, 0.698}
        %                         \definecolor{color16}{rgb}{0.369, 0.31, 0.635}

        %                         \newcommand{\mriwidth}{4.3cm}
        %                         \newcommand{\gap}{-0.2cm}

        %                         \newcommand{\correlationplot}[4]{
        %                             \begin{tikzpicture}
        %                                 \begin{axis}[
        %                                     height=1.1 * \mriwidth,
        %                                     width=1.365 * \mriwidth,
        %                                     xmajorticks=false,
        %                                     ylabel=####3,
        %                                     ytick={0, 2, 4, 6, 8},
        %                                     yticklabels=####2,
        %                                     xmin=-1,
        %                                     xmax=17,
        %                                     ymin=0,
        %                                     ymax=9,
        %                                     every tick label/.append style={font=\footnotesize},
        %                                     ytick pos=left,
        %                                     scatter/classes={
        %                                         ADNI_EF={color0, draw=black},
        %                                         ADNI_MEM={color1, draw=black},
        %                                         CDCARE={color2, draw=black},
        %                                         CDCOMMUN={color3, draw=black},
        %                                         CDGLOBAL={color4, draw=black},
        %                                         CDHOME={color5, draw=black},
        %                                         CDJUDGE={color6, draw=black},
        %                                         CDMEMORY={color7, draw=black},
        %                                         CDORIENT={color8, draw=black},
        %                                         FAQTOTAL={color9, draw=black},
        %                                         GDTOTAL={color10, draw=black},
        %                                         MMSCORE={color11, draw=black},
        %                                         NPISCORE={color12, draw=black},
        %                                         PHC_EXF={color13, draw=black},
        %                                         PHC_LAN={color14, draw=black},
        %                                         PHC_MEM={color15, draw=black},
        %                                         PHC_VSP={color16, draw=black}
        %                                     },
        %                                     y label style={at={(-0.1,0.5)}},
        %                                     ymajorgrids=true,
        %                                     ytick style={draw=none},
        %                                     clip=false,
        %                                     grid style={draw=gray!20},
        %                                     axis line style={draw=gray!70}
        %                                 ]
        %                                     \addplot[
        %                                         only marks,
        %                                         scatter,
        %                                         scatter src=explicit symbolic,
        %                                         mark size=4pt
        %                                     ] table [
        %                                         col sep=comma,
        %                                         x=index,
        %                                         y=component_####1,
        %                                         meta=symptom
        %                                     ] {data/correlations.csv};
        %                                     \addplot[dashed,red, thick] coordinates {
        %                                         (-1, 2.76)
        %                                         (17, 2.76)
        %                                     };
        %                                     ####4
        %                                 \end{axis}
        %                             \end{tikzpicture}
        %                         }

        %                         \newsavebox{\firstcorrelations}
        %                         \sbox{\firstcorrelations}{%
        %                             \correlationplot{0}{{0, 2, 4, 6, 8}}{\footnotesize{$-log_{10}(p)$}}{
        %                                 \node[] at (axis cs: 14, 6.77) {\footnotesize{PHC LAN}};
        %                             }
        %                         }
        %                         \newsavebox{\secondcorrelations}
        %                         \sbox{\secondcorrelations}{%
        %                             \correlationplot{1}{{,,}}{{}}{
        %                                 \node[] at (axis cs: 9, 4.3) {\footnotesize{FAQTOTAL}};
        %                             }
        %                         }
        %                         \newsavebox{\thirdcorrelations}
        %                         \sbox{\thirdcorrelations}{%
        %                             \correlationplot{2}{{,,}}{{}}{
        %                                 \node[] at (axis cs: 0, 7) {\footnotesize{ADNI EF}};
        %                                 \node[] at (axis cs: 13, 8.51) {\footnotesize{PHC EXF}};
        %                             }
        %                         }

        %                         \begin{tikzpicture}
        %                             \node[] (first) at (0, 0) {
        %                                 \includegraphics[
        %                                     width=\mriwidth,
        %                                     clip=true,
        %                                     trim = 192mm 232mm 0mm 0mm
        %                                 ]{data/components/component_0.png}
        %                             };
        %                             \node[anchor=north west] (first-correlation) at ($ (first.south west) + (-1.75, 0.3) $) {
        %                                 \usebox{\firstcorrelations}
        %                             };

        %                             \node[anchor=west] (second) at ($ (first.east) + (\gap, 0) $) {
        %                                 \includegraphics[
        %                                     width=\mriwidth,
        %                                     clip=true,
        %                                     trim = 192mm 232mm 0mm 0mm
        %                                 ]{data/components/component_1.png}
        %                             };
        %                             \node[anchor=north west] (second-correlation) at ($ (first-correlation.north east) - (2.02, 0.11) $) {
        %                                 \usebox{\secondcorrelations}
        %                             };

        %                             \node[anchor=west] (third) at ($ (second.east) + (\gap, 0) $) {
        %                                 \includegraphics[
        %                                     width=\mriwidth,
        %                                     clip=true,
        %                                     trim = 192mm 232mm 0mm 0mm
        %                                 ]{data/components/component_2.png}
        %                             };
        %                             \node[anchor=north west] (third-correlation) at ($ (second-correlation.north east) - (1.86, -0.29) $) {
        %                                 \usebox{\thirdcorrelations}
        %                             };
        %                         \end{tikzpicture}
        %                         \caption{%
        %                             \parbox{\textwidth}{\justify%
        %                                 \textbf{Figure~\thefigure:}~ Three prototypical heatmaps and the strength of
        %                                 their associations with performance on cognitive
        %                                 tests. PHC LAN: Composite language score; FAQTOTAL:
        %                                 Functional Activities Questionnaire; ADNI EF/PHC
        %                                 EXF: Executive function scores
        %                             }
        %                         }\label{fig:associations}
        %                     \end{figure}
        %                 \end{itemize}
        %                 }
        %             \end{column}
        %         \end{columns}
        %     \end{block}
        % \end{column}
        % \end{columns}

        % \vspace{\verticalspace}

        % \begin{columns}
        %     \begin{column}{.6\textwidth}
        %         \centering
        %         \definecolor{cases-default}{HTML}{EB5353}
        %         \definecolor{controls-default}{HTML}{0079FF}
        %         \definecolor{healthy-default}{HTML}{36AE7C}
        %         \definecolor{baseline}{HTML}{FAEAB1}
        %         \definecolor{preds}{HTML}{E5BA73}
        %         \definecolor{maps}{HTML}{C58940}

        %         \def\marksize{8pt}

        %         \newsavebox{\resultsbox}
        %             \sbox{\resultsbox}{%
        %             \begin{tikzpicture}
        %                 \begin{axis}[
        %                     height=7cm,
        %                     width=12.5cm,
        %                     xmajorticks=false,
        %                     xmin=0.5,
        %                     xmax=5.5,
        %                     ymin=0,
        %                     ymax=1,
        %                     ylabel=AUC,
        %                     ymajorticks=false,
        %                     ymajorgrids=true,
        %                     ytick={0.25, 0.50, 0.75},
        %                     axis background/.style={fill=white}
        %                 ]
        %                     \addplot[mark=*, draw=black, mark options={fill=baseline}, mark size=\marksize] coordinates {
        %                         (1, 0.506)
        %                         (2, 0.474)
        %                         (3, 0.536)
        %                         (4, 0.529)
        %                         (5, 0.515)
        %                     };\label{trace:baseline}
        %                     % \addplot[mark=*, draw=black, mark options={fill=preds}, mark size=\marksize, opacity=0.5] coordinates {
        %                     %     (1, 0.666)
        %                     %     (2, 0.742)
        %                     %     (3, 0.797)
        %                     %     (4, 0.844)
        %                     %     (5, 0.889)
        %                     % };
        %                     \addplot[mark=*, draw=black, mark options={fill=maps}, mark size=\marksize] coordinates {
        %                         (1, 0.743)
        %                         (2, 0.786)
        %                         (3, 0.808)
        %                         (4, 0.867)
        %                         (5, 0.903)
        %                     };\label{trace:maps}
        %                     \node[anchor=north, inner sep=12pt] at (axis cs: 1, 0.506) {\footnotesize{0.50}};
        %                     \node[anchor=north, inner sep=12pt] at (axis cs: 2, 0.474) {\footnotesize{0.47}};
        %                     \node[anchor=north, inner sep=12pt] at (axis cs: 3, 0.536) {\footnotesize{0.53}};
        %                     \node[anchor=north, inner sep=12pt] at (axis cs: 4, 0.529) {\footnotesize{0.52}};
        %                     \node[anchor=north, inner sep=12pt] at (axis cs: 5, 0.515) {\footnotesize{0.51}};
        %                     \node[anchor=north, inner sep=12pt] at (axis cs: 1, 0.743) {\footnotesize{0.74}};
        %                     \node[anchor=north, inner sep=12pt] at (axis cs: 2, 0.786) {\footnotesize{0.78}};
        %                     \node[anchor=north, inner sep=12pt] at (axis cs: 3, 0.808) {\footnotesize{0.80}};
        %                     \node[anchor=north, inner sep=12pt] at (axis cs: 4, 0.867) {\footnotesize{0.86}};
        %                     \node[anchor=north, inner sep=12pt] at (axis cs: 5, 0.903) {\footnotesize{0.90}};
        %                 \end{axis}
        %             \end{tikzpicture}
        %         }
        %         \begin{figure}
        %         \begin{tikzpicture}
        %             \begin{axis}[
        %                 height=0.408\textwidth,
        %                 width=0.9\textwidth,
        %                 xlabel={Age},
        %                 ylabel={Cognitive function},
        %                 ticks=none,
        %                 axis x line=bottom,
        %                 axis y line=left,
        %                 y axis line style={-|},
        %                 xmin=0,
        %                 xmax=1.4,
        %                 ymin=0,
        %                 ymax=1,
        %                 clip=false
        %             ]
        %             \addplot[draw=healthy-default, smooth, line width=15pt, opacity=0.5] coordinates {
        %                 (0, 0.9)
        %                 (0.25, 0.87)
        %                 (0.5, 0.77)
        %                 (0.6, 0.72)
        %                 (0.8, 0.63)
        %                 (0.9, 0.72)
        %                 (1.4, 0.67)
        %             };
        %             \addplot[draw=controls-default, smooth, line width=15pt, opacity=0.5] coordinates {
        %                 (0, 0.9)
        %                 (0.25, 0.87)
        %                 (0.5, 0.77)
        %                 (0.6, 0.72)
        %                 (0.8, 0.63)
        %                 (0.9, 0.61)
        %                 (1.4, 0.54)
        %             };
        %             \addplot[draw=cases-default, smooth, line width=15pt, opacity=0.5] coordinates {
        %                 (0, 0.9)
        %                 (0.25, 0.87)
        %                 (0.5, 0.77)
        %                 (0.6, 0.72)
        %                 (0.8, 0.625)
        %                 (1.1, 0.48)
        %                 (1.4, 0.3)
        %             };
        %             \addplot[dashed] coordinates {
        %                 (0, 0.65)
        %                 (1.4, 0.65)
        %             };
        %             \addplot[dashed] coordinates {
        %                 (0, 0.4)
        %                 (1.4, 0.4)
        %             };
        %             \node[anchor=south west] at (axis cs: 0.1, 0.64) {Normal cognition};
        %             \node[anchor=north west] at (axis cs: 0.1, 0.66) {Mild cognitive impairment};
        %             \node[anchor=north west] at (axis cs: 0.1, 0.41) {Dementia};
        %             \node[
        %                 align=center,
        %                 font=\linespread{0},
        %                 text=healthy-default
        %             ] at (axis cs: 1.51, 0.67) {Improving\\[-10pt](n=80)};
        %             \node[
        %                 align=center,
        %                 font=\linespread{0},
        %                 text=controls-default
        %             ] at (axis cs: 1.51, 0.53) {Stable\\[-10pt](n=754)};
        %             \node[
        %                 align=center,
        %                 font=\linespread{0},
        %                 text=cases-default
        %             ] at (axis cs: 1.51, 0.3) {Progressive\\[-10pt](n=304)};
        %             \draw[-{Stealth[length=10pt, width=6pt, inset=3pt]}, red, thick] (axis cs: 0.8, 0.8) -- (axis cs: 0.8, 0.67);
        %             \node[anchor=south] at (axis cs: 0.8, 0.8) {\textcolor{red}{t}};
        %             \draw[densely dotted] (axis cs: 0.9, 0.8) -- (axis cs: 0.9, 0.3);
        %             \draw[densely dotted] (axis cs: 1, 0.8) -- (axis cs: 1, 0.3);
        %             \draw[densely dotted] (axis cs: 1.1, 0.8) -- (axis cs: 1.1, 0.3);
        %             \draw[densely dotted] (axis cs: 1.2, 0.8) -- (axis cs: 1.2, 0.3);
        %             \draw[densely dotted] (axis cs: 1.3, 0.8) -- (axis cs: 1.3, 0.3);
        %             \node[anchor=south] at (axis cs: 0.9, 0.8) {t+1};
        %             \node[anchor=south] at (axis cs: 1, 0.8) {t+2};
        %             \node[anchor=south] at (axis cs: 1.1, 0.8) {t+3};
        %             \node[anchor=south] at (axis cs: 1.2, 0.8) {t+4};
        %             \node[anchor=south] at (axis cs: 1.3, 0.8) {t+5};
        %             \node[] at (axis cs: 1.08, 0.155) {
        %                 \usebox{\resultsbox}
        %             };
        %             \node[] at (axis cs: 1.7, 0.6) {};
        %             \end{axis}
        %         \end{tikzpicture}
        %         \centering
        %         \caption{%
        %             \parbox{\textwidth}{\justify%
        %                 \textbf{Figure~\thefigure:}~Clinical trajectories observed in the
        %                 MCI sample. Embedded is the performance of the prognostic models
        %                 for each year, the baseline model (\ref{trace:baseline}) and the
        %                 model employing information from the pipeline (\ref{trace:maps}).
        %             }
        %         }\label{fig:prognosis}
        %     \end{figure}
        %     \end{column}
        %     \begin{column}{.3\textwidth}
        %         \begin{block}{Conclusion}
        %             \parbox{\textwidth}{\justify%
        %             Our explainable pipeline for dementia prediction allowed us to accurately
        %             \textbf{characterize the manifestation of dementia in individual patients}. When
        %             employing the pipeline in a sample of patients with MCI,
        %             information derived from it allowed us to \textbf{predict progression of the disease},
        %             and revealed \textbf{associations between heterogeneity in the brain and impairments in distinct cognitive domains}.
        %             Our study presents an empirical foundation for further investigations into how
        %             explainable artificial intelligence can play an important role in precise personalized
        %             diagnosis of heterogeneous neurological disorders.
        %             }
        %         \end{block}
        %     \end{column}
        % \end{columns}

        % \vspace{\verticalspace}

        % \vfill
        % \begin{beamercolorbox}[sep=0pt,wd=\textwidth]{title}
        %     \begin{tikzpicture}
        %         \node[] at (-29, 0) {};
        %         \node[] at (0, 0) {
        %             \includegraphics[height=3cm]{
        %                 data/qr.png
        %             }
        %         };
        %         \node[anchor=east] at (29, 0) {
        %             \includegraphics[height=3cm]{data/uio.png}
        %         };
        %     \end{tikzpicture}
        % \end{beamercolorbox}

    \end{frame}

\end{document}
