
\documentclass{beamer}

\usetheme{PSY9511}

\usepackage{emoji}
\usepackage{pgfplots}
\usepackage{tikz}
\usepackage{xcolor}

\usepgfplotslibrary{colormaps}

\usetikzlibrary{calc}
\usetikzlibrary{matrix}

\title{PSY9511: Seminar 5}
\subtitle{Beyond linearity: Extensions of linear models and tree-based models}
\author{Esten H. Leonardsen}
\date{27.10.24}

\pgfmathdeclarefunction{gamma}{1}{%
  \pgfmathparse{(#1-1)!}%
}

\pgfplotsset{
    colormap={blue red}{
        rgb255(0cm)=(0,0,255);
        rgb255(1cm)=(255,0,0)
    }
}

\begin{document}
	\begin{frame}
	 	\maketitle
	\end{frame}

    \begin{frame}{Outline}
        \begin{enumerate}
            \item Exercise 3
            \item Exercise 4
            \item Recap
            \item Extensions of linear models
            \begin{enumerate}
                \item Generalized linear models (GLMs)
                \item Generalized additive models (GAMs)
            \end{enumerate}
            \item Tree-based models
            \begin{enumerate}
                \item Decision trees
                \item Random forests
                \alert<2>{\item Gradient boosting (XGBoost)}
            \end{enumerate}
            \item Exercise 5
        \end{enumerate}
    \end{frame}

    \begin{frame}{Exercise solutions}
        \centering
        \vfill
        \small{\url{https://uio.instructure.com/courses/59550}}
        \vfill
    \end{frame}

    \begin{frame}{Exercise 3: Solution}
        \centering
        \vfill
        \scriptsize{\url{http://localhost:8888/notebooks/notebooks/Solution\%203.ipynb}}
        \vfill
    \end{frame}

    \section{Exercise 4}

\begin{frame}{Exercise 4: Solution}
    \centering
    \vfill
    \scriptsize{\url{http://localhost:8888/notebooks/notebooks/Solution\%204.ipynb}}
    \vfill
\end{frame}

    \newcommand{\mpgvsweight}[1]{
    \begin{tikzpicture}
        \begin{axis}[
            height=7cm,
            width=10cm,
            xlabel=$x$ (horsepower),
            ylabel=$y$ (mpg),
            ytick pos=left,
            xtick pos=bottom,
            xmin=30,
            xmax=245,
            ymin=6,
            ymax=51
        ]

            \addplot[
                only marks,
                cyan,
                opacity=0.5
            ] table [
                col sep=comma,
                x=horsepower,
                y=mpg
            ] {data/Auto.csv};

            \ifnum#1=1
                \draw[very thick, red] (axis cs: 30, 39-0.157*30) -- (axis cs: 245, 39-0.157*245);
                \node[text=red] at (axis cs: 180, 42) {
                    $\hat{f}(x) = 39 - 0.157x$
                };
            \fi
            \ifnum#1=2
                \addplot[very thick, red, domain=30:245, samples=1000] {0.0012*x^2-0.46*x+56};
                \node[text=red] at (axis cs: 180, 42) {
                    $\hat{f}(x) = 56+0.0012x^2-0.46x$
                };
            \fi
        \end{axis}
    \end{tikzpicture}
}

\newcommand{\modeltypes}{
    \begin{tikzpicture}
        \begin{axis}[
            height=5cm,
            width=10cm,
            xlabel=$x$ (horsepower),
            ylabel=$y$ (mpg),
            ytick pos=left,
            xtick pos=bottom,
            xmin=30,
            xmax=245,
            ymin=6,
            ymax=51
        ]

        \addplot[
            only marks,
            cyan,
            opacity=0.5
        ] table [
            col sep=comma,
            x=horsepower,
            y=mpg
        ] {data/Auto.csv};

        \draw[very thick, red] (axis cs: 30, 39-0.157*30) -- (axis cs: 245, 39-0.157*245);
        \node[align=left] at (axis cs: 180, 35) {
            $
            \begin{aligned}
                {\color{red}\hat{f}(x) }&{\color{red}= 39 - 0.157x}\\
                {\color{teal}\hat{f}(x) }&{\color{teal}= \ldots}\\
            \end{aligned}
            $
        };

        \addplot[very thick, teal, smooth] coordinates {
            (6.0, 243.2069751806266)
            (15.958333333333334, 118.86132183998052)
            (25.916666666666668, 55.54830308937471)
            (35.875, 33.03657403682814)
            (45.833333333333336, 31.094789790600284)
            (55.79166666666667, 32.62203312809545)
            (65.75, 32.322995894942245)
            (75.70833333333334, 26.789726618159705)
            (85.66666666666667, 25.56443372228785)
            (95.625, 23.411114113578773)
            (105.58333333333334, 21.972291293527256)
            (115.54166666666667, 22.0202279029072)
            (125.5, 19.623191768387127)
            (135.45833333333334, 17.796535374305538)
            (145.41666666666669, 15.42464063855285)
            (155.375, 13.83975395598323)
            (165.33333333333334, 13.58266973340266)
            (175.29166666666669, 14.577234897250934)
            (185.25, 13.44873671948327)
            (195.20833333333334, 11.13569858732588)
            (205.16666666666669, 10.948027264126887)
            (215.125, 12.746078206932541)
            (225.08333333333334, 14.710541046885707)
            (235.04166666666669, 16.775438769652766)
            (245.0, 18.301538120974897)
        };

        \end{axis}
    \end{tikzpicture}
}


\newcommand{\tradeoffplot}[1]{
    \definecolor{full}{HTML}{ef476f}
    \definecolor{bias}{HTML}{26547c}
    \definecolor{variance}{HTML}{06d6a0}
    \definecolor{train}{HTML}{ffd166}

    \begin{tikzpicture}
        \begin{axis}[
            height=6cm,
            width=9cm,
            ymajorticks=false,
            xmajorticks=false,
            ylabel=Model error,
            xlabel=Flexibility,
            axis lines=left,
            xmin=0,
            xmax=1,
            ymin=0,
            ymax=2.5,
            clip=false
        ]
            \node[] at (axis cs: 1.34, 2.9) {};

            \ifnum#1>0
                \draw[dotted] (axis cs: 0.1, 0) -- (axis cs: 0.1, 2.5);
                \node[anchor=south, align=center, font=\footnotesize\linespread{0.9}\selectfont] at (axis cs: 0.1, 2.5) {
                    Simple\\model
                };
                \addplot[
                    only marks,
                    mark options={
                        draw=black,
                        fill=bias,
                        scale=2
                    }
                ] coordinates {
                    (0.1, 0.5)
                    (1.05, 2.4)
                };
                \node[anchor=west] at (axis cs: 1.065, 2.4) {
                    Bias
                };

                \addplot[
                    only marks,
                    mark options={
                        draw=black,
                        fill=variance,
                        scale=2
                    }
                ] coordinates {
                    (0.1, 0.01114)
                    (1.05, 2.2)
                };
                \node[anchor=west] at (axis cs: 1.065, 2.2) {
                    Variance
                };
            \fi
            \ifnum#1>1
                \draw[dotted] (axis cs: 0.75, 0) -- (axis cs: 0.75, 2.5);
                \node[anchor=south, align=center, font=\footnotesize\linespread{0.9}\selectfont] at (axis cs: 0.75, 2.5) {
                    Complex\\model
                };
                \addplot[
                    only marks,
                    mark options={
                        draw=black,
                        fill=bias,
                        scale=2
                    }
                ] coordinates {
                    (0.75, 0.117)
                };

                \addplot[
                    only marks,
                    mark options={
                        draw=black,
                        fill=variance,
                        scale=2
                    }
                ] coordinates {
                    (0.75, 0.2873)
                };
            \fi
            \ifnum#1>2
                \addplot[
                    bias,
                    very thick,
                    domain=0:1,
                    samples=100
                ] {(1/(x+0.1))/10};
                \addplot[
                    variance,
                    very thick,
                    domain=0:1,
                    samples=100
                ] {(exp(x*5))/148};
            \fi
            \ifnum#1>3
                \draw[dashed] (axis cs: 0, 1.2) -- (axis cs: 1, 1.2);
                \node[anchor=west, align=left, font=\footnotesize\linespread{0.9}\selectfont] at (axis cs: 1, 1.2) {
                    Irreducible\\error
                };
            \fi
            \ifnum#1>4
                \addplot[
                    full,
                    very thick,
                    domain=0:1,
                    samples=100
                ] {((1/(x+0.1))/10) + ((exp(x*5))/148) + 1.2};

                \addplot[
                    only marks,
                    mark options={
                        draw=black,
                        fill=full,
                        scale=2
                    }
                ] coordinates {
                    (0.1, 1.71114)
                    (0.75, 1.6043)
                    (1.05, 2)
                };
                \ifnum#1<6
                    \node[anchor=west] at (axis cs: 1.065, 2) {
                        Total
                    };
                \fi
            \fi
            \ifnum#1>5
                \addplot[
                    train,
                    very thick,
                    domain=0:1,
                    samples=100
                ] {((1/(x+0.1))/10) + 1.2};

                \node[anchor=west] at (axis cs: 1.065, 2) {
                    Total (test)
                };

                \addplot[
                    only marks,
                    mark options={
                        draw=black,
                        fill=train,
                        scale=2
                    }
                ] coordinates {
                    (0.1, 1.7)
                    (0.75, 1.317)
                    (1.05, 1.8)
                };
                \node[anchor=west] at (axis cs: 1.065, 1.8) {
                    Total (train)
                };
            \fi

        \end{axis}
    \end{tikzpicture}
}

\begin{frame}{Recap}
    \begin{tikzpicture}
        \node[draw=black] at (-5.25, -3.5) {};
        \node[draw=black] at (5.25, 3.5) {};

        \visible<1-4>{
            \node[anchor=west, text width=10cm] at (-5, 0) {
                \textbf{What is statistical learning?}
                \begin{itemize}
                    \item<2-> \underline{Inferentiental view:} \alert<4>{Finding a function $\hat{f}(X)$} that describes the relationship between some input variables $X$ and an output variable $y$.
                    \item<3->{\underline{Predictive view:} \alert<4>{Finding a function $\hat{f}(X)$} that, when given a new set of inputs $X$ allows us to predict an output $y$.}
                \end{itemize}
            };
        }
        \visible<5>{
            \node[] at (0, 0) {
                \mpgvsweight{0}
            };
        }
        \visible<6>{
            \node[] at (0, 0) {
                \mpgvsweight{1}
            };
        }
        \visible<7>{
            \node[] at (0, 0) {
                \mpgvsweight{2}
            };
        }
        \visible<8>{
            \node[inner sep=0pt, draw=black, anchor=west] (x1) at (-5.25, 2.5) {
                \includegraphics[width=1cm]{data/cats_and_dogs/cat.2.jpg}
            };
            \node[inner sep=0pt, draw=black] (x2) at ($ (x1) - (0, 1.5) $) {
                \includegraphics[width=1cm]{data/cats_and_dogs/cat.0.jpg}
            };
            \node[inner sep=0pt, draw=black] (x3) at ($ (x1) - (0, 3) $) {
                \includegraphics[width=1cm]{data/cats_and_dogs/cat.1.jpg}
            };
            \node[inner sep=0pt, draw=black] (x3) at ($ (x1) - (0, 4.5) $) {
                \includegraphics[width=1cm]{data/cats_and_dogs/dog.0.jpg}
            };

            \node[inner sep=0pt, draw=black] (x5) at ($ (x1) - (-1.25, 0.75) $) {
                \includegraphics[width=1cm]{data/cats_and_dogs/dog.1.jpg}
            };
            \node[inner sep=0pt, draw=black] (x6) at ($ (x1) - (-1.25, 2.25) $) {
                \includegraphics[width=1cm]{data/cats_and_dogs/dog.2.jpg}
            };
            \node[inner sep=0pt, draw=black] (x7) at ($ (x1) - (-1.25, 3.75) $) {
                \includegraphics[width=1cm]{data/cats_and_dogs/cat.3.jpg}
            };

            \node[align=center, font=\scriptsize, draw=black] (sm) at ($ (x1) + (3.5, -0.55) $) {
                Supervised\\model
            };

            \draw[-stealth, gray!50, line width=3pt] (x6) to [in=270, out=0] (sm);
            \draw[-stealth, gray!50, line width=3pt] (sm) -- ($ (sm.east) + (1.1, 0) $);

            \node[inner sep=0pt, draw=black] (y1) at ($ (sm.center) + (5, 0.5) $) {
                \includegraphics[width=1cm]{data/cats_and_dogs/dog.0.jpg}
            };
            \node[anchor=west, inner sep=0pt, draw=black] (y2) at (y1.east) {
                \includegraphics[width=1cm]{data/cats_and_dogs/dog.1.jpg}
            };
            \node[anchor=north, inner sep=0pt, draw=black] at (y1.south) {
                \includegraphics[width=1cm]{data/cats_and_dogs/dog.2.jpg}
            };

            \node[inner sep=0pt, draw=black] (y4) at ($ (sm) + (2.5, 0.5) $) {
                \includegraphics[width=1cm]{data/cats_and_dogs/cat.0.jpg}
            };
            \node[anchor=west, inner sep=0pt, draw=black] (y5) at (y4.east) {
                \includegraphics[width=1cm]{data/cats_and_dogs/cat.1.jpg}
            };
            \node[anchor=north, inner sep=0pt, draw=black] (y6) at (y4.south) {
                \includegraphics[width=1cm]{data/cats_and_dogs/cat.2.jpg}
            };
            \node[anchor=north, inner sep=0pt, draw=black] (y7) at (y5.south) {
                \includegraphics[width=1cm]{data/cats_and_dogs/cat.3.jpg}
            };

            \node[anchor=south, text depth=0] at ($ (y1.north)!0.5!(y2.north) $) {
                Dogs
            };
            \node[anchor=south, text depth=0] at ($ (y4.north)!0.5!(y5.north) $) {
                Cats
            };

            \node[align=center, font=\scriptsize, draw=black] (um) at ($ (x1) + (3.5, -3.95) $) {
                Unsupervised\\model
            };

            \draw[-stealth, gray!50, line width=3pt] (x6) to [in=90, out=0] (um);
            \draw[-stealth, gray!50, line width=3pt] (um) -- ($ (um.east) + (1.1, 0) $);

            \node[inner sep=0pt, draw=black] at ($ (um.center) + (6, 0.55) $) {
                \includegraphics[width=1cm]{data/cats_and_dogs/cat.0.jpg}
            };
            \node[inner sep=0pt, draw=black] at ($ (um.center) + (5.95, -0.65) $) {
                \includegraphics[width=1cm]{data/cats_and_dogs/cat.1.jpg}
            };
            \node[inner sep=0pt, draw=black] at ($ (um.center) + (4.8, 0.4) $) {
                \includegraphics[width=1cm]{data/cats_and_dogs/dog.0.jpg}
            };
            \node[inner sep=0pt, draw=black] at ($ (um.center) + (4.7, -0.7) $) {
                \includegraphics[width=1cm]{data/cats_and_dogs/dog.1.jpg}
            };

            \node[inner sep=0pt, draw=black] at ($ (um.center) + (2.7, 1.2) $) {
                \includegraphics[width=1cm]{data/cats_and_dogs/dog.2.jpg}
            };
            \node[inner sep=0pt, draw=black] at ($ (um.center) + (2.55, 0) $) {
                \includegraphics[width=1cm]{data/cats_and_dogs/cat.2.jpg}
            };
            \node[inner sep=0pt, draw=black] at ($ (um.center) + (2.65, -1.1) $) {
                \includegraphics[width=1cm]{data/cats_and_dogs/cat.3.jpg}
            };
            \node[] at ($ (um.center) + (2.65, -1.87) $) {
                Cluster 1
            };
            \node[] at ($ (um.center) + (5.35, -1.45) $) {
                Cluster 2
            };
        }
        \visible<9>{
            \draw[dashed] (0, -3.5) -- (0, 3.5);
            \node[anchor=north] at (-2.675, 3.5) {\textbf{Regression}};
            \node[anchor=north] at (2.675, 3.5) {\textbf{Classification}};

            \node[ampersand replacement=\&, inner sep=0pt] (regdata) at (-2.675, 1) {
                \begin{tabular}{|c|}
                    \hline
                    $\boldsymbol{y}$\\
                    \hline
                    18\\
                    15\\
                    18\\
                    16\\
                    17\\
                    \hline
                \end{tabular}
            };
            \node[text width=5cm, align=flush center, font=\small\linespread{0.9}\selectfont, anchor=north] at (-2.675, -2.1) {
                The predictive target $y$ is a \textit{continuous} (or \textit{quantitative}) variable.
            };

            \node[ampersand replacement=\&, inner sep=0pt] (classdata) at (2.675, 1) {
                \begin{tabular}{|c|}
                    \hline
                    $\boldsymbol{y}$\\
                    \hline
                    cat\\
                    cat\\
                    dog\\
                    cat\\
                    dog\\
                    \hline
                \end{tabular}
            };
            \node[text width=5cm, align=flush center, font=\small\linespread{0.9}\selectfont, anchor=north] at (2.675, -2.1) {
                The predictive target $y$ is a \textit{categorical} (or \textit{qualitative}) variable.
            };
        }
        \visible<10>{
            \node[] at (0, 1.3) {
                \modeltypes
            };
            \node[text width=10.5cm] at (0, -2.2) {
                \begin{itemize}
                    \item \textbf{Parametric models} The function ${\color{red}\hat{f}(X)}$ is relatively simple and can be described by a small number of parameters.
                    \begin{itemize}
                        \item Linear regression: $\hat{f}(X) = \beta_0 + \beta_1x$
                    \end{itemize}
                    \item \textbf{Non-parametric models} The function ${\color{teal}\hat{f}(X)}$ is more complex and often relies directly on the data.
                \end{itemize}
            };
        }
        \visible<11>{
            \node[] at (0, 0) {
                \tradeoffplot{3}
            };
        }
        \visible<12>{
            \node[] at (0, 0) {
                \tradeoffplot{6}
            };
        }
    \end{tikzpicture}
\end{frame}

    \section{Extensions of linear models}


\newcommand{\datasetplot}[1]{
    \begin{tikzpicture}
        \begin{axis}[
            height=5cm,
            width=8cm,
            xmin=0,
            xmax=1,
            xtick pos=bottom,
            ytick pos=left,
            ymajorticks=false
        ]
            \ifnum#1=0
                \addplot[
                    only marks,
                    blue,
                    samples=100,
                    domain=0:1,
                    opacity=0.5
                ] (x, 20 + 5 * x + rand);
            \fi
            \ifnum#1=1
                \addplot[
                    only marks,
                    blue,
                    samples=100,
                    domain=0:1,
                    opacity=0.5
                ] (x, 20 + x^3 * 5 + rand);
            \fi
            \ifnum#1=2
                \addplot[
                    only marks,
                    blue,
                    samples=100,
                    domain=-1:1,
                    opacity=0.5
                ] coordinates {
                    (0.1, 0)
                    (0.15, 0)
                    (0.22, 0)
                    (0.4, 0)
                    (0.25, 0)
                    (0.55,0)
                    (0.43, 0)
                    (0.51, 0)
                };
                \addplot[
                    only marks,
                    blue,
                    samples=100,
                    domain=-1:1,
                    opacity=0.5
                ] coordinates {
                    (0.3, 1)
                    (0.47, 1)
                    (0.52, 1)
                    (0.65, 1)
                    (0.81, 1)
                    (0.84,1)
                    (0.91, 1)
                    (0.99, 1)
                };
            \fi
            \ifnum#1=3
                \addplot[
                    only marks,
                    blue,
                    samples=100,
                    domain=-1:1,
                    opacity=0.5
                ] coordinates {
                    (0.1, 0)
                    (0.15, 0)
                    (0.22, 0)
                    (0.4, 0)
                    (0.25, 0)
                    (0.55,0)
                    (0.43, 0)
                    (0.51, 0)
                };
                \addplot[
                    only marks,
                    blue,
                    samples=100,
                    domain=-1:1,
                    opacity=0.5
                ] coordinates {
                    (0.3, 1)
                    (0.47, 1)
                    (0.52, 1)
                    (0.65, 1)
                    (0.81, 1)
                    (0.84,1)
                    (0.91, 1)
                    (0.99, 1)
                };
                \addplot[
                    domain=0:1,
                    samples=100,
                    thick,
                    red
                ] {1 / (1 + exp(-(20 * (x - 0.5))))};
            \fi
            \ifnum#1=4
                \addplot[
                    only marks,
                    blue,
                    samples=100,
                    domain=0:1,
                    opacity=0.5
                ] (x, 20 + x^3 * 5 + rand);
                \addplot[
                    thick,
                    red,
                    samples=100,
                    domain=0:1
                ] (x, 20+x^3*5);
            \fi
        \end{axis}
    \end{tikzpicture}
}

\newsavebox{\linearbox}
\sbox{\linearbox}{
    \datasetplot{0}
}

\newsavebox{\exponentialbox}
\sbox{\exponentialbox}{
    \datasetplot{1}
}

\newsavebox{\logisticbox}
\sbox{\logisticbox}{
    \datasetplot{2}
}

\newsavebox{\logregbox}
\sbox{\logregbox}{
    \datasetplot{3}
}

\newsavebox{\expregbox}
\sbox{\expregbox}{
    \datasetplot{4}
}

\begin{frame}{Extensions of linear models: Motivation}
    \begin{tikzpicture}
        \node[draw=black] at (-5.25, -3.5) {};
        \node[draw=black] at (5.25, 3.5) {};

        \visible<1-3,5>{
            \node[] at (0, -2.5) {
                $\hat{y}=$$\beta_0+\sum\limits_{i=0}^p \beta_ix_i$
            };
        }

        \visible<2>{
            \node[] at (0, 1) {
                \usebox{\linearbox}
            };
        }
        \visible<3-4,12>{
            \node[] at (0, 1) {
                \usebox{\exponentialbox}
            };
        }
        \visible<4>{
            \node[] at (0, -2.5) {
                $\hat{y}=$\textcolor{red}{$\beta_0+\sum\limits_{i=0}^p \beta_ix_i$}
            };
        }
        \visible<5-7>{
            \node[] at (0, 1) {
                \usebox{\logisticbox}
            };
        }
        \visible<6>{
            \node[] at (0, -2.5) {
                $\log\left(\dfrac{p(X)}{1 - p(X)}\right)=\beta_0+\sum\limits_{i=0}^p \beta_ix_i$
            };
        }
        \visible<7-8>{
            \node[] at (0, -2.5) {
                $p(X)=\dfrac{e^{\left(\beta_0+\sum\limits_{i=0}^p \beta_ix_i\right)}}{1 + e^{\left(\beta_0+\sum\limits_{i=0}^p \beta_ix_i\right)}}$
            };
        }
        \visible<8-11>{
            \node[] at (0, 1) {
                \usebox{\logregbox}
            };
        }
        \visible<9>{
            \node[] at (0, -2.5) {
                $p(X)=\dfrac{e^{\left({\color{red}\beta_0+\sum\limits_{i=0}^p \beta_ix_i} \right)}}{1 + e^{\left({\color{red}\beta_0+\sum\limits_{i=0}^p \beta_ix_i}\right)}}$
            };
        }
        \visible<10>{
            \node[] at (0, -2.5) {
                $p(X)=f(\beta_0+\sum\limits_{i=0}^p \beta_ix_i)$
            };
        }
        \visible<11-12>{
            \node[] at (0, -2.5) {
                $f(\hat{y})=\beta_0+\sum\limits_{i=0}^p \beta_ix_i$
            };
        }
        \visible<13>{
            \node[] at (0, 1) {
                \usebox{\expregbox}
            };
            \node[] at (0, -2.5) {
                $\log(\hat{y})=\beta_0+\sum\limits_{i=0}^p \beta_ix_i$
            };
        }
        \visible<14>{
            \node[align=flush left, text width=10.5cm] at (0, 0) {
                \textbf{Generalized linear models (GLMs):}\\
                Extends upon the regular linear model by associating the predictors to the response via a non-linear link function.
            };
        }
    \end{tikzpicture}
\end{frame}

    % \section{Tree-based models}

\newcommand{\treedataplot}[1]{
    \begin{tikzpicture}
        \begin{axis}[
            height=5cm,
            width=8cm,
            xlabel=$x_1$,
            ylabel=$x_2$,
            zlabel=$y$,
            view={75}{15}
        ]
            \addplot3 table [
                only marks,
                opacity=0.5,
                col sep=comma,
                x=x,
                y=y,
                z=value
            ] {data/treedata.csv};

            \ifnum#1>0
                \addplot3[
                    surf,
                    opacity=0.1,
                    red,
                    domain=0:0.5, y domain=0:0.5
                ] {0.8};

                \addplot3[
                    surf,
                    opacity=0.1,
                    red,
                    domain=0:0.5, y domain=0.5:1
                ] {0.9};
                \addplot3[
                    surf,
                    opacity=0.1,
                    red,
                    domain=0.5:1, y domain=0:0.5
                ] {0.5};
                \addplot3[
                    surf,
                    opacity=0.1,
                    red,
                    domain=0.5:1, y domain=0.5:1
                ] {0.1};
            \fi
        \end{axis}
    \end{tikzpicture}
}

\newcommand{\polyplot}[1]{
    \begin{tikzpicture}
        \begin{axis}[
            height=5cm,
            width=8cm,
            xlabel=$x$,
            ylabel=$y$,
            xmin=-0.1,
            xmax=1,
            xtick={-0.1, 0.175, 0.45, 0.725, 1},
            xticklabels={0, 0.25, 0.5, 0.75, 1},
            ymajorticks=false,
            ymin=-10,
            ymax=2.5,
            xtick pos=bottom
        ]
            \ifnum#1<3
                \addplot[
                    only marks,
                    blue,
                    samples=100,
                    domain=-0.1:1,
                    opacity=0.5
                ] (x, 300*x^5 - 750 * x^4 + 630 * x^3 - 210 * x^2 + 20 * x + 1 + rand);
            \fi

            \ifnum#1=1
                \addplot[
                    very thick,
                    red
                ] coordinates {
                    (-0.1,-3)
                    (-0.05, -3)
                    (-0.05, 0)
                    (0, 0)
                    (0, 1.5)
                    (0.2, 1.5)
                    (0.2, -0.5)
                    (0.65, -0.5)
                    (0.65, -1)
                    (0.7, -1)
                    (0.7, -3)
                    (0.85, -3)
                    (0.85, -6)
                    (0.95, -6)
                    (0.95, -9)
                    (1, -9)
                };
            \fi
            \ifnum#1>1
                \addplot[
                    very thick,
                    red
                ] coordinates {
                    (-0.1,-3)
                    (-0.05, -3)
                    (-0.05, 0)
                    (0, 0)
                    (0, 1)
                    (0.05, 1)
                    (0.05, 1.5)
                    (0.15, 1.5)
                    (0.15, 1)
                    (0.2, 1)
                    (0.2, 0)
                    (0.4, 0)
                    (0.4, -0.5)
                    (0.65, -0.5)
                    (0.65, -1)
                    (0.7, -1)
                    (0.7, -2.5)
                    (0.75, -2.5)
                    (0.75, -4)
                    (0.85, -4)
                    (0.85, -6)
                    (0.95, -6)
                    (0.95, -9)
                    (1, -9)
                };
            \fi
            \ifnum#1=4
                \node[anchor=south, font=\scriptsize, inner sep=2pt] at (axis cs: 0.525, -0.5) {
                    $\hat{y}=-0.5$
                };
                \draw[red] (axis cs: 0.4, 2.5) -- (axis cs: 0.4, -10);
                \draw[red] (axis cs: 0.65, 2.5) -- (axis cs: 0.65, -10);

                \node[anchor=south, font=\scriptsize, inner sep=2pt] at (axis cs: 0.3, -0.0) {
                    $\hat{y}=0.0$
                };
                \draw[red] (axis cs: 0.2, 2.5) -- (axis cs: 0.2, -10);
                \draw[red] (axis cs: 0.4, 2.5) -- (axis cs: 0.4, -10);
            \fi
        \end{axis}
    \end{tikzpicture}
}

\newsavebox{\treedata}
\sbox{\treedata}{
    \treedataplot{0}
}

\newsavebox{\treestep}
\sbox{\treestep}{
    \treedataplot{1}
}

\newsavebox{\polydata}
\sbox{\polydata}{
    \polyplot{0}
}

\newsavebox{\polystep}
\sbox{\polystep}{
    \polyplot{1}
}

\newsavebox{\polyprecise}
\sbox{\polyprecise}{
    \polyplot{2}
}
\newsavebox{\polyconstant}
\sbox{\polyconstant}{
    \polyplot{3}
}
\newsavebox{\polyregions}
\sbox{\polyregions}{
    \polyplot{4}
}

\newsavebox{\emptyplot}
\sbox{\emptyplot}{
    \begin{tikzpicture}
        \begin{axis}[
            height=5cm,
            width=8cm,
            xlabel=$x$,
            ylabel=$y$,
            ymajorticks=false,
            xmajorticks=false,
            xmin=0,
            xmax=1,
            ymin=0,
            ymax=1
        ]
            \node[] at (axis cs: 0.5, 0.5) {
                \Huge{?}
            };
        \end{axis}
    \end{tikzpicture}
}

\begin{frame}{Tree-based models: Motivation}
    \begin{tikzpicture}
        \node[draw=black] at (-5.25, -3.5) {};
        \node[draw=black] at (5.25, 3.5) {};

        \visible<1-2>{
            \node[] at (0, 1) {
                \usebox{\treedata}
            };
        }
        \visible<2>{
            \node[font=\scriptsize\selectfont] at (0, -2.5) {
                $y=
                \begin{cases}
                    0.8&x_1\leq0.5\ \&\ x_2\leq0.5\\
                    0.9&x_1\leq0.5\ \&\ x_2>0.5\\
                    0.5&x_1>0.5\ \&\ x_2\leq0.5\\
                    0.1&x_1>0.5\ \&\ x_2>0.5\\
                \end{cases}
                $
            };
        }
        \visible<3-4>{
            \node[] at (0, 1) {
                \usebox{\polydata}
            };
        }
        \visible<4>{
            \node[] at (0, -2.5) {
                $y=?x^5+?x^4+?x^3+?x^2+?x+?$
            };
        }
        \visible<5>{
            \node[] at (0, 1) {
                \usebox{\emptyplot}
            };
        }
        \visible<6>{
            \node[] at (0, 1) {
                \usebox{\treestep}
            };
        }
        \visible<7>{
            \node[] at (0, 1) {
                \usebox{\polystep}
            };
        }
        \visible<8>{
            \node[] at (0, 1) {
                \usebox{\polyprecise}
            };
        }
        \visible<9>{
            \node[] at (0, 1) {
                \usebox{\polyconstant}
            };
        }
        \visible<10-11>{
            \node[] at (0, 1) {
                \usebox{\polyregions}
            };
        }
        \visible<11>{
            \node[font=\footnotesize\selectfont] at (0, -2.5) {
                $\hat{y}=
                \begin{cases}
                    ...\\
                    0.0&x\geq0.27\ \& \ x<0.45\\
                    -0.5&x\geq0.45\ \& \ x<0.69\\
                    ...\\
                \end{cases}
                $
            };
        }
    \end{tikzpicture}
\end{frame}

\newcommand{\stratificationplot}[1]{
    \begin{tikzpicture}
        \begin{axis}[
            height=5cm,
            width=5cm,
            xlabel=\scriptsize{$x_1$},
            ylabel=\scriptsize{$x_2$},
            ticklabel style={font=\scriptsize\selectfont},
            tick style={draw=none},
            clip=false
        ]
        \end{axis}
    \end{tikzpicture}
}

\newsavebox{\stratificationregions}
\sbox{\stratificationregions}{
    \stratificationplot{0}
}

\begin{frame}{Tree-based models: Decision trees}
    \begin{tikzpicture}
        \node[draw=black] at (-5.25, -3.5) {};
        \node[draw=black] at (5.25, 3.5) {};

        \visible<1>{
            \node[anchor=east] at (5, 0) {
                \usebox{\stratificationregions}
            };
        }

    \end{tikzpicture}
\end{frame}

    % \newsavebox{\innercv}
\sbox{\innercv}{%
    \begin{tikzpicture}[transform canvas={scale=0.3}]
        \node[] at (-4.5, 1) {};
        \node[] at (6.5, -5) {};

        \node[minimum height=0.5cm, minimum width=7.5cm, draw=black, fill=gray!20] (full) at (0, 0) {};
        \node[minimum height=0.5cm, minimum width=0.375cm, draw=black, fill=green!20, anchor=west, inner sep=0pt, outer sep=0pt] (p1) at (full.west) {};
        \node[minimum height=0.5cm, minimum width=0.375cm, draw=black, fill=green!20, anchor=west, inner sep=0pt, outer sep=0pt] (p2) at (p1.east) {};
        \node[minimum height=0.5cm, minimum width=0.375cm, draw=black, fill=green!20, anchor=west, inner sep=0pt, outer sep=0pt] (p3) at (p2.east) {};
        \node[minimum height=0.5cm, minimum width=0.375cm, draw=black, fill=green!20, anchor=west, inner sep=0pt, outer sep=0pt] (p4) at (p3.east) {};
        \node[minimum height=0.5cm, minimum width=0.375cm, draw=black, fill=green!20, anchor=west, inner sep=0pt, outer sep=0pt] (p5) at (p4.east) {};
        \node[minimum height=0.5cm, minimum width=0.375cm, draw=black, fill=green!20, anchor=west, inner sep=0pt, outer sep=0pt] (p6) at (p5.east) {};
        \node[minimum height=0.5cm, minimum width=0.375cm, draw=black, fill=green!20, anchor=west, inner sep=0pt, outer sep=0pt] (p7) at (p6.east) {};
        \node[minimum height=0.5cm, minimum width=0.375cm, draw=black, fill=green!20, anchor=west, inner sep=0pt, outer sep=0pt] (p8) at (p7.east) {};
        \node[minimum height=0.5cm, minimum width=0.375cm, draw=black, fill=green!20, anchor=west, inner sep=0pt, outer sep=0pt] (p9) at (p8.east) {};
        \node[minimum height=0.5cm, minimum width=0.375cm, draw=black, fill=green!20, anchor=west, inner sep=0pt, outer sep=0pt] (p10) at (p9.east) {};
        \node[minimum height=0.5cm, minimum width=0.375cm, draw=black, fill=green!20, anchor=west, inner sep=0pt, outer sep=0pt] (p11) at (p10.east) {};
        \node[minimum height=0.5cm, minimum width=0.375cm, draw=black, fill=green!20, anchor=west, inner sep=0pt, outer sep=0pt] (p12) at (p11.east) {};
        \node[minimum height=0.5cm, minimum width=0.375cm, draw=black, fill=green!20, anchor=west, inner sep=0pt, outer sep=0pt] (p13) at (p12.east) {};
        \node[minimum height=0.5cm, minimum width=0.375cm, draw=black, fill=green!20, anchor=west, inner sep=0pt, outer sep=0pt] (p14) at (p13.east) {};
        \node[minimum height=0.5cm, minimum width=0.375cm, draw=black, fill=green!20, anchor=west, inner sep=0pt, outer sep=0pt] (p15) at (p14.east) {};
        \node[minimum height=0.5cm, minimum width=0.375cm, draw=black, fill=green!20, anchor=west, inner sep=0pt, outer sep=0pt] (p16) at (p15.east) {};
        \node[minimum height=0.5cm, minimum width=0.375cm, draw=black, fill=blue!20, anchor=west, inner sep=0pt, outer sep=0pt] (p17) at (p16.east) {};
        \node[minimum height=0.5cm, minimum width=0.375cm, draw=black, fill=blue!20, anchor=west, inner sep=0pt, outer sep=0pt] (p18) at (p17.east) {};
        \node[minimum height=0.5cm, minimum width=0.375cm, draw=black, fill=blue!20, anchor=west, inner sep=0pt, outer sep=0pt] (p19) at (p18.east) {};
        \node[minimum height=0.5cm, minimum width=0.375cm, draw=black, fill=blue!20, anchor=west, inner sep=0pt, outer sep=0pt] (p20) at (p19.east) {};

        \node[anchor=west] (error1) at ($ (full.east) + (0.5, 0) $) {Error};
        \draw[->] (full.east) -- (error1.west);

        \node[minimum height=0.5cm, minimum width=7.5cm, draw=black, fill=gray!20] (full2) at (0, -1) {};
        \node[minimum height=0.5cm, minimum width=0.375cm, draw=black, fill=green!20, anchor=west, inner sep=0pt, outer sep=0pt] (p21) at (full2.west) {};
        \node[minimum height=0.5cm, minimum width=0.375cm, draw=black, fill=green!20, anchor=west, inner sep=0pt, outer sep=0pt] (p22) at (p21.east) {};
        \node[minimum height=0.5cm, minimum width=0.375cm, draw=black, fill=green!20, anchor=west, inner sep=0pt, outer sep=0pt] (p23) at (p22.east) {};
        \node[minimum height=0.5cm, minimum width=0.375cm, draw=black, fill=green!20, anchor=west, inner sep=0pt, outer sep=0pt] (p24) at (p23.east) {};
        \node[minimum height=0.5cm, minimum width=0.375cm, draw=black, fill=green!20, anchor=west, inner sep=0pt, outer sep=0pt] (p25) at (p24.east) {};
        \node[minimum height=0.5cm, minimum width=0.375cm, draw=black, fill=green!20, anchor=west, inner sep=0pt, outer sep=0pt] (p26) at (p25.east) {};
        \node[minimum height=0.5cm, minimum width=0.375cm, draw=black, fill=green!20, anchor=west, inner sep=0pt, outer sep=0pt] (p27) at (p26.east) {};
        \node[minimum height=0.5cm, minimum width=0.375cm, draw=black, fill=green!20, anchor=west, inner sep=0pt, outer sep=0pt] (p28) at (p27.east) {};
        \node[minimum height=0.5cm, minimum width=0.375cm, draw=black, fill=green!20, anchor=west, inner sep=0pt, outer sep=0pt] (p29) at (p28.east) {};
        \node[minimum height=0.5cm, minimum width=0.375cm, draw=black, fill=green!20, anchor=west, inner sep=0pt, outer sep=0pt] (p210) at (p29.east) {};
        \node[minimum height=0.5cm, minimum width=0.375cm, draw=black, fill=green!20, anchor=west, inner sep=0pt, outer sep=0pt] (p211) at (p210.east) {};
        \node[minimum height=0.5cm, minimum width=0.375cm, draw=black, fill=green!20, anchor=west, inner sep=0pt, outer sep=0pt] (p212) at (p211.east) {};
        \node[minimum height=0.5cm, minimum width=0.375cm, draw=black, fill=blue!20, anchor=west, inner sep=0pt, outer sep=0pt] (p213) at (p212.east) {};
        \node[minimum height=0.5cm, minimum width=0.375cm, draw=black, fill=blue!20, anchor=west, inner sep=0pt, outer sep=0pt] (p214) at (p213.east) {};
        \node[minimum height=0.5cm, minimum width=0.375cm, draw=black, fill=blue!20, anchor=west, inner sep=0pt, outer sep=0pt] (p215) at (p214.east) {};
        \node[minimum height=0.5cm, minimum width=0.375cm, draw=black, fill=blue!20, anchor=west, inner sep=0pt, outer sep=0pt] (p216) at (p215.east) {};
        \node[minimum height=0.5cm, minimum width=0.375cm, draw=black, fill=green!20, anchor=west, inner sep=0pt, outer sep=0pt] (p217) at (p216.east) {};
        \node[minimum height=0.5cm, minimum width=0.375cm, draw=black, fill=green!20, anchor=west, inner sep=0pt, outer sep=0pt] (p218) at (p217.east) {};
        \node[minimum height=0.5cm, minimum width=0.375cm, draw=black, fill=green!20, anchor=west, inner sep=0pt, outer sep=0pt] (p219) at (p218.east) {};
        \node[minimum height=0.5cm, minimum width=0.375cm, draw=black, fill=green!20, anchor=west, inner sep=0pt, outer sep=0pt] (p220) at (p219.east) {};

        \node[anchor=west] (error2) at ($ (full2.east) + (0.5, 0) $) {Error};
        \draw[->] (full2.east) -- (error2.west);

        \node[minimum height=0.5cm, minimum width=7.5cm, draw=black, fill=gray!20] (full3) at (0, -2) {};
        \node[minimum height=0.5cm, minimum width=0.375cm, draw=black, fill=green!20, anchor=west, inner sep=0pt, outer sep=0pt] (p31) at (full3.west) {};
        \node[minimum height=0.5cm, minimum width=0.375cm, draw=black, fill=green!20, anchor=west, inner sep=0pt, outer sep=0pt] (p32) at (p31.east) {};
        \node[minimum height=0.5cm, minimum width=0.375cm, draw=black, fill=green!20, anchor=west, inner sep=0pt, outer sep=0pt] (p33) at (p32.east) {};
        \node[minimum height=0.5cm, minimum width=0.375cm, draw=black, fill=green!20, anchor=west, inner sep=0pt, outer sep=0pt] (p34) at (p33.east) {};
        \node[minimum height=0.5cm, minimum width=0.375cm, draw=black, fill=green!20, anchor=west, inner sep=0pt, outer sep=0pt] (p35) at (p34.east) {};
        \node[minimum height=0.5cm, minimum width=0.375cm, draw=black, fill=green!20, anchor=west, inner sep=0pt, outer sep=0pt] (p36) at (p35.east) {};
        \node[minimum height=0.5cm, minimum width=0.375cm, draw=black, fill=green!20, anchor=west, inner sep=0pt, outer sep=0pt] (p37) at (p36.east) {};
        \node[minimum height=0.5cm, minimum width=0.375cm, draw=black, fill=green!20, anchor=west, inner sep=0pt, outer sep=0pt] (p38) at (p37.east) {};
        \node[minimum height=0.5cm, minimum width=0.375cm, draw=black, fill=blue!20, anchor=west, inner sep=0pt, outer sep=0pt] (p39) at (p38.east) {};
        \node[minimum height=0.5cm, minimum width=0.375cm, draw=black, fill=blue!20, anchor=west, inner sep=0pt, outer sep=0pt] (p310) at (p39.east) {};
        \node[minimum height=0.5cm, minimum width=0.375cm, draw=black, fill=blue!20, anchor=west, inner sep=0pt, outer sep=0pt] (p311) at (p310.east) {};
        \node[minimum height=0.5cm, minimum width=0.375cm, draw=black, fill=blue!20, anchor=west, inner sep=0pt, outer sep=0pt] (p312) at (p311.east) {};
        \node[minimum height=0.5cm, minimum width=0.375cm, draw=black, fill=green!20, anchor=west, inner sep=0pt, outer sep=0pt] (p313) at (p312.east) {};
        \node[minimum height=0.5cm, minimum width=0.375cm, draw=black, fill=green!20, anchor=west, inner sep=0pt, outer sep=0pt] (p314) at (p313.east) {};
        \node[minimum height=0.5cm, minimum width=0.375cm, draw=black, fill=green!20, anchor=west, inner sep=0pt, outer sep=0pt] (p315) at (p314.east) {};
        \node[minimum height=0.5cm, minimum width=0.375cm, draw=black, fill=green!20, anchor=west, inner sep=0pt, outer sep=0pt] (p316) at (p315.east) {};
        \node[minimum height=0.5cm, minimum width=0.375cm, draw=black, fill=green!20, anchor=west, inner sep=0pt, outer sep=0pt] (p317) at (p316.east) {};
        \node[minimum height=0.5cm, minimum width=0.375cm, draw=black, fill=green!20, anchor=west, inner sep=0pt, outer sep=0pt] (p318) at (p317.east) {};
        \node[minimum height=0.5cm, minimum width=0.375cm, draw=black, fill=green!20, anchor=west, inner sep=0pt, outer sep=0pt] (p319) at (p318.east) {};
        \node[minimum height=0.5cm, minimum width=0.375cm, draw=black, fill=green!20, anchor=west, inner sep=0pt, outer sep=0pt] (p320) at (p319.east) {};

        \node[anchor=west] (error3) at ($ (full3.east) + (0.5, 0) $) {Error};
        \draw[->] (full3.east) -- (error3.west);

        \node[minimum height=0.5cm, minimum width=7.5cm, draw=black, fill=gray!20] (full4) at (0, -3) {};
        \node[minimum height=0.5cm, minimum width=0.375cm, draw=black, fill=green!20, anchor=west, inner sep=0pt, outer sep=0pt] (p41) at (full4.west) {};
        \node[minimum height=0.5cm, minimum width=0.375cm, draw=black, fill=green!20, anchor=west, inner sep=0pt, outer sep=0pt] (p42) at (p41.east) {};
        \node[minimum height=0.5cm, minimum width=0.375cm, draw=black, fill=green!20, anchor=west, inner sep=0pt, outer sep=0pt] (p43) at (p42.east) {};
        \node[minimum height=0.5cm, minimum width=0.375cm, draw=black, fill=green!20, anchor=west, inner sep=0pt, outer sep=0pt] (p44) at (p43.east) {};
        \node[minimum height=0.5cm, minimum width=0.375cm, draw=black, fill=blue!20, anchor=west, inner sep=0pt, outer sep=0pt] (p45) at (p44.east) {};
        \node[minimum height=0.5cm, minimum width=0.375cm, draw=black, fill=blue!20, anchor=west, inner sep=0pt, outer sep=0pt] (p46) at (p45.east) {};
        \node[minimum height=0.5cm, minimum width=0.375cm, draw=black, fill=blue!20, anchor=west, inner sep=0pt, outer sep=0pt] (p47) at (p46.east) {};
        \node[minimum height=0.5cm, minimum width=0.375cm, draw=black, fill=blue!20, anchor=west, inner sep=0pt, outer sep=0pt] (p48) at (p47.east) {};
        \node[minimum height=0.5cm, minimum width=0.375cm, draw=black, fill=green!20, anchor=west, inner sep=0pt, outer sep=0pt] (p49) at (p48.east) {};
        \node[minimum height=0.5cm, minimum width=0.375cm, draw=black, fill=green!20, anchor=west, inner sep=0pt, outer sep=0pt] (p410) at (p49.east) {};
        \node[minimum height=0.5cm, minimum width=0.375cm, draw=black, fill=green!20, anchor=west, inner sep=0pt, outer sep=0pt] (p411) at (p410.east) {};
        \node[minimum height=0.5cm, minimum width=0.375cm, draw=black, fill=green!20, anchor=west, inner sep=0pt, outer sep=0pt] (p412) at (p411.east) {};
        \node[minimum height=0.5cm, minimum width=0.375cm, draw=black, fill=green!20, anchor=west, inner sep=0pt, outer sep=0pt] (p413) at (p412.east) {};
        \node[minimum height=0.5cm, minimum width=0.375cm, draw=black, fill=green!20, anchor=west, inner sep=0pt, outer sep=0pt] (p414) at (p413.east) {};
        \node[minimum height=0.5cm, minimum width=0.375cm, draw=black, fill=green!20, anchor=west, inner sep=0pt, outer sep=0pt] (p415) at (p414.east) {};
        \node[minimum height=0.5cm, minimum width=0.375cm, draw=black, fill=green!20, anchor=west, inner sep=0pt, outer sep=0pt] (p416) at (p415.east) {};
        \node[minimum height=0.5cm, minimum width=0.375cm, draw=black, fill=green!20, anchor=west, inner sep=0pt, outer sep=0pt] (p417) at (p416.east) {};
        \node[minimum height=0.5cm, minimum width=0.375cm, draw=black, fill=green!20, anchor=west, inner sep=0pt, outer sep=0pt] (p418) at (p417.east) {};
        \node[minimum height=0.5cm, minimum width=0.375cm, draw=black, fill=green!20, anchor=west, inner sep=0pt, outer sep=0pt] (p419) at (p418.east) {};
        \node[minimum height=0.5cm, minimum width=0.375cm, draw=black, fill=green!20, anchor=west, inner sep=0pt, outer sep=0pt] (p420) at (p419.east) {};

        \node[anchor=west] (error4) at ($ (full4.east) + (0.5, 0) $) {Error};
        \draw[->] (full4.east) -- (error4.west);

        \node[minimum height=0.5cm, minimum width=7.5cm, draw=black, fill=gray!20] (full5) at (0, -4) {};
        \node[minimum height=0.5cm, minimum width=0.375cm, draw=black, fill=blue!20, anchor=west, inner sep=0pt, outer sep=0pt] (p51) at (full5.west) {};
        \node[minimum height=0.5cm, minimum width=0.375cm, draw=black, fill=blue!20, anchor=west, inner sep=0pt, outer sep=0pt] (p52) at (p51.east) {};
        \node[minimum height=0.5cm, minimum width=0.375cm, draw=black, fill=blue!20, anchor=west, inner sep=0pt, outer sep=0pt] (p53) at (p52.east) {};
        \node[minimum height=0.5cm, minimum width=0.375cm, draw=black, fill=blue!20, anchor=west, inner sep=0pt, outer sep=0pt] (p54) at (p53.east) {};
        \node[minimum height=0.5cm, minimum width=0.375cm, draw=black, fill=green!20, anchor=west, inner sep=0pt, outer sep=0pt] (p55) at (p54.east) {};
        \node[minimum height=0.5cm, minimum width=0.375cm, draw=black, fill=green!20, anchor=west, inner sep=0pt, outer sep=0pt] (p56) at (p55.east) {};
        \node[minimum height=0.5cm, minimum width=0.375cm, draw=black, fill=green!20, anchor=west, inner sep=0pt, outer sep=0pt] (p57) at (p56.east) {};
        \node[minimum height=0.5cm, minimum width=0.375cm, draw=black, fill=green!20, anchor=west, inner sep=0pt, outer sep=0pt] (p58) at (p57.east) {};
        \node[minimum height=0.5cm, minimum width=0.375cm, draw=black, fill=green!20, anchor=west, inner sep=0pt, outer sep=0pt] (p59) at (p58.east) {};
        \node[minimum height=0.5cm, minimum width=0.375cm, draw=black, fill=green!20, anchor=west, inner sep=0pt, outer sep=0pt] (p510) at (p59.east) {};
        \node[minimum height=0.5cm, minimum width=0.375cm, draw=black, fill=green!20, anchor=west, inner sep=0pt, outer sep=0pt] (p511) at (p510.east) {};
        \node[minimum height=0.5cm, minimum width=0.375cm, draw=black, fill=green!20, anchor=west, inner sep=0pt, outer sep=0pt] (p512) at (p511.east) {};
        \node[minimum height=0.5cm, minimum width=0.375cm, draw=black, fill=green!20, anchor=west, inner sep=0pt, outer sep=0pt] (p513) at (p512.east) {};
        \node[minimum height=0.5cm, minimum width=0.375cm, draw=black, fill=green!20, anchor=west, inner sep=0pt, outer sep=0pt] (p514) at (p513.east) {};
        \node[minimum height=0.5cm, minimum width=0.375cm, draw=black, fill=green!20, anchor=west, inner sep=0pt, outer sep=0pt] (p515) at (p514.east) {};
        \node[minimum height=0.5cm, minimum width=0.375cm, draw=black, fill=green!20, anchor=west, inner sep=0pt, outer sep=0pt] (p516) at (p515.east) {};
        \node[minimum height=0.5cm, minimum width=0.375cm, draw=black, fill=green!20, anchor=west, inner sep=0pt, outer sep=0pt] (p517) at (p516.east) {};
        \node[minimum height=0.5cm, minimum width=0.375cm, draw=black, fill=green!20, anchor=west, inner sep=0pt, outer sep=0pt] (p518) at (p517.east) {};
        \node[minimum height=0.5cm, minimum width=0.375cm, draw=black, fill=green!20, anchor=west, inner sep=0pt, outer sep=0pt] (p519) at (p518.east) {};
        \node[minimum height=0.5cm, minimum width=0.375cm, draw=black, fill=green!20, anchor=west, inner sep=0pt, outer sep=0pt] (p520) at (p519.east) {};

        \node[anchor=west] (error5) at ($ (full5.east) + (0.5, 0) $) {Error};
        \draw[->] (full5.east) -- (error5.west);

        \draw[red, dashed] (error1.north east) rectangle (error5.south west);
        \node[anchor=north] at (error5.south) {\textcolor{red}{Average}};
    \end{tikzpicture}
}

\newsavebox{\edabox}
\sbox{\edabox}{
    \begin{tikzpicture}
        \begin{groupplot}[
            group style={
                group size=3 by 1,
                horizontal sep=0.1cm
            },
            height=3cm,
            width=3cm,
            xmajorticks=false,
            ymajorticks=false
        ]
            \nextgroupplot[]
                \addplot[
                    only marks,
                    blue,
                    opacity=0.5,
                    mark size=1
                ] table [
                    col sep=comma,
                    x=Income,
                    y=Balance
                ] {data/Credit.csv};
                \nextgroupplot[]
                    \addplot[
                        only marks,
                        blue,
                        opacity=0.5,
                        mark size=1
                    ] table [
                        col sep=comma,
                        x=Rating,
                        y=Balance
                    ] {data/Credit.csv};
                \nextgroupplot[]
                    \addplot[
                        only marks,
                        blue,
                        opacity=0.5,
                        mark size=1
                    ] table [
                        col sep=comma,
                        x=Age,
                        y=Balance
                    ] {data/Credit.csv};
        \end{groupplot}
    \end{tikzpicture}
}

\begin{frame}{Exercise 5}
    \begin{tikzpicture}
        \node[draw=black] at (-5.25, 3.5) {};
        \node[draw=black] at (5.25, -3.5) {};

        \visible<1-2>{
            \node[minimum height=0.5cm, minimum width=7.5cm, draw=black, fill=gray!20, label=\small{Dataset}] (full) at (-0.5, 0) {};
            \node[minimum height=0.5cm, minimum width=0.375cm, draw=black, anchor=west, inner sep=0pt, outer sep=0pt, shading=axis, left color=green!40, right color=blue!40, shading angle=45] (p1) at (full.west) {};
            \node[minimum height=0.5cm, minimum width=0.375cm, draw=black, anchor=west, inner sep=0pt, outer sep=0pt, shading=axis, left color=green!40, right color=blue!40, shading angle=45] (p2) at (p1.east) {};
            \node[minimum height=0.5cm, minimum width=0.375cm, draw=black, anchor=west, inner sep=0pt, outer sep=0pt, shading=axis, left color=green!40, right color=blue!40, shading angle=45] (p3) at (p2.east) {};
            \node[minimum height=0.5cm, minimum width=0.375cm, draw=black, anchor=west, inner sep=0pt, outer sep=0pt, shading=axis, left color=green!40, right color=blue!40, shading angle=45] (p4) at (p3.east) {};
            \node[minimum height=0.5cm, minimum width=0.375cm, draw=black, anchor=west, inner sep=0pt, outer sep=0pt, shading=axis, left color=green!40, right color=blue!40, shading angle=45] (p5) at (p4.east) {};
            \node[minimum height=0.5cm, minimum width=0.375cm, draw=black, anchor=west, inner sep=0pt, outer sep=0pt, shading=axis, left color=green!40, right color=blue!40, shading angle=45] (p6) at (p5.east) {};
            \node[minimum height=0.5cm, minimum width=0.375cm, draw=black, anchor=west, inner sep=0pt, outer sep=0pt, shading=axis, left color=green!40, right color=blue!40, shading angle=45] (p7) at (p6.east) {};
            \node[minimum height=0.5cm, minimum width=0.375cm, draw=black, anchor=west, inner sep=0pt, outer sep=0pt, shading=axis, left color=green!40, right color=blue!40, shading angle=45] (p8) at (p7.east) {};
            \node[minimum height=0.5cm, minimum width=0.375cm, draw=black, anchor=west, inner sep=0pt, outer sep=0pt, shading=axis, left color=green!40, right color=blue!40, shading angle=45] (p9) at (p8.east) {};
            \node[minimum height=0.5cm, minimum width=0.375cm, draw=black, anchor=west, inner sep=0pt, outer sep=0pt, shading=axis, left color=green!40, right color=blue!40, shading angle=45] (p10) at (p9.east) {};
            \node[minimum height=0.5cm, minimum width=0.375cm, draw=black, anchor=west, inner sep=0pt, outer sep=0pt, shading=axis, left color=green!40, right color=blue!40, shading angle=45] (p11) at (p10.east) {};
            \node[minimum height=0.5cm, minimum width=0.375cm, draw=black, anchor=west, inner sep=0pt, outer sep=0pt, shading=axis, left color=green!40, right color=blue!40, shading angle=45] (p12) at (p11.east) {};
            \node[minimum height=0.5cm, minimum width=0.375cm, draw=black, anchor=west, inner sep=0pt, outer sep=0pt, shading=axis, left color=green!40, right color=blue!40, shading angle=45] (p13) at (p12.east) {};
            \node[minimum height=0.5cm, minimum width=0.375cm, draw=black, anchor=west, inner sep=0pt, outer sep=0pt, shading=axis, left color=green!40, right color=blue!40, shading angle=45] (p14) at (p13.east) {};
            \node[minimum height=0.5cm, minimum width=0.375cm, draw=black, anchor=west, inner sep=0pt, outer sep=0pt, shading=axis, left color=green!40, right color=blue!40, shading angle=45] (p15) at (p14.east) {};
            \node[minimum height=0.5cm, minimum width=0.375cm, draw=black, anchor=west, inner sep=0pt, outer sep=0pt, shading=axis, left color=green!40, right color=blue!40, shading angle=45] (p16) at (p15.east) {};
            \node[minimum height=0.5cm, minimum width=0.375cm, draw=black, fill=red!20, anchor=west, inner sep=0pt, outer sep=0pt] (p17) at (p16.east) {};
            \node[minimum height=0.5cm, minimum width=0.375cm, draw=black, fill=red!20, anchor=west, inner sep=0pt, outer sep=0pt] (p18) at (p17.east) {};
            \node[minimum height=0.5cm, minimum width=0.375cm, draw=black, fill=red!20, anchor=west, inner sep=0pt, outer sep=0pt] (p19) at (p18.east) {};
            \node[minimum height=0.5cm, minimum width=0.375cm, draw=black, fill=red!20, anchor=west, inner sep=0pt, outer sep=0pt] (p20) at (p19.east) {};

            \node[] (innercv) at ($ (full.south) - (0, 1.63) $) {
                \usebox{\innercv}
            };
            \draw[-stealth, line width=3pt, gray] (full.south) -- (innercv.north);

        }
        \visible<2>{
            \node[] (eda) at (-0.5, 2.5) {
                \usebox{\edabox}
            };
            \draw[-stealth, line width=3pt, gray] ($ (full.north) + (0, 0.5) $) -- (eda.south);
        }
        \visible<3-4>{
            \node[minimum height=0.5cm, minimum width=7.5cm, draw=black, fill=gray!20, label=below:\small{\textbf{Random forest}}] (full) at (-0.5, 0) {};
            \node[minimum height=0.5cm, minimum width=0.375cm, draw=black, anchor=west, inner sep=0pt, outer sep=0pt, shading=axis, left color=green!40, right color=blue!40, shading angle=45] (p1) at (full.west) {};
            \node[minimum height=0.5cm, minimum width=0.375cm, draw=black, anchor=west, inner sep=0pt, outer sep=0pt, shading=axis, left color=green!40, right color=blue!40, shading angle=45] (p2) at (p1.east) {};
            \node[minimum height=0.5cm, minimum width=0.375cm, draw=black, anchor=west, inner sep=0pt, outer sep=0pt, shading=axis, left color=green!40, right color=blue!40, shading angle=45] (p3) at (p2.east) {};
            \node[minimum height=0.5cm, minimum width=0.375cm, draw=black, anchor=west, inner sep=0pt, outer sep=0pt, shading=axis, left color=green!40, right color=blue!40, shading angle=45] (p4) at (p3.east) {};
            \node[minimum height=0.5cm, minimum width=0.375cm, draw=black, anchor=west, inner sep=0pt, outer sep=0pt, shading=axis, left color=green!40, right color=blue!40, shading angle=45] (p5) at (p4.east) {};
            \node[minimum height=0.5cm, minimum width=0.375cm, draw=black, anchor=west, inner sep=0pt, outer sep=0pt, shading=axis, left color=green!40, right color=blue!40, shading angle=45] (p6) at (p5.east) {};
            \node[minimum height=0.5cm, minimum width=0.375cm, draw=black, anchor=west, inner sep=0pt, outer sep=0pt, shading=axis, left color=green!40, right color=blue!40, shading angle=45] (p7) at (p6.east) {};
            \node[minimum height=0.5cm, minimum width=0.375cm, draw=black, anchor=west, inner sep=0pt, outer sep=0pt, shading=axis, left color=green!40, right color=blue!40, shading angle=45] (p8) at (p7.east) {};
            \node[minimum height=0.5cm, minimum width=0.375cm, draw=black, anchor=west, inner sep=0pt, outer sep=0pt, shading=axis, left color=green!40, right color=blue!40, shading angle=45] (p9) at (p8.east) {};
            \node[minimum height=0.5cm, minimum width=0.375cm, draw=black, anchor=west, inner sep=0pt, outer sep=0pt, shading=axis, left color=green!40, right color=blue!40, shading angle=45] (p10) at (p9.east) {};
            \node[minimum height=0.5cm, minimum width=0.375cm, draw=black, anchor=west, inner sep=0pt, outer sep=0pt, shading=axis, left color=green!40, right color=blue!40, shading angle=45] (p11) at (p10.east) {};
            \node[minimum height=0.5cm, minimum width=0.375cm, draw=black, anchor=west, inner sep=0pt, outer sep=0pt, shading=axis, left color=green!40, right color=blue!40, shading angle=45] (p12) at (p11.east) {};
            \node[minimum height=0.5cm, minimum width=0.375cm, draw=black, anchor=west, inner sep=0pt, outer sep=0pt, shading=axis, left color=green!40, right color=blue!40, shading angle=45] (p13) at (p12.east) {};
            \node[minimum height=0.5cm, minimum width=0.375cm, draw=black, anchor=west, inner sep=0pt, outer sep=0pt, shading=axis, left color=green!40, right color=blue!40, shading angle=45] (p14) at (p13.east) {};
            \node[minimum height=0.5cm, minimum width=0.375cm, draw=black, anchor=west, inner sep=0pt, outer sep=0pt, shading=axis, left color=green!40, right color=blue!40, shading angle=45] (p15) at (p14.east) {};
            \node[minimum height=0.5cm, minimum width=0.375cm, draw=black, anchor=west, inner sep=0pt, outer sep=0pt, shading=axis, left color=green!40, right color=blue!40, shading angle=45] (p16) at (p15.east) {};
            \node[minimum height=0.5cm, minimum width=0.375cm, draw=black, fill=red!20, anchor=west, inner sep=0pt, outer sep=0pt] (p17) at (p16.east) {};
            \node[minimum height=0.5cm, minimum width=0.375cm, draw=black, fill=red!20, anchor=west, inner sep=0pt, outer sep=0pt] (p18) at (p17.east) {};
            \node[minimum height=0.5cm, minimum width=0.375cm, draw=black, fill=red!20, anchor=west, inner sep=0pt, outer sep=0pt] (p19) at (p18.east) {};
            \node[minimum height=0.5cm, minimum width=0.375cm, draw=black, fill=red!20, anchor=west, inner sep=0pt, outer sep=0pt] (p20) at (p19.east) {};

            \node[minimum height=0.5cm, minimum width=7.5cm, draw=black, fill=gray!20, label=below:\small{\textbf{Generalized Additive Model}}] (full) at (-0.5, 1.5) {};
            \node[minimum height=0.5cm, minimum width=0.375cm, draw=black, anchor=west, inner sep=0pt, outer sep=0pt, shading=axis, left color=green!40, right color=blue!40, shading angle=45] (p1) at (full.west) {};
            \node[minimum height=0.5cm, minimum width=0.375cm, draw=black, anchor=west, inner sep=0pt, outer sep=0pt, shading=axis, left color=green!40, right color=blue!40, shading angle=45] (p2) at (p1.east) {};
            \node[minimum height=0.5cm, minimum width=0.375cm, draw=black, anchor=west, inner sep=0pt, outer sep=0pt, shading=axis, left color=green!40, right color=blue!40, shading angle=45] (p3) at (p2.east) {};
            \node[minimum height=0.5cm, minimum width=0.375cm, draw=black, anchor=west, inner sep=0pt, outer sep=0pt, shading=axis, left color=green!40, right color=blue!40, shading angle=45] (p4) at (p3.east) {};
            \node[minimum height=0.5cm, minimum width=0.375cm, draw=black, anchor=west, inner sep=0pt, outer sep=0pt, shading=axis, left color=green!40, right color=blue!40, shading angle=45] (p5) at (p4.east) {};
            \node[minimum height=0.5cm, minimum width=0.375cm, draw=black, anchor=west, inner sep=0pt, outer sep=0pt, shading=axis, left color=green!40, right color=blue!40, shading angle=45] (p6) at (p5.east) {};
            \node[minimum height=0.5cm, minimum width=0.375cm, draw=black, anchor=west, inner sep=0pt, outer sep=0pt, shading=axis, left color=green!40, right color=blue!40, shading angle=45] (p7) at (p6.east) {};
            \node[minimum height=0.5cm, minimum width=0.375cm, draw=black, anchor=west, inner sep=0pt, outer sep=0pt, shading=axis, left color=green!40, right color=blue!40, shading angle=45] (p8) at (p7.east) {};
            \node[minimum height=0.5cm, minimum width=0.375cm, draw=black, anchor=west, inner sep=0pt, outer sep=0pt, shading=axis, left color=green!40, right color=blue!40, shading angle=45] (p9) at (p8.east) {};
            \node[minimum height=0.5cm, minimum width=0.375cm, draw=black, anchor=west, inner sep=0pt, outer sep=0pt, shading=axis, left color=green!40, right color=blue!40, shading angle=45] (p10) at (p9.east) {};
            \node[minimum height=0.5cm, minimum width=0.375cm, draw=black, anchor=west, inner sep=0pt, outer sep=0pt, shading=axis, left color=green!40, right color=blue!40, shading angle=45] (p11) at (p10.east) {};
            \node[minimum height=0.5cm, minimum width=0.375cm, draw=black, anchor=west, inner sep=0pt, outer sep=0pt, shading=axis, left color=green!40, right color=blue!40, shading angle=45] (p12) at (p11.east) {};
            \node[minimum height=0.5cm, minimum width=0.375cm, draw=black, anchor=west, inner sep=0pt, outer sep=0pt, shading=axis, left color=green!40, right color=blue!40, shading angle=45] (p13) at (p12.east) {};
            \node[minimum height=0.5cm, minimum width=0.375cm, draw=black, anchor=west, inner sep=0pt, outer sep=0pt, shading=axis, left color=green!40, right color=blue!40, shading angle=45] (p14) at (p13.east) {};
            \node[minimum height=0.5cm, minimum width=0.375cm, draw=black, anchor=west, inner sep=0pt, outer sep=0pt, shading=axis, left color=green!40, right color=blue!40, shading angle=45] (p15) at (p14.east) {};
            \node[minimum height=0.5cm, minimum width=0.375cm, draw=black, anchor=west, inner sep=0pt, outer sep=0pt, shading=axis, left color=green!40, right color=blue!40, shading angle=45] (p16) at (p15.east) {};
            \node[minimum height=0.5cm, minimum width=0.375cm, draw=black, fill=red!20, anchor=west, inner sep=0pt, outer sep=0pt] (p17) at (p16.east) {};
            \node[minimum height=0.5cm, minimum width=0.375cm, draw=black, fill=red!20, anchor=west, inner sep=0pt, outer sep=0pt] (p18) at (p17.east) {};
            \node[minimum height=0.5cm, minimum width=0.375cm, draw=black, fill=red!20, anchor=west, inner sep=0pt, outer sep=0pt] (p19) at (p18.east) {};
            \node[minimum height=0.5cm, minimum width=0.375cm, draw=black, fill=red!20, anchor=west, inner sep=0pt, outer sep=0pt] (p20) at (p19.east) {};

            \node[minimum height=0.5cm, minimum width=7.5cm, draw=black, fill=gray!20, label=below:\small{\textbf{XGBoost}}] (full) at (-0.5, -1.5) {};
            \node[minimum height=0.5cm, minimum width=0.375cm, draw=black, anchor=west, inner sep=0pt, outer sep=0pt, shading=axis, left color=green!40, right color=blue!40, shading angle=45] (p1) at (full.west) {};
            \node[minimum height=0.5cm, minimum width=0.375cm, draw=black, anchor=west, inner sep=0pt, outer sep=0pt, shading=axis, left color=green!40, right color=blue!40, shading angle=45] (p2) at (p1.east) {};
            \node[minimum height=0.5cm, minimum width=0.375cm, draw=black, anchor=west, inner sep=0pt, outer sep=0pt, shading=axis, left color=green!40, right color=blue!40, shading angle=45] (p3) at (p2.east) {};
            \node[minimum height=0.5cm, minimum width=0.375cm, draw=black, anchor=west, inner sep=0pt, outer sep=0pt, shading=axis, left color=green!40, right color=blue!40, shading angle=45] (p4) at (p3.east) {};
            \node[minimum height=0.5cm, minimum width=0.375cm, draw=black, anchor=west, inner sep=0pt, outer sep=0pt, shading=axis, left color=green!40, right color=blue!40, shading angle=45] (p5) at (p4.east) {};
            \node[minimum height=0.5cm, minimum width=0.375cm, draw=black, anchor=west, inner sep=0pt, outer sep=0pt, shading=axis, left color=green!40, right color=blue!40, shading angle=45] (p6) at (p5.east) {};
            \node[minimum height=0.5cm, minimum width=0.375cm, draw=black, anchor=west, inner sep=0pt, outer sep=0pt, shading=axis, left color=green!40, right color=blue!40, shading angle=45] (p7) at (p6.east) {};
            \node[minimum height=0.5cm, minimum width=0.375cm, draw=black, anchor=west, inner sep=0pt, outer sep=0pt, shading=axis, left color=green!40, right color=blue!40, shading angle=45] (p8) at (p7.east) {};
            \node[minimum height=0.5cm, minimum width=0.375cm, draw=black, anchor=west, inner sep=0pt, outer sep=0pt, shading=axis, left color=green!40, right color=blue!40, shading angle=45] (p9) at (p8.east) {};
            \node[minimum height=0.5cm, minimum width=0.375cm, draw=black, anchor=west, inner sep=0pt, outer sep=0pt, shading=axis, left color=green!40, right color=blue!40, shading angle=45] (p10) at (p9.east) {};
            \node[minimum height=0.5cm, minimum width=0.375cm, draw=black, anchor=west, inner sep=0pt, outer sep=0pt, shading=axis, left color=green!40, right color=blue!40, shading angle=45] (p11) at (p10.east) {};
            \node[minimum height=0.5cm, minimum width=0.375cm, draw=black, anchor=west, inner sep=0pt, outer sep=0pt, shading=axis, left color=green!40, right color=blue!40, shading angle=45] (p12) at (p11.east) {};
            \node[minimum height=0.5cm, minimum width=0.375cm, draw=black, anchor=west, inner sep=0pt, outer sep=0pt, shading=axis, left color=green!40, right color=blue!40, shading angle=45] (p13) at (p12.east) {};
            \node[minimum height=0.5cm, minimum width=0.375cm, draw=black, anchor=west, inner sep=0pt, outer sep=0pt, shading=axis, left color=green!40, right color=blue!40, shading angle=45] (p14) at (p13.east) {};
            \node[minimum height=0.5cm, minimum width=0.375cm, draw=black, anchor=west, inner sep=0pt, outer sep=0pt, shading=axis, left color=green!40, right color=blue!40, shading angle=45] (p15) at (p14.east) {};
            \node[minimum height=0.5cm, minimum width=0.375cm, draw=black, anchor=west, inner sep=0pt, outer sep=0pt, shading=axis, left color=green!40, right color=blue!40, shading angle=45] (p16) at (p15.east) {};
            \node[minimum height=0.5cm, minimum width=0.375cm, draw=black, fill=red!20, anchor=west, inner sep=0pt, outer sep=0pt] (p17) at (p16.east) {};
            \node[minimum height=0.5cm, minimum width=0.375cm, draw=black, fill=red!20, anchor=west, inner sep=0pt, outer sep=0pt] (p18) at (p17.east) {};
            \node[minimum height=0.5cm, minimum width=0.375cm, draw=black, fill=red!20, anchor=west, inner sep=0pt, outer sep=0pt] (p19) at (p18.east) {};
            \node[minimum height=0.5cm, minimum width=0.375cm, draw=black, fill=red!20, anchor=west, inner sep=0pt, outer sep=0pt] (p20) at (p19.east) {};

            \node[] at (4.5, 1.5) {
                140
            };
            \draw[-stealth] (3.5, 1.5) -- (4.1, 1.5);

            \draw[-stealth] (3.5, 0) -- (4.1, 0);

            \node[] at (4.5, -1.5) {
                130
            };
            \draw[-stealth] (3.5, -1.5) -- (4.1, -1.5);
        }
        \visible<3->{
            \node[] (mae) at (4.5, 0) {
                \alert<4>{125}
            };
        }
        \visible<5->{
            \draw[-stealth] (mae) -- (4.5, 2);
        }
        \visible<5>{
            \node[anchor=south] at (4.5, 2) {
                ???
            };
        }
        \visible<6->{
            \draw[|-|] (-5, 2) -- (5, 2);
            \node[anchor=north] at (-5, 2) {1999};
            \node[anchor=north] at (5, 2) {0};
        }
        \visible<7>{
            \node[circle, draw=black, fill=red!20] at (2, 2) {};

            \PythonInputNode{1}{(-4.1, -1)}{import}{0.9\textwidth}{7}{
from sklearn.dummy import DummyRegressor^^J
^^J
dummy = DummyRegressor(strategy='mean')^^J
dummy.fit(train[predictors], train[target])^^J
predictions = dummy.predict(test[predictors])^^J
mae = mean_absolute_error(test[target], predictors)^^J
            }
        }
        \visible<8>{
            \node[circle, draw=black, fill=red!20] at (2.5, 2) {};

            \PythonInputNode{1}{(-4.1, -1)}{import}{0.9\textwidth}{7}{
from sklearn.linear_model import LinearRegression^^J
^^J
dummy = LinearRegression()^^J
dummy.fit(train['Age'], train[target])^^J
predictions = dummy.predict(test['Age'])^^J
mae = mean_absolute_error(test[target], predictors)^^J
            }
        }
    \end{tikzpicture}
\end{frame}


\end{document}
