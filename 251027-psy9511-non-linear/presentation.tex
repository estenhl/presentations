
\documentclass{beamer}

\usetheme{PSY9511}

\usepackage{emoji}
\usepackage{pgfplots}
\usepackage{tikz}
\usepackage{xcolor}

\usepgfplotslibrary{colormaps}

\usetikzlibrary{calc}
\usetikzlibrary{matrix}

\title{PSY9511: Seminar 5}
\subtitle{Beyond linearity: Extensions of linear models and tree-based models}
\author{Esten H. Leonardsen}
\date{27.10.24}

\pgfmathdeclarefunction{gamma}{1}{%
  \pgfmathparse{(#1-1)!}%
}

\pgfplotsset{
    colormap={blue red}{
        rgb255(0cm)=(0,0,255);
        rgb255(1cm)=(255,0,0)
    }
}

\begin{document}
	\begin{frame}
	 	\maketitle
	\end{frame}

    \begin{frame}{Outline}
        \begin{enumerate}
            \item Exercise 3
            \item Exercise 4
            \item Recap
            \item Extensions of linear models
            \begin{enumerate}
                \item Generalized linear models (GLMs)
                \item Generalized additive models (GAMs)
            \end{enumerate}
            \item Tree-based models
            \begin{enumerate}
                \item Decision trees
                \item Random forests
                \alert<2>{\item Gradient boosting (XGBoost)}
            \end{enumerate}
            \item Exercise 5
        \end{enumerate}
    \end{frame}

    \begin{frame}{Exercise solutions}
        \centering
        \vfill
        \small{\url{https://uio.instructure.com/courses/59550}}
        \vfill
    \end{frame}

    \begin{frame}{Exercise 3: Solution}
        \centering
        \vfill
        \scriptsize{\url{http://localhost:8888/notebooks/notebooks/Solution\%203.ipynb}}
        \vfill
    \end{frame}

    \section{Exercise 4}

\begin{frame}{Exercise 4: Solution}
    \centering
    \vfill
    \scriptsize{\url{http://localhost:8888/notebooks/notebooks/Solution\%204.ipynb}}
    \vfill
\end{frame}

    \section{Recap}

\begin{frame}{Recap}
    Whenever a modelling choice is made on the basis of
    performance in a dataset, \textbf{we have to assume the
    performance achieved by the chosen model is inflated}
    \begin{itemize}
        \item[!] Don't use a single train/validation split, cross-validation or bootstrapping with only train/validation sets to select the best model and then reports its performance
        \item[\textrightarrow] Hold out a test set
        \item[\textrightarrow] Use a nested cross-validation
        \item[\textrightarrow] Bootstrap train/validation/test sets
    \end{itemize}
\end{frame}

    \section{Extensions of linear models}


\newcommand{\datasetplot}[1]{
    \begin{tikzpicture}
        \begin{axis}[
            height=6cm,
            width=8cm,
            xmin=0,
            xmax=1,
            xtick pos=bottom,
            ytick pos=left
        ]
            \ifnum#1=0
                \addplot[
                    only marks,
                    blue,
                    samples=100,
                    domain=0:1,
                    opacity=0.5
                ] (x, 20 + 5 * x + rand);
            \fi
            \ifnum#1=1
                \addplot[
                    only marks,
                    blue,
                    samples=100,
                    domain=0:1,
                    opacity=0.5
                ] (x, x^2 + gauss(0,0.1));
            \fi


        \end{axis}
    \end{tikzpicture}
}

\newsavebox{\linearbox}
\sbox{\linearbox}{
    \datasetplot{0}
}

\newsavebox{\exponentialbox}
\sbox{\exponentialbox}{
    \datasetplot{1}
}

\begin{frame}{Extensions of linear models: Motivation}
    \begin{tikzpicture}
        \node[draw=black] at (-5.25, -3.5) {};
        \node[draw=black] at (5.25, 3.5) {};

        \node[] at (0, -3) {
            $\hat{y}=\beta_0+\sum\limits_{i=0}^p \beta_ix_i$
        };

        \visible<2>{
            \node[] at (0, 0.5) {
                \usebox{\linearbox}
            };
        }
        \visible<3>{
            \node[] at (0, 0.5) {
                \usebox{\exponentialbox}
            };
        }
    \end{tikzpicture}
\end{frame}

    % \section{Tree-based models}

\newcommand{\treedataplot}[1]{
    \begin{tikzpicture}
        \begin{axis}[
            height=5cm,
            width=8cm,
            xlabel=$x_1$,
            ylabel=$x_2$,
            zlabel=$y$,
            view={75}{15}
        ]
            \addplot3 table [
                only marks,
                opacity=0.5,
                col sep=comma,
                x=x,
                y=y,
                z=value
            ] {data/treedata.csv};

            \ifnum#1>0
                \addplot3[
                    surf,
                    opacity=0.1,
                    red,
                    domain=0:0.5, y domain=0:0.5
                ] {0.8};

                \addplot3[
                    surf,
                    opacity=0.1,
                    red,
                    domain=0:0.5, y domain=0.5:1
                ] {0.9};
                \addplot3[
                    surf,
                    opacity=0.1,
                    red,
                    domain=0.5:1, y domain=0:0.5
                ] {0.5};
                \addplot3[
                    surf,
                    opacity=0.1,
                    red,
                    domain=0.5:1, y domain=0.5:1
                ] {0.1};
            \fi
        \end{axis}
    \end{tikzpicture}
}

\newcommand{\polyplot}[1]{
    \begin{tikzpicture}
        \begin{axis}[
            height=5cm,
            width=8cm,
            xlabel=$x$,
            ylabel=$y$,
            xmin=-0.1,
            xmax=1,
            xtick={-0.1, 0.175, 0.45, 0.725, 1},
            xticklabels={0, 0.25, 0.5, 0.75, 1},
            ymajorticks=false,
            ymin=-10,
            ymax=2.5,
            xtick pos=bottom
        ]
            \ifnum#1<3
                \addplot[
                    only marks,
                    blue,
                    samples=100,
                    domain=-0.1:1,
                    opacity=0.5
                ] (x, 300*x^5 - 750 * x^4 + 630 * x^3 - 210 * x^2 + 20 * x + 1 + rand);
            \fi

            \ifnum#1=1
                \addplot[
                    very thick,
                    red
                ] coordinates {
                    (-0.1,-3)
                    (-0.05, -3)
                    (-0.05, 0)
                    (0, 0)
                    (0, 1.5)
                    (0.2, 1.5)
                    (0.2, -0.5)
                    (0.65, -0.5)
                    (0.65, -1)
                    (0.7, -1)
                    (0.7, -3)
                    (0.85, -3)
                    (0.85, -6)
                    (0.95, -6)
                    (0.95, -9)
                    (1, -9)
                };
            \fi
            \ifnum#1>1
                \addplot[
                    very thick,
                    red
                ] coordinates {
                    (-0.1,-3)
                    (-0.05, -3)
                    (-0.05, 0)
                    (0, 0)
                    (0, 1)
                    (0.05, 1)
                    (0.05, 1.5)
                    (0.15, 1.5)
                    (0.15, 1)
                    (0.2, 1)
                    (0.2, 0)
                    (0.4, 0)
                    (0.4, -0.5)
                    (0.65, -0.5)
                    (0.65, -1)
                    (0.7, -1)
                    (0.7, -2.5)
                    (0.75, -2.5)
                    (0.75, -4)
                    (0.85, -4)
                    (0.85, -6)
                    (0.95, -6)
                    (0.95, -9)
                    (1, -9)
                };
            \fi
            \ifnum#1=4
                \node[anchor=south, font=\scriptsize, inner sep=2pt] at (axis cs: 0.525, -0.5) {
                    $\hat{y}=-0.5$
                };
                \draw[red] (axis cs: 0.4, 2.5) -- (axis cs: 0.4, -10);
                \draw[red] (axis cs: 0.65, 2.5) -- (axis cs: 0.65, -10);

                \node[anchor=south, font=\scriptsize, inner sep=2pt] at (axis cs: 0.3, -0.0) {
                    $\hat{y}=0.0$
                };
                \draw[red] (axis cs: 0.2, 2.5) -- (axis cs: 0.2, -10);
                \draw[red] (axis cs: 0.4, 2.5) -- (axis cs: 0.4, -10);
            \fi
        \end{axis}
    \end{tikzpicture}
}

\newsavebox{\treedata}
\sbox{\treedata}{
    \treedataplot{0}
}

\newsavebox{\treestep}
\sbox{\treestep}{
    \treedataplot{1}
}

\newsavebox{\polydata}
\sbox{\polydata}{
    \polyplot{0}
}

\newsavebox{\polystep}
\sbox{\polystep}{
    \polyplot{1}
}

\newsavebox{\polyprecise}
\sbox{\polyprecise}{
    \polyplot{2}
}
\newsavebox{\polyconstant}
\sbox{\polyconstant}{
    \polyplot{3}
}
\newsavebox{\polyregions}
\sbox{\polyregions}{
    \polyplot{4}
}

\newsavebox{\emptyplot}
\sbox{\emptyplot}{
    \begin{tikzpicture}
        \begin{axis}[
            height=5cm,
            width=8cm,
            xlabel=$x$,
            ylabel=$y$,
            ymajorticks=false,
            xmajorticks=false,
            xmin=0,
            xmax=1,
            ymin=0,
            ymax=1
        ]
            \node[] at (axis cs: 0.5, 0.5) {
                \Huge{?}
            };
        \end{axis}
    \end{tikzpicture}
}

\begin{frame}{Tree-based models: Motivation}
    \begin{tikzpicture}
        \node[draw=black] at (-5.25, -3.5) {};
        \node[draw=black] at (5.25, 3.5) {};

        \visible<1-2>{
            \node[] at (0, 1) {
                \usebox{\treedata}
            };
        }
        \visible<2>{
            \node[font=\scriptsize\selectfont] at (0, -2.5) {
                $y=
                \begin{cases}
                    0.8&x_1\leq0.5\ \&\ x_2\leq0.5\\
                    0.9&x_1\leq0.5\ \&\ x_2>0.5\\
                    0.5&x_1>0.5\ \&\ x_2\leq0.5\\
                    0.1&x_1>0.5\ \&\ x_2>0.5\\
                \end{cases}
                $
            };
        }
        \visible<3-4>{
            \node[] at (0, 1) {
                \usebox{\polydata}
            };
        }
        \visible<4>{
            \node[] at (0, -2.5) {
                $y=?x^5+?x^4+?x^3+?x^2+?x+?$
            };
        }
        \visible<5>{
            \node[] at (0, 1) {
                \usebox{\emptyplot}
            };
        }
        \visible<6>{
            \node[] at (0, 1) {
                \usebox{\treestep}
            };
        }
        \visible<7>{
            \node[] at (0, 1) {
                \usebox{\polystep}
            };
        }
        \visible<8>{
            \node[] at (0, 1) {
                \usebox{\polyprecise}
            };
        }
        \visible<9>{
            \node[] at (0, 1) {
                \usebox{\polyconstant}
            };
        }
        \visible<10-11>{
            \node[] at (0, 1) {
                \usebox{\polyregions}
            };
        }
        \visible<11>{
            \node[font=\footnotesize\selectfont] at (0, -2.5) {
                $\hat{y}=
                \begin{cases}
                    ...\\
                    0.0&x\geq0.27\ \& \ x<0.45\\
                    -0.5&x\geq0.45\ \& \ x<0.69\\
                    ...\\
                \end{cases}
                $
            };
        }
    \end{tikzpicture}
\end{frame}

\newcommand{\stratificationplot}[1]{
    \begin{tikzpicture}
        \begin{axis}[
            height=5cm,
            width=5cm,
            xlabel=\scriptsize{$x_1$},
            ylabel=\scriptsize{$x_2$},
            ticklabel style={font=\scriptsize\selectfont},
            tick style={draw=none},
            clip=false
        ]
        \end{axis}
    \end{tikzpicture}
}

\newsavebox{\stratificationregions}
\sbox{\stratificationregions}{
    \stratificationplot{0}
}

\begin{frame}{Tree-based models: Decision trees}
    \begin{tikzpicture}
        \node[draw=black] at (-5.25, -3.5) {};
        \node[draw=black] at (5.25, 3.5) {};

        \visible<1>{
            \node[anchor=east] at (5, 0) {
                \usebox{\stratificationregions}
            };
        }

    \end{tikzpicture}
\end{frame}

    % \begin{frame}{Exercise 5}

\begin{enumerate}
    \item Download the Credit.csv dataset from ISL
    \item Preprocess the dataset as you see fit.
    \item Fit one or more non-linear models, choosing model types and hyperparameters according to what you've learned
    \item Report the performance of the best model
    \item Reflect on the choices you made
    \item Reflect upon the performance of the model(s)
\end{enumerate}

\end{frame}


\end{document}
