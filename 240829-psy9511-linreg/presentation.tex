\documentclass[10pt]{beamer}

\usetheme{metropolis}

\usepackage[export]{adjustbox}
\usepackage{array}
\usepackage{bbm}
\usepackage{emoji}
\usepackage{etoolbox}
\usepackage{graphicx}
\usepackage{hyperref}
\usepackage{listings}
\usepackage{pgfplots}
\usepackage{tikz}
\usepackage{xcolor}

\usepgfplotslibrary{fillbetween}

\usetikzlibrary{arrows.meta}
\usetikzlibrary{calc}
\usetikzlibrary{patterns}

\hypersetup{
    colorlinks=true,
    linkcolor=white,
    urlcolor=blue!80
}

\definecolor{uiored}{HTML}{DD0000}
\definecolor{uiolightred}{HTML}{FB6666}
\definecolor{uioredtone}{HTML}{FEE0E0}
\definecolor{uioblue}{HTML}{3E31D6}
\definecolor{uiolightblue}{HTML}{86A4F7}
\definecolor{uioblueone}{HTML}{E6ECFF}
\definecolor{uiogreen}{HTML}{2EC483}
\definecolor{uiolightgreen}{HTML}{6CE1AB}
\definecolor{uiogreentone}{HTML}{CEFFDF}
\definecolor{uioorange}{HTML}{FEA11B}
\definecolor{uiolightorange}{HTML}{FDCB87}
\definecolor{uioorangetone}{HTML}{FFE8D4}
\definecolor{uioyellow}{HTML}{FFFEA7}
\definecolor{uiogray}{HTML}{B2B3B7}

\colorlet{mainbackground}{uiored}

\setbeamercolor{frametitle}{bg=mainbackground, fg=white}
\setbeamercolor{title separator}{fg=mainbackground}
\setbeamercolor{progress bar in section page}{fg=white, bg=uiogray}

\def\logowidth{4cm}

\makeatletter
\setbeamertemplate{section page}
{
  \begingroup

    \vspace{4.3cm}
    {\usebeamercolor[fg]{section title}\usebeamerfont{section title}\insertsectionhead}\\[-1ex]
    {\centering\color{white}\rule{\linewidth}{1pt}\par} % the horizontal line

    \vspace*{3.1cm}
    \begin{center}
        \includegraphics[width=\logowidth,valign=c]{data/uio_logo_full_white.png} % Adjust width and path to your logo as needed
    \end{center}

  \endgroup
}
\makeatother

\AtBeginSection{
  {
    \setbeamercolor{background canvas}{bg=uiored}
    \setbeamercolor{section title}{fg=white}
    \frame[plain,c,noframenumbering]{\sectionpage}
    \setbeamercolor{background canvas}{bg=black!2}
  }
}



\setbeamertemplate{footline}{
    \ifnum\insertframenumber=1
        % Title page, no footer
    \else
        \begin{tikzpicture}[remember picture,overlay]
            \fill[mainbackground] (current page.south west) rectangle ([yshift=0.45cm]current page.south east); % Draw filled rectangle

            % Logo
            \node[anchor=west, yshift=0.225cm] at (current page.south west) {\includegraphics[height=1.2cm]{data/uio_logo_white.png}};

            % Title and subtitle
            \node[align=center, yshift=0.225cm] at (current page.south) {\textcolor{white}{\textbf{\inserttitle}}\\[0.05cm]\textcolor{white}{\insertsubtitle}};

            % Page number
            \node[anchor=east, yshift=0.225cm, xshift=-0.2cm, align=right] at (current page.south east) {\textcolor{white}{\insertframenumber/\inserttotalframenumber}};
        \end{tikzpicture}
    \fi
}

\lstdefinestyle{Core}{
    identifierstyle=\color[RGB]{0, 0, 0},
    stringstyle=\color[RGB]{205, 49, 49},
    showstringspaces=false,
    breaklines,
    xleftmargin=3pt,
    xrightmargin=3pt,
    framesep=3pt,
    aboveskip=-1.5pt,
    belowskip=-0.5pt,
    showlines=true,
}

\lstdefinestyle{RInput}{
    style=Core,
    language=R,
    keywordstyle=\color[RGB]{17, 115, 187},
    commentstyle=\color[RGB]{0, 128, 0},
    identifierstyle=\color[RGB]{0, 0, 0},
    stringstyle=\color[RGB]{205, 49, 49},
    backgroundcolor=\color[RGB]{255, 255, 255},
    frame=single,
    rulecolor=\color[RGB]{0, 0, 0},
}

\lstdefinestyle{PythonInput}{
    style=Core,
    language=Python,
    keywordstyle=\color[RGB]{26, 13, 171},
    commentstyle=\color[RGB]{0, 128, 0},
    backgroundcolor=\color[RGB]{245, 245, 245},
    rulecolor=\color[RGB]{192, 192, 192},
    frame=tblr,
}

\lstdefinestyle{ROutput}{
    style=Core,
    language=R,
    backgroundcolor=\color[RGB]{255, 255, 255},
    commentstyle=\color[HTML]{009900},
    stringstyle=\color[HTML]{0000FF},
    keywordstyle=\color[HTML]{000000},
    numberstyle=\tiny\color[HTML]{000000},
    breakatwhitespace=true,
    frame=single,
    rulecolor=\color{black},
}

\lstdefinestyle{PythonOutput}{
    backgroundcolor=\color[RGB]{255, 255, 255},
    rulecolor=\color[RGB]{192, 192, 192},
    frame=single,
    numbers=none,
    showstringspaces=false,
    breakatwhitespace=true,
    keywordstyle=\color[RGB]{255, 0, 0},
    morekeywords={AttributeError}
}

\newcommand{\PythonInputNode}[6]{%
    \node[
        minimum width=#4,
        text width=#4,
        align=left,
        inner sep=0pt,
        outer sep=0pt,
        anchor=north west,
        label={[blue,
                anchor=north east,
                font=\ttfamily\fontsize{\the\numexpr#5-1\relax}{#5}\selectfont,
                inner sep=0pt,
                outer sep=0pt,
                xshift=-3pt,
                yshift=-3pt
                ]north west:In{[}#1{]}:},
    ] (#3) at #2 {
    \begin{lstlisting}[
        style=PythonInput,
        linewidth=#4,
        basicstyle=\ttfamily\fontsize{\the\numexpr#5-1\relax}{#5}\selectfont,
        numberstyle=\fontsize{\the\numexpr#5-1\relax}{#5}\selectfont\color[RGB]{128, 128, 128},
    ]^^J
        #6
    \end{lstlisting}
    };
}

\newcommand{\PythonOutputNode}[6]{%
    \node[
        minimum width=#4,
        text width=#4,
        align=left,
        inner sep=0pt,
        outer sep=0pt,
        anchor=north west,
        label={[red,
                anchor=north east,
                font=\ttfamily\fontsize{\the\numexpr#5-1\relax}{#5}\selectfont,
                inner sep=0pt,
                outer sep=0pt,
                xshift=-6pt,
                yshift=-11pt
                ]north west:Out{[}#1{]}:}
    ] (pythonoutput) (#3) at #2 {
        \begin{lstlisting}[
            style=PythonOutput,
            basicstyle=\ttfamily\fontsize{\the\numexpr#5-1\relax}{#5}\selectfont,
            numberstyle=\fontsize{\the\numexpr#5-1\relax}{#5}\selectfont\color[RGB]{128, 128, 128},
        ]^^J
            #6
        \end{lstlisting}
    };
}

\newcommand{\RInputNode}[5]{
    \node[
        minimum width=#3,
        text width=#3,
        align=left,
        inner sep=0pt,
        outer sep=0pt,
        draw=black,
        anchor=north west
    ] (#2) at #1 {
        \begin{lstlisting}[
            style=RInput,
            linewidth=\textwidth,
            basicstyle=\ttfamily\fontsize{\the\numexpr#4-1\relax}{#4}\selectfont,
            numberstyle=\fontsize{\the\numexpr#4-1\relax}{#4}\selectfont\color[RGB]{128, 128, 128},
        ]^^J
            #5
        \end{lstlisting}
    };
}

\title{PSY9511: Seminar 4}
\subtitle{The basics of regression and classification}
\author{Esten H. Leonardsen}
\date{29.05.2024}

\titlegraphic{
	\centering
	\vspace{7.7cm}
	\includegraphics[width=\logowidth]{data/uio_logo_full.png}
}

\begin{document}
	\begin{frame}
	 	\titlepage
	\end{frame}

    \begin{frame}[t]{Recap}
        \only<1-4>{
            \vspace{0.5cm}
            \underline{What is statistical learning?}\\
            \vspace{0.5cm}


            \only<2>{\small{\textbf{Inferentiental view:} Finding a function $\hat{f}(X)$ that describes the relationship between some input variables $X$ and an output variable $y$.}\\}
            \only<3>{\textcolor{gray!20}{\small{\textbf{Inferentiental view:} Finding a function $\hat{f}(X)$ that describes the relationship between some input variables $X$ and an output variable $y$.}\\}}
            \only<4>{\small{\textcolor{gray!20}{\textbf{Inferentiental view:}} Finding a function $\hat{f}(X)$ \textcolor{gray!20}{that describes the relationship between some input variables $X$ and an output variable $y$.}\\}}
            \vspace{0.25cm}
            \vspace{0.25cm}
            \only<3>{\small{\textbf{Predictive view:} Finding a function $\hat{f}(X)$ that, when given a new set of inputs $X$ allows us to predict an output $y$.}}
            \only<4>{\small{\textcolor{gray!20}{\textbf{Predictive view:}} Finding a function $\hat{f}(X)$ \textcolor{gray!20}{ that, when given a new set of inputs $X$ allows us to predict an output $y$.}}}
        }

        \only<5-9>{
            \centering
            \begin{tikzpicture}
                \begin{axis}[
                    xtick pos=bottom,
                    ytick pos=left,
                    xmin=30,
                    xmax=240,
                    ymin=3,
                    ymax=50,
                    xlabel=Horsepower (x),
                    ylabel=Miles per gallon (y),
                    height=6.8cm,
                    width=11cm
                ]
                    \addplot[
                        only marks,
                        opacity=0.6
                    ] table [
                        col sep=comma,
                        x=horsepower,
                        y=mpg
                    ] {data/Auto.csv};
                    \only<7>{
                        \addplot[
                            domain=30:240,
                            samples=100,
                            color=red,
                            thick
                        ] {39.93 - 0.1578*x};
                    }
                    \only<8>{
                        \addplot[
                            domain=30:240,
                            samples=100,
                            color=red,
                            thick
                        ] {23.44};
                    }
                    \only<9>{
                        \addplot[
                            domain=30:240,
                            samples=100,
                            color=red,
                            thick
                        ] {exp(3.86-0.0073*x)};
                    }
                \end{axis}
            \end{tikzpicture}\\
            \vspace{0.2cm}
            \hspace{1.3cm}
            \only<6>{
                \textcolor{red}{$\hat{y}=\hat{f}(x)$}
            }
            \only<7>{
                \textcolor{red}{$\hat{y}=39.93-0.1578x$}
            }
            \only<8>{
                \textcolor{red}{$\hat{y}=23.44$}
            }
            \only<9>{
                \textcolor{red}{$\hat{y}=e^{3.86-0.0073x}$}
            }
        }
        \only<10>{
            \centering
            \begin{tikzpicture}
                \node[inner sep=0pt, draw=black] {
                    \includegraphics[width=6cm]{data/bias-variance.png}
                };
            \end{tikzpicture}
        }
    \end{frame}

    \begin{frame}[t]{Outline}
        \underline{Plan for the day:}
        \begin{itemize}
            \item Different types of outputs $y$: Regression vs classification
            \item Linear regression: Restricting the scope of $\hat{f}(X)$
            \begin{itemize}
                \item Live coding
            \end{itemize}
            \item k-Nearest Neighbours
            \item Logistic regression: Extending linear regression to classification
            \begin{itemize}
                \item Live coding
            \end{itemize}
            \item Generative models
        \end{itemize}

        \underline{Plan for future lectures:}
        \begin{itemize}
            \item How do we evaluate how good our models are? (Lecture 3)
            \item Complex solutions to regression and classification problems (Lecture 4 and onwards)
        \end{itemize}
    \end{frame}

    \newcommand{\weightnode}[3]{
        \node[circle, draw=black, fill=teal!60, opacity=#3] at (#1, #2*3.5) {};
    }

    \newcommand{\weightplot}[1]{
        \begin{tikzpicture}
            \ifnum#1<3
                \draw[|-Latex] (0, 3.3*3.5) -- (0, 4.6*3.5);
                \weightnode{0}{3.850}{1.0}
                \weightnode{0}{4.354}{1.0}
            \fi
            \ifnum#1>2
                \draw[|-Latex,black!20] (0, 3.3*3.5) -- (0, 4.6*3.5);
                \weightnode{0}{3.850}{0.2}
                \weightnode{0}{4.354}{0.2}
            \fi
            \ifnum#1=1
                \weightnode{0}{3.504}{1.0}
                \weightnode{0}{3.693}{1.0}
                \weightnode{0}{3.436}{1.0}
                \weightnode{0}{3.433}{1.0}
                \weightnode{0}{3.449}{1.0}
                \weightnode{0}{4.341}{1.0}
                \weightnode{0}{4.354}{1.0}
                \weightnode{0}{4.312}{1.0}
                \weightnode{0}{4.425}{1.0}
            \fi
            \ifnum#1>1
                \weightnode{0}{3.504}{0.2}
                \weightnode{0}{3.693}{0.2}
                \weightnode{0}{3.436}{0.2}
                \weightnode{0}{3.433}{0.2}
                \weightnode{0}{3.449}{0.2}
                \weightnode{0}{4.341}{0.2}
                \weightnode{0}{4.354}{0.2}
                \weightnode{0}{4.312}{0.2}
                \weightnode{0}{4.425}{0.2}

                \ifnum#1=2
                    \draw[|-|, dashed] (0.5, 3.850*3.5) -- (0.5, 4.354*3.5);
                    \node[anchor=west] at (0.5, 4.102*3.5) {\small{504}};
                \fi
            \fi
        \end{tikzpicture}
    }

    \newsavebox{\weightboxfirst}
    \sbox{\weightboxfirst}{
        \weightplot{1}
    }

    \newsavebox{\weightboxsecond}
    \sbox{\weightboxsecond}{
        \weightplot{2}
    }

    \newsavebox{\weightboxhidden}
    \sbox{\weightboxhidden}{
        \weightplot{3}
    }

    \newcommand{\manufacturernode}[5]{
        \node[circle, draw=black, fill=#3, opacity=#4] (#5) at (#1, #2) {};
    }


    \newcommand{\manufacturerplot}[1]{
        \begin{tikzpicture}
            \colorlet{chewy}{blue!60}
            \colorlet{ford}{red!60}
            \colorlet{pontiac}{green!60}

            \manufacturernode{0}{0}{chewy}{1.0}{c0}
            \manufacturernode{1}{-1.2}{ford}{1.0}{f0}
            \manufacturernode{-0.3}{-1.1}{pontiac}{1.0}{p0}

            \ifnum#1=1
                \manufacturernode{0.5}{0.4}{chewy}{1.0}{}
                \manufacturernode{0.1}{0.5}{chewy}{1.0}{}

                \manufacturernode{1.4}{-1.0}{ford}{1.0}{}
                \manufacturernode{1.5}{-1.5}{ford}{1.0}{}
                \manufacturernode{0.9}{-1.6}{ford}{1.0}{}

                \manufacturernode{-0.1}{-1.5}{pontiac}{1.0}{}
                \manufacturernode{0.2}{-1.05}{pontiac}{1.0}{}
            \fi
            \ifnum#1=2
                \manufacturernode{0.5}{0.4}{chewy}{0.2}{}
                \manufacturernode{0.1}{0.5}{chewy}{0.2}{}

                \manufacturernode{1.4}{-1.0}{ford}{0.2}{}
                \manufacturernode{1.5}{-1.5}{ford}{0.2}{}
                \manufacturernode{0.9}{-1.6}{ford}{0.2}{}

                \manufacturernode{-0.1}{-1.5}{pontiac}{0.2}{}
                \manufacturernode{0.2}{-1.05}{pontiac}{0.2}{}

                \draw[->, dashed] (c0) -- (f0) node[pos=0.7, above=0.15cm] {\footnotesize{?}};
                \draw[->, dashed] (c0) -- (p0) node[pos=0.9, above=0.2cm] {\footnotesize{?}};;
            \fi

            \node[anchor=south] at (0.25, 0.7) {\textbf{\small{Chevrolet}}};
            \node[anchor=north] at (1.2, -1.8) {\textbf{\small{Ford}}};
            \node[anchor=north] at (0.05, -1.7) {\textbf{\small{Pontiac}}};

        \end{tikzpicture}
    }

    \newsavebox{\manufacturerboxfirst}
    \sbox{\manufacturerboxfirst}{
        \manufacturerplot{1}
    }

    \newsavebox{\manufacturerboxsecond}
    \sbox{\manufacturerboxsecond}{
        \manufacturerplot{2}
    }

    \begin{frame}{Regression vs. classification}
        \only<1-5>{
            \begin{tikzpicture}
                \node[] at (-5.25, -3) {};
                \node[] at (5.25, 3) {};
                \only<1-2,4>{
                    \node[] at (0, 0) {
                        \begin{tabular}{cc}
                            \textbf{Weight}&\textbf{Manufacturer}\\
                            3504&Chevrolet\\
                            3693&Ford\\
                            3436&Pontiac\\
                            3433&Pontiac\\
                            3449&Ford\\
                            4341&Ford\\
                            4354&Chevrolet\\
                            4312&Ford\\
                            4425&Pontiac\\
                            3850&Chevrolet\\
                        \end{tabular}
                    };
                }
                \onslide<2,4>{
                    \node[anchor=west] at (-4, 0) {
                        \usebox{\weightboxfirst}
                    };
                }
                \only<3>{
                    \node[anchor=west] at (-4, 0) {
                        \usebox{\weightboxsecond}
                    };
                    \node[] at (0, 0) {
                        \begin{tabular}{cc}
                            \textbf{Weight}&\textcolor{gray!20}{\textbf{Manufacturer}}\\
                            \textcolor{gray!20}{3504}&\textcolor{gray!20}{Chevrolet}\\
                            \textcolor{gray!20}{3693}&\textcolor{gray!20}{Ford}\\
                            \textcolor{gray!20}{3436}&\textcolor{gray!20}{Pontiac}\\
                            \textcolor{gray!20}{3433}&\textcolor{gray!20}{Pontiac}\\
                            \textcolor{gray!20}{3449}&\textcolor{gray!20}{Ford}\\
                            \textcolor{gray!20}{4341}&\textcolor{gray!20}{Ford}\\
                            4354&\textcolor{gray!20}{Chevrolet}\\
                            \textcolor{gray!20}{4312}&\textcolor{gray!20}{Ford}\\
                            \textcolor{gray!20}{4425}&\textcolor{gray!20}{Pontiac}\\
                            3850&\textcolor{gray!20}{Chevrolet}\\
                        \end{tabular}
                    };
                }
                \only<4>{
                    \node[anchor=east] at (5, 0) {
                        \usebox{\manufacturerboxfirst}
                    };
                }
                \only<5>{
                    \node[anchor=west] at (-4, 0) {
                        \usebox{\weightboxhidden}
                    };
                    \node[anchor=east] at (5, 0) {
                        \usebox{\manufacturerboxsecond}
                    };
                    \node[] at (0, 0) {
                        \begin{tabular}{cc}
                            \textcolor{gray!20}{\textbf{Weight}}&\textbf{Manufacturer}\\
                            \textcolor{gray!20}{3504}&\textcolor{gray!20}{Chevrolet}\\
                            \textcolor{gray!20}{3693}&Ford\\
                            \textcolor{gray!20}{3436}&\textcolor{gray!20}{Pontiac}\\
                            \textcolor{gray!20}{3433}&\textcolor{gray!20}{Pontiac}\\
                            \textcolor{gray!20}{3449}&\textcolor{gray!20}{Ford}\\
                            \textcolor{gray!20}{4341}&\textcolor{gray!20}{Ford}\\
                            \textcolor{gray!20}{4354}&Chevrolet\\
                            \textcolor{gray!20}{4312}&\textcolor{gray!20}{Ford}\\
                            \textcolor{gray!20}{4425}&Pontiac\\
                            \textcolor{gray!20}{3850}&\textcolor{gray!20}{Chevrolet}\\
                        \end{tabular}
                    };

                }
            \end{tikzpicture}
        }
        \only<6>{
            \centering
            \begin{tikzpicture}
                \node[align=center, anchor=north] at (0.5, 0) {
                    \underline{\small{Mean squared error (MSE):}}\\[0.3cm]
                    $\frac{1}{n}\sum_{i=1}^{n}(y_i-\hat{y}_i)^2$
                };
                \node[align=center, anchor=north] at (6.5, 0) {
                    \underline{\small{Accuracy}}\\[0.3cm]
                    $\frac{1}{n}\sum_{i=1}^{n}\mathbbm{1}_{eq} (y_i,\hat{y}_i)$,\\
                    $\mathbbm{1}_{eq}(a, b)=\begin{cases}1 & \text{if } a=b\\0 & \text{if } a\neq b\end{cases}$
                };
            \end{tikzpicture}
        }
        \only<7>{
            Regression:
            \begin{itemize}
                \item Predicting reaction time of a cognitive task based on sleep scores
                \item Predicting the age of an individual based on a brain scan
                \item Predicting the anxiety scores based on questionnaire data
            \end{itemize}
            Classification:
            \begin{itemize}
                \item Predicting whether an individual is depressed based on cell phone usage data.
                \item Predicting if a patient has dementia based on a brain scan
                \item Predicting whether a patient is happy based on their facial expression
            \end{itemize}
        }
        \only<8-9>{
            \centering
            \begin{tikzpicture}
                \node[] at (-3, -3) {};
                \node[] at (3, 3) {};

                \node[] (l) at (0, 1.5) {Large};
                \node[] (m) at (0, 0) {Medium};
                \node[] (s) at (0, -1.5) {Small};

                \only<9>{
                    \draw[-stealth, line width=1mm] (-1, -1.5) -- (-1, 1.5);
                    \draw[stealth-stealth] (l) -- (m) node[midway, right] {?};
                    \draw[stealth-stealth] (m) -- (s) node[midway, right] {?};
                }
            \end{tikzpicture}
        }
        \only<10-11>{
            \centering
            \begin{tikzpicture}
                \node[] at (-4.5, 1) {};
                \node[] at (5.5, -4) {};
                \node[] (sentence) at (0, 0) {
                    The quick brown fox jumps over the lazy
                };
                \draw[red] ($ (sentence.south east) + (0, 0.15) $) -- ($ (sentence.south east) + (1, 0.15) $);
                \only<11>{
                    \draw (2,-2) ellipse (1.8cm and 0.9cm);
                    \node[] at (3, -2.4) {\footnotesize{cat}};
                    \node[] at (1, -2.2) {\textbf{\footnotesize{dog}}};
                    \node[] at (2.1, -1.6) {\footnotesize{hurdle}};
                    \node[] at (2, -3.3) {\small{Classification}};
                }
            \end{tikzpicture}
        }
        \only<12-13>{
            \centering
            \begin{tikzpicture}
                \node[] at (-5.5, 2) {};
                \node[] at (5, -4) {};

                \node[align=center] (prompt) at (-4, 0) {
                    "Students taking\\a machine learning\\class"
                };
                \node[fill=white, draw=black] (model) at (-1, 0) {
                    \includegraphics[width=1cm]{data/chatgpt.png}
                };
                \node[] (output)at (2.5, 0) {
                    \includegraphics[width=4cm]{data/ml-students.png}
                };
                \draw[-Latex] (prompt) -- (model);
                \draw[-Latex] (model) -- (output);

                \only<13>{
                    \node[draw=black, fill=red!50, inner sep=5pt, opacity=0.9] at (-2, -3) {};
                    \node[draw=black, fill=green!75, inner sep=5pt, opacity=0.9] (green) at (-1.9, -3.1) {};
                    \node[draw=black, fill=blue!25, inner sep=5pt, opacity=0.9] at (-1.8, -3.2) {};
                    \node[anchor=west] at ($ (green.east) + (0.2, 0) $) {(128, 192, 64)};
                }
            \end{tikzpicture}
        }
        \only<14>{
            \begin{tikzpicture}
                \node[anchor=west] at (0, 0) {
                    \includegraphics[
                        height=4.5cm,
                        trim={0 0 13.8cm 0},
                        clip
                    ]{data/tumor.png}
                };
            \end{tikzpicture}
        }
        \only<15>{
            \begin{tikzpicture}
                \node[anchor=west] at (0, 0) {
                    \includegraphics[height=4.5cm]{data/tumor.png}
                };
            \end{tikzpicture}
        }
        \only<16-17>{
            \begin{tikzpicture}
                \node[] at (-5.5, 3.5) {};
                \node[] at (5, -3.5) {};
                \node[] at (0, 0) {
                    \includegraphics[width=5cm]{data/dog.png}
                };
                \draw[orange, thick] (-0.8, 1.4) -- (1.7, 1.4) -- (1.7, -1.8) -- (-0.8, -1.8) -- cycle;

                \only<17>{
                    \node[anchor=south, orange] at (0.45, 1.4) {Dog};
                    \node[anchor=south, orange] at (-0.8, 1.4) {\footnotesize{(-0.8, 1.4)}};
                    \node[anchor=south, orange] at (1.7, 1.4) {\footnotesize{(1.7, 1.4)}};
                    \node[anchor=north, orange] at (-0.8, -1.8) {\footnotesize{(-0.8, -1.8)}};
                    \node[anchor=north, orange] at (1.7, -1.8) {\footnotesize{(1.7, -1.8)}};
                }

            \end{tikzpicture}
        }
        \only<18-19>{
            \begin{tikzpicture}
                \node[] at (-5.5, 3.5) {};
                \node[] at (5, -3.5) {};

                \node[inner sep=0pt, draw=black] at (0, 1) {
                    \includegraphics[width=3.5cm]{data/robot.png}
                };
                \only<19>{
                    \node[inner sep=0pt, draw=black] at (0, -2.5) {
                        \includegraphics[width=5cm]{data/sound.png}
                    };
                }
            \end{tikzpicture}
        }
        \only<20>{
            Different types of outputs $y$ require us to use different mathematical formulations of the problem we want to solve.
            \begin{itemize}
                \item Problems with quantitative outputs are solved via regression, often by minimizing the mean squared error.
                \item Problems with qualitative outputs are solved by classification, often by maximizing accuracy.
                \item Ordinal regression falls between the two, with qualitative classes that have some kind of order.
                \item A variety of other types of problems can be seen as special cases of these two.
            \end{itemize}
        }
    \end{frame}

    \newsavebox{\linreg}
    \sbox{\linreg}{
        \begin{tikzpicture}
            \begin{axis}[
                xmin=30,
                xmax=240,
                ymin=3,
                ymax=50,
                width=5.2cm,
                height=5.2cm,
                ylabel=mpg,
                xlabel=horsepower
            ]
                \addplot[
                    only marks,
                    opacity=0.3
                ] table [
                    col sep=comma,
                    x=horsepower,
                    y=mpg
                ] {data/Auto.csv};
                \addplot[
                    domain=30:240,
                    samples=100,
                    color=red,
                    thick
                ] {39.93 - 0.1578*x};
            \end{axis}
        \end{tikzpicture}
    }

    \newsavebox{\residuals}
    \sbox{\residuals}{
        \begin{tikzpicture}
            \begin{axis}[
                xmin=30,
                xmax=240,
                ymin=-20,
                ymax=20,
                width=5.2cm,
                height=5.2cm,
                ylabel=$y-\hat{y}$,
                xlabel=horsepower
            ]
                \addplot[
                    only marks,
                    opacity=0.3
                ] table [
                    col sep=comma,
                    x=horsepower,
                    y=residuals
                ] {data/residuals.csv};
                \addplot[
                    domain=30:240,
                    samples=100,
                    color=red,
                    thick
                ] {0};
            \end{axis}
        \end{tikzpicture}
    }

    \newsavebox{\multilinreg}
    \sbox{\multilinreg}{
        \begin{tikzpicture}
            \begin{axis}[
                width=8cm,
                height=7cm, % Size of the plot
                xlabel={Horsepower ($x_1$)},
                ylabel={Weight ($x_2$)},
                zlabel={Miles per gallon (y)},
                view={25}{25}, % Adjust viewing angle colormap/cool,
                xmin=30,
                xmax=240,
                ymin=1500,
                ymax=5000,
                zmin=3,
                zmax=50,
                xlabel style={sloped},
                ylabel style={sloped},
                xmajorgrids=true,
                ymajorgrids=true,
                zmajorgrids=true,
                xtick style={draw=none},
                ytick style={draw=none},
                ztick style={draw=none}
            ]
                \addplot3[
                    surf,
                    samples=20,
                    domain=30:240,
                    y domain=1500:5000,
                ] {45.64-0.04*x-0.005*y};
                \addplot3[
                    only marks,
                    mark size=2.5pt,
                    opacity=0.25
                ] table [
                    x=horsepower,
                    y=weight,
                    z=mpg,
                    col sep=comma
                ] {data/Auto.csv};
            \end{axis}
        \end{tikzpicture}
    }

    \newcommand{\binaryplot}[1]{
        \begin{tikzpicture}
            \begin{axis}[
                xmin=-0.5,
                xmax=1.5,
                xtick={0, 1},
                xticklabels={Ford, Chevrolet},
                ylabel=mpg,
                height=7cm,
                width=7cm
            ]
                \addplot[
                    only marks,
                    fill=blue,
                    opacity=0.6
                ] coordinates {
                    (-0.065, 18)
                    (-0.080, 15)
                    (-0.038, 18)
                    (-0.139, 16)
                    (0.028, 17)
                    (-0.140, 15)
                    (-0.149, 14)
                    (0.137, 14)
                    (0.043, 14)
                    (-0.101, 15)
                    (0.007, 15)
                    (-0.145, 14)
                    (-0.106, 15)
                    (-0.013, 14)
                    (-0.120, 24)
                    (0.150, 22)
                    (-0.166, 18)
                    (-0.052, 21)
                    (0.031, 27)
                };
                \addplot[
                    only marks,
                    fill=red,
                    opacity=0.6
                ] coordinates {
                    (0.985, 38)
                    (0.943, 36)
                    (1.085, 36)
                    (0.971, 36)
                    (0.990, 34)
                    (1.029, 38)
                    (0.958, 32)
                    (1.018, 38)
                    (1.054, 25)
                    (1.064, 38)
                    (1.070, 26)
                    (0.938, 22)
                    (1.131, 32)
                    (1.009, 36)
                    (0.835, 27)
                    (1.109, 27)
                    (1.032, 44)
                    (1.142, 32)
                    (1.115, 28)
                    (1.009, 31)
                };
                \ifnum#1=2
                    \draw[thick, dashed] (axis cs: -0.3, 17.158) -- (axis cs: 0.3, 17.158);
                    \node[anchor=south east, inner sep=1pt] at (axis cs: 0.3, 17.158) {\small{17.1}};
                    \draw[thick, dashed] (axis cs: 0.7, 32.8) -- (axis cs: 1.3, 32.8);
                    \node[anchor=south west, inner sep=1pt] at (axis cs: 0.7, 32.8) {\small{32.8}};
                \fi
            \end{axis}
        \end{tikzpicture}
    }

    \newsavebox{\binary}
    \savebox{\binary}{
        \binaryplot{1}
    }
    \newsavebox{\binarymeans}
    \savebox{\binarymeans}{
        \binaryplot{2}
    }

    \newsavebox{\multilevel}
    \savebox{\multilevel}{
        \begin{tikzpicture}
            \begin{axis}[
                xmin=-0.5,
                xmax=2.5,
                xtick={0, 1, 2},
                xticklabels={Ford, Chevrolet, Pontiac},
                ylabel=mpg,
                height=7cm,
                width=7cm
            ]
                \addplot[
                    only marks,
                    fill=blue,
                    opacity=0.6
                ] coordinates {
                    (-0.065, 18)
                    (-0.080, 15)
                    (-0.038, 18)
                    (-0.139, 16)
                    (0.028, 17)
                    (-0.140, 15)
                    (-0.149, 14)
                    (0.137, 14)
                    (0.043, 14)
                    (-0.101, 15)
                    (0.007, 15)
                    (-0.145, 14)
                    (-0.106, 15)
                    (-0.013, 14)
                    (-0.120, 24)
                    (0.150, 22)
                    (-0.166, 18)
                    (-0.052, 21)
                    (0.031, 27)
                };
                \addplot[
                    only marks,
                    fill=red,
                    opacity=0.6
                ] coordinates {
                    (0.985, 38)
                    (0.943, 36)
                    (1.085, 36)
                    (0.971, 36)
                    (0.990, 34)
                    (1.029, 38)
                    (0.958, 32)
                    (1.018, 38)
                    (1.054, 25)
                    (1.064, 38)
                    (1.070, 26)
                    (0.938, 22)
                    (1.131, 32)
                    (1.009, 36)
                    (0.835, 27)
                    (1.109, 27)
                    (1.032, 44)
                    (1.142, 32)
                    (1.115, 28)
                    (1.009, 31)
                };
                \addplot[
                    only marks,
                    fill=green,
                    opacity=0.6
                ] coordinates {
                    (1.961, 30.200)
                    (1.919, 28.200)
                    (2.061, 28.200)
                    (1.947, 28.200)
                    (1.966, 26.200)
                    (2.005, 30.200)
                    (1.934, 24.200)
                    (1.994, 30.200)
                    (2.030, 17.200)
                    (2.040, 30.200)
                    (2.046, 18.200)
                    (1.914, 14.200)
                    (2.107, 24.200)
                    (1.985, 28.200)
                    (1.811, 19.200)
                    (2.085, 19.200)
                    (2.008, 36.200)
                    (2.118, 24.200)
                    (2.091, 20.200)
                    (1.985, 23.200)
                };

                \draw[thick, dashed] (axis cs: -0.4, 17.158) -- (axis cs: 0.4, 17.158);
                \node[anchor=south east, inner sep=1pt] at (axis cs: 0.4, 17.158) {\small{17.1}};
                \draw[thick, dashed] (axis cs: 0.6, 32.8) -- (axis cs: 1.4, 32.8);
                \node[anchor=south west, inner sep=1pt] at (axis cs: 0.6, 32.8) {\small{32.8}};
                \draw[thick, dashed] (axis cs: 1.6, 25) -- (axis cs: 2.4, 25);
                \node[anchor=south west, inner sep=1pt] at (axis cs: 1.6, 25) {\small{25.1}};

            \end{axis}
        \end{tikzpicture}
    }

    \newcommand{\combinationplot}[1]{
        \begin{tikzpicture}
            \begin{axis}[
                xtick pos=bottom,
                ytick pos=left,
                xmin=30,
                xmax=240,
                ymin=3,
                ymax=50,
                xlabel=Horsepower,
                ylabel=Miles per gallon,
                height=6.5cm,
                width=10.5cm
            ]
                \addplot[
                    only marks,
                    opacity=0.6
                ] table [
                    col sep=comma,
                    x=horsepower,
                    y=mpg
                ] {data/Auto.csv};

                \ifnum#1=1
                    \addplot[
                        domain=30:240,
                        samples=100,
                        color=red,
                        thick
                    ] {40 - 0.1566*x};
                    \addplot[
                        domain=30:240,
                        samples=100,
                        color=blue,
                        thick
                    ] {40 - 0.1566*x-1.85};
                \fi
                \ifnum#1=2
                    \addplot[
                        domain=30:240,
                        samples=100,
                        color=red,
                        thick
                    ] {40 - 0.1566*x};
                    \addplot[
                        domain=30:240,
                        samples=100,
                        color=blue,
                        thick
                    ] {38 - 0.13*x-1.85};
                \fi
            \end{axis}
        \end{tikzpicture}
    }

    \newsavebox{\combinationbox}
    \sbox{\combinationbox}{
        \combinationplot{1}
    }
    \newsavebox{\interactionbox}
    \sbox{\interactionbox}{
        \combinationplot{2}
    }

    \newcommand{\cisplot}[1]{
        \begin{tikzpicture}
            \begin{axis}[
                xtick pos=bottom,
                ytick pos=left,
                xmin=30,
                xmax=240,
                ymin=3,
                ymax=50,
                xlabel=Horsepower,
                ylabel=Miles per gallon,
                height=6.5cm,
                width=10.5cm
            ]
                \addplot[
                    domain=30:240,
                    samples=100,
                    color=red,
                    thick
                ] {40 - 0.1566*x};

                \ifnum#1>1
                    \addplot[
                        only marks,
                    ] coordinates {
                        (105, 23.5)
                    };
                    \draw[|-|, dashed] (axis cs: 105, 17.5) -- (axis cs: 105, 29.5);
                    \node[anchor=south, align=center, font=\scriptsize\linespread{0.8}\selectfont] at (axis cs: 105, 29.5) {
                        Prediction\\
                        interval
                    };
                \fi

                \ifnum#1>2
                    \addplot[
                        only marks,
                        opacity=0.2
                    ] coordinates {
                        (140, 22)
                        (140, 21)
                        (140, 19)
                        (140, 18)
                        (140, 15)
                    };

                    \draw[|-|, densely dotted, red, thick] (axis cs: 140, 16) -- (axis cs: 140, 20);
                    \node[anchor=south, align=center, red, font=\scriptsize\linespread{0.8}\selectfont] at (axis cs: 140, 23) {
                        Confidence\\
                        interval
                    };
                \fi

            \end{axis}
        \end{tikzpicture}
    }

    \newsavebox{\cisdata}
    \sbox{\cisdata}{
        \cisplot{1}
    }

    \newsavebox{\cisprediction}
    \sbox{\cisprediction}{
        \cisplot{2}
    }

    \newsavebox{\cisconfidence}
    \sbox{\cisconfidence}{
        \cisplot{3}
    }

    \section{Linear regression (via ordinary least squares)}

    \begin{frame}[t]{Linear regression (via ordinary least squares)}
        \only<1-16>{
            \begin{tikzpicture}
                \begin{axis}[
                    xtick pos=bottom,
                    ytick pos=left,
                    xmin=30,
                    xmax=240,
                    ymin=3,
                    ymax=50,
                    xlabel=Horsepower (x),
                    ylabel=Miles per gallon (y),
                    height=6.8cm,
                    width=10.7cm
                ]
                    \only<1-5>{
                        \addplot[
                            only marks,
                            opacity=0.6
                        ] table [
                            col sep=comma,
                            x=horsepower,
                            y=mpg
                        ] {data/Auto.csv};
                    }
                    \only<6-16>{
                        \addplot[
                            only marks,
                            opacity=0.05
                        ] table [
                            col sep=comma,
                            x=horsepower,
                            y=mpg
                        ] {data/Auto.csv};
                        \addplot[
                            only marks,
                            opacity=0.8
                        ] coordinates {
                            (115, 28.9)
                        };
                    }
                    \only<7-10>{
                        \draw[dashed] (axis cs: 115, 3) -- (axis cs: 115, 50);
                        \node[anchor=south, draw=black, fill=white, inner sep=2pt] at (axis cs: 115, 5) {\small{115}};
                    }
                    \only<3-16>{
                        \addplot[
                            domain=30:240,
                            samples=100,
                            color=red,
                            thick
                        ] {39.93 - 0.1578*x};
                    }
                    \only<10-16>{
                        \addplot[
                            only marks,
                            red,
                            opacity=0.6
                        ] coordinates {
                            (115, 21.78)
                        };
                    }
                    \only<11-13>{
                        \node[anchor=north] at (axis cs: 115, 21.78) {\small{$\hat{y}$}};
                        \node[anchor=south] at (axis cs: 115, 28.9) {\small{$y$}};
                    }
                    \only<12>{
                        \draw[] (axis cs: 115, 21.78) -- (axis cs: 115, 28.9);
                        \node[anchor=west] at (axis cs: 115, 25.34) {\small{$|y-\hat{y}|$}};
                    }
                    \only<13>{
                        \draw[] (axis cs: 115, 21.78) --
                                                    (axis cs: 115, 28.9) --
                                                    (axis cs: 134, 28.9) --
                                                    (axis cs: 134, 21.78) -- cycle;
                        \node[anchor=west] at (axis cs: 134, 25.34) {\small{$(y-\hat{y})^2$}};
                    }
                    \only<14-16>{
                        \draw[] (axis cs: 115, 21.78) --
                                                    (axis cs: 115, 28.9) --
                                                    (axis cs: 134, 28.9) --
                                                    (axis cs: 134, 21.78) -- cycle;
                        \draw[] (axis cs: 60, 27) --
                                                    (axis cs: 60, 30.462) --
                                                    (axis cs: 69, 30.462) --
                                                    (axis cs: 69, 27) -- cycle;
                        \draw[] (axis cs: 75, 26) --
                                                    (axis cs: 75, 28.095) --
                                                    (axis cs: 81, 28.095) --
                                                    (axis cs: 81, 26) -- cycle;
                        \draw[] (axis cs: 165.3, 17.7) --
                                                    (axis cs: 165.3, 13.85) --
                                                    (axis cs: 176, 13.85) --
                                                    (axis cs: 176, 17.7) -- cycle;
                        \draw[] (axis cs: 198, 15) --
                                                    (axis cs: 198, 8.68) --
                                                    (axis cs: 215, 8.68) --
                                                    (axis cs: 215, 15) -- cycle;

                        \addplot[
                            only marks,
                            opacity=0.8
                        ] coordinates {
                            (60, 27)
                            (75, 26)
                            (165.3, 17.7)
                            (198, 15)
                        };
                        \addplot[
                            only marks,
                            red,
                            opacity=0.6
                        ] coordinates {
                            (60, 30.462)
                            (75, 28.095)
                            (165.3, 13.85)
                            (198, 8.68)
                        };
                    }
                    \only<4>{
                        \addplot[
                            domain=30:240,
                            samples=100,
                            color=blue,
                            thick
                        ] {36 - 0.1*x};
                    }

                \end{axis}
            \end{tikzpicture}\\
            \centering
            \only<2>{
                \hspace{0.8cm}\textcolor{red}{$\hat{y}=\beta_0-\beta_1x$}\\
            }
            \only<3-7>{
                \hspace{0.75cm}\textcolor{red}{$\hat{y}=39.93-0.1578x$}\\
            }
            \only<4>{
                \hspace{1.4cm}\textcolor{blue}{$\hat{y}=36-0.1x$}
            }
            \only<8>{
                \hspace{0.8cm}\textcolor{red}{$\hat{y}=39.93-0.1578*115$}\\
            }
            \only<9-10>{
                \hspace{0.8cm}\textcolor{red}{$21.78=39.93-0.1578*115$}\\
            }
            \only<15>{
                \hspace{0.8cm}\textcolor{red}{$MSE=\frac{1}{n}\sum_{i=1}^{n}(y_i-\hat{y}_i)^2$}\\
            }
            \only<16>{
                \hspace{0.8cm}\textcolor{red}{$RSS=\sum_{i=1}^{n}(y_i-\hat{y}_i)^2$}\\
            }
        }
        \only<17-47>{
            \begin{tikzpicture}
                \node[] at (-5.25, -3) {};
                \node[] at (5.25, 3) {};
                \only<17,21-22,24>{
                    \node[font=\LARGE] at (0, 0) {
                        $\hat{y}=$
                        $\beta_0$
                        $+$
                        $\beta_1$
                        $x$
                    };
                }
                \only<18>{
                    \node[font=\LARGE] at (0, 0) {
                        $\hat{y}=$
                        \textcolor{red}{$\beta_0$}
                        $+$
                        \textcolor{red}{$\beta_1$}
                        $x$
                    };
                }
                \only<19-20>{
                    \node[] at (-2.7, 0) {
                        \usebox{\linreg}
                    };
                }
                \only<20>{
                    \node[] at (2.7, 0) {
                        \usebox{\residuals}
                    };
                }
                \only<22>{
                    \node[font=\LARGE] at (0, -1) {
                        $MSE=\frac{1}{n}\sum_{i=1}^{n}(y_i-\hat{y}_i)^2$
                    };
                }
                \only<23>{
                    \node[font=\LARGE] at (0, -0.5) {
                        $MSE=\frac{1}{n}\sum_{i=1}^{n}(y_i-\beta_0+\beta_1x_i)^2$
                    };
                }
                \only<25>{
                    \node[font=\LARGE] at (0, 0) {
                        $\hat{y}=\beta_0+\beta_1x_1+\beta_2x_2+\ldots+\beta_px_p$
                    };
                }
                \only<26>{
                    \node[] at (0, 0) {
                        \usebox{\multilinreg}
                    };
                }
                \only<27>{
                    \node[] at (0, 0) {
                        \begin{tabular}{cc}
                            \textbf{mpg}&\textbf{manufacturer}\\
                            36&Chevrolet\\
                            15&Ford\\
                            25&Chevrolet\\
                            26&Chevrolet\\
                            17&Ford\\
                            15&Ford\\
                            32&Chevrolet\\
                            14&Ford\\
                            14&Ford\\
                            28&Chevrolet\\
                        \end{tabular}
                    };
                    \node[] at (0, -3.5) {
                        $\widehat{\text{mpg}}=\beta_0+\beta_1\times\text{manufacturer}$
                    };
                }
                \only<28>{
                    \node[] at (0, 0) {
                        \begin{tabular}{ccc}
                            \textbf{mpg}&\textbf{manufacturer}&\textbf{chevrolet}\\
                            36&Chevrolet&1\\
                            15&Ford&0\\
                            25&Chevrolet&1\\
                            26&Chevrolet&1\\
                            17&Ford&0\\
                            15&Ford&0\\
                            32&Chevrolet&1\\
                            14&Ford&0\\
                            14&Ford&0\\
                            28&Chevrolet&1\\
                        \end{tabular}
                    };
                    \only<28>{
                        \node[] at (0, -3.5) {
                            $\widehat{\text{mpg}}=\beta_0+\beta_1\times\text{chevrolet}$
                        };
                    }
                }
                \only<29>{
                    \node[] at (0, 0) {
                        \usebox{\binary}
                    };
                }
                \only<30-31>{
                    \node[] at (0, 0) {
                        \usebox{\binarymeans}
                    };
                }
                \only<31>{
                    \node[] at (0.57, -3.5) {
                        \textbf{Blackboard!}
                    };
                }
                \only<32>{
                    \node[] at (0, 0) {
                        \begin{tabular}{cc}
                            \textbf{mpg}&\textbf{manufacturer}\\
                            36&Chevrolet\\
                            15&Ford\\
                            25&Chevrolet\\
                            26&Pontiac\\
                            17&Ford\\
                            15&Ford\\
                            32&Pontiac\\
                            14&Ford\\
                            14&Pontiac\\
                            28&Chevrolet\\
                        \end{tabular}
                    };
                    \node[] at (0, -3.5) {
                        $\widehat{\text{mpg}}=\beta_0+\beta_1\times\text{manufacturer}$
                    };
                }
                \only<33>{
                    \node[] at (0, 0) {
                        \begin{tabular}{cccc}
                            \textbf{mpg}&\textbf{manufacturer}&\textbf{chevrolet}&\textbf{pontiac}\\
                            36&Chevrolet&1&0\\
                            15&Ford&0&0\\
                            25&Chevrolet&1&0\\
                            26&Pontiac&0&1\\
                            17&Ford&0&0\\
                            15&Ford&0&0\\
                            32&Pontiac&0&1\\
                            14&Ford&0&0\\
                            14&Pontiac&0&1\\
                            28&Chevrolet&1&0\\
                        \end{tabular}
                    };
                    \node[] at (0, -3.5) {
                        $\widehat{\text{mpg}}=\beta_0+\beta_1\times\text{chevrolet}+\beta_2\times\text{pontiac}$
                    };
                }
                \only<34>{
                    \PythonInputNode{1}{(-4, 2)}{pythonnode}{0.9\textwidth}{7}{
                        import pandas as pd^^J
                        ^^J
                        df = pd.DataFrame(...)^^J
                        print(f'Columns before: \{df.columns.values\}')^^J
                        df = pd.get_dummies(df)^^J
                        print(f'Columns after: \{df.columns.values\})^^J
                    }
                    \PythonOutputNode{1}{(-3.895, 0)}{out}{0.79\textwidth}{7}{
                        Columns before: ['manufacturer']^^J
                        Columns after: ['manufacturer_chevrolet' 'manufacturer_ford']^^J
                    }
                }
                \only<35>{
                    \node[] at (0, 0) {
                        \usebox{\multilevel}
                    };
                }
                \only<36-37>{
                    \node[] at (0, 0) {
                        \begin{tabular}{ccc}
                            \textbf{mpg}&\textbf{chevrolet}&\textbf{horsepower}\\
                            36&1&130\\
                            15&0&165\\
                            25&1&150\\
                            26&1&150\\
                            17&0&140\\
                            15&0&198\\
                            32&1&220\\
                            14&0&215\\
                            14&0&225\\
                            28&1&212\\
                        \end{tabular}
                    };
                    \only<36>{
                        \node[] at (0, -3.5) {
                            $\widehat{\text{mpg}}=\beta_0+\beta_1\times\text{chevrolet}+\beta_2\times\text{horsepower}$
                        };
                    }
                }
                \only<37>{
                    \node[align=center] at (0, -3.8) {
                        $\widehat{mpg}=
                        \begin{cases}
                            \beta_0+\beta_1+\beta_2\times\text{horsepower} & \text{if chevrolet}\\
                            \beta_0+\beta_2\times\text{horsepower} & \text{else}\\
                        \end{cases}$
                    };

                }
                \only<38>{
                    \node[] at (0, 0) {
                        \usebox{\combinationbox}
                    };
                    \node[align=center] at (0, -3.8) {
                        $\widehat{mpg}=
                        \begin{cases}
                            \textcolor{blue}{\beta_0+\beta_1+\beta_2\times\text{horsepower}} & \text{if chevrolet}\\
                            \textcolor{red}{\beta_0+\beta_2\times\text{horsepower}} & \text{else}\\
                        \end{cases}$
                    };
                }
                \only<39-41>{
                    \node[] at (0, 0) {
                        \usebox{\interactionbox}
                    };
                }
                \only<40>{
                    \node[anchor=north] at (0, -3.4) {
                        $\widehat{mpg}=\beta_0+\beta_1\times\text{chevrolet}+\beta_2\times\text{horsepower}$
                    };
                }
                \only<41>{
                    \node[align=center, anchor=north] at (0, -3.4) {
                        $\widehat{mpg}=\beta_0+\beta_1\times\text{chevrolet}+\beta_2\times\text{horsepower}$\\
                        $+\beta_3\times\text{chevrolet}\times\text{horsepower}$
                    };
                }
                \only<42>{
                    \node[] at (0, 0) {
                        \usebox{\cisdata}
                    };
                }
                \only<43>{
                    \node[] at (0, 0) {
                        \usebox{\cisprediction}
                    };
                }
                \only<44>{
                    \node[] at (0, 0) {
                        \usebox{\cisconfidence}
                    };
                }
                \only<45>{
                    \RInputNode{(-4.5, 2)}{in1}{0.8\textwidth}{7}{
predict(fit, newdata=data.frame(horsepower=105),^^J
{ }{ }{ }{ }{ }{ }{ }{ }interval='prediction', level=0.95)^^J
                    }
                    \RInputNode{(-4.5, 1)}{out1}{0.8\textwidth}{7}{
{ }{ }{ }{ }{ }fit{ }{ }{ }{ }lwr{ }{ }{ }{ }upr^^J
1 23.535 17.158 29.912^^J
                    }
                    \RInputNode{(-4.5, -1)}{in2}{0.8\textwidth}{7}{
predict(fit, newdata=data.frame(horsepower=105),^^J
{ }{ }{ }{ }{ }{ }{ }{ }interval='confidence', level=0.95)^^J
                    }
                    \RInputNode{(-4.5, -2)}{out1}{0.8\textwidth}{7}{
                        { }{ }{ }{ }{ }fit{ }{ }{ }{ }lwr{ }{ }{ }{ }upr^^J
                        1 23.535 23.023 24.047^^J
                    }
                }
                \only<46>{
                    \PythonInputNode{1}{(-4.5, 2)}{pythonnode}{0.95\textwidth}{7}{
                        import statsmodels.api as sm^^J
                        ^^J
                        model = sm.OLS(df['mpg'], sm.add_constant(df[['horsepower']]))^^J
                        fit = model.fit()^^J
                        new_input = sm.add_constant(pd.DataFrame(\{'horsepower': [105, 106]\}))^^J
                        intervals = fit.get_prediction(new_input).summary_frame()^^J
                        print(intervals)^^J
                    }
                    \PythonOutputNode{1}{(-4.395, -0.5)}{out}{0.882\textwidth}{7}{
                        mean   mean_se  mean_ci_lower  mean_ci_upper  obs_ci_lower  obs_ci_upper^^J
                        0  24.467077  0.251262      23.973079      24.961075     14.809396     34.124758^^J
                        1  31.096556  0.398740      30.312607      31.880505     21.419710     40.773402^^J
                    }
                }

                \only<47>{
                    \node[] at (0, 1) {
                        \Huge{\emoji{partying-face}}
                    };

                    \node[] at (0, -1) {
                        \scriptsize{\url{http://localhost:8888/notebooks/notebooks\%2FLinear\%20regression.ipynb}}
                    };
                }
            \end{tikzpicture}
        }
        \only<48>{
            Linear regression: The true workhorse of machine learning
            \begin{itemize}
                \item Models the relationship between (either singular or mulitple) inputs $X$ and (a continuous) output $y$ as a linear function
                \begin{itemize}
                    \item Inputs can be both continuous and categorical
                \end{itemize}
                \item A strict parametric form limits the expressivity of the model
                \begin{itemize}
                    \item More advanced terms can be explicitly added
                    \item The strictness allows for further interpretation, such as confidence intervals
                    \item \textbf{Makes the model human interpretable}
                \end{itemize}
            \end{itemize}
        }
    \end{frame}

    \newcommand{\knnplot}[1]{
        \begin{tikzpicture}
            \begin{axis}[
                width=10cm,
                height=6cm,
                xmin=-0.2,
                xmax=1.2,
                ymin=-0.2,
                ymax=1.2,
                xmajorticks=false,
                ymajorticks=false,
                ylabel=y,
                xlabel=x,
                clip=false
            ]
                \ifnum#1=1
                    \addplot[
                        only marks,
                        opacity=0.5,
                        mark size=4pt
                    ] coordinates {
                        (0.000, -0.075)
                        (0.105, 0.075)
                        (0.211, 0.156)
                        (0.316, 0.295)
                        (0.421, 0.645)
                        (0.526, 0.685)
                        (0.684, 0.499)
                        (0.789, 0.602)
                        (0.895, 0.899)
                        (1.000, 1.140)
                    };
                \fi
                \ifnum#1>1
                    \ifnum#1<4
                        \addplot[
                            only marks,
                            opacity=0.1,
                            mark size=4pt
                        ] coordinates {
                            (0.000, -0.075)
                            (0.105, 0.075)
                            (0.211, 0.156)
                            (0.316, 0.295)
                            (0.421, 0.645)
                            (0.526, 0.685)
                            (0.684, 0.499)
                            (0.789, 0.602)
                            (0.895, 0.899)
                            (1.000, 1.140)
                        };
                        \draw[dashed] (axis cs: 0.6, -0.2) -- (axis cs: 0.6, 1.2);
                        \node[anchor=south] at (axis cs: 0.6, 1.2) {\small{?}};

                        \ifnum#1=3
                            \addplot[
                                only marks,
                                red,
                                mark size=4pt,
                                opacity=0.6
                            ] coordinates {
                                (0.6, 0.685)
                            };
                            \draw[-stealth,red] (axis cs: 0.60, 0.685) -- (axis cs: 0.55, 0.685);
                        \fi
                    \fi
                \fi

                \ifnum#1=4
                    \addplot[
                        only marks,
                        opacity=0.5,
                        mark size=4pt
                    ] coordinates {
                        (0.000, -0.075)
                        (0.105, 0.075)
                        (0.211, 0.156)
                        (0.316, 0.295)
                        (0.421, 0.645)
                        (0.526, 0.685)
                        (0.684, 0.499)
                        (0.789, 0.602)
                        (0.895, 0.899)
                        (1.000, 1.140)
                    };
                    \addplot[dashed] coordinates {
                        (-0.2, -0.075)
                        (0.0525, -0.075)
                        (0.0525, 0.075)
                        (0.1575, 0.075)
                        (0.1575, 0.156)
                        (0.263, 0.156)
                        (0.263, 0.295)
                        (0.368, 0.295)
                        (0.368, 0.645)
                        (0.473, 0.645)
                        (0.473, 0.685)
                        (0.578, 0.685)
                        (0.578, 0.499)
                        (0.7375, 0.499)
                        (0.7375, 0.602)
                        (0.842, 0.602)
                        (0.842, 0.899)
                        (0.9475, 0.899)
                        (0.9475, 1.140)
                        (1.2, 1.140)
                    };
                \fi

                \ifnum#1=5
                    \addplot[
                        only marks,
                        opacity=0.1,
                        mark size=4pt
                    ] coordinates {
                        (0.000, -0.075)
                        (0.105, 0.075)
                        (0.211, 0.156)
                        (0.316, 0.295)
                        (0.421, 0.645)
                        (0.526, 0.685)
                        (0.684, 0.499)
                        (0.789, 0.602)
                        (0.895, 0.899)
                        (1.000, 1.140)
                    };
                    \draw[dashed] (axis cs: 0.6, -0.2) -- (axis cs: 0.6, 1.2);
                    \node[anchor=south] at (axis cs: 0.6, 1.2) {\small{?}};
                    \addplot[
                        only marks,
                        red,
                        mark size=4pt,
                        opacity=0.6
                    ] coordinates {
                        (0.6, 0.592)
                    };
                    \draw[-stealth,red] (axis cs: 0.60, 0.592) -- (axis cs: 0.55, 0.66);
                    \draw[-stealth,red] (axis cs: 0.60, 0.592) -- (axis cs: 0.65, 0.52);
                \fi
                \ifnum#1=6
                    \addplot[
                        only marks,
                        opacity=0.5,
                        mark size=4pt
                    ] coordinates {
                        (0.000, -0.075)
                        (0.105, 0.075)
                        (0.211, 0.156)
                        (0.316, 0.295)
                        (0.421, 0.645)
                        (0.526, 0.685)
                        (0.684, 0.499)
                        (0.789, 0.602)
                        (0.895, 0.899)
                        (1.000, 1.140)
                    };
                    \addplot[dashed] coordinates {
                        (-0.2, 0)
                        (0.105, 0)
                        (0.105, 0.115)
                        (0.2105, 0.115)
                        (0.2105, 0.225)
                        (0.316, 0.225)
                        (0.316, 0.47)
                        (0.421, 0.47)
                        (0.421, 0.665)
                        (0.5525, 0.665)
                        (0.5525, 0.592)
                        (0.6575, 0.592)
                        (0.6575, 0.55)
                        (0.79, 0.55)
                        (0.79, 0.7505)
                        (0.894, 0.7505)
                        (0.894, 1.01)
                        (1.2, 1.01)


                    };
                \fi
                \ifnum#1=7
                \addplot[
                    only marks,
                    opacity=0.5,
                    mark size=4pt
                ] coordinates {
                    (0.000, -0.075)
                    (0.105, 0.075)
                    (0.211, 0.156)
                    (0.316, 0.295)
                    (0.421, 0.645)
                    (0.526, 0.685)
                    (0.684, 0.499)
                    (0.789, 0.602)
                    (0.895, 0.899)
                    (1.000, 1.140)
                };
                \addplot[dashed] coordinates {
                    (-0.2, 0.4921)
                    (1.2, 0.4921)
                };
            \fi

            \addplot[] coordinates {
                (-0.2, -0.2)
                (1.2, 1.2)
            };

            \end{axis}
        \end{tikzpicture}
    }

    \newsavebox{\knndata}
    \sbox{\knndata}{
        \knnplot{1}
    }
    \newsavebox{\knninput}
    \sbox{\knninput}{
        \knnplot{2}
    }
    \newsavebox{\knnpredone}
    \sbox{\knnpredone}{
        \knnplot{3}
    }
    \newsavebox{\knnformone}
    \sbox{\knnformone}{
        \knnplot{4}
    }
    \newsavebox{\knnpredtwo}
    \sbox{\knnpredtwo}{
        \knnplot{5}
    }
    \newsavebox{\knnformtwo}
    \sbox{\knnformtwo}{
        \knnplot{6}
    }
    \newsavebox{\knnall}
    \sbox{\knnall}{
        \knnplot{7}
    }

    \newcommand{\dimplot}[1]{
        \begin{tikzpicture}[scale=2]
            \node[] at (-1.5, -1.5) {};
            \node[] at (1.5, 1.5) {};

            \draw[-stealth,gray!50] (-1, -1) -- (1, -1);
            \node[anchor=north,,gray!50] at (0, -1) {$x_1$};
            \node[
                inner sep=2pt,
                circle,
                draw=black,
                fill=teal!60
            ] (n1) at (-0.8, -0.8) {};

            \ifnum#1=1
                \node[
                    inner sep=2pt,
                    circle,
                    draw=black,
                    fill=teal!60
                ] (n2) at (0.2, -0.8) {};
                \draw[dashed,stealth-stealth] (-0.75, -0.8) -- (0.15, -0.8) node [midway, above] {\small{1}};
            \fi

            \ifnum#1>1
                \draw[-stealth,gray!50] (-1, -1) -- (-1, 1);
                \node[anchor=east,gray!50] at (-1, 0) {$x_2$};
            \fi

            \ifnum#1=2
                \node[
                    inner sep=2pt,
                    circle,
                    draw=black,
                    fill=teal!60
                ] (n2) at (0.2, 0.2) {};
                \draw[dashed,stealth-] (-0.75, -0.8) -- (0.2, -0.8) node [midway, above] {\small{1}};
                \draw[dashed,-stealth] (0.2, -0.8) -- (0.2, 0.15) node [midway, right] {\small{1}};
                \draw[-stealth,red] (n1) -- (n2) node [pos=0.45, above=0.15cm] {\small{$\sqrt{2}$}};
            \fi

            \ifnum#1=3
                \draw[-stealth,gray!50] (-1, -1) -- (0.5, 0.0);
                \node[anchor=north, rotate=30,gray!50] at (-0.25, -0.5) {$x_3$};
                \node[
                    inner sep=2pt,
                    circle,
                    draw=black,
                    fill=teal!60
                ] (n2) at (0.7, 0.53) {};
                \draw[dashed,stealth-] (-0.75, -0.8) -- (0.2, -0.8) node [midway, above] {\small{1}};
                \draw[dashed,-stealth] (0.2, -0.8) -- (0.2, 0.2) node [midway, right] {\small{1}};
                \draw[dashed,-stealth] (0.2, 0.2) -- (0.7, 0.53) node [pos=0.4, below=0.15cm] {\small{1}};
                \draw[-stealth,red] (n1) -- (n2) node [pos=0.45, above=0.15cm] {\small{$\sqrt{3}$}};
            \fi

        \end{tikzpicture}
    }

    \newsavebox{\dimone}
    \sbox{\dimone}{
        \dimplot{1}
    }
    \newsavebox{\dimtwo}
    \sbox{\dimtwo}{
        \dimplot{2}
    }
    \newsavebox{\dimthree}
    \sbox{\dimthree}{
        \dimplot{3}
    }

    \section{K-Nearest Neighbours}

    \begin{frame}{K-Nearest Neighbours}
        \only<1-13>{
            \begin{tikzpicture}
                \node[] at (-5.25, -3) {};
                \node[] at (5.25, 3) {};

                \only<1-2>{
                    \node[anchor=west, align=left] at (-5, 1.5) {
                        \underline{Linear regression:}\\[0.25cm]
                        $\hat{f}(X)=\beta_0+\beta_1X_1+\beta_2X_2+\ldots+\beta_pX_p$
                    };

                    \node[anchor=west, align=left] at (-5, -1.5) {
                        \underline{K-Nearest Neighbours:}\\[0.25cm]
                        $\hat{f}(X)=\frac{1}{K}\sum\limits_{x_i \in \mathcal{N}}y_i$
                    };
                }
                \only<2>{
                    \node[rotate=45] at (0, 0) {
                        \Huge{Blackboard!}
                    };
                }
                \only<3>{
                    \node[] at (0, 0) {
                        \usebox{\knndata}
                    };
                }
                \only<4>{
                    \node[] at (0, 0) {
                        \usebox{\knninput}
                    };
                }
                \only<5>{
                    \node[] at (0, 0) {
                        \usebox{\knnpredone}
                    };
                }
                \only<6>{
                    \node[] at (0, 0) {
                        \usebox{\knnformone}
                    };
                }
                \only<7>{
                    \node[] at (0, 0) {
                        \usebox{\knnpredtwo}
                    };
                }
                \only<8>{
                    \node[] at (0, 0) {
                        \usebox{\knnformtwo}
                    };
                }
                \only<9>{
                    \node[] at (0, 0) {
                        \usebox{\knnall}
                    };
                }
                \only<10>{
                    \node[align=left] at (0, 0) {
                        \underline{Blackboard exercise}:\\
                        How does the bias-variance trade-off relate to the choice of $K$?
                    };
                };
                \only<11>{
                    \node[] at (0, 0) {
                        \usebox{\dimone}
                    };
                }
                \only<12>{
                    \node[] at (0, 0) {
                        \usebox{\dimtwo}
                    };
                }
                \only<13>{
                    \node[] at (0, 0) {
                        \usebox{\dimthree}
                    };
                }
            \end{tikzpicture}
        }
        \only<14>{
            K-Nearest Neighbours: An intuitive model relying on similar datapoints to make a prediction
            \begin{itemize}
                \item No assumptions about the functional relationship between inputs and outputs
                \item Directly trades off bias and variance through the choice of $K$
                \item Does not work well in high dimensions as the neighbourhoods get sparse
            \end{itemize}
        }
    \end{frame}

    \newcommand{\logisticplot}[1]{
        \begin{tikzpicture}
            \begin{axis}[
                height=5cm,
                width=10cm,
                ytick={0, 1},
                ytick style={draw=none},
                xtick pos=bottom,
                xlabel=mpg (x),
                ylabel=chevrolet (y),
                ymin=-0.5,
                ymax=1.5,
                xmin=13,
                xmax=38
            ]
                \addplot[
                    only marks,
                    fill=red,
                    mark size=3pt
                ] coordinates {
                    (36, 1)
                    (25, 1)
                    (26, 1)
                    (32, 1)
                    (28, 1)
                    (29, 1)
                    (29.5, 1)
                    (21, 1)
                };
                \addplot[
                    only marks,
                    fill=blue,
                    mark size=3pt
                ] coordinates {
                    (15, 0)
                    (17, 0)
                    (21, 0)
                    (14, 0)
                    (14.5, 0)
                    (16, 0)
                    (19, 0)
                    (24, 0)
                    (27, 0)
                };

                \ifnum#1=2
                    \addplot[
                        domain=10:40,
                        samples=100,
                        color=red,
                        thick
                    ] {-0.87+0.06*x};
                \fi
                \ifnum#1=3
                    \draw[dashed] (axis cs: 25.5, -0.5) -- (axis cs: 25.5, 1.5);
                    \draw[] (axis cs: 25.5, 1) -- (axis cs: 38, 1);
                    \draw[] (axis cs: 25.5, 0) -- (axis cs: 13, 0);
                \fi
                \ifnum#1=4
                    \addplot[
                        domain=13:38,
                        samples=100,
                        color=red,
                        thick
                    ] {exp(-10+0.42*x)/(1+exp(-10+0.42*x))};

                \fi
            \end{axis}
        \end{tikzpicture}
    }

    \newsavebox{\logisticdata}
    \sbox{\logisticdata}{
        \logisticplot{1}
    }
    \newsavebox{\logisticlinear}
    \sbox{\logisticlinear}{
        \logisticplot{2}
    }
    \newsavebox{\logisticstep}
    \sbox{\logisticstep}{
        \logisticplot{3}
    }
    \newsavebox{\logisticlog}
    \sbox{\logisticlog}{
        \logisticplot{4}
    }

    \section{Logistic regression}

    \begin{frame}{Logistic regression}
        \only<1-20>{
            \begin{tikzpicture}
                \node[draw=black] at (-5.25, -4) {};
                \node[draw=black] at (5.25, 3) {};

                \only<1-2>{
                    \node[] at (0, 0) {
                        \begin{tabular}{ccc}
                            \textbf{mpg}&\textbf{manufacturer}&\textbf{chevrolet}\\
                            36&Chevrolet&1\\
                            15&Ford&0\\
                            25&Chevrolet&1\\
                            26&Chevrolet&1\\
                            17&Ford&0\\
                            15&Ford&0\\
                            32&Chevrolet&1\\
                            14&Ford&0\\
                            14&Ford&0\\
                            28&Chevrolet&1\\
                        \end{tabular}
                    };
                }
                \only<1>{
                    \node[] at (0, -3.5) {
                        $\widehat{\text{mpg}}=\beta_0+\beta_1\times\text{chevrolet}$
                    };
                }
                \only<2>{
                    \node[] at (0, -3.5) {
                        $\widehat{\text{chevrolet}}=\beta_0+\beta_1\times\text{mpg}$
                    };
                }
                \only<3>{
                    \node[] at (0, 0) {
                        \usebox{\logisticdata}
                    };
                }
                \only<4>{
                    \node[] at (0, 0) {
                        \usebox{\logisticlinear}
                    };
                    \node[] at (0, -3) {
                        $\widehat{\text{chevrolet}}=-0.87+0.06\times\text{mpg}$
                    };
                }
                \only<5>{
                    \node[] at (0, 0) {
                        \usebox{\logisticstep}
                    };
                }
                \only<6>{
                    \node[] at (0, 0) {
                        \usebox{\logisticlog}
                    };
                    \node[] at (0, -3) {
                        $\widehat{\text{chevrolet}}=\dfrac{e^{-10.22+0.42\times\text{mpg}}}{1 + e^{-10.22+0.42\times\text{mpg}}}$
                    };
                }
                \only<7-20>{
                    \node[] at (-0.9, 1) {
                        $\hat{y}=$
                    };
                    \draw[] (-0.5, 1) -- (1.1, 1);
                    \node[] at (-0.3, 0.75) {
                        $1+$
                    };
                    \node[] at (0.45, 0.8) {
                        $e^{\beta_0+\beta_1x}$
                    };
                    \node[] at (0.3, 1.2) {
                        $e^{\beta_0+\beta_1x}$
                    };
                }
                \only<8-13>{
                    \node[anchor=east] at (0, -1) {
                        $e^{\beta_0+\beta_1x} \rightarrow 0$
                    };
                }
                \only<9-13>{
                    \node[anchor=west] at (0, -1) {
                        $\implies \hat{y} = 0$
                    };
                }
                \only<10-13>{
                    \node[anchor=east] at (0, -1.5) {
                        $e^{\beta_0+\beta_1x} \rightarrow \infty$
                    };
                }
                \only<11-13>{
                    \node[anchor=west] at (0, -1.5) {
                        $\implies \hat{y} = 1$
                    };
                }
                \only<12-13>{
                    \node[anchor=east] at (0, -2.5) {
                        $0\leq\hat{y}\leq1$
                    };
                }
                \only<13>{
                    \node[anchor=west] at (0, -2.5) {
                        $\implies\hat{y}=Pr(Y=1|X)$
                    };
                }
                \only<15-16>{
                    \node[] at (0, -1) {
                        $log \left( \dfrac{p(X)}{1-p(X)} \right) = \beta_0+\beta_1x$
                    };
                }
                \only<16>{
                    \node[] at (0, -3) {
                        "\textit{... counterintuitive and challenging to interpret.}" - James Jaccard
                    };
                }
                \only<18-20>{
                    \node[] at (0, 0) {
                        More than one class?
                    };
                }
                \only<19-20>{
                    \node[] at (0, -2) {
                        $Pr(Y=k|X=x)=\dfrac{e^{\beta_{0k}+\beta_{1k}x}}{\sum\limits_{l=1}^{K}e^{\beta_{0l}+\beta_{1l}x}}$
                    };
                }
                \only<20>{
                    \node[rotate=45] at (0, 0) {
                        \Huge{Blackboard!}
                    };
                }
            \end{tikzpicture}
        }
        \only<21>{
            Logistic regression: Extending linear regression to classification.
            \begin{itemize}
                \item Treats all members of a class (approximately) equally.
                \item Ouputs an understandable quantity: The probability of a sample belonging to the positive class.
                \item Somewhat interpretable, although not as much as linear regression
                \item Can trivially be extended to include multiple classes
            \end{itemize}
        }
        \only<22>{
            \scriptsize{\url{http://localhost:8888/notebooks/notebooks\%2FLogistic\%20regression.ipynb}}
        }
    \end{frame}

    \section{Generative models for classification}

    \begin{frame}[t]{Generative models}
        \only<1-2>{
            Generative models for classification:
        }
        \only<1>{
            \begin{itemize}
                \item Logistic regression models the probability of a class given predictors: $P(Y|X)$
                \item Generative models instead consider the probability of the predictors values given the class, $P(X|Y)$
                \item How would a given class generate the datapoint?
            \end{itemize}
        }
        \only<2>{
            \begin{itemize}
                \item Bayes' theorem: $P(Y|X)=\dfrac{P(X|Y)P(Y)}{P(X)}$
                \item $P(Y|X)$ is the posterior probability of the class given the data
                \item $P(X|Y)$ is the likelihood of the data given the class
                \item $P(Y)$ is the prior probability of the class
                \item $P(X)$ is the marginal likelihood of the data
            \end{itemize}
        }
        \only<3>{
            Linear Discriminant Analysis
            \begin{itemize}
                \item Assume the predictors are normally distributed within each class separately
                \begin{itemize}
                    \item $P(X|Y=Class 1)$
                    \item $P(X|Y=Class 2)$
                    \item etc
                \end{itemize}
                \item Class probabilities $P(Y)$ estimated from training data
                \item Marginal probability of data $P(X)$ can be simplified away
                \item Posterior probability $P(Y|X)$ can be calculated using Bayes' theorem
            \end{itemize}
        }
    \end{frame}

    \section{Classification metrics}

    \begin{frame}{Classification metrics}
        The two most common, severe, mistakes made in machine learning studies in psychology and neuroscience (in my opinion) are:
        \begin{itemize}
            \item Poor validation and testing strategies (Next lecture)
            \item Using incorrect performance measures, most commonly accuracy (Blackboard!)
        \end{itemize}
    \end{frame}

    \begin{frame}{Assignment 2}
        \scriptsize{\url{https://uio.instructure.com/courses/53357/assignments/116394?module_item_id=946345}}
    \end{frame}

\end{document}
