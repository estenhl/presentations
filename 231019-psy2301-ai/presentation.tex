\documentclass[8pt]{beamer}

\usetheme{metropolis}

\usepackage{emoji}
\usepackage{pgfplots}
\usepackage{pgfplotstable}
\usepackage{tikz}
\usepackage{transparent}

\usepgfplotslibrary{fillbetween}
\usepgfplotslibrary{groupplots}

\usetikzlibrary{arrows.meta}
\usetikzlibrary{calc}
\usetikzlibrary{matrix}

\date{19.10.23}
\title{PSY2301: The psychology of judgement and decision making}
\subtitle{Artificial Intelligence and decision-making}
\author{Esten H. Leonardsen}

\def\logoheight{1.2cm}
\def\logosep{0.5cm}

\definecolor{headerbackground}{HTML}{23373b}
\definecolor{headerforeground}{HTML}{F4F4F4}
\setbeamercolor{footer}{bg=headerbackground, fg=headerforeground}

\defbeamertemplate*{footline}{mycin}{%
  \leavevmode%
  \hbox{%
  \begin{beamercolorbox}[wd=\paperwidth,ht=2.25ex,dp=1ex,center]{footer}%
	\href{https://www.sciencedirect.com/science/article/pii/S0020737378800492}{MYCIN: a knowledge-based consultation program for infectious disease diagnosis, William van Melle, \textit{International Journal of Man-Machine Studies}, 1978}
  \end{beamercolorbox}%
  }%
  \vskip0pt%
}

\defbeamertemplate*{footline}{backprop}{%
  \leavevmode%
  \hbox{%
  \begin{beamercolorbox}[wd=\paperwidth,ht=2.25ex,dp=1ex,center]{footer}%
	\href{https://www.nature.com/articles/323533a0}{Learning representations by back-propagating errors, Rumelhart, D., Hinton, G. \& Williams, R, \textit{Nature}, 1986}
  \end{beamercolorbox}%
  }%
  \vskip0pt%
}

\defbeamertemplate*{footline}{eliza}{%
  \leavevmode%
  \hbox{%
  \begin{beamercolorbox}[wd=\paperwidth,ht=2.25ex,dp=1ex,center]{footer}%
	\href{https://web.njit.edu/~ronkowit/eliza.html}{ELIZA: a very basic Rogerian psychotherapist chatbot}
  \end{beamercolorbox}%
  }%
  \vskip0pt%
}

\defbeamertemplate*{footline}{neuron}{%
  \leavevmode%
  \hbox{%
  \begin{beamercolorbox}[wd=\paperwidth,ht=2.25ex,dp=1ex,center]{footer}%
	\href{https://opentextbc.ca/introductiontopsychology/chapter/3-1-the-neuron-is-the-building-block-of-the-nervous-system/}{The neuron is the building block of the nervous system}
  \end{beamercolorbox}%
  }%
  \vskip0pt%
}

\defbeamertemplate*{footline}{turing}{%
  \leavevmode%
  \hbox{%
  \begin{beamercolorbox}[wd=\paperwidth,ht=2.25ex,dp=1ex,center]{footer}%
	\href{https://link.springer.com/chapter/10.1007/978-1-4020-6710-5_3}{Computing Machinery and Intelligence, A. M. Turing, \textit{Mind}, 1950.}
  \end{beamercolorbox}%
  }%
  \vskip0pt%
}

\defbeamertemplate*{footline}{compas}{%
  \leavevmode%
  \hbox{%
  \begin{beamercolorbox}[wd=\paperwidth,ht=2.25ex,dp=1ex,center]{footer}%
	\href{https://www.science.org/doi/10.1126/sciadv.aao5580}{The accuracy, fairness, and limits of predicting recidivism, Julia Dressel \& Hany Farid, \textit{Science Advances}, 2018.}
  \end{beamercolorbox}%
  }%
  \vskip0pt%
}

\defbeamertemplate*{footline}{humanbias}{%
  \leavevmode%
  \hbox{%
  \begin{beamercolorbox}[wd=\paperwidth,ht=2.25ex,dp=1ex,center]{footer}%
	\href{https://www.nber.org/papers/w9873}{Are Emily and Greg More Employable than Lakisha and Jamal? ..., Marianne Bertrand \& Sendhil Mullainathan, \textit{American economic review}, 2004.}
  \end{beamercolorbox}%
  }%
  \vskip0pt%
}

\defbeamertemplate*{footline}{theoryofmind}{%
  \leavevmode%
  \hbox{%
  \begin{beamercolorbox}[wd=\paperwidth,ht=2.25ex,dp=1ex,center]{footer}%
	\href{https://arxiv.org/abs/2302.02083}{Theory of Mind Might Have Spontaneously Emerged in Large Language Models, Michal Kosinski, \textit{arXiv}, 2023.}
  \end{beamercolorbox}%
  }%
  \vskip0pt%
}

\defbeamertemplate*{footline}{pedestrian}{%
  \leavevmode%
  \hbox{%
  \begin{beamercolorbox}[wd=\paperwidth,ht=2.25ex,dp=1ex,center]{footer}%
	\href{https://ieeexplore.ieee.org/abstract/document/9559998}{A Survey on Motion Prediction of Pedestrians and Vehicles for Autonomous Driving, Gulzar, Mahir, Yar Muhammad, and Naveed Muhammad, \textit{IEEE Access 9}, 2021.}
  \end{beamercolorbox}%
  }%
  \vskip0pt%
}

\defbeamertemplate*{footline}{sparks}{%
  \leavevmode%
  \hbox{%
  \begin{beamercolorbox}[wd=\paperwidth,ht=2.25ex,dp=1ex,center]{footer}%
	\href{https://arxiv.org/pdf/2303.12712.pdf}{Sparks of artificial general intelligence: Early experiments with gpt-4, Bubeck, Sébastien, et al., \textit{arxiv}, 2023.}
  \end{beamercolorbox}%
  }%
  \vskip0pt%
}

\defbeamertemplate*{footline}{creativity}{%
  \leavevmode%
  \hbox{%
  \begin{beamercolorbox}[wd=\paperwidth,ht=2.25ex,dp=1ex,center]{footer}%
	\href{https://www.mdpi.com/2504-2289/7/1/35}{"What Can ChatGPT Do?” Analyzing Early Reactions to the Innovative AI Chatbot on Twitter, Taecharungroj, Viriya, \textit{Big Data and Cognitive Computing}, 2023.}
  \end{beamercolorbox}%
  }%
  \vskip0pt%
}

\defbeamertemplate*{footline}{vg}{%
  \leavevmode%
  \hbox{%
  \begin{beamercolorbox}[wd=\paperwidth,ht=2.25ex,dp=1ex,center]{footer}%
	\href{https://www.vg.no/nyheter/innenriks/i/9zdmBM/fikk-haanden-analysert-av-kunstig-intelligens-resultatet-kom-saa-raskt}{https://www.vg.no/nyheter/innenriks/i/9zdmBM/fikk-haanden-analysert-av-kunstig-intelligens-resultatet-kom-saa-raskt}
  \end{beamercolorbox}%
  }%
  \vskip0pt%
}

\defbeamertemplate*{footline}{boneview}{%
  \leavevmode%
  \hbox{%
  \begin{beamercolorbox}[wd=\paperwidth,ht=2.25ex,dp=1ex,center]{footer}%
	\href{https://pubs.rsna.org/doi/full/10.1148/radiol.210937}{Improving Radiographic Fracture Recognition Performance and Efficiency Using Artificial Intelligence, Guermazi, Ali et al., \textit{Radiology}, 2022.}
  \end{beamercolorbox}%
  }%
  \vskip0pt%
}

\defbeamertemplate*{footline}{covid}{%
  \leavevmode%
  \hbox{%
  \begin{beamercolorbox}[wd=\paperwidth,ht=2.25ex,dp=1ex,center]{footer}%
	\href{https://www.sciencedirect.com/science/article/pii/S0004370222001795}{Assessing the communication gap between AI models and healthcare professionals ..., Wysocki, Oskar et al., \textit{Artificial Intelligence}, 2023.}
  \end{beamercolorbox}%
  }%
  \vskip0pt%
}

\defbeamertemplate*{footline}{adverserial}{%
  \leavevmode%
  \hbox{%
  \begin{beamercolorbox}[wd=\paperwidth,ht=2.25ex,dp=1ex,center]{footer}%
	\href{https://arxiv.org/abs/1412.6572}{Explaining and Harnessing Adversarial Examples, Goodfellow, Ian J., Jonathon Shlens, and Christian Szegedy, \textit{arXiv}, 2014.}
  \end{beamercolorbox}%
  }%
  \vskip0pt%
}

\defbeamertemplate*{footline}{medical_adverserial}{%
  \leavevmode%
  \hbox{%
  \begin{beamercolorbox}[wd=\paperwidth,ht=2.25ex,dp=1ex,center]{footer}%
	\href{https://doi.org/10.1126/science.aaw4399}{Adversarial attacks on medical machine learning, Finlayson, Samuel G., et al, \textit{Science}, 2019.}
  \end{beamercolorbox}%
  }%
  \vskip0pt%
}

\defbeamertemplate*{footline}{gradcam}{%
  \leavevmode%
  \hbox{%
  \begin{beamercolorbox}[wd=\paperwidth,ht=2.25ex,dp=1ex,center]{footer}%
	\href{https://arxiv.org/abs/1610.02391}{Grad-cam: Visual explanations from deep networks via gradient-based localization, Selvaraju, Ramprasaath R., et al., \textit{Proceedings of the IEEE ICCV}, 2017.}
  \end{beamercolorbox}%
  }%
  \vskip0pt%
}

\defbeamertemplate*{footline}{positivism}{%
  \leavevmode%
  \hbox{%
  \begin{beamercolorbox}[wd=\paperwidth,ht=2.25ex,dp=1ex,center]{footer}%
	\href{https://link.springer.com/article/10.1007/s00146-019-00931-w}{Perceptions about automated decision-making by artificial intelligence, Araujo, Theo, et al., \textit{AI \& society}, 2020.}
  \end{beamercolorbox}%
  }%
  \vskip0pt%
}

\defbeamertemplate*{footline}{decisiontype}{%
  \leavevmode%
  \hbox{%
  \begin{beamercolorbox}[wd=\paperwidth,ht=2.25ex,dp=1ex,center]{footer}%
	\href{https://journals.sagepub.com/doi/full/10.1177/2053951718756684}{Understanding perception of algorithmic decisions: Fairness, trust, and emotion in response to algorithmic management, Lee, Min Kyung, \textit{Big Data \& Society}, 2018.}
  \end{beamercolorbox}%
  }%
  \vskip0pt%
}

\defbeamertemplate*{footline}{moraloutrage}{%
  \leavevmode%
  \hbox{%
  \begin{beamercolorbox}[wd=\paperwidth,ht=2.25ex,dp=1ex,center]{footer}%
	\href{https://pubmed.ncbi.nlm.nih.gov/35758989/}{Algorithmic discrimination causes less moral outrage than human discrimination, Bigman, Yochanan E., et al., \textit{Journal of Experimental Psychology}, 2023.}
  \end{beamercolorbox}%
  }%
  \vskip0pt%
}

\defbeamertemplate*{footline}{aitrust}{%
  \leavevmode%
  \hbox{%
  \begin{beamercolorbox}[wd=\paperwidth,ht=2.25ex,dp=1ex,center]{footer}%
	\href{https://journals.sagepub.com/doi/full/10.1177/00187208211013988}{Trust in artificial intelligence: Meta-analytic findings, Kaplan, Alexandra D., et al., \textit{Human Factors}, 2023.}
  \end{beamercolorbox}%
  }%
  \vskip0pt%
}

\defbeamertemplate*{footline}{humanblackbox}{%
  \leavevmode%
  \hbox{%
  \begin{beamercolorbox}[wd=\paperwidth,ht=2.25ex,dp=1ex,center]{footer}%
	\href{https://psycnet.apa.org/record/2022-29891-001}{The human black-box ..., Kaplan, Bonezzi, Andrea, Massimiliano Ostinelli, and Johann Melzner, \textit{ournal of Experimental Psychology: General}, 2022.}
  \end{beamercolorbox}%
  }%
  \vskip0pt%
}

\defbeamertemplate*{footline}{publicattitudes}{%
  \leavevmode%
  \hbox{%
  \begin{beamercolorbox}[wd=\paperwidth,ht=2.25ex,dp=1ex,center]{footer}%
	\href{https://www.pewresearch.org/internet/2018/11/16/public-attitudes-toward-computer-algorithms/}{Public attitudes toward computer algorithms, Smith, Aaron, 2018.}
  \end{beamercolorbox}%
  }%
  \vskip0pt%
}

\defbeamertemplate*{footline}{moralaversion}{%
  \leavevmode%
  \hbox{%
  \begin{beamercolorbox}[wd=\paperwidth,ht=2.25ex,dp=1ex,center]{footer}%
	\href{https://www.pewresearch.org/internet/2018/11/16/public-attitudes-toward-computer-algorithms/}{People are averse to machines making moral decisions, Bigman, Yochanan E., and Kurt Gray, \textit{Cognition}, 2018.}
  \end{beamercolorbox}%
  }%
  \vskip0pt%
}


\colorlet{nodefill}{teal!20}
\colorlet{background}{gray!10}

\begin{document}
  	\setbeamertemplate{footline}[default]

	\begin{frame}
		\maketitle
	\end{frame}

	\begin{frame}{Outline}
		\begin{enumerate}
			\item The history of artificial intelligence (AI).
			\item Terminology and concepts.
			\item How does AI make decisions?
			\item How can AI be used to support judgment and decision-making processes?
			\item How are decisions made by AIs perceived?
		\end{enumerate}
	\end{frame}

	\definecolor{activehistory}{HTML}{E71D36}
	\definecolor{passivehistory}{HTML}{939597}

	\section{The history of artificial intelligence}

	\begin{frame}{The history of artificial intelligence} % Turing
		\begin{tikzpicture}
			\node[] at (0, 0) {};
			\node[] at (10.5, -7.5) {};

			\draw[very thick, gray] (0.5, -1)  -- (1.65, -1) {};

			\node[
				circle,
				draw=activehistory,
				fill=activehistory
			] at (1.65, -1) {};
			\node[
				anchor=north,
				activehistory,
				align=center,
				font=\small\linespread{0.9}\selectfont,
				text height=9pt,
				text depth=3pt,
				inner sep=0pt,
				align=center
			] at (1.65, -1.588) {Turing\\test\\(1950)};

			\node[inner sep=0pt, draw=black, label=below:{Alan Turing}] (patient) at (5.25, -4.25) {
				\includegraphics[width=4cm]{data/turing.jpeg}
			};
		\end{tikzpicture}
	\end{frame}

	\setbeamertemplate{footline}[turing]

	\begin{frame}{The history of artificial intelligence} % Turing test
		\begin{tikzpicture}
			\node[] at (0, 0) {};
			\node[] at (10.5, -7.5) {};

			\draw[very thick, gray] (0.5, -1)  -- (1.65, -1) {};

			\node[
				circle,
				draw=activehistory,
				fill=activehistory
			] at (1.65, -1) {};
			\node[
				anchor=north,
				activehistory,
				align=center,
				font=\small\linespread{0.9}\selectfont,
				text height=9pt,
				text depth=3pt,
				inner sep=0pt,
				align=center
			] at (1.65, -1.588) {Turing\\test\\(1950)};

			\node[inner sep=0pt, draw=black] (patient) at (5.25, -4.25) {
				\includegraphics[width=5cm]{data/thinking.png}
			};
		\end{tikzpicture}
	\end{frame}

	\begin{frame}{The history of artificial intelligence} % Dartmouth
		\begin{tikzpicture}
			\node[] at (0, 0) {};
			\node[] at (10.5, -7.5) {};

			\draw[very thick, gray] (0.5, -1)  -- (2.35, -1) {};

			\node[
				circle,
				draw=passivehistory,
				fill=passivehistory
			] at (1.65, -1) {};
			\node[
				anchor=north,
				passivehistory,
				align=center,
				font=\small\linespread{0.9}\selectfont,
				text height=9pt,
				text depth=3pt,
				inner sep=0pt,
				align=center
			] at (1.65, -1.588) {Turing\\test\\(1950)};

			\node[
				circle,
				draw=activehistory,
				fill=activehistory,
			] at (2.35, -1) {};
			\node[
				anchor=south,
				activehistory,
				font=\small\linespread{0.9}\selectfont,
				text height=9pt,
				text depth=3pt,
				inner sep=0pt,
				align=center
			] at (2.35, -0.87) {Dartmouth\\(1956)};

			\node[inner sep=0pt] (patient) at (5.25, -4.75) {
				\includegraphics[width=6cm]{data/dartmouth.png}
			};
		\end{tikzpicture}
	\end{frame}

	\setbeamertemplate{footline}[neuron]

	\begin{frame}{The history of artificial intelligence} % Perceptron: Neuron
		\begin{tikzpicture}
			\node[] at (0, 0) {};
			\node[] at (10.5, -7.5) {};

			\draw[very thick, gray] (0.5, -1)  -- (2.68, -1) {};

			\node[
				circle,
				draw=passivehistory,
				fill=passivehistory
			] at (1.65, -1) {};
			\node[
				anchor=north,
				passivehistory,
				align=center,
				font=\small\linespread{0.9}\selectfont,
				text height=9pt,
				text depth=3pt,
				inner sep=0pt,
				align=center
			] at (1.65, -1.588) {Turing\\test\\(1950)};

			\node[
				circle,
				draw=passivehistory,
				fill=passivehistory,
			] at (2.35, -1) {};
			\node[
				anchor=south,
				passivehistory,
				font=\small\linespread{0.9}\selectfont,
				text height=9pt,
				text depth=3pt,
				inner sep=0pt,
				align=center
			] at (2.35, -0.87) {Dartmouth\\(1956)};

			\node[
				circle,
				draw=activehistory,
				fill=activehistory,
			] at (2.68, -1) {};
			\node[
				anchor=north,
				activehistory,
				font=\small\linespread{0.9}\selectfont,
				text height=9pt,
				text depth=3pt,
				inner sep=0pt,
				align=center
			] at (2.68, -1.375) {Perceptron\\(1958)};

			\node[draw=black, inner sep=5pt, fill=white] (patient) at (2.875, -4.75) {
				\includegraphics[width=4cm]{data/neuron.png}
			};
		\end{tikzpicture}
	\end{frame}

	\begin{frame}{The history of artificial intelligence} % Perceptron
		\begin{tikzpicture}
			\node[] at (0, 0) {};
			\node[] at (10.5, -7.5) {};

			\draw[very thick, gray] (0.5, -1)  -- (2.68, -1) {};

			\node[
				circle,
				draw=passivehistory,
				fill=passivehistory
			] at (1.65, -1) {};
			\node[
				anchor=north,
				passivehistory,
				align=center,
				font=\small\linespread{0.9}\selectfont,
				text height=9pt,
				text depth=3pt,
				inner sep=0pt,
				align=center
			] at (1.65, -1.588) {Turing\\test\\(1950)};

			\node[
				circle,
				draw=passivehistory,
				fill=passivehistory,
			] at (2.35, -1) {};
			\node[
				anchor=south,
				passivehistory,
				font=\small\linespread{0.9}\selectfont,
				text height=9pt,
				text depth=3pt,
				inner sep=0pt,
				align=center
			] at (2.35, -0.87) {Dartmouth\\(1956)};

			\node[
				circle,
				draw=activehistory,
				fill=activehistory,
			] at (2.68, -1) {};
			\node[
				anchor=north,
				activehistory,
				font=\small\linespread{0.9}\selectfont,
				text height=9pt,
				text depth=3pt,
				inner sep=0pt,
				align=center
			] at (2.68, -1.375) {Perceptron\\(1958)};

			\node[draw=black, inner sep=5pt, fill=white] (patient) at (2.875, -4.75) {
				\includegraphics[width=4cm]{data/neuron.png}
			};

			\node[circle, draw=black, fill=nodefill, inner sep=2pt, text depth=0] (node) at (7.625, -4.75) {+};

			\node[] (x0) at (6.375, -3.75) {$\mathrm{input}_0$};
			\node[] (x1) at (6.375, -4.75) {$\mathrm{input}_1$};
			\node[] (x2) at (6.375, -5.75) {$\mathrm{input}_2$};

			\node[align=center] (out) at (8.875, -4.75) {$\mathrm{output}$\\$\mathrm{(0/1)}$};

			\draw[-Latex] (x0) -- (node) node [midway, above] {\small{$w_0$}};
			\draw[-Latex] (x1) -- (node) node [midway, below] {\small{$w_1$}};;
			\draw[-Latex] (x2) -- (node) node [midway, below] {\small{$w_2$}};;
			\draw[-Latex] (node) -- (out);
		\end{tikzpicture}
	\end{frame}

	\setbeamertemplate{footline}[eliza]

	\begin{frame}{The history of artificial intelligence} % Eliza
		\begin{tikzpicture}
			\node[] at (0, 0) {};
			\node[] at (10.5, -7.5) {};

			\draw[very thick, gray] (0.5, -1)  -- (3.28, -1) {};

			\node[
				circle,
				draw=passivehistory,
				fill=passivehistory
			] at (1.65, -1) {};
			\node[
				anchor=north,
				passivehistory,
				align=center,
				font=\small\linespread{0.9}\selectfont,
				text height=9pt,
				text depth=3pt,
				inner sep=0pt,
				align=center
			] at (1.65, -1.588) {Turing\\test\\(1950)};

			\node[
				circle,
				draw=passivehistory,
				fill=passivehistory,
			] at (2.35, -1) {};
			\node[
				anchor=south,
				passivehistory,
				font=\small\linespread{0.9}\selectfont,
				text height=9pt,
				text depth=3pt,
				inner sep=0pt,
				align=center
			] at (2.35, -0.87) {Dartmouth\\(1956)};

			\node[
				circle,
				draw=passivehistory,
				fill=passivehistory,
			] at (2.68, -1) {};
			\node[
				anchor=north,
				passivehistory,
				font=\small\linespread{0.9}\selectfont,
				text height=9pt,
				text depth=3pt,
				inner sep=0pt,
				align=center
			] at (2.68, -1.375) {Perceptron\\(1958)};

			\node[
				circle,
				draw=activehistory,
				fill=activehistory
			] at (3.28, -1) {};
			\node[
				anchor=south,
				activehistory,
				font=\small\linespread{0.9}\selectfont,
				text height=9pt,
				text depth=3pt,
				inner sep=0pt,
				align=center
			] at (3.28, -0.87) {Eliza\\(1964)};

			\node[draw=black, inner sep=5pt, fill=white] (patient) at (5.25, -4.75) {
				\includegraphics[width=7cm]{data/eliza.png}
			};
		\end{tikzpicture}
	\end{frame}

	\setbeamertemplate{footline}[mycin]

	\begin{frame}{The history of artificial intelligence} % Expert systems: Overview
		\begin{tikzpicture}
			\node[] at (0, 0) {};
			\node[] at (10.5, -7.5) {};

			\draw[very thick, gray] (0.5, -1)  -- (5.13, -1) {};

			\node[
				circle,
				draw=passivehistory,
				fill=passivehistory
			] at (1.65, -1) {};
			\node[
				anchor=north,
				passivehistory,
				align=center,
				font=\small\linespread{0.9}\selectfont,
				text height=9pt,
				text depth=3pt,
				inner sep=0pt,
				align=center
			] at (1.65, -1.588) {Turing\\test\\(1950)};

			\node[
				circle,
				draw=passivehistory,
				fill=passivehistory,
			] at (2.35, -1) {};
			\node[
				anchor=south,
				passivehistory,
				font=\small\linespread{0.9}\selectfont,
				text height=9pt,
				text depth=3pt,
				inner sep=0pt,
				align=center
			] at (2.35, -0.87) {Dartmouth\\(1956)};

			\node[
				circle,
				draw=passivehistory,
				fill=passivehistory,
			] at (2.68, -1) {};
			\node[
				anchor=north,
				passivehistory,
				font=\small\linespread{0.9}\selectfont,
				text height=9pt,
				text depth=3pt,
				inner sep=0pt,
				align=center
			] at (2.68, -1.375) {Perceptron\\(1958)};

			\node[
				circle,
				draw=passivehistory,
				fill=passivehistory
			] at (3.28, -1) {};
			\node[
				circle,
				draw=passivehistory,
				fill=passivehistory,
			] at (2.35, -1) {};
			\node[
				anchor=south,
				passivehistory,
				font=\small\linespread{0.9}\selectfont,
				text height=9pt,
				text depth=3pt,
				inner sep=0pt,
				align=center
			] at (3.28, -0.87) {Eliza\\(1964)};

			\node[
				circle,
				draw=activehistory,
				fill=activehistory
			] at (5.13, -1) {};
			\node[
				anchor=north,
				activehistory,
				font=\small\linespread{0.9}\selectfont,
				text height=9pt,
				text depth=3pt,
				inner sep=0pt,
				align=center
			] at (5.13, -1.625) {Expert\\systems\\(1980s)};

			\node[] (patient) at (5.25, -3.25) {
				\Huge{\emoji{face-with-medical-mask}}
			};
			\node[draw=black, align=center ] (interface) at ($ (patient.south) - (0, 0.8) $) {
				User interface
			};

			\node[draw=black] (inference) at ($ (interface.south) - (0, 0.5) $) {
				Inference engine
			};
			\node[draw=black] (database) at ($ (inference.south) - (0, 0.5) $) {
				Knowledge database
			};
			\node[] (doctor) at ($ (database.south) - (0, 0.9) $) {
				\Huge{\emoji{woman-scientist}}
			};

			\draw[-Latex] ($ (patient.south) - (0.1, 0) $) -- ($ (interface.north) - (0.1, 0) $);
			\draw[Latex-] ($ (patient.south) + (0.1, 0) $) -- ($ (interface.north) + (0.1, 0) $);
			\draw[-Latex] ($ (interface.south) - (0.1, 0) $) -- ($ (inference.north) - (0.1, 0) $);
			\draw[Latex-] ($ (interface.south) + (0.1, 0) $) -- ($ (inference.north) + (0.1, 0) $);
			\draw[Latex-] (inference.south) -- (database.north);
			\draw[Latex-, dashed] (database.south) -- (doctor.north);

			\draw[densely dotted] ($ (interface.north west) + (-1.3, 0.1) $) rectangle ($ (database.south east) + (1, -0.1) $);
			\node[anchor=south west] at ($ (interface.north west) + (-1.3, 0.1) $) {MYCIN};

			\node[anchor=north east] at ($ (patient.south) - (0.1, 0.05) $) {\small{Query}};
			\node[anchor=north west] at ($ (patient.south) - (-0.1, 0.05) $) {\small{Response}};
		\end{tikzpicture}
	\end{frame}

	\begin{frame}{The history of artificial intelligence} % Expert systems: Overview
		\begin{tikzpicture}
			\node[] at (0, 0) {};
			\node[] at (10.5, -7.5) {};

			\draw[very thick, gray] (0.5, -1)  -- (5.13, -1) {};

			\node[
				circle,
				draw=passivehistory,
				fill=passivehistory
			] at (1.65, -1) {};
			\node[
				anchor=north,
				passivehistory,
				align=center,
				font=\small\linespread{0.9}\selectfont,
				text height=9pt,
				text depth=3pt,
				inner sep=0pt,
				align=center
			] at (1.65, -1.588) {Turing\\test\\(1950)};

			\node[
				circle,
				draw=passivehistory,
				fill=passivehistory,
			] at (2.35, -1) {};
			\node[
				anchor=south,
				passivehistory,
				font=\small\linespread{0.9}\selectfont,
				text height=9pt,
				text depth=3pt,
				inner sep=0pt,
				align=center
			] at (2.35, -0.87) {Dartmouth\\(1956)};

			\node[
				circle,
				draw=passivehistory,
				fill=passivehistory,
			] at (2.68, -1) {};
			\node[
				anchor=north,
				passivehistory,
				font=\small\linespread{0.9}\selectfont,
				text height=9pt,
				text depth=3pt,
				inner sep=0pt,
				align=center
			] at (2.68, -1.375) {Perceptron\\(1958)};

			\node[
				circle,
				draw=passivehistory,
				fill=passivehistory
			] at (3.28, -1) {};
			\node[
				circle,
				draw=passivehistory,
				fill=passivehistory,
			] at (2.35, -1) {};
			\node[
				anchor=south,
				passivehistory,
				font=\small\linespread{0.9}\selectfont,
				text height=9pt,
				text depth=3pt,
				inner sep=0pt,
				align=center
			] at (3.28, -0.87) {Eliza\\(1964)};

			\node[
				circle,
				draw=activehistory,
				fill=activehistory
			] at (5.13, -1) {};
			\node[
				anchor=north,
				activehistory,
				font=\small\linespread{0.9}\selectfont,
				text height=9pt,
				text depth=3pt,
				inner sep=0pt,
				align=center
			] at (5.13, -1.625) {Expert\\systems\\(1980s)};

			\node[inner sep=0pt, draw=black] (patient) at (5.25, -4.75) {
				\includegraphics[width=7cm]{data/mycin.png}
			};
		\end{tikzpicture}
	\end{frame}

	\setbeamertemplate{footline}[backprop]

	\begin{frame}{The history of artificial intelligence} % ANNs: Paper
		\begin{tikzpicture}
			\node[] at (0, 0) {};
			\node[] at (10.5, -7.5) {};

			\draw[very thick, gray] (0.5, -1)  -- (5.82, -1) {};

			\node[
				circle,
				draw=passivehistory,
				fill=passivehistory
			] at (1.65, -1) {};
			\node[
				anchor=north,
				passivehistory,
				align=center,
				font=\small\linespread{0.9}\selectfont,
				text height=9pt,
				text depth=3pt,
				inner sep=0pt,
				align=center
			] at (1.65, -1.588) {Turing\\test\\(1950)};

			\node[
				circle,
				draw=passivehistory,
				fill=passivehistory,
			] at (2.35, -1) {};
			\node[
				anchor=south,
				passivehistory,
				font=\small\linespread{0.9}\selectfont,
				text height=9pt,
				text depth=3pt,
				inner sep=0pt,
				align=center
			] at (2.35, -0.87) {Dartmouth\\(1956)};

			\node[
				circle,
				draw=passivehistory,
				fill=passivehistory,
			] at (2.68, -1) {};
			\node[
				anchor=north,
				passivehistory,
				font=\small\linespread{0.9}\selectfont,
				text height=9pt,
				text depth=3pt,
				inner sep=0pt,
				align=center
			] at (2.68, -1.375) {Perceptron\\(1958)};

			\node[
				circle,
				draw=passivehistory,
				fill=passivehistory
			] at (3.28, -1) {};
			\node[
				circle,
				draw=passivehistory,
				fill=passivehistory,
			] at (2.35, -1) {};
			\node[
				anchor=south,
				passivehistory,
				font=\small\linespread{0.9}\selectfont,
				text height=9pt,
				text depth=3pt,
				inner sep=0pt,
				align=center
			] at (3.28, -0.87) {Eliza\\(1964)};

			\node[
				circle,
				draw=passivehistory,
				fill=passivehistory
			] at (5.13, -1) {};
			\node[
				anchor=north,
				passivehistory,
				font=\small\linespread{0.9}\selectfont,
				text height=9pt,
				text depth=3pt,
				inner sep=0pt,
				align=center
			] at (5.13, -1.625) {Expert\\systems\\(1980s)};

			\node[
				circle,
				draw=activehistory,
				fill=activehistory
			] at (5.82, -1) {};
			\node[
				anchor=south,
				activehistory,
				font=\small\linespread{0.9}\selectfont,
				text height=9pt,
				text depth=3pt,
				inner sep=0pt,
				align=center
			] at (5.82, -0.87) {ANNs\\(1986)};

			\node[inner sep=0pt, draw=black] (img3) at (5.25, -4.75) {
				\includegraphics[width=4cm]{data/backprop.png}
			};
		\end{tikzpicture}
	\end{frame}

	\begin{frame}{The history of artificial intelligence} % ANNs: Perceptron
		\begin{tikzpicture}
			\node[] at (0, 0) {};
			\node[] at (10.5, -7.5) {};

			\draw[very thick, gray] (0.5, -1)  -- (5.82, -1) {};

			\node[
				circle,
				draw=passivehistory,
				fill=passivehistory
			] at (1.65, -1) {};
			\node[
				anchor=north,
				passivehistory,
				align=center,
				font=\small\linespread{0.9}\selectfont,
				text height=9pt,
				text depth=3pt,
				inner sep=0pt,
				align=center
			] at (1.65, -1.588) {Turing\\test\\(1950)};

			\node[
				circle,
				draw=passivehistory,
				fill=passivehistory,
			] at (2.35, -1) {};
			\node[
				anchor=south,
				passivehistory,
				font=\small\linespread{0.9}\selectfont,
				text height=9pt,
				text depth=3pt,
				inner sep=0pt,
				align=center
			] at (2.35, -0.87) {Dartmouth\\(1956)};

			\node[
				circle,
				draw=passivehistory,
				fill=passivehistory,
			] at (2.68, -1) {};
			\node[
				anchor=north,
				passivehistory,
				font=\small\linespread{0.9}\selectfont,
				text height=9pt,
				text depth=3pt,
				inner sep=0pt,
				align=center
			] at (2.68, -1.375) {Perceptron\\(1958)};

			\node[
				circle,
				draw=passivehistory,
				fill=passivehistory
			] at (3.28, -1) {};
			\node[
				circle,
				draw=passivehistory,
				fill=passivehistory,
			] at (2.35, -1) {};
			\node[
				anchor=south,
				passivehistory,
				font=\small\linespread{0.9}\selectfont,
				text height=9pt,
				text depth=3pt,
				inner sep=0pt,
				align=center
			] at (3.28, -0.87) {Eliza\\(1964)};

			\node[
				circle,
				draw=passivehistory,
				fill=passivehistory
			] at (5.13, -1) {};
			\node[
				anchor=north,
				passivehistory,
				font=\small\linespread{0.9}\selectfont,
				text height=9pt,
				text depth=3pt,
				inner sep=0pt,
				align=center
			] at (5.13, -1.625) {Expert\\systems\\(1980s)};

			\node[
				circle,
				draw=activehistory,
				fill=activehistory
			] at (5.82, -1) {};
			\node[
				anchor=south,
				activehistory,
				font=\small\linespread{0.9}\selectfont,
				text height=9pt,
				text depth=3pt,
				inner sep=0pt,
				align=center
			] at (5.82, -0.87) {ANNs\\(1986)};

			\node[inner sep=0pt, minimum size=0.4cm, draw=black, circle, fill=nodefill] (n00) at (5.25, -4.75) {};

			\node[] (x0) at (4, -4) {$\mathrm{input}_0$};
			\node[] (x1) at (4, -4.75) {$\mathrm{input}_1$};
			\node[] (x2) at (4, -5.5) {$\mathrm{input}_2$};

			\node[] (out) at (6.5, -4.75) {$\mathrm{output}$};

			\draw[-] (x0.east) -- (n00);
			\draw[-] (x1.east) -- (n00);
			\draw[-] (x2.east) -- (n00);

			\draw[->] (n00) -- (out);
		\end{tikzpicture}
	\end{frame}

	\begin{frame}{The history of artificial intelligence} % ANNs: Neural net
		\begin{tikzpicture}
			\node[] at (0, 0) {};
			\node[] at (10.5, -7.5) {};

			\draw[very thick, gray] (0.5, -1)  -- (5.82, -1) {};

			\node[
				circle,
				draw=passivehistory,
				fill=passivehistory
			] at (1.65, -1) {};
			\node[
				anchor=north,
				passivehistory,
				align=center,
				font=\small\linespread{0.9}\selectfont,
				text height=9pt,
				text depth=3pt,
				inner sep=0pt,
				align=center
			] at (1.65, -1.588) {Turing\\test\\(1950)};

			\node[
				circle,
				draw=passivehistory,
				fill=passivehistory,
			] at (2.35, -1) {};
			\node[
				anchor=south,
				passivehistory,
				font=\small\linespread{0.9}\selectfont,
				text height=9pt,
				text depth=3pt,
				inner sep=0pt,
				align=center
			] at (2.35, -0.87) {Dartmouth\\(1956)};

			\node[
				circle,
				draw=passivehistory,
				fill=passivehistory,
			] at (2.68, -1) {};
			\node[
				anchor=north,
				passivehistory,
				font=\small\linespread{0.9}\selectfont,
				text height=9pt,
				text depth=3pt,
				inner sep=0pt,
				align=center
			] at (2.68, -1.375) {Perceptron\\(1958)};

			\node[
				circle,
				draw=passivehistory,
				fill=passivehistory
			] at (3.28, -1) {};
			\node[
				circle,
				draw=passivehistory,
				fill=passivehistory,
			] at (2.35, -1) {};
			\node[
				anchor=south,
				passivehistory,
				font=\small\linespread{0.9}\selectfont,
				text height=9pt,
				text depth=3pt,
				inner sep=0pt,
				align=center
			] at (3.28, -0.87) {Eliza\\(1964)};

			\node[
				circle,
				draw=passivehistory,
				fill=passivehistory
			] at (5.13, -1) {};
			\node[
				anchor=north,
				passivehistory,
				font=\small\linespread{0.9}\selectfont,
				text height=9pt,
				text depth=3pt,
				inner sep=0pt,
				align=center
			] at (5.13, -1.625) {Expert\\systems\\(1980s)};

			\node[
				circle,
				draw=activehistory,
				fill=activehistory
			] at (5.82, -1) {};
			\node[
				anchor=south,
				activehistory,
				font=\small\linespread{0.9}\selectfont,
				text height=9pt,
				text depth=3pt,
				inner sep=0pt,
				align=center
			] at (5.82, -0.87) {ANNs\\(1986)};

			\node[] (x0) at (2.5, -4) {$\mathrm{input}_0$};
			\node[] (x1) at (2.5, -4.75) {$\mathrm{input}_1$};
			\node[] (x2) at (2.5, -5.5) {$\mathrm{input}_2$};

			\node[inner sep=0pt, minimum size=0.4cm, draw=black, circle, fill=nodefill] (n00) at (3.75, -3.75) {};
			\node[inner sep=0pt, minimum size=0.4cm, draw=black, circle, fill=nodefill] (n01) at (3.75, -4.25) {};
			\node[inner sep=0pt, minimum size=0.4cm, draw=black, circle, fill=nodefill] (n02) at (3.75, -4.75) {};
			\node[inner sep=0pt, minimum size=0.4cm, draw=black, circle, fill=nodefill] (n03) at (3.75, -5.25) {};
			\node[inner sep=0pt, minimum size=0.4cm, draw=black, circle, fill=nodefill] (n04) at (3.75, -5.75) {};

			\node[inner sep=0pt, minimum size=0.4cm, draw=black, circle, fill=nodefill] (n10) at (4.5, -4) {};
			\node[inner sep=0pt, minimum size=0.4cm, draw=black, circle, fill=nodefill] (n11) at (4.5, -4.5) {};
			\node[inner sep=0pt, minimum size=0.4cm, draw=black, circle, fill=nodefill] (n12) at (4.5, -5) {};
			\node[inner sep=0pt, minimum size=0.4cm, draw=black, circle, fill=nodefill] (n13) at (4.5, -5.5) {};

			\node[inner sep=0pt, minimum size=0.4cm, draw=black, circle, fill=nodefill] (n20) at (5.25, -4.25) {};
			\node[inner sep=0pt, minimum size=0.4cm, draw=black, circle, fill=nodefill] (n21) at (5.25, -4.75) {};
			\node[inner sep=0pt, minimum size=0.4cm, draw=black, circle, fill=nodefill] (n22) at (5.25, -5.25) {};

			\node[inner sep=0pt, minimum size=0.4cm, draw=black, circle, fill=nodefill] (n30) at (6, -4.5) {};
			\node[inner sep=0pt, minimum size=0.4cm, draw=black, circle, fill=nodefill] (n31) at (6, -5) {};

			\node[inner sep=0pt, minimum size=0.4cm, draw=black, circle, fill=nodefill, text depth=0] (n40) at (6.75, -4.75) {};
			\node[] (out) at (8, -4.75) {$\mathrm{output}$};

			\draw[-] (x0.east) -- (n00);
			\draw[-] (x0.east) -- (n01);
			\draw[-] (x0.east) -- (n02);
			\draw[-] (x0.east) -- (n03);
			\draw[-] (x0.east) -- (n04);
			\draw[-] (x1.east) -- (n00);
			\draw[-] (x1.east) -- (n01);
			\draw[-] (x1.east) -- (n02);
			\draw[-] (x1.east) -- (n03);
			\draw[-] (x1.east) -- (n04);
			\draw[-] (x2.east) -- (n00);
			\draw[-] (x2.east) -- (n01);
			\draw[-] (x2.east) -- (n02);
			\draw[-] (x2.east) -- (n03);
			\draw[-] (x2.east) -- (n04);

			\draw[-] (n00) -- (n10);
			\draw[-] (n00) -- (n11);
			\draw[-] (n00) -- (n12);
			\draw[-] (n00) -- (n13);
			\draw[-] (n01) -- (n10);
			\draw[-] (n01) -- (n11);
			\draw[-] (n01) -- (n12);
			\draw[-] (n01) -- (n13);
			\draw[-] (n02) -- (n10);
			\draw[-] (n02) -- (n11);
			\draw[-] (n02) -- (n12);
			\draw[-] (n02) -- (n13);
			\draw[-] (n03) -- (n10);
			\draw[-] (n03) -- (n11);
			\draw[-] (n03) -- (n12);
			\draw[-] (n03) -- (n13);
			\draw[-] (n04) -- (n10);
			\draw[-] (n04) -- (n11);
			\draw[-] (n04) -- (n12);
			\draw[-] (n04) -- (n13);

			\draw[-] (n10) -- (n20);
			\draw[-] (n10) -- (n21);
			\draw[-] (n10) -- (n22);
			\draw[-] (n11) -- (n20);
			\draw[-] (n11) -- (n21);
			\draw[-] (n11) -- (n22);
			\draw[-] (n12) -- (n20);
			\draw[-] (n12) -- (n21);
			\draw[-] (n12) -- (n22);
			\draw[-] (n13) -- (n20);
			\draw[-] (n13) -- (n21);
			\draw[-] (n13) -- (n22);

			\draw[-] (n20) -- (n30);
			\draw[-] (n20) -- (n31);
			\draw[-] (n21) -- (n30);
			\draw[-] (n21) -- (n31);
			\draw[-] (n22) -- (n30);
			\draw[-] (n22) -- (n31);

			\draw[-] (n30) -- (n40);
			\draw[-] (n31) -- (n40);

			\draw[->] (n40) -- (out);
		\end{tikzpicture}
	\end{frame}

	\begin{frame}{The history of artificial intelligence} % ANNs: Backprop
		\begin{tikzpicture}
			\node[] at (0, 0) {};
			\node[] at (10.5, -7.5) {};

			\draw[very thick, gray] (0.5, -1)  -- (5.82, -1) {};

			\node[
				circle,
				draw=passivehistory,
				fill=passivehistory
			] at (1.65, -1) {};
			\node[
				anchor=north,
				passivehistory,
				align=center,
				font=\small\linespread{0.9}\selectfont,
				text height=9pt,
				text depth=3pt,
				inner sep=0pt,
				align=center
			] at (1.65, -1.588) {Turing\\test\\(1950)};

			\node[
				circle,
				draw=passivehistory,
				fill=passivehistory,
			] at (2.35, -1) {};
			\node[
				anchor=south,
				passivehistory,
				font=\small\linespread{0.9}\selectfont,
				text height=9pt,
				text depth=3pt,
				inner sep=0pt,
				align=center
			] at (2.35, -0.87) {Dartmouth\\(1956)};

			\node[
				circle,
				draw=passivehistory,
				fill=passivehistory,
			] at (2.68, -1) {};
			\node[
				anchor=north,
				passivehistory,
				font=\small\linespread{0.9}\selectfont,
				text height=9pt,
				text depth=3pt,
				inner sep=0pt,
				align=center
			] at (2.68, -1.375) {Perceptron\\(1958)};

			\node[
				circle,
				draw=passivehistory,
				fill=passivehistory
			] at (3.28, -1) {};
			\node[
				circle,
				draw=passivehistory,
				fill=passivehistory,
			] at (2.35, -1) {};
			\node[
				anchor=south,
				passivehistory,
				font=\small\linespread{0.9}\selectfont,
				text height=9pt,
				text depth=3pt,
				inner sep=0pt,
				align=center
			] at (3.28, -0.87) {Eliza\\(1964)};

			\node[
				circle,
				draw=passivehistory,
				fill=passivehistory
			] at (5.13, -1) {};
			\node[
				anchor=north,
				passivehistory,
				font=\small\linespread{0.9}\selectfont,
				text height=9pt,
				text depth=3pt,
				inner sep=0pt,
				align=center
			] at (5.13, -1.625) {Expert\\systems\\(1980s)};

			\node[
				circle,
				draw=activehistory,
				fill=activehistory
			] at (5.82, -1) {};
			\node[
				anchor=south,
				activehistory,
				font=\small\linespread{0.9}\selectfont,
				text height=9pt,
				text depth=3pt,
				inner sep=0pt,
				align=center
			] at (5.82, -0.87) {ANNs\\(1986)};

			\node[] (x0) at (2.5, -4) {$\mathrm{input}_0$};
			\node[] (x1) at (2.5, -4.75) {$\mathrm{input}_1$};
			\node[] (x2) at (2.5, -5.5) {$\mathrm{input}_2$};

			\node[inner sep=0pt, minimum size=0.4cm, draw=black, circle, fill=nodefill] (n00) at (3.75, -3.75) {};
			\node[inner sep=0pt, minimum size=0.4cm, draw=black, circle, fill=nodefill] (n01) at (3.75, -4.25) {};
			\node[inner sep=0pt, minimum size=0.4cm, draw=black, circle, fill=nodefill] (n02) at (3.75, -4.75) {};
			\node[inner sep=0pt, minimum size=0.4cm, draw=black, circle, fill=nodefill] (n03) at (3.75, -5.25) {};
			\node[inner sep=0pt, minimum size=0.4cm, draw=black, circle, fill=nodefill] (n04) at (3.75, -5.75) {};

			\node[inner sep=0pt, minimum size=0.4cm, draw=black, circle, fill=nodefill] (n10) at (4.5, -4) {};
			\node[inner sep=0pt, minimum size=0.4cm, draw=black, circle, fill=nodefill] (n11) at (4.5, -4.5) {};
			\node[inner sep=0pt, minimum size=0.4cm, draw=black, circle, fill=nodefill] (n12) at (4.5, -5) {};
			\node[inner sep=0pt, minimum size=0.4cm, draw=black, circle, fill=nodefill] (n13) at (4.5, -5.5) {};

			\node[inner sep=0pt, minimum size=0.4cm, draw=black, circle, fill=nodefill] (n20) at (5.25, -4.25) {};
			\node[inner sep=0pt, minimum size=0.4cm, draw=black, circle, fill=nodefill] (n21) at (5.25, -4.75) {};
			\node[inner sep=0pt, minimum size=0.4cm, draw=black, circle, fill=nodefill] (n22) at (5.25, -5.25) {};

			\node[inner sep=0pt, minimum size=0.4cm, draw=black, circle, fill=nodefill] (n30) at (6, -4.5) {};
			\node[inner sep=0pt, minimum size=0.4cm, draw=black, circle, fill=nodefill] (n31) at (6, -5) {};

			\node[inner sep=0pt, minimum size=0.4cm, draw=black, circle, fill=nodefill, text depth=0] (n40) at (6.75, -4.75) {};
			\node[] (out) at (8, -4.75) {$\mathrm{output}$};

			\draw[-] (x0.east) -- (n00);
			\draw[-] (x0.east) -- (n01);
			\draw[-] (x0.east) -- (n02);
			\draw[-] (x0.east) -- (n03);
			\draw[-] (x0.east) -- (n04);
			\draw[-] (x1.east) -- (n00);
			\draw[-] (x1.east) -- (n01);
			\draw[-] (x1.east) -- (n02);
			\draw[-] (x1.east) -- (n03);
			\draw[-] (x1.east) -- (n04);
			\draw[-] (x2.east) -- (n00);
			\draw[-] (x2.east) -- (n01);
			\draw[-] (x2.east) -- (n02);
			\draw[-] (x2.east) -- (n03);
			\draw[-] (x2.east) -- (n04);

			\draw[Latex-, red] (n00) -- (n10);
			\draw[Latex-, red] (n00) -- (n11);
			\draw[Latex-, red] (n00) -- (n12);
			\draw[Latex-, red] (n00) -- (n13);
			\draw[Latex-, red] (n01) -- (n10);
			\draw[Latex-, red] (n01) -- (n11);
			\draw[Latex-, red] (n01) -- (n12);
			\draw[Latex-, red] (n01) -- (n13);
			\draw[Latex-, red] (n02) -- (n10);
			\draw[Latex-, red] (n02) -- (n11);
			\draw[Latex-, red] (n02) -- (n12);
			\draw[Latex-, red] (n02) -- (n13);
			\draw[Latex-, red] (n03) -- (n10);
			\draw[Latex-, red] (n03) -- (n11);
			\draw[Latex-, red] (n03) -- (n12);
			\draw[Latex-, red] (n03) -- (n13);
			\draw[Latex-, red] (n04) -- (n10);
			\draw[Latex-, red] (n04) -- (n11);
			\draw[Latex-, red] (n04) -- (n12);
			\draw[Latex-, red] (n04) -- (n13);

			\draw[Latex-, red] (n10) -- (n20);
			\draw[Latex-, red] (n10) -- (n21);
			\draw[Latex-, red] (n10) -- (n22);
			\draw[Latex-, red] (n11) -- (n20);
			\draw[Latex-, red] (n11) -- (n21);
			\draw[Latex-, red] (n11) -- (n22);
			\draw[Latex-, red] (n12) -- (n20);
			\draw[Latex-, red] (n12) -- (n21);
			\draw[Latex-, red] (n12) -- (n22);
			\draw[Latex-, red] (n13) -- (n20);
			\draw[Latex-, red] (n13) -- (n21);
			\draw[Latex-, red] (n13) -- (n22);

			\draw[Latex-, red] (n20) -- (n30);
			\draw[Latex-, red] (n20) -- (n31);
			\draw[Latex-, red] (n21) -- (n30);
			\draw[Latex-, red] (n21) -- (n31);
			\draw[Latex-, red] (n22) -- (n30);
			\draw[Latex-, red] (n22) -- (n31);

			\draw[Latex-, red] (n30) -- (n40);
			\draw[Latex-, red] (n31) -- (n40);

			\draw[Latex-, red] (n40) -- (out);
		\end{tikzpicture}
	\end{frame}

	\setbeamertemplate{footline}[default]

	\begin{frame}{The history of artificial intelligence} % Deep blue
		\begin{tikzpicture}
			\node[] at (0, 0) {};
			\node[] at (10.5, -7.5) {};

			\draw[very thick, gray] (0.5, -1)  -- (7.1, -1) {};

			\node[
				circle,
				draw=passivehistory,
				fill=passivehistory
			] at (1.65, -1) {};
			\node[
				anchor=north,
				passivehistory,
				align=center,
				font=\small\linespread{0.9}\selectfont,
				text height=9pt,
				text depth=3pt,
				inner sep=0pt,
				align=center
			] at (1.65, -1.588) {Turing\\test\\(1950)};

			\node[
				circle,
				draw=passivehistory,
				fill=passivehistory,
			] at (2.35, -1) {};
			\node[
				anchor=south,
				passivehistory,
				font=\small\linespread{0.9}\selectfont,
				text height=9pt,
				text depth=3pt,
				inner sep=0pt,
				align=center
			] at (2.35, -0.87) {Dartmouth\\(1956)};

			\node[
				circle,
				draw=passivehistory,
				fill=passivehistory,
			] at (2.68, -1) {};
			\node[
				anchor=north,
				passivehistory,
				font=\small\linespread{0.9}\selectfont,
				text height=9pt,
				text depth=3pt,
				inner sep=0pt,
				align=center
			] at (2.68, -1.375) {Perceptron\\(1958)};

			\node[
				circle,
				draw=passivehistory,
				fill=passivehistory
			] at (3.28, -1) {};
			\node[
				circle,
				draw=passivehistory,
				fill=passivehistory,
			] at (2.35, -1) {};
			\node[
				anchor=south,
				passivehistory,
				font=\small\linespread{0.9}\selectfont,
				text height=9pt,
				text depth=3pt,
				inner sep=0pt,
				align=center
			] at (3.28, -0.87) {Eliza\\(1964)};

			\node[
				circle,
				draw=passivehistory,
				fill=passivehistory
			] at (5.13, -1) {};
			\node[
				anchor=north,
				passivehistory,
				font=\small\linespread{0.9}\selectfont,
				text height=9pt,
				text depth=3pt,
				inner sep=0pt,
				align=center
			] at (5.13, -1.625) {Expert\\systems\\(1980s)};

			\node[
				circle,
				draw=passivehistory,
				fill=passivehistory
			] at (5.82, -1) {};
			\node[
				anchor=south,
				passivehistory,
				font=\small\linespread{0.9}\selectfont,
				text height=9pt,
				text depth=3pt,
				inner sep=0pt,
				align=center
			] at (5.82, -0.87) {ANNs\\(1986)};

			\node[
				circle,
				draw=activehistory,
				fill=activehistory
			] at (7.10, -1) {};
			\node[
				anchor=north,
				activehistory,
				font=\small\linespread{0.9}\selectfont,
				text height=9pt,
				text depth=3pt,
				inner sep=0pt,
				align=center
			] at (7.10, -1.372) {Deep blue\\(1997)};

			\node[inner sep=0pt, draw=black, label=below:{\small{DALL-E: "A robot playing chess"}}] (img) at (2.75, -4.75) {
				\includegraphics[width=4cm]{data/chess.png}
			};
			\node[anchor=north west, align=left] at ($ (img.north east)  + (0.1, 0) $) {
				\bullet\hspace{0.1cm}IBMs Deep Blue became the first computer\\ to beat the reigning human world champion\\ in chess.\\
				\bullet\hspace{0.1cm}Deep blue won with 3\textonehalf { }points to Garry\\Kasparovs 2\textonehalf{ } after six matches.\\
				\bullet\hspace{0.1cm}Kasparov famously stated that\\"Deep Blue was intelligent the way your\\programmable alarm clock is intelligent."\\
				\bullet\hspace{0.1cm}Combination of machine learning and\\preprogrammed knowledge from experts.
			};
		\end{tikzpicture}
	\end{frame}

	\begin{frame}{The history of artificial intelligence} %DL (image classification)
		\begin{tikzpicture}
			\node[] at (0, 0) {};
			\node[] at (10.5, -7.5) {};

			\draw[very thick, gray] (0.5, -1)  -- (8.84, -1) {};

			\node[
				circle,
				draw=passivehistory,
				fill=passivehistory
			] at (1.65, -1) {};
			\node[
				anchor=north,
				passivehistory,
				align=center,
				font=\small\linespread{0.9}\selectfont,
				text height=9pt,
				text depth=3pt,
				inner sep=0pt,
				align=center
			] at (1.65, -1.588) {Turing\\test\\(1950)};

			\node[
				circle,
				draw=passivehistory,
				fill=passivehistory,
			] at (2.35, -1) {};
			\node[
				anchor=south,
				passivehistory,
				font=\small\linespread{0.9}\selectfont,
				text height=9pt,
				text depth=3pt,
				inner sep=0pt,
				align=center
			] at (2.35, -0.87) {Dartmouth\\(1956)};

			\node[
				circle,
				draw=passivehistory,
				fill=passivehistory,
			] at (2.68, -1) {};
			\node[
				anchor=north,
				passivehistory,
				font=\small\linespread{0.9}\selectfont,
				text height=9pt,
				text depth=3pt,
				inner sep=0pt,
				align=center
			] at (2.68, -1.375) {Perceptron\\(1958)};

			\node[
				circle,
				draw=passivehistory,
				fill=passivehistory
			] at (3.28, -1) {};
			\node[
				circle,
				draw=passivehistory,
				fill=passivehistory,
			] at (2.35, -1) {};
			\node[
				anchor=south,
				passivehistory,
				font=\small\linespread{0.9}\selectfont,
				text height=9pt,
				text depth=3pt,
				inner sep=0pt,
				align=center
			] at (3.28, -0.87) {Eliza\\(1964)};

			\node[
				circle,
				draw=passivehistory,
				fill=passivehistory
			] at (5.13, -1) {};
			\node[
				anchor=north,
				passivehistory,
				font=\small\linespread{0.9}\selectfont,
				text height=9pt,
				text depth=3pt,
				inner sep=0pt,
				align=center
			] at (5.13, -1.625) {Expert\\systems\\(1980s)};

			\node[
				circle,
				draw=passivehistory,
				fill=passivehistory
			] at (5.82, -1) {};
			\node[
				anchor=south,
				passivehistory,
				font=\small\linespread{0.9}\selectfont,
				text height=9pt,
				text depth=3pt,
				inner sep=0pt,
				align=center
			] at (5.82, -0.87) {ANNs\\(1986)};

			\node[
				circle,
				draw=passivehistory,
				fill=passivehistory
			] at (7.10, -1) {};
			\node[
				anchor=north,
				passivehistory,
				font=\small\linespread{0.9}\selectfont,
				text height=9pt,
				text depth=3pt,
				inner sep=0pt,
				align=center
			] at (7.10, -1.372) {Deep blue\\(1997)};

			\node[
				circle,
				draw=activehistory,
				fill=activehistory
			] at (8.84, -1) {};
			\node[
				anchor=south,
				activehistory,
				font=\small\linespread{0.9}\selectfont,
				text height=9pt,
				text depth=3pt,
				inner sep=0pt,
				align=center
			] at (8.84, -0.87) {Deep learning\\(2012)};

			\node[inner sep=0pt, label=below:\small{Cat}] (img3) at (5.25, -4.75) {
				\includegraphics[width=1.5cm]{data/cat.jpeg}
			};
		\end{tikzpicture}
	\end{frame}

	\begin{frame}{The history of artificial intelligence} %DL (image classification)
		\begin{tikzpicture}
			\node[] at (0, 0) {};
			\node[] at (10.5, -7.5) {};

			\draw[very thick, gray] (0.5, -1)  -- (8.84, -1) {};

			\node[
				circle,
				draw=passivehistory,
				fill=passivehistory
			] at (1.65, -1) {};
			\node[
				anchor=north,
				passivehistory,
				align=center,
				font=\small\linespread{0.9}\selectfont,
				text height=9pt,
				text depth=3pt,
				inner sep=0pt,
				align=center
			] at (1.65, -1.588) {Turing\\test\\(1950)};

			\node[
				circle,
				draw=passivehistory,
				fill=passivehistory,
			] at (2.35, -1) {};
			\node[
				anchor=south,
				passivehistory,
				font=\small\linespread{0.9}\selectfont,
				text height=9pt,
				text depth=3pt,
				inner sep=0pt,
				align=center
			] at (2.35, -0.87) {Dartmouth\\(1956)};

			\node[
				circle,
				draw=passivehistory,
				fill=passivehistory,
			] at (2.68, -1) {};
			\node[
				anchor=north,
				passivehistory,
				font=\small\linespread{0.9}\selectfont,
				text height=9pt,
				text depth=3pt,
				inner sep=0pt,
				align=center
			] at (2.68, -1.375) {Perceptron\\(1958)};

			\node[
				circle,
				draw=passivehistory,
				fill=passivehistory
			] at (3.28, -1) {};
			\node[
				circle,
				draw=passivehistory,
				fill=passivehistory,
			] at (2.35, -1) {};
			\node[
				anchor=south,
				passivehistory,
				font=\small\linespread{0.9}\selectfont,
				text height=9pt,
				text depth=3pt,
				inner sep=0pt,
				align=center
			] at (3.28, -0.87) {Eliza\\(1964)};

			\node[
				circle,
				draw=passivehistory,
				fill=passivehistory
			] at (5.13, -1) {};
			\node[
				anchor=north,
				passivehistory,
				font=\small\linespread{0.9}\selectfont,
				text height=9pt,
				text depth=3pt,
				inner sep=0pt,
				align=center
			] at (5.13, -1.625) {Expert\\systems\\(1980s)};

			\node[
				circle,
				draw=passivehistory,
				fill=passivehistory
			] at (5.82, -1) {};
			\node[
				anchor=south,
				passivehistory,
				font=\small\linespread{0.9}\selectfont,
				text height=9pt,
				text depth=3pt,
				inner sep=0pt,
				align=center
			] at (5.82, -0.87) {ANNs\\(1986)};

			\node[
				circle,
				draw=passivehistory,
				fill=passivehistory
			] at (7.10, -1) {};
			\node[
				anchor=north,
				passivehistory,
				font=\small\linespread{0.9}\selectfont,
				text height=9pt,
				text depth=3pt,
				inner sep=0pt,
				align=center
			] at (7.10, -1.372) {Deep blue\\(1997)};

			\node[
				circle,
				draw=activehistory,
				fill=activehistory
			] at (8.84, -1) {};
			\node[
				anchor=south,
				activehistory,
				font=\small\linespread{0.9}\selectfont,
				text height=9pt,
				text depth=3pt,
				inner sep=0pt,
				align=center
			] at (8.84, -0.87) {Deep learning\\(2012)};

			\node[inner sep=0pt, label=below:\small{Cat}] (img3) at (5.25, -4.75) {
				\includegraphics[width=1.5cm]{data/cat.jpeg}
			};
		\end{tikzpicture}
	\end{frame}

	\begin{frame}{The history of artificial intelligence} %DL (imagenet)
		\begin{tikzpicture}
			\node[] at (0, 0) {};
			\node[] at (10.5, -7.5) {};

			\draw[very thick, gray] (0.5, -1)  -- (8.84, -1) {};

			\node[
				circle,
				draw=passivehistory,
				fill=passivehistory
			] at (1.65, -1) {};
			\node[
				anchor=north,
				passivehistory,
				align=center,
				font=\small\linespread{0.9}\selectfont,
				text height=9pt,
				text depth=3pt,
				inner sep=0pt,
				align=center
			] at (1.65, -1.588) {Turing\\test\\(1950)};

			\node[
				circle,
				draw=passivehistory,
				fill=passivehistory,
			] at (2.35, -1) {};
			\node[
				anchor=south,
				passivehistory,
				font=\small\linespread{0.9}\selectfont,
				text height=9pt,
				text depth=3pt,
				inner sep=0pt,
				align=center
			] at (2.35, -0.87) {Dartmouth\\(1956)};

			\node[
				circle,
				draw=passivehistory,
				fill=passivehistory,
			] at (2.68, -1) {};
			\node[
				anchor=north,
				passivehistory,
				font=\small\linespread{0.9}\selectfont,
				text height=9pt,
				text depth=3pt,
				inner sep=0pt,
				align=center
			] at (2.68, -1.375) {Perceptron\\(1958)};

			\node[
				circle,
				draw=passivehistory,
				fill=passivehistory
			] at (3.28, -1) {};
			\node[
				circle,
				draw=passivehistory,
				fill=passivehistory,
			] at (2.35, -1) {};
			\node[
				anchor=south,
				passivehistory,
				font=\small\linespread{0.9}\selectfont,
				text height=9pt,
				text depth=3pt,
				inner sep=0pt,
				align=center
			] at (3.28, -0.87) {Eliza\\(1964)};

			\node[
				circle,
				draw=passivehistory,
				fill=passivehistory
			] at (5.13, -1) {};
			\node[
				anchor=north,
				passivehistory,
				font=\small\linespread{0.9}\selectfont,
				text height=9pt,
				text depth=3pt,
				inner sep=0pt,
				align=center
			] at (5.13, -1.625) {Expert\\systems\\(1980s)};

			\node[
				circle,
				draw=passivehistory,
				fill=passivehistory
			] at (5.82, -1) {};
			\node[
				anchor=south,
				passivehistory,
				font=\small\linespread{0.9}\selectfont,
				text height=9pt,
				text depth=3pt,
				inner sep=0pt,
				align=center
			] at (5.82, -0.87) {ANNs\\(1986)};

			\node[
				circle,
				draw=passivehistory,
				fill=passivehistory
			] at (7.10, -1) {};
			\node[
				anchor=north,
				passivehistory,
				font=\small\linespread{0.9}\selectfont,
				text height=9pt,
				text depth=3pt,
				inner sep=0pt,
				align=center
			] at (7.10, -1.372) {Deep blue\\(1997)};

			\node[
				circle,
				draw=activehistory,
				fill=activehistory
			] at (8.84, -1) {};
			\node[
				anchor=south,
				activehistory,
				font=\small\linespread{0.9}\selectfont,
				text height=9pt,
				text depth=3pt,
				inner sep=0pt,
				align=center
			] at (8.84, -0.87) {Deep learning\\(2012)};

			\node[inner sep=0pt, label=below:\small{Cat}] (img3) at (5.25, -4.75) {
				\includegraphics[width=1.5cm]{data/cat.jpeg}
			};
			\node[inner sep=0pt, label=below:\small{Airplane}, anchor=west] (img4) at ($ (img3.east) + (0.1, 0) $) {
				\includegraphics[width=1.5cm]{data/airplane.jpeg}
			};
			\node[inner sep=0pt, label=below:\small{Shark}, anchor=west] (img5) at ($ (img4.east) + (0.1, 0) $) {
				\includegraphics[width=1.5cm]{data/shark.jpeg}
			};
			\node[inner sep=0pt, label=below:\small{Ladybug}, anchor=east] (img2) at ($ (img3.west) - (0.1, 0) $) {
				\includegraphics[width=1.5cm]{data/ladybug.png}
			};
			\node[inner sep=0pt, label=below:\small{Sunflower}, anchor=east] (img1) at ($ (img2.west) - (0.1, 0) $) {
				\includegraphics[width=1.5cm]{data/sunflower.jpeg}
			};
			\node[] at (5.25, -6.5) {
				ImageNet: $\sim$14m images, $\sim$22k categories
			};
		\end{tikzpicture}
	\end{frame}

	\begin{frame}{The history of artificial intelligence} %DL (old)
		\newsavebox{\imagenetold}
		\sbox{\imagenetold}{%
			\begin{tikzpicture}
				\begin{axis}[
					ylabel={Error rate},
					xlabel={Year},
					xtick={2010, 2012, 2014, 2016, 2018, 2020},
					xticklabels={2010, 2012, 2014, 2016, 2018, 2020},
					ytick={0, 10, 20, 30},
					yticklabels={0\%, 10\%, 20\%, 30\%},
					ytick style={draw=none},
					ytick pos=left,
					xtick pos=bottom,
					ymajorgrids=true,
					ymax=29,
					ymin=0,
					xmin=2009,
					xmax=2021,
					width=8cm,
					height=5cm
				]
					\addplot[mark=*, purple,thick] coordinates {
						(2010, 28.2)
						(2011, 25.8)
					};

					\node[anchor=north, inner sep=5pt] at (axis cs: 2010, 28.2) {
						\small{28.2}
					};
					\node[anchor=north, inner sep=5pt] at (axis cs: 2011, 25.8) {
						\small{25.8}
					};
				\end{axis}
			\end{tikzpicture}
		}

		\begin{tikzpicture}
			\node[] at (0, 0) {};
			\node[] at (10.5, -7.5) {};

			\draw[very thick, gray] (0.5, -1)  -- (8.84, -1) {};

			\node[
				circle,
				draw=passivehistory,
				fill=passivehistory
			] at (1.65, -1) {};
			\node[
				anchor=north,
				passivehistory,
				align=center,
				font=\small\linespread{0.9}\selectfont,
				text height=9pt,
				text depth=3pt,
				inner sep=0pt,
				align=center
			] at (1.65, -1.588) {Turing\\test\\(1950)};

			\node[
				circle,
				draw=passivehistory,
				fill=passivehistory,
			] at (2.35, -1) {};
			\node[
				anchor=south,
				passivehistory,
				font=\small\linespread{0.9}\selectfont,
				text height=9pt,
				text depth=3pt,
				inner sep=0pt,
				align=center
			] at (2.35, -0.87) {Dartmouth\\(1956)};

			\node[
				circle,
				draw=passivehistory,
				fill=passivehistory,
			] at (2.68, -1) {};
			\node[
				anchor=north,
				passivehistory,
				font=\small\linespread{0.9}\selectfont,
				text height=9pt,
				text depth=3pt,
				inner sep=0pt,
				align=center
			] at (2.68, -1.375) {Perceptron\\(1958)};

			\node[
				circle,
				draw=passivehistory,
				fill=passivehistory
			] at (3.28, -1) {};
			\node[
				circle,
				draw=passivehistory,
				fill=passivehistory,
			] at (2.35, -1) {};
			\node[
				anchor=south,
				passivehistory,
				font=\small\linespread{0.9}\selectfont,
				text height=9pt,
				text depth=3pt,
				inner sep=0pt,
				align=center
			] at (3.28, -0.87) {Eliza\\(1964)};

			\node[
				circle,
				draw=passivehistory,
				fill=passivehistory
			] at (5.13, -1) {};
			\node[
				anchor=north,
				passivehistory,
				font=\small\linespread{0.9}\selectfont,
				text height=9pt,
				text depth=3pt,
				inner sep=0pt,
				align=center
			] at (5.13, -1.625) {Expert\\systems\\(1980s)};

			\node[
				circle,
				draw=passivehistory,
				fill=passivehistory
			] at (5.82, -1) {};
			\node[
				anchor=south,
				passivehistory,
				font=\small\linespread{0.9}\selectfont,
				text height=9pt,
				text depth=3pt,
				inner sep=0pt,
				align=center
			] at (5.82, -0.87) {ANNs\\(1986)};

			\node[
				circle,
				draw=passivehistory,
				fill=passivehistory
			] at (7.10, -1) {};
			\node[
				anchor=north,
				passivehistory,
				font=\small\linespread{0.9}\selectfont,
				text height=9pt,
				text depth=3pt,
				inner sep=0pt,
				align=center
			] at (7.10, -1.372) {Deep blue\\(1997)};

			\node[
				circle,
				draw=activehistory,
				fill=activehistory
			] at (8.84, -1) {};
			\node[
				anchor=south,
				activehistory,
				font=\small\linespread{0.9}\selectfont,
				text height=9pt,
				text depth=3pt,
				inner sep=0pt,
				align=center
			] at (8.84, -0.87) {Deep learning\\(2012)};

			\node[inner sep=0pt] at (5.25, -4.75) {
				\usebox{\imagenetold}
			};
		\end{tikzpicture}
	\end{frame}

	\begin{frame}{The history of artificial intelligence} %DL (CNN)
		\newsavebox{\imagenetcnn}
		\sbox{\imagenetcnn}{%
			\begin{tikzpicture}
				\begin{axis}[
					ylabel={Error rate},
					xlabel={Year},
					xtick={2010, 2012, 2014, 2016, 2018, 2020},
					xticklabels={2010, 2012, 2014, 2016, 2018, 2020},
					ytick={0, 10, 20, 30},
					yticklabels={0\%, 10\%, 20\%, 30\%},
					ytick style={draw=none},
					ytick pos=left,
					xtick pos=bottom,
					ymajorgrids=true,
					ymax=29,
					ymin=0,
					xmin=2009,
					xmax=2021,
					width=8cm,
					height=5cm
				]
					\addplot[mark=*, purple,thick] coordinates {
						(2010, 28.2)
						(2011, 25.8)
						(2012, 16.4)
					};

					\node[anchor=north, inner sep=5pt] at (axis cs: 2010, 28.2) {
						\small{28.2}
					};
					\node[anchor=north, inner sep=5pt] at (axis cs: 2011, 25.8) {
						\small{25.8}
					};
					\node[anchor=north, inner sep=5pt] at (axis cs: 2012, 16.4) {
						\small{16.4}
					};

					\addplot[densely dotted] coordinates {
						(2011.5, 30)
						(2011.5, 0)
					};
				\end{axis}
			\end{tikzpicture}
		}

		\begin{tikzpicture}
			\node[] at (0, 0) {};
			\node[] at (10.5, -7.5) {};

			\draw[very thick, gray] (0.5, -1)  -- (8.84, -1) {};

			\node[
				circle,
				draw=passivehistory,
				fill=passivehistory
			] at (1.65, -1) {};
			\node[
				anchor=north,
				passivehistory,
				align=center,
				font=\small\linespread{0.9}\selectfont,
				text height=9pt,
				text depth=3pt,
				inner sep=0pt,
				align=center
			] at (1.65, -1.588) {Turing\\test\\(1950)};

			\node[
				circle,
				draw=passivehistory,
				fill=passivehistory,
			] at (2.35, -1) {};
			\node[
				anchor=south,
				passivehistory,
				font=\small\linespread{0.9}\selectfont,
				text height=9pt,
				text depth=3pt,
				inner sep=0pt,
				align=center
			] at (2.35, -0.87) {Dartmouth\\(1956)};

			\node[
				circle,
				draw=passivehistory,
				fill=passivehistory,
			] at (2.68, -1) {};
			\node[
				anchor=north,
				passivehistory,
				font=\small\linespread{0.9}\selectfont,
				text height=9pt,
				text depth=3pt,
				inner sep=0pt,
				align=center
			] at (2.68, -1.375) {Perceptron\\(1958)};

			\node[
				circle,
				draw=passivehistory,
				fill=passivehistory
			] at (3.28, -1) {};
			\node[
				circle,
				draw=passivehistory,
				fill=passivehistory,
			] at (2.35, -1) {};
			\node[
				anchor=south,
				passivehistory,
				font=\small\linespread{0.9}\selectfont,
				text height=9pt,
				text depth=3pt,
				inner sep=0pt,
				align=center
			] at (3.28, -0.87) {Eliza\\(1964)};

			\node[
				circle,
				draw=passivehistory,
				fill=passivehistory
			] at (5.13, -1) {};
			\node[
				anchor=north,
				passivehistory,
				font=\small\linespread{0.9}\selectfont,
				text height=9pt,
				text depth=3pt,
				inner sep=0pt,
				align=center
			] at (5.13, -1.625) {Expert\\systems\\(1980s)};

			\node[
				circle,
				draw=passivehistory,
				fill=passivehistory
			] at (5.82, -1) {};
			\node[
				anchor=south,
				passivehistory,
				font=\small\linespread{0.9}\selectfont,
				text height=9pt,
				text depth=3pt,
				inner sep=0pt,
				align=center
			] at (5.82, -0.87) {ANNs\\(1986)};

			\node[
				circle,
				draw=passivehistory,
				fill=passivehistory
			] at (7.10, -1) {};
			\node[
				anchor=north,
				passivehistory,
				font=\small\linespread{0.9}\selectfont,
				text height=9pt,
				text depth=3pt,
				inner sep=0pt,
				align=center
			] at (7.10, -1.372) {Deep blue\\(1997)};

			\node[
				circle,
				draw=activehistory,
				fill=activehistory
			] at (8.84, -1) {};
			\node[
				anchor=south,
				activehistory,
				font=\small\linespread{0.9}\selectfont,
				text height=9pt,
				text depth=3pt,
				inner sep=0pt,
				align=center
			] at (8.84, -0.87) {Deep learning\\(2012)};

			\node[inner sep=0pt] at (5.25, -4.75) {
				\usebox{\imagenetcnn}
			};
		\end{tikzpicture}
	\end{frame}

	\begin{frame}{The history of artificial intelligence} %DL (2020)
		\newsavebox{\imagenetlatest}
		\sbox{\imagenetlatest}{%
			\begin{tikzpicture}
				\begin{axis}[
					ylabel={Error rate},
					xlabel={Year},
					xtick={2010, 2012, 2014, 2016, 2018, 2020},
					xticklabels={2010, 2012, 2014, 2016, 2018, 2020},
					ytick={0, 10, 20, 30},
					yticklabels={0\%, 10\%, 20\%, 30\%},
					ytick style={draw=none},
					ytick pos=left,
					xtick pos=bottom,
					ymajorgrids=true,
					ymax=29,
					ymin=0,
					xmin=2009,
					xmax=2021,
					width=8cm,
					height=5cm
				]
					\addplot[mark=*, purple,thick] coordinates {
						(2010, 28.2)
						(2011, 25.8)
						(2012, 16.4)
						(2013, 11.7)
						(2014, 7.3)
						(2015, 3.5)
						(2016, 3.0)
						(2017, 2.3)
						(2018, 1.8)
						(2019, 1.3)
						(2020, 0.9)
					};

					\node[anchor=north, inner sep=5pt] at (axis cs: 2010, 28.2) {
						\small{28.2}
					};
					\node[anchor=north, inner sep=5pt] at (axis cs: 2011, 25.8) {
						\small{25.8}
					};
					\node[anchor=north, inner sep=5pt] at (axis cs: 2012, 16.4) {
						\small{16.4}
					};
					\node[anchor=north, inner sep=5pt] at (axis cs: 2013, 11.7) {
						\small{11.7}
					};
					\node[anchor=north, inner sep=5pt] at (axis cs: 2014, 7.3) {
						\small{7.3}
					};
					\node[anchor=north, inner sep=5pt] at (axis cs: 2015, 3.5) {
						\small{3.5}
					};
					\node[anchor=south, inner sep=5pt] at (axis cs: 2016, 3.0) {
						\small{3.0}
					};
					\node[anchor=south, inner sep=5pt] at (axis cs: 2017, 2.3) {
						\small{2.3}
					};
					\node[anchor=south, inner sep=5pt] at (axis cs: 2018, 1.8) {
						\small{1.8}
					};
					\node[anchor=south, inner sep=5pt] at (axis cs: 2019, 1.3) {
						\small{1.3}
					};
					\node[anchor=south, inner sep=5pt] at (axis cs: 2020, 0.9) {
						\small{\textbf{0.9}}
					};

					\addplot[densely dotted] coordinates {
						(2011.5, 30)
						(2011.5, 0)
					};
				\end{axis}
			\end{tikzpicture}
		}

		\begin{tikzpicture}
			\node[] at (0, 0) {};
			\node[] at (10.5, -7.5) {};

			\draw[very thick, gray] (0.5, -1)  -- (8.84, -1) {};

			\node[
				circle,
				draw=passivehistory,
				fill=passivehistory
			] at (1.65, -1) {};
			\node[
				anchor=north,
				passivehistory,
				align=center,
				font=\small\linespread{0.9}\selectfont,
				text height=9pt,
				text depth=3pt,
				inner sep=0pt,
				align=center
			] at (1.65, -1.588) {Turing\\test\\(1950)};

			\node[
				circle,
				draw=passivehistory,
				fill=passivehistory,
			] at (2.35, -1) {};
			\node[
				anchor=south,
				passivehistory,
				font=\small\linespread{0.9}\selectfont,
				text height=9pt,
				text depth=3pt,
				inner sep=0pt,
				align=center
			] at (2.35, -0.87) {Dartmouth\\(1956)};

			\node[
				circle,
				draw=passivehistory,
				fill=passivehistory,
			] at (2.68, -1) {};
			\node[
				anchor=north,
				passivehistory,
				font=\small\linespread{0.9}\selectfont,
				text height=9pt,
				text depth=3pt,
				inner sep=0pt,
				align=center
			] at (2.68, -1.375) {Perceptron\\(1958)};

			\node[
				circle,
				draw=passivehistory,
				fill=passivehistory
			] at (3.28, -1) {};
			\node[
				circle,
				draw=passivehistory,
				fill=passivehistory,
			] at (2.35, -1) {};
			\node[
				anchor=south,
				passivehistory,
				font=\small\linespread{0.9}\selectfont,
				text height=9pt,
				text depth=3pt,
				inner sep=0pt,
				align=center
			] at (3.28, -0.87) {Eliza\\(1964)};

			\node[
				circle,
				draw=passivehistory,
				fill=passivehistory
			] at (5.13, -1) {};
			\node[
				anchor=north,
				passivehistory,
				font=\small\linespread{0.9}\selectfont,
				text height=9pt,
				text depth=3pt,
				inner sep=0pt,
				align=center
			] at (5.13, -1.625) {Expert\\systems\\(1980s)};

			\node[
				circle,
				draw=passivehistory,
				fill=passivehistory
			] at (5.82, -1) {};
			\node[
				anchor=south,
				passivehistory,
				font=\small\linespread{0.9}\selectfont,
				text height=9pt,
				text depth=3pt,
				inner sep=0pt,
				align=center
			] at (5.82, -0.87) {ANNs\\(1986)};

			\node[
				circle,
				draw=passivehistory,
				fill=passivehistory
			] at (7.10, -1) {};
			\node[
				anchor=north,
				passivehistory,
				font=\small\linespread{0.9}\selectfont,
				text height=9pt,
				text depth=3pt,
				inner sep=0pt,
				align=center
			] at (7.10, -1.372) {Deep blue\\(1997)};

			\node[
				circle,
				draw=activehistory,
				fill=activehistory
			] at (8.84, -1) {};
			\node[
				anchor=south,
				activehistory,
				font=\small\linespread{0.9}\selectfont,
				text height=9pt,
				text depth=3pt,
				inner sep=0pt,
				align=center
			] at (8.84, -0.87) {Deep learning\\(2012)};

			\node[inner sep=0pt] at (5.25, -4.75) {
				\usebox{\imagenetlatest}
			};
		\end{tikzpicture}
	\end{frame}

	\begin{frame}{The history of artificial intelligence} %DL (human)
		\newsavebox{\imagenethuman}
		\sbox{\imagenethuman}{%
			\begin{tikzpicture}
				\begin{axis}[
					ylabel={Error rate},
					xlabel={Year},
					xtick={2010, 2012, 2014, 2016, 2018, 2020},
					xticklabels={2010, 2012, 2014, 2016, 2018, 2020},
					ytick={0, 10, 20, 30},
					yticklabels={0\%, 10\%, 20\%, 30\%},
					ytick style={draw=none},
					ytick pos=left,
					xtick pos=bottom,
					ymajorgrids=true,
					ymax=29,
					ymin=0,
					xmin=2009,
					xmax=2021,
					width=8cm,
					height=5cm
				]
					\addplot[mark=*, purple,thick] coordinates {
						(2010, 28.2)
						(2011, 25.8)
						(2012, 16.4)
						(2013, 11.7)
						(2014, 7.3)
						(2015, 3.5)
						(2016, 3.0)
						(2017, 2.3)
						(2018, 1.8)
						(2019, 1.3)
						(2020, 0.9)
					};

					\node[anchor=north, inner sep=5pt] at (axis cs: 2010, 28.2) {
						\small{28.2}
					};
					\node[anchor=north, inner sep=5pt] at (axis cs: 2011, 25.8) {
						\small{25.8}
					};
					\node[anchor=north, inner sep=5pt] at (axis cs: 2012, 16.4) {
						\small{16.4}
					};
					\node[anchor=north, inner sep=5pt] at (axis cs: 2013, 11.7) {
						\small{11.7}
					};
					\node[anchor=north, inner sep=5pt] at (axis cs: 2014, 7.3) {
						\small{7.3}
					};
					\node[anchor=north, inner sep=5pt] at (axis cs: 2015, 3.5) {
						\small{3.5}
					};
					\node[anchor=south, inner sep=5pt] at (axis cs: 2016, 3.0) {
						\small{3.0}
					};
					\node[anchor=south, inner sep=5pt] at (axis cs: 2017, 2.3) {
						\small{2.3}
					};
					\node[anchor=south, inner sep=5pt] at (axis cs: 2018, 1.8) {
						\small{1.8}
					};
					\node[anchor=south, inner sep=5pt] at (axis cs: 2019, 1.3) {
						\small{1.3}
					};
					\node[anchor=south, inner sep=5pt] at (axis cs: 2020, 0.9) {
						\small{\textbf{0.9}}
					};

					\addplot[densely dotted] coordinates {
						(2011.5, 30)
						(2011.5, 0)
					};

					\addplot[dashed, red] coordinates {
						(2009, 5.1)
						(2021, 5.1)
					};

					\node[anchor=south east] at (axis cs: 2021, 5.1) {
						\textcolor{red}{\small{Human level}}
					};
				\end{axis}
			\end{tikzpicture}
		}

		\begin{tikzpicture}
			\node[] at (0, 0) {};
			\node[] at (10.5, -7.5) {};

			\draw[very thick, gray] (0.5, -1)  -- (8.84, -1) {};

			\node[
				circle,
				draw=passivehistory,
				fill=passivehistory
			] at (1.65, -1) {};
			\node[
				anchor=north,
				passivehistory,
				align=center,
				font=\small\linespread{0.9}\selectfont,
				text height=9pt,
				text depth=3pt,
				inner sep=0pt,
				align=center
			] at (1.65, -1.588) {Turing\\test\\(1950)};

			\node[
				circle,
				draw=passivehistory,
				fill=passivehistory,
			] at (2.35, -1) {};
			\node[
				anchor=south,
				passivehistory,
				font=\small\linespread{0.9}\selectfont,
				text height=9pt,
				text depth=3pt,
				inner sep=0pt,
				align=center
			] at (2.35, -0.87) {Dartmouth\\(1956)};

			\node[
				circle,
				draw=passivehistory,
				fill=passivehistory,
			] at (2.68, -1) {};
			\node[
				anchor=north,
				passivehistory,
				font=\small\linespread{0.9}\selectfont,
				text height=9pt,
				text depth=3pt,
				inner sep=0pt,
				align=center
			] at (2.68, -1.375) {Perceptron\\(1958)};

			\node[
				circle,
				draw=passivehistory,
				fill=passivehistory
			] at (3.28, -1) {};
			\node[
				circle,
				draw=passivehistory,
				fill=passivehistory,
			] at (2.35, -1) {};
			\node[
				anchor=south,
				passivehistory,
				font=\small\linespread{0.9}\selectfont,
				text height=9pt,
				text depth=3pt,
				inner sep=0pt,
				align=center
			] at (3.28, -0.87) {Eliza\\(1964)};

			\node[
				circle,
				draw=passivehistory,
				fill=passivehistory
			] at (5.13, -1) {};
			\node[
				anchor=north,
				passivehistory,
				font=\small\linespread{0.9}\selectfont,
				text height=9pt,
				text depth=3pt,
				inner sep=0pt,
				align=center
			] at (5.13, -1.625) {Expert\\systems\\(1980s)};

			\node[
				circle,
				draw=passivehistory,
				fill=passivehistory
			] at (5.82, -1) {};
			\node[
				anchor=south,
				passivehistory,
				font=\small\linespread{0.9}\selectfont,
				text height=9pt,
				text depth=3pt,
				inner sep=0pt,
				align=center
			] at (5.82, -0.87) {ANNs\\(1986)};

			\node[
				circle,
				draw=passivehistory,
				fill=passivehistory
			] at (7.10, -1) {};
			\node[
				anchor=north,
				passivehistory,
				font=\small\linespread{0.9}\selectfont,
				text height=9pt,
				text depth=3pt,
				inner sep=0pt,
				align=center
			] at (7.10, -1.372) {Deep blue\\(1997)};

			\node[
				circle,
				draw=activehistory,
				fill=activehistory
			] at (8.84, -1) {};
			\node[
				anchor=south,
				activehistory,
				font=\small\linespread{0.9}\selectfont,
				text height=9pt,
				text depth=3pt,
				inner sep=0pt,
				align=center
			] at (8.84, -0.87) {Deep learning\\(2012)};

			\node[inner sep=0pt] at (5.25, -4.75) {
				\usebox{\imagenethuman}
			};
		\end{tikzpicture}
	\end{frame}

	\begin{frame}{The history of artificial intelligence} %ChatGPT
		\begin{tikzpicture}
			\node[] at (0, 0) {};
			\node[] at (10.5, -7.5) {};

			\draw[very thick, gray] (0.5, -1)  -- (10, -1) {};

			\node[
				circle,
				draw=passivehistory,
				fill=passivehistory
			] at (1.65, -1) {};
			\node[
				anchor=north,
				passivehistory,
				align=center,
				font=\small\linespread{0.9}\selectfont,
				text height=9pt,
				text depth=3pt,
				inner sep=0pt,
				align=center
			] at (1.65, -1.588) {Turing\\test\\(1950)};

			\node[
				circle,
				draw=passivehistory,
				fill=passivehistory,
			] at (2.35, -1) {};
			\node[
				anchor=south,
				passivehistory,
				font=\small\linespread{0.9}\selectfont,
				text height=9pt,
				text depth=3pt,
				inner sep=0pt,
				align=center
			] at (2.35, -0.87) {Dartmouth\\(1956)};

			\node[
				circle,
				draw=passivehistory,
				fill=passivehistory,
			] at (2.68, -1) {};
			\node[
				anchor=north,
				passivehistory,
				font=\small\linespread{0.9}\selectfont,
				text height=9pt,
				text depth=3pt,
				inner sep=0pt,
				align=center
			] at (2.68, -1.375) {Perceptron\\(1958)};

			\node[
				circle,
				draw=passivehistory,
				fill=passivehistory
			] at (3.28, -1) {};
			\node[
				circle,
				draw=passivehistory,
				fill=passivehistory,
			] at (2.35, -1) {};
			\node[
				anchor=south,
				passivehistory,
				font=\small\linespread{0.9}\selectfont,
				text height=9pt,
				text depth=3pt,
				inner sep=0pt,
				align=center
			] at (3.28, -0.87) {Eliza\\(1964)};

			\node[
				circle,
				draw=passivehistory,
				fill=passivehistory
			] at (5.13, -1) {};
			\node[
				anchor=north,
				passivehistory,
				font=\small\linespread{0.9}\selectfont,
				text height=9pt,
				text depth=3pt,
				inner sep=0pt,
				align=center
			] at (5.13, -1.625) {Expert\\systems\\(1980s)};

			\node[
				circle,
				draw=passivehistory,
				fill=passivehistory
			] at (5.82, -1) {};
			\node[
				anchor=south,
				passivehistory,
				font=\small\linespread{0.9}\selectfont,
				text height=9pt,
				text depth=3pt,
				inner sep=0pt,
				align=center
			] at (5.82, -0.87) {ANNs\\(1986)};

			\node[
				circle,
				draw=passivehistory,
				fill=passivehistory
			] at (7.10, -1) {};
			\node[
				anchor=north,
				passivehistory,
				font=\small\linespread{0.9}\selectfont,
				text height=9pt,
				text depth=3pt,
				inner sep=0pt,
				align=center
			] at (7.10, -1.372) {Deep blue\\(1997)};

			\node[
				circle,
				draw=passivehistory,
				fill=passivehistory
			] at (8.84, -1) {};
			\node[
				anchor=south,
				passivehistory,
				font=\small\linespread{0.9}\selectfont,
				text height=9pt,
				text depth=3pt,
				inner sep=0pt,
				align=center
			] at (8.84, -0.87) {Deep learning\\(2012)};

			\node[
				circle,
				draw=activehistory,
				fill=activehistory
			] at (10, -1) {};
			\node[
				anchor=north,
				activehistory,
				font=\small\linespread{0.9}\selectfont,
				text height=9pt,
				text depth=3pt,
				inner sep=0pt,
				align=center
			] at (10, -1.33) {ChatGPT\\(2022)};

			\node[inner sep=0pt, draw=black] at (5.25, -4.75) {
				\includegraphics[width=4cm]{data/chatgpt.png}
			};
		\end{tikzpicture}
	\end{frame}

	\section{Terminology and concepts}

	\begin{frame}{Terminology and concepts: Taxonomy} % Taxonomy, Artificial intelligence
		\centering
		\vfill
		\begin{tikzpicture}
			\node[circle, fill=blue!60, minimum size=6cm] (ai) at (0, 0) {};
			\node[text=white, anchor=north] at ($ (ai.north) - (0, 0.3) $) {\textbf{Artificial intelligence}};
			\node[anchor=north west, align=left, font=\small] (ai-text) at ($ (ai.north) + (3.5, 0.2) $) {\textbf{Artificial intelligence:}\\Machines that solve tasks\\requiring intelligence};
			\node[] at (-3, 3) {};
			\node[] at (7.7, -3.2) {};
		\end{tikzpicture}
		\vfill
	\end{frame}

	\begin{frame}{Terminology and concepts: Taxonomy} % Taxonomy, Symbolic AI
		\centering
		\vfill
		\begin{tikzpicture}
			\node[circle, fill=blue!60, minimum size=6cm] (ai) at (0, 0) {};
			\node[text=white, anchor=north] at ($ (ai.north) - (0, 0.3) $) {\textbf{Artificial intelligence}};
			\node[text=white] at ($ (ai.north) - (-1.1, 1.2) $) {Symbolic AI};
			\node[anchor=north west, align=left, font=\small, text=gray!40] (ai-text) at ($ (ai.north) + (3.5, 0.2) $) {\textbf{Artificial intelligence:}\\Machines that solve tasks\\requiring intelligence};

			\node[] at (-3, 3) {};
			\node[] at (7.7, -3.2) {};
		\end{tikzpicture}
		\vfill
	\end{frame}

	\begin{frame}{Terminology and concepts: Taxonomy} % Taxonomy, Machine learning explanation
		\centering
		\vfill
		\begin{tikzpicture}
			\node[circle, fill=blue!60, minimum size=6cm] (ai) at (0, 0) {};
			\node[text=white, anchor=north] at ($ (ai.north) - (0, 0.3) $) {\textbf{Artificial intelligence}};
			\node[text=white] at ($ (ai.north) - (-1.1, 1.2) $) {Symbolic AI};
			\node[circle, fill=purple!60, minimum size=4.5cm, anchor=south] (ml) at ($ (ai.south) + (0, 0.05) $) {};
			\node[text=white, anchor=north] at ($ (ml.north) - (0, 0.3) $) {\textbf{Machine learning}};
			\node[anchor=north west, align=left, font=\small, text=gray!40] (ai-text) at ($ (ai.north) + (3.5, 0.2) $) {\textbf{Artificial intelligence:}\\Machines that solve tasks\\requiring intelligence};
			\node[anchor=north west, align=left, font=\small] (ml-text) at ($ (ai-text.south west) - (0, 0) $) {\textbf{Machine learning:}\\Machines that learn to\\solve tasks through\\learning patterns from data};
			\node[] at (-3, 3) {};
			\node[] at (7.7, -3.2) {};
		\end{tikzpicture}
		\vfill
	\end{frame}

	\begin{frame}{Terminology and concepts: Taxonomy} % Taxonomy, Deep learning
		\centering
		\vfill
		\begin{tikzpicture}
			\node[circle, fill=blue!60, minimum size=6cm] (ai) at (0, 0) {};
			\node[text=white, anchor=north] at ($ (ai.north) - (0, 0.3) $) {\textbf{Artificial intelligence}};
			\node[text=white] at ($ (ai.north) - (-1.1, 1.2) $) {Symbolic AI};
			\node[circle, fill=purple!60, minimum size=4.5cm, anchor=south] (ml) at ($ (ai.south) + (0, 0.05) $) {};
			\node[text=white, anchor=north] at ($ (ml.north) - (0, 0.3) $) {\textbf{Machine learning}};
			\node[text=white, align=center, font=\linespread{0.5}\selectfont] at ($ (ml.north) - (1, 1.1) $) {Linear\\regression};
			\node[anchor=north west, align=left, font=\small, text=gray!40] (ai-text) at ($ (ai.north) + (3.5, 0.2) $) {\textbf{Artificial intelligence:}\\Machines that solve tasks\\requiring intelligence};
			\node[anchor=north west, align=left, font=\small, text=gray!40] (ml-text) at ($ (ai-text.south west) - (0, 0) $) {\textbf{Machine learning:}\\Machines that learn to\\solve tasks through\\learning patterns from data};

			\node[] at (-3, 3) {};
			\node[] at (7.7, -3.2) {};
		\end{tikzpicture}
		\vfill
	\end{frame}

	\begin{frame}{Terminology and concepts: Taxonomy} % Taxonomy, Deep learning
		\centering
		\vfill
		\begin{tikzpicture}
			\node[circle, fill=blue!60, minimum size=6cm] (ai) at (0, 0) {};
			\node[text=white, anchor=north] at ($ (ai.north) - (0, 0.3) $) {\textbf{Artificial intelligence}};
			\node[text=white] at ($ (ai.north) - (-1.1, 1.2) $) {Symbolic AI};
			\node[circle, fill=purple!60, minimum size=4.5cm, anchor=south] (ml) at ($ (ai.south) + (0, 0.05) $) {};
			\node[text=white, anchor=north] at ($ (ml.north) - (0, 0.3) $) {\textbf{Machine learning}};
			\node[text=white, align=center, font=\linespread{0.5}\selectfont] at ($ (ml.north) - (1, 1.1) $) {Linear\\regression};
			\node[circle, fill=red!60, minimum size=3cm, anchor=south] (dl) at ($ (ai.south) + (0, 0.1) $) {};
			\node[text=white, anchor=north] at ($ (dl.north) - (0, 0.3) $) {\textbf{Deep learning}};
			\node[anchor=north west, align=left, font=\small, text=gray!40] (ai-text) at ($ (ai.north) + (3.5, 0.2) $) {\textbf{Artificial intelligence:}\\Machines that solve tasks\\requiring intelligence};
			\node[anchor=north west, align=left, font=\small, text=gray!40] (ml-text) at ($ (ai-text.south west) - (0, 0) $) {\textbf{Machine learning:}\\Machines that learn to\\solve tasks through\\learning patterns from data};
			\node[anchor=north west, align=left, font=\small] (dl-text) at ($ (ml-text.south west) - (0, 0) $) {\textbf{Deep learning:}\\Machine learning models\\ organized in hierarchies\\($\approx$ deep neural networks)\\inspired by the brain};
			\node[] at (-3, 3) {};
			\node[] at (7.7, -3.2) {};
		\end{tikzpicture}
		\vfill
	\end{frame}

	\begin{frame}{Terminology and concepts: Taxonomy} % Taxonomy, CNNs and LLMs
		\centering
		\vfill
		\begin{tikzpicture}
			\node[circle, fill=blue!60, minimum size=6cm] (ai) at (0, 0) {};
			\node[text=white, anchor=north] at ($ (ai.north) - (0, 0.3) $) {\textbf{Artificial intelligence}};
			\node[text=white] at ($ (ai.north) - (-1.1, 1.2) $) {Symbolic AI};
			\node[circle, fill=purple!60, minimum size=4.5cm, anchor=south] (ml) at ($ (ai.south) + (0, 0.05) $) {};
			\node[text=white, anchor=north] at ($ (ml.north) - (0, 0.3) $) {\textbf{Machine learning}};
			\node[text=white, align=center, font=\linespread{0.5}\selectfont] at ($ (ml.north) - (1, 1.1) $) {Linear\\regression};
			\node[circle, fill=red!60, minimum size=3cm, anchor=south] (dl) at ($ (ai.south) + (0, 0.1) $) {};
			\node[text=white, anchor=north] at ($ (dl.north) - (0, 0.3) $) {\textbf{Deep learning}};
			\node[align=center, text=white, font=\linespread{0.5}\selectfont] at ($ (dl.north) - (-0.2, 1.2) $) {Convolutional\\neural networks};
			\node[align=center, text=white, font=\linespread{0.5}\selectfont] at ($ (dl.north) - (0.2, 2.1) $) {Large language\\models};
			\node[anchor=north west, align=left, font=\small, text=gray!40] (ai-text) at ($ (ai.north) + (3.5, 0.2) $) {\textbf{Artificial intelligence:}\\Machines that solve tasks\\requiring intelligence};
			\node[anchor=north west, align=left, font=\small, text=gray!40] (ml-text) at ($ (ai-text.south west) - (0, 0) $) {\textbf{Machine learning:}\\Machines that learn to\\solve tasks through\\learning patterns from data};
			\node[anchor=north west, align=left, font=\small, text=gray!40] (dl-text) at ($ (ml-text.south west) - (0, 0) $) {\textbf{Deep learning:}\\Machine learning models\\ organized in hierarchies\\($\approx$ deep neural networks)\\inspired by the brain};
			\node[anchor=north west, align=left, font=\small] (cnn-text) at ($ (dl-text.south west) - (0, 0) $) {\textbf{Convolutional neural nets:}\\Neural networks for image\\data};
			\node[anchor=north west, align=left, font=\small] at ($ (cnn-text.south west) - (0, 0) $) {\textbf{Large language models:}\\Neural networks for natural\\language (ChatGPT)};
			\node[] at (-3, 3) {};
			\node[] at (7.7, -3.2) {};
		\end{tikzpicture}
		\vfill
	\end{frame}

	\begin{frame}{Terminology: Supervision} % Supervised vs unsupervised
		\centering
		\vfill
		\begin{tikzpicture}
			\node[align=center, anchor=north] (supervised) at (0, 0) {Supervised learning};
			\node[align=center, anchor=north] (unsupervised) at (7, 0) {Unsupervised learning};
			\draw[] (3.5, 0) -- (3.5, -7.5);
			\node[] (cat1) at ($ (supervised.south) + (-1, -0.8) $) {
				\includegraphics[width=1.2cm]{data/cat.1.jpg}
			};
			\node[anchor=west] (cattext1) at ($ (cat1.east) + (1.2, 0) $) {Cat};
			\draw[->] (cat1) -- (cattext1);
			\node[anchor=north] (dog1) at ($ (cat1.south) + (0, -0.1) $) {
				\includegraphics[width=1.2cm]{data/dog.0.jpg}
			};
			\node[anchor=west] (dogtext1) at ($ (dog1.east) + (1.2, 0) $) {Dog};
			\draw[->] (dog1) -- (dogtext1);
			\node[anchor=north] (cat2) at ($ (dog1.south) + (0, -0.1) $) {
				\includegraphics[width=1.2cm]{data/cat.2.jpg}
			};
			\node[anchor=west] (cattext2) at ($ (cat2.east) + (1.2, 0) $) {Cat};
			\draw[->] (cat2) -- (cattext2);
			\node[anchor=north] (dog2) at ($ (cat2.south) + (0, -0.1) $) {
				\includegraphics[width=1.2cm]{data/dog.1.jpg}
			};
			\node[anchor=west] (dogtext2) at ($ (dog2.east) + (1.2, 0) $) {Dog};
			\draw[->] (dog2) -- (dogtext2);

			\node[] (cat1) at ($ (unsupervised.south) + (-1.2, -2.4) $) {
				\includegraphics[width=0.8cm]{data/cat.1.jpg}
			};
			\node[] (cat2) at ($ (cat1) + (-0.9, 0.2) $) {
				\includegraphics[width=0.8cm]{data/cat.2.jpg}
			};
			\node[] (cat3) at ($ (cat1) + (-0.5, -0.8) $) {
				\includegraphics[width=0.8cm]{data/cat.3.jpg}
			};
			\node[] (cat4) at ($ (cat1) + (0.9, -0.1) $) {
				\includegraphics[width=0.8cm]{data/cat.4.jpg}
			};

			\node[] (dog1) at ($ (cat1) + (1.8, -1.5) $) {
				\includegraphics[width=0.8cm]{data/dog.0.jpg}
			};
			\node[] (dog2) at ($ (dog1) + (0.9, 0.1) $) {
				\includegraphics[width=0.8cm]{data/dog.1.jpg}
			};
			\node[] (dog3) at ($ (dog1) + (0.3, -0.8) $) {
				\includegraphics[width=0.8cm]{data/dog.3.jpg}
			};
			\node[] (dog4) at ($ (dog1) + (-0.9, -0.3) $) {
				\includegraphics[width=0.8cm]{data/dog.4.jpg}
			};
			\draw[dashed, thick, red] ($ (cat1) + (-0.9, -1.7) $) -- ($ (dog1) + (1.2, 1.7) $);
		\end{tikzpicture}
		\vfill
	\end{frame}

	\begin{frame}{Terminology: Strong and weak AI} % Strong vs weak, axis
		\centering
		\vfill
		\begin{tikzpicture}
			\draw[<->] (0, 0) -- (10, 0);
			\node[anchor=north west] at (0, -0.1) {More specific};
			\node[anchor=north east] at (10, -0.1) {More general};
			\node[anchor=north west] at (0, 6) {Narrow (weak)};
			\node[anchor=north east] at (10, 6) {General (strong)};

			\draw[red, ->, thick] (2, 4) -- (8, 4);
			\node[anchor=south,align=center,text=red] at (5, 4.1) {Able to solve a broader spectrum of\\problems in a wider array of domains};

			\node[] at (10.2, -0.4) {};
			\node[] at (-0.2, 5.9) {};
		\end{tikzpicture}
		\vfill
	\end{frame}

	\begin{frame}{Terminology: Strong and weak AI} % Strong vs weak, dichotomi
		\centering
		\vfill
		\begin{tikzpicture}
			\draw[<->] (0, 0) -- (10, 0);
			\node[anchor=north west] at (0, -0.1) {More specific};
			\node[anchor=north east] at (10, -0.1) {More general};
			\node[anchor=north west] at (0, 6) {Narrow (weak)};
			\node[anchor=north east] at (10, 6) {General (strong)};

			\draw[red, ->, thick] (2, 4) -- (8, 4);
			\node[anchor=south,align=center,text=red] at (5, 4.1) {Able to solve a broader spectrum of\\problems in a wider array of domains};
			\node[anchor=south east] at (9.5, 0.2) {
				\includegraphics[width=1.3cm]{data/human.png}
			};
			\node[anchor=south west] at (0.5, 0.2) {
				\includegraphics[width=1.3cm]{data/laptop.png}
			};
			\node[] at (10.2, -0.4) {};
			\node[] at (-0.2, 5.9) {};
		\end{tikzpicture}
		\vfill
	\end{frame}

	\begin{frame}{Terminology: Strong and weak AI} % Strong vs weak, weak
		\centering
		\vfill
		\begin{tikzpicture}
			\draw[<->] (0, 0) -- (10, 0);
			\node[anchor=north west] at (0, -0.1) {More specific};
			\node[anchor=north east] at (10, -0.1) {More general};
			\node[anchor=north west] at (0, 6) {Narrow (weak)};
			\node[anchor=north east] at (10, 6) {General (strong)};

			\draw[red, ->, thick] (2, 4) -- (8, 4);
			\node[anchor=south,align=center,text=red] at (5, 4.1) {Able to solve a broader spectrum of\\problems in a wider array of domains};
			\node[anchor=south east] at (9.5, 0.2) {
				\includegraphics[width=1.3cm]{data/human.png}
			};
			\node[anchor=south west] at (0.5, 0.2) {
				\includegraphics[width=1.3cm]{data/laptop.png}
			};
			\node[align=center,font=\small, fill=blue!60, text=white, minimum width=1.6cm, rounded corners=.1cm] at (3, 2.1) {
				Image\\
				diagnostics
			};
			\node[align=center,font=\small, fill=blue!60, text=white, minimum width=1.6cm, rounded corners=.1cm] at (3, 1.3) {
				Insurance\\
				pricing
			};
			\node[align=center,font=\small, fill=blue!60, text=white, minimum width=1.6cm, rounded corners=.1cm] at (3, 0.5) {
				Document\\
				reading
			};
			\node[] at (10.2, -0.4) {};
			\node[] at (-0.2, 5.9) {};
		\end{tikzpicture}
		\vfill
	\end{frame}

	\begin{frame}{Terminology: Strong and weak AI} % Strong vs weak, stronger
		\centering
		\vfill
		\begin{tikzpicture}
			\draw[<->] (0, 0) -- (10, 0);
			\node[anchor=north west] at (0, -0.1) {More specific};
			\node[anchor=north east] at (10, -0.1) {More general};
			\node[anchor=north west] at (0, 6) {Narrow (weak)};
			\node[anchor=north east] at (10, 6) {General (strong)};

			\draw[red, ->, thick] (2, 4) -- (8, 4);
			\node[anchor=south,align=center,text=red] at (5, 4.1) {Able to solve a broader spectrum of\\problems in a wider array of domains};
			\node[anchor=south east] at (9.5, 0.2) {
				\includegraphics[width=1.3cm]{data/human.png}
			};
			\node[anchor=south west] at (0.5, 0.2) {
				\includegraphics[width=1.3cm]{data/laptop.png}
			};
			\node[align=center,font=\small, fill=blue!60, text=white, minimum width=1.6cm, rounded corners=.1cm] at (3, 2.1) {
				Image\\
				diagnostics
			};
			\node[align=center,font=\small, fill=blue!60, text=white, minimum width=1.6cm, rounded corners=.1cm] at (3, 1.3) {
				Insurance\\
				pricing
			};
			\node[align=center,font=\small, fill=blue!60, text=white, minimum width=1.6cm, rounded corners=.1cm] at (3, 0.5) {
				Document\\
				reading
			};
			\node[align=center,font=\small, fill=blue!60, text=white, minimum width=1.6cm, rounded corners=.1cm] at (4.8, 0.5) {
				Tesla
			};
			\node[align=center,font=\small, fill=blue!60, text=white, minimum width=1.6cm, rounded corners=.1cm] at (5.4, 1.1) {
				ChatGPT
			};
			\node[] at (10.2, -0.4) {};
			\node[] at (-0.2, 5.9) {};
		\end{tikzpicture}
		\vfill
	\end{frame}

	\section{How does AI make decisions?}

	\begin{frame}{Decision making: Expert systems vs. machine learning} % Expert system
		\begin{tikzpicture}
			\node[draw=black, dashed] (in) at (-4, -0.75) {Laboratory report};

			\node[draw=black, fill=background] (n00) at (0, 0) {
				gram stain = gramneg
			};
			\node[draw=black, fill=background] (n01) at (0, -0.75) {
				morphology = rod
			};
			\node[draw=black, fill=background] (n02) at (0, -1.5) {
				aerobicity = anaerobic
			};

			\node[] (out) at (4, -0.75) {bacteroides};

			\draw[-Latex] (in.east) -- (n00.west);
			\draw[-Latex] (in.east) -- (n01.west);
			\draw[-Latex] (in.east) -- (n02.west);
			\draw[-Latex] (n00.east) -- (out.west);
			\draw[-Latex] (n01.east) -- (out.west);
			\draw[-Latex] (n02.east) -- (out.west);

			\node[anchor=south west] at (-5.3, -2.2) {Expert system};

			\node[] at (-5.3, 0.75) {};
			\node[] at (5.3, -5.5) {};

		\end{tikzpicture}
	\end{frame}

	\begin{frame}{Decision making: Expert systems vs. machine learning} % Black box
		\begin{tikzpicture}
			\node[draw=black, dashed] (in) at (-4, -0.75) {Laboratory report};

			\node[draw=black, fill=background] (n00) at (0, 0) {
				gram stain = gramneg
			};
			\node[draw=black, fill=background] (n01) at (0, -0.75) {
				morphology = rod
			};
			\node[draw=black, fill=background] (n02) at (0, -1.5) {
				aerobicity = anaerobic
			};

			\node[] (out) at (4, -0.75) {bacteroides};

			\draw[-Latex] (in.east) -- (n00.west);
			\draw[-Latex] (in.east) -- (n01.west);
			\draw[-Latex] (in.east) -- (n02.west);
			\draw[-Latex] (n00.east) -- (out.west);
			\draw[-Latex] (n01.east) -- (out.west);
			\draw[-Latex] (n02.east) -- (out.west);

			\draw[fill=background] (-1.85, -2.65) rectangle (1.85, -5.35);

			\node[draw=black, dashed] (in) at (-4, -4) {Laboratory report};

			\node[] (out) at (4, -4) {bacteroides};

			\draw[-Latex] (in.east) -- (-1.85, -4);

			\draw[-Latex] (1.85, -4) -- (out);

			\draw[densely dotted] (-5.3, -2.2) -- (5.3, -2.2);
			\node[anchor=south west] at (-5.3, -2.2) {Expert system};
			\node[anchor=north west] at (-5.3, -2.2) {Machine learning};

			\node[] at (-5.3, 0.75) {};
			\node[] at (5.3, -5.5) {};

		\end{tikzpicture}
	\end{frame}

	\begin{frame}{Decision making: Expert systems vs. machine learning} % Comparison
		\begin{tikzpicture}
			\node[draw=black, dashed] (in) at (-4, -0.75) {Laboratory report};

			\node[draw=black, fill=background] (n00) at (0, 0) {
				gram stain = gramneg
			};
			\node[draw=black, fill=background] (n01) at (0, -0.75) {
				morphology = rod
			};
			\node[draw=black, fill=background] (n02) at (0, -1.5) {
				aerobicity = anaerobic
			};

			\node[] (out) at (4, -0.75) {bacteroides};

			\draw[-Latex] (in.east) -- (n00.west);
			\draw[-Latex] (in.east) -- (n01.west);
			\draw[-Latex] (in.east) -- (n02.west);
			\draw[-Latex] (n00.east) -- (out.west);
			\draw[-Latex] (n01.east) -- (out.west);
			\draw[-Latex] (n02.east) -- (out.west);

			\draw[fill=background] (-1.85, -2.65) rectangle (1.85, -5.35);
			\node[anchor=north east] at (1.85, -2.65) {\small{Neural network}};

			\node[draw=black, dashed] (in) at (-4, -4) {Laboratory report};

			\node[draw=black, fill=nodefill, circle, inner sep=4pt] (n00) at (-1.5, -3) {};
			\node[draw=black, fill=nodefill, circle, inner sep=4pt] (n01) at (-1.5, -3.5) {};
			\node[draw=black, fill=nodefill, circle, inner sep=4pt] (n02) at (-1.5, -4) {};
			\node[draw=black, fill=nodefill, circle, inner sep=4pt] (n03) at (-1.5, -4.5) {};
			\node[draw=black, fill=nodefill, circle, inner sep=4pt] (n04) at (-1.5, -5) {};

			\node[draw=black, fill=nodefill, circle, inner sep=4pt] (n10) at (-0.75, -3.25) {};
			\node[draw=black, fill=nodefill, circle, inner sep=4pt] (n11) at (-0.75, -3.75) {};
			\node[draw=black, fill=nodefill, circle, inner sep=4pt] (n12) at (-0.75, -4.25) {};
			\node[draw=black, fill=nodefill, circle, inner sep=4pt] (n13) at (-0.75, -4.75) {};

			\node[draw=black, fill=nodefill, circle, inner sep=4pt] (n20) at (0, -3.5) {};
			\node[draw=black, fill=nodefill, circle, inner sep=4pt] (n21) at (0, -4) {};
			\node[draw=black, fill=nodefill, circle, inner sep=4pt] (n22) at (0, -4.5) {};

			\node[draw=black, fill=nodefill, circle, inner sep=4pt] (n30) at (0.75, -3.75) {};
			\node[draw=black, fill=nodefill, circle, inner sep=4pt] (n31) at (0.75, -4.25) {};

			\node[draw=black, fill=nodefill, circle, inner sep=4pt] (n40) at (1.5, -4) {};

			\node[] (out) at (4, -4) {bacteroides};

			\draw[-Latex] (in.east) -- (n00);
			\draw[-Latex] (in.east) -- (n01);
			\draw[-Latex] (in.east) -- (n02);
			\draw[-Latex] (in.east) -- (n03);
			\draw[-Latex] (in.east) -- (n04);

			\draw[] (n00) -- (n10);
			\draw[] (n00) -- (n11);
			\draw[] (n00) -- (n12);
			\draw[] (n00) -- (n13);
			\draw[] (n01) -- (n10);
			\draw[] (n01) -- (n11);
			\draw[] (n01) -- (n12);
			\draw[] (n01) -- (n13);
			\draw[] (n02) -- (n10);
			\draw[] (n02) -- (n11);
			\draw[] (n02) -- (n12);
			\draw[] (n02) -- (n13);
			\draw[] (n03) -- (n10);
			\draw[] (n03) -- (n11);
			\draw[] (n03) -- (n12);
			\draw[] (n03) -- (n13);
			\draw[] (n04) -- (n10);
			\draw[] (n04) -- (n11);
			\draw[] (n04) -- (n12);
			\draw[] (n04) -- (n13);

			\draw[] (n10) -- (n20);
			\draw[] (n10) -- (n21);
			\draw[] (n10) -- (n22);
			\draw[] (n11) -- (n20);
			\draw[] (n11) -- (n21);
			\draw[] (n11) -- (n22);
			\draw[] (n12) -- (n20);
			\draw[] (n12) -- (n21);
			\draw[] (n12) -- (n22);
			\draw[] (n13) -- (n20);
			\draw[] (n13) -- (n21);
			\draw[] (n13) -- (n22);

			\draw[] (n20) -- (n30);
			\draw[] (n20) -- (n31);
			\draw[] (n21) -- (n30);
			\draw[] (n21) -- (n31);
			\draw[] (n22) -- (n30);
			\draw[] (n22) -- (n31);

			\draw[] (n30) -- (n40);
			\draw[] (n31) -- (n40);

			\draw[-Latex] (n40) -- (out);

			\draw[densely dotted] (-5.3, -2.2) -- (5.3, -2.2);
			\node[anchor=south west] at (-5.3, -2.2) {Expert system};
			\node[anchor=north west] at (-5.3, -2.2) {Machine learning};

			\node[] at (-5.3, 0.75) {};
			\node[] at (5.3, -5.5) {};

		\end{tikzpicture}
	\end{frame}

	\begin{frame}[t]{Decision making: Loss functions} % Loss definition
		\vspace{2cm}
		A loss function formalizes what we want the machine learning model to do:\\
	\end{frame}

	\begin{frame}[t]{Decision making: Loss functions} % Classification
		\vspace{2cm}
		A loss function formalizes what we want the machine learning model to do:\\
		\begin{itemize}
			\item \textbf{\underline{Classification}}\\
			What category does the input belong to?\\
		\end{itemize}
	\end{frame}

	\begin{frame}[t]{Decision making: Loss functions} % Probabilities
		\vspace{2cm}
		A loss function formalizes what we want the machine learning model to do:\\
		\begin{itemize}
			\item \textbf{\underline{Classification}}\\
			What category does the input belong to?\\
			\rightarrow \hspace{0.2cm} What is the probability that input is a cat/dog/giraffe?
		\end{itemize}
	\end{frame}

	\begin{frame}[t]{Decision making: Loss functions} % Formula
		\vspace{2cm}
		A loss function formalizes what we want the machine learning model to do:\\
		\begin{itemize}
			\item \textbf{\underline{Classification}}\\
			What category does the input belong to?\\
			\rightarrow \hspace{0.2cm} What is the probability that input is a cat/dog/giraffe?\\
			\rightarrow \hspace{0.2cm} $-\dfrac{1}{N}\sum\limits_{i=0}^N \left[ y_i \log \hat{y}_i + (1 - y_i) \log (1 - \hat{y}_i) \right]$\\[0.2cm]
			where $y_i$ is the correct label and $\hat{y}_i$ is the predicted probability.
		\end{itemize}
	\end{frame}

	\begin{frame}[t]{Decision making: Loss functions} % Regression
		\vspace{2cm}
		A loss function formalizes what we want the machine learning model to do:\\
		\begin{itemize}
			\item \textbf{\underline{Classification}}\\
			What category does the input belong to?\\
			\rightarrow \hspace{0.2cm} What is the probability that input is a cat/dog/giraffe?\\
			\rightarrow \hspace{0.2cm} $-\dfrac{1}{N}\sum\limits_{i=0}^N \left[ y_i \log \hat{y}_i + (1 - y_i) \log (1 - \hat{y}_i) \right]$\\[0.2cm]
			where $y_i$ is the correct label and $\hat{y}_i$ is the predicted probability.
			\item \textbf{\underline{Regression}}\\
			What is the correct (continuous) output for the input?\\
			\rightarrow \hspace{0.2cm} $\left(y - \hat{y}\right)^2$.
		\end{itemize}
	\end{frame}

	\begin{frame}{Decision making: Learning} % Model
		\centering
		\begin{tikzpicture}

			\node[draw=black, fill=nodefill, circle, inner sep=4pt] (n00) at (-1.5, -3) {};
			\node[draw=black, fill=nodefill, circle, inner sep=4pt] (n01) at (-1.5, -3.5) {};
			\node[draw=black, fill=nodefill, circle, inner sep=4pt] (n02) at (-1.5, -4) {};
			\node[draw=black, fill=nodefill, circle, inner sep=4pt] (n03) at (-1.5, -4.5) {};
			\node[draw=black, fill=nodefill, circle, inner sep=4pt] (n04) at (-1.5, -5) {};

			\node[draw=black, fill=nodefill, circle, inner sep=4pt] (n10) at (-0.75, -3.25) {};
			\node[draw=black, fill=nodefill, circle, inner sep=4pt] (n11) at (-0.75, -3.75) {};
			\node[draw=black, fill=nodefill, circle, inner sep=4pt] (n12) at (-0.75, -4.25) {};
			\node[draw=black, fill=nodefill, circle, inner sep=4pt] (n13) at (-0.75, -4.75) {};

			\node[draw=black, fill=nodefill, circle, inner sep=4pt] (n20) at (0, -3.5) {};
			\node[draw=black, fill=nodefill, circle, inner sep=4pt] (n21) at (0, -4) {};
			\node[draw=black, fill=nodefill, circle, inner sep=4pt] (n22) at (0, -4.5) {};

			\node[draw=black, fill=nodefill, circle, inner sep=4pt] (n30) at (0.75, -3.75) {};
			\node[draw=black, fill=nodefill, circle, inner sep=4pt] (n31) at (0.75, -4.25) {};

			\node[draw=black, fill=nodefill, circle, inner sep=4pt] (n40) at (1.5, -4) {};

			\draw[] (n00) -- (n10);
			\draw[] (n00) -- (n11);
			\draw[] (n00) -- (n12);
			\draw[] (n00) -- (n13);
			\draw[] (n01) -- (n10);
			\draw[] (n01) -- (n11);
			\draw[] (n01) -- (n12);
			\draw[] (n01) -- (n13);
			\draw[] (n02) -- (n10);
			\draw[] (n02) -- (n11);
			\draw[] (n02) -- (n12);
			\draw[] (n02) -- (n13);
			\draw[] (n03) -- (n10);
			\draw[] (n03) -- (n11);
			\draw[] (n03) -- (n12);
			\draw[] (n03) -- (n13);
			\draw[] (n04) -- (n10);
			\draw[] (n04) -- (n11);
			\draw[] (n04) -- (n12);
			\draw[] (n04) -- (n13);

			\draw[] (n10) -- (n20);
			\draw[] (n10) -- (n21);
			\draw[] (n10) -- (n22);
			\draw[] (n11) -- (n20);
			\draw[] (n11) -- (n21);
			\draw[] (n11) -- (n22);
			\draw[] (n12) -- (n20);
			\draw[] (n12) -- (n21);
			\draw[] (n12) -- (n22);
			\draw[] (n13) -- (n20);
			\draw[] (n13) -- (n21);
			\draw[] (n13) -- (n22);

			\draw[] (n20) -- (n30);
			\draw[] (n20) -- (n31);
			\draw[] (n21) -- (n30);
			\draw[] (n21) -- (n31);
			\draw[] (n22) -- (n30);
			\draw[] (n22) -- (n31);

			\draw[] (n30) -- (n40);
			\draw[] (n31) -- (n40);

			\node[] at (-5, -3) {};
			\node[] at (3.75, -6.2) {};
		\end{tikzpicture}
	\end{frame}

	\begin{frame}{Decision making: Learning} % Input
		\centering
		\begin{tikzpicture}
			\node[draw=black, inner sep=0pt, label=below:cat] (in) at (-4, -4) {
				\includegraphics[width=2cm]{data/cat.jpeg}
			};

			\node[draw=black, fill=nodefill, circle, inner sep=4pt] (n00) at (-1.5, -3) {};
			\node[draw=black, fill=nodefill, circle, inner sep=4pt] (n01) at (-1.5, -3.5) {};
			\node[draw=black, fill=nodefill, circle, inner sep=4pt] (n02) at (-1.5, -4) {};
			\node[draw=black, fill=nodefill, circle, inner sep=4pt] (n03) at (-1.5, -4.5) {};
			\node[draw=black, fill=nodefill, circle, inner sep=4pt] (n04) at (-1.5, -5) {};

			\node[draw=black, fill=nodefill, circle, inner sep=4pt] (n10) at (-0.75, -3.25) {};
			\node[draw=black, fill=nodefill, circle, inner sep=4pt] (n11) at (-0.75, -3.75) {};
			\node[draw=black, fill=nodefill, circle, inner sep=4pt] (n12) at (-0.75, -4.25) {};
			\node[draw=black, fill=nodefill, circle, inner sep=4pt] (n13) at (-0.75, -4.75) {};

			\node[draw=black, fill=nodefill, circle, inner sep=4pt] (n20) at (0, -3.5) {};
			\node[draw=black, fill=nodefill, circle, inner sep=4pt] (n21) at (0, -4) {};
			\node[draw=black, fill=nodefill, circle, inner sep=4pt] (n22) at (0, -4.5) {};

			\node[draw=black, fill=nodefill, circle, inner sep=4pt] (n30) at (0.75, -3.75) {};
			\node[draw=black, fill=nodefill, circle, inner sep=4pt] (n31) at (0.75, -4.25) {};

			\node[draw=black, fill=nodefill, circle, inner sep=4pt] (n40) at (1.5, -4) {};

			\draw[-Latex] (in.east) -- (n00);
			\draw[-Latex] (in.east) -- (n01);
			\draw[-Latex] (in.east) -- (n02);
			\draw[-Latex] (in.east) -- (n03);
			\draw[-Latex] (in.east) -- (n04);

			\draw[] (n00) -- (n10);
			\draw[] (n00) -- (n11);
			\draw[] (n00) -- (n12);
			\draw[] (n00) -- (n13);
			\draw[] (n01) -- (n10);
			\draw[] (n01) -- (n11);
			\draw[] (n01) -- (n12);
			\draw[] (n01) -- (n13);
			\draw[] (n02) -- (n10);
			\draw[] (n02) -- (n11);
			\draw[] (n02) -- (n12);
			\draw[] (n02) -- (n13);
			\draw[] (n03) -- (n10);
			\draw[] (n03) -- (n11);
			\draw[] (n03) -- (n12);
			\draw[] (n03) -- (n13);
			\draw[] (n04) -- (n10);
			\draw[] (n04) -- (n11);
			\draw[] (n04) -- (n12);
			\draw[] (n04) -- (n13);

			\draw[] (n10) -- (n20);
			\draw[] (n10) -- (n21);
			\draw[] (n10) -- (n22);
			\draw[] (n11) -- (n20);
			\draw[] (n11) -- (n21);
			\draw[] (n11) -- (n22);
			\draw[] (n12) -- (n20);
			\draw[] (n12) -- (n21);
			\draw[] (n12) -- (n22);
			\draw[] (n13) -- (n20);
			\draw[] (n13) -- (n21);
			\draw[] (n13) -- (n22);

			\draw[] (n20) -- (n30);
			\draw[] (n20) -- (n31);
			\draw[] (n21) -- (n30);
			\draw[] (n21) -- (n31);
			\draw[] (n22) -- (n30);
			\draw[] (n22) -- (n31);

			\draw[] (n30) -- (n40);
			\draw[] (n31) -- (n40);

			\node[] at (-5, -3) {};
			\node[] at (3.75, -6.2) {};
		\end{tikzpicture}
	\end{frame}

	\begin{frame}{Decision making: Learning} % Prediction
		\centering
		\begin{tikzpicture}
			\node[draw=black, inner sep=0pt, label=below:cat] (in) at (-4, -4) {
				\includegraphics[width=2cm]{data/cat.jpeg}
			};

			\node[draw=black, fill=nodefill, circle, inner sep=4pt] (n00) at (-1.5, -3) {};
			\node[draw=black, fill=nodefill, circle, inner sep=4pt] (n01) at (-1.5, -3.5) {};
			\node[draw=black, fill=nodefill, circle, inner sep=4pt] (n02) at (-1.5, -4) {};
			\node[draw=black, fill=nodefill, circle, inner sep=4pt] (n03) at (-1.5, -4.5) {};
			\node[draw=black, fill=nodefill, circle, inner sep=4pt] (n04) at (-1.5, -5) {};

			\node[draw=black, fill=nodefill, circle, inner sep=4pt] (n10) at (-0.75, -3.25) {};
			\node[draw=black, fill=nodefill, circle, inner sep=4pt] (n11) at (-0.75, -3.75) {};
			\node[draw=black, fill=nodefill, circle, inner sep=4pt] (n12) at (-0.75, -4.25) {};
			\node[draw=black, fill=nodefill, circle, inner sep=4pt] (n13) at (-0.75, -4.75) {};

			\node[draw=black, fill=nodefill, circle, inner sep=4pt] (n20) at (0, -3.5) {};
			\node[draw=black, fill=nodefill, circle, inner sep=4pt] (n21) at (0, -4) {};
			\node[draw=black, fill=nodefill, circle, inner sep=4pt] (n22) at (0, -4.5) {};

			\node[draw=black, fill=nodefill, circle, inner sep=4pt] (n30) at (0.75, -3.75) {};
			\node[draw=black, fill=nodefill, circle, inner sep=4pt] (n31) at (0.75, -4.25) {};

			\node[draw=black, fill=nodefill, circle, inner sep=4pt] (n40) at (1.5, -4) {};

			\node[] (out) at (3, -4) {dog};

			\draw[-Latex] (in.east) -- (n00);
			\draw[-Latex] (in.east) -- (n01);
			\draw[-Latex] (in.east) -- (n02);
			\draw[-Latex] (in.east) -- (n03);
			\draw[-Latex] (in.east) -- (n04);

			\draw[] (n00) -- (n10);
			\draw[] (n00) -- (n11);
			\draw[] (n00) -- (n12);
			\draw[] (n00) -- (n13);
			\draw[] (n01) -- (n10);
			\draw[] (n01) -- (n11);
			\draw[] (n01) -- (n12);
			\draw[] (n01) -- (n13);
			\draw[] (n02) -- (n10);
			\draw[] (n02) -- (n11);
			\draw[] (n02) -- (n12);
			\draw[] (n02) -- (n13);
			\draw[] (n03) -- (n10);
			\draw[] (n03) -- (n11);
			\draw[] (n03) -- (n12);
			\draw[] (n03) -- (n13);
			\draw[] (n04) -- (n10);
			\draw[] (n04) -- (n11);
			\draw[] (n04) -- (n12);
			\draw[] (n04) -- (n13);

			\draw[] (n10) -- (n20);
			\draw[] (n10) -- (n21);
			\draw[] (n10) -- (n22);
			\draw[] (n11) -- (n20);
			\draw[] (n11) -- (n21);
			\draw[] (n11) -- (n22);
			\draw[] (n12) -- (n20);
			\draw[] (n12) -- (n21);
			\draw[] (n12) -- (n22);
			\draw[] (n13) -- (n20);
			\draw[] (n13) -- (n21);
			\draw[] (n13) -- (n22);

			\draw[] (n20) -- (n30);
			\draw[] (n20) -- (n31);
			\draw[] (n21) -- (n30);
			\draw[] (n21) -- (n31);
			\draw[] (n22) -- (n30);
			\draw[] (n22) -- (n31);

			\draw[] (n30) -- (n40);
			\draw[] (n31) -- (n40);

			\draw[-Latex] (n40) -- (out);

			\node[] at (-5, -3) {};
			\node[] at (3.75, -6.2) {};
		\end{tikzpicture}
	\end{frame}


	\begin{frame}{Decision making: Learning} % Loss
		\centering
		\begin{tikzpicture}
			\node[draw=black, inner sep=0pt, label=below:cat] (in) at (-4, -4) {
				\includegraphics[width=2cm]{data/cat.jpeg}
			};

			\node[draw=black, fill=nodefill, circle, inner sep=4pt] (n00) at (-1.5, -3) {};
			\node[draw=black, fill=nodefill, circle, inner sep=4pt] (n01) at (-1.5, -3.5) {};
			\node[draw=black, fill=nodefill, circle, inner sep=4pt] (n02) at (-1.5, -4) {};
			\node[draw=black, fill=nodefill, circle, inner sep=4pt] (n03) at (-1.5, -4.5) {};
			\node[draw=black, fill=nodefill, circle, inner sep=4pt] (n04) at (-1.5, -5) {};

			\node[draw=black, fill=nodefill, circle, inner sep=4pt] (n10) at (-0.75, -3.25) {};
			\node[draw=black, fill=nodefill, circle, inner sep=4pt] (n11) at (-0.75, -3.75) {};
			\node[draw=black, fill=nodefill, circle, inner sep=4pt] (n12) at (-0.75, -4.25) {};
			\node[draw=black, fill=nodefill, circle, inner sep=4pt] (n13) at (-0.75, -4.75) {};

			\node[draw=black, fill=nodefill, circle, inner sep=4pt] (n20) at (0, -3.5) {};
			\node[draw=black, fill=nodefill, circle, inner sep=4pt] (n21) at (0, -4) {};
			\node[draw=black, fill=nodefill, circle, inner sep=4pt] (n22) at (0, -4.5) {};

			\node[draw=black, fill=nodefill, circle, inner sep=4pt] (n30) at (0.75, -3.75) {};
			\node[draw=black, fill=nodefill, circle, inner sep=4pt] (n31) at (0.75, -4.25) {};

			\node[draw=black, fill=nodefill, circle, inner sep=4pt] (n40) at (1.5, -4) {};

			\node[] (out) at (3, -4) {dog};

			\draw[-Latex] (in.east) -- (n00);
			\draw[-Latex] (in.east) -- (n01);
			\draw[-Latex] (in.east) -- (n02);
			\draw[-Latex] (in.east) -- (n03);
			\draw[-Latex] (in.east) -- (n04);

			\draw[] (n00) -- (n10);
			\draw[] (n00) -- (n11);
			\draw[] (n00) -- (n12);
			\draw[] (n00) -- (n13);
			\draw[] (n01) -- (n10);
			\draw[] (n01) -- (n11);
			\draw[] (n01) -- (n12);
			\draw[] (n01) -- (n13);
			\draw[] (n02) -- (n10);
			\draw[] (n02) -- (n11);
			\draw[] (n02) -- (n12);
			\draw[] (n02) -- (n13);
			\draw[] (n03) -- (n10);
			\draw[] (n03) -- (n11);
			\draw[] (n03) -- (n12);
			\draw[] (n03) -- (n13);
			\draw[] (n04) -- (n10);
			\draw[] (n04) -- (n11);
			\draw[] (n04) -- (n12);
			\draw[] (n04) -- (n13);

			\draw[] (n10) -- (n20);
			\draw[] (n10) -- (n21);
			\draw[] (n10) -- (n22);
			\draw[] (n11) -- (n20);
			\draw[] (n11) -- (n21);
			\draw[] (n11) -- (n22);
			\draw[] (n12) -- (n20);
			\draw[] (n12) -- (n21);
			\draw[] (n12) -- (n22);
			\draw[] (n13) -- (n20);
			\draw[] (n13) -- (n21);
			\draw[] (n13) -- (n22);

			\draw[] (n20) -- (n30);
			\draw[] (n20) -- (n31);
			\draw[] (n21) -- (n30);
			\draw[] (n21) -- (n31);
			\draw[] (n22) -- (n30);
			\draw[] (n22) -- (n31);

			\draw[] (n30) -- (n40);
			\draw[] (n31) -- (n40);

			\draw[-Latex] (n40) -- (out);
			\node[draw=black, dotted, label=below:\small{loss}] (loss) at ($ (out.south) - (0, 1.3) $) {\textcolor{red}{dog $\neq$ cat}};

			\draw[-Latex] (out) -- (loss);

			\node[] at (-5, -3) {};
			\node[] at (3.75, -6.2) {};
		\end{tikzpicture}
	\end{frame}

	\begin{frame}{Decision making: Learning} % Update
		\centering
		\begin{tikzpicture}
			\node[draw=black, inner sep=0pt, label=below:cat] (in) at (-4, -4) {
				\includegraphics[width=2cm]{data/cat.jpeg}
			};

			\node[draw=black, fill=nodefill, circle, inner sep=4pt] (n00) at (-1.5, -3) {};
			\node[draw=black, fill=nodefill, circle, inner sep=4pt] (n01) at (-1.5, -3.5) {};
			\node[draw=black, fill=nodefill, circle, inner sep=4pt] (n02) at (-1.5, -4) {};
			\node[draw=black, fill=nodefill, circle, inner sep=4pt] (n03) at (-1.5, -4.5) {};
			\node[draw=black, fill=nodefill, circle, inner sep=4pt] (n04) at (-1.5, -5) {};

			\node[draw=black, fill=nodefill, circle, inner sep=4pt] (n10) at (-0.75, -3.25) {};
			\node[draw=black, fill=nodefill, circle, inner sep=4pt] (n11) at (-0.75, -3.75) {};
			\node[draw=black, fill=nodefill, circle, inner sep=4pt] (n12) at (-0.75, -4.25) {};
			\node[draw=black, fill=nodefill, circle, inner sep=4pt] (n13) at (-0.75, -4.75) {};

			\node[draw=black, fill=nodefill, circle, inner sep=4pt] (n20) at (0, -3.5) {};
			\node[draw=black, fill=nodefill, circle, inner sep=4pt] (n21) at (0, -4) {};
			\node[draw=black, fill=nodefill, circle, inner sep=4pt] (n22) at (0, -4.5) {};

			\node[draw=black, fill=nodefill, circle, inner sep=4pt] (n30) at (0.75, -3.75) {};
			\node[draw=black, fill=nodefill, circle, inner sep=4pt] (n31) at (0.75, -4.25) {};

			\node[draw=black, fill=nodefill, circle, inner sep=4pt] (n40) at (1.5, -4) {};

			\node[] (out) at (3, -4) {dog};

			\draw[-Latex] (in.east) -- (n00);
			\draw[-Latex] (in.east) -- (n01);
			\draw[-Latex] (in.east) -- (n02);
			\draw[-Latex] (in.east) -- (n03);
			\draw[-Latex] (in.east) -- (n04);

			\draw[red] (n00) -- (n10);
			\draw[red] (n00) -- (n11);
			\draw[red] (n00) -- (n12);
			\draw[red] (n00) -- (n13);
			\draw[red] (n01) -- (n10);
			\draw[red] (n01) -- (n11);
			\draw[red] (n01) -- (n12);
			\draw[red] (n01) -- (n13);
			\draw[red] (n02) -- (n10);
			\draw[red] (n02) -- (n11);
			\draw[red] (n02) -- (n12);
			\draw[red] (n02) -- (n13);
			\draw[red] (n03) -- (n10);
			\draw[red] (n03) -- (n11);
			\draw[red] (n03) -- (n12);
			\draw[red] (n03) -- (n13);
			\draw[red] (n04) -- (n10);
			\draw[red] (n04) -- (n11);
			\draw[red] (n04) -- (n12);
			\draw[red] (n04) -- (n13);

			\draw[red] (n10) -- (n20);
			\draw[red] (n10) -- (n21);
			\draw[red] (n10) -- (n22);
			\draw[red] (n11) -- (n20);
			\draw[red] (n11) -- (n21);
			\draw[red] (n11) -- (n22);
			\draw[red] (n12) -- (n20);
			\draw[red] (n12) -- (n21);
			\draw[red] (n12) -- (n22);
			\draw[red] (n13) -- (n20);
			\draw[red] (n13) -- (n21);
			\draw[red] (n13) -- (n22);

			\draw[red] (n20) -- (n30);
			\draw[red] (n20) -- (n31);
			\draw[red] (n21) -- (n30);
			\draw[red] (n21) -- (n31);
			\draw[red] (n22) -- (n30);
			\draw[red] (n22) -- (n31);

			\draw[red] (n30) -- (n40);
			\draw[red] (n31) -- (n40);

			\draw[Latex-,red] (n40) -- (out);
			\node[draw=black, dotted, label=below:\small{loss}] (loss) at ($ (out.south) - (0, 1.3) $) {\textcolor{red}{dog $\neq$ cat}};

			\draw[Latex-,red] (out) -- (loss);

			\node[] at (-5, -3) {};
			\node[] at (3.75, -6.2) {};
		\end{tikzpicture}
	\end{frame}

	\begin{frame}[t]{Decision making: Summary}
		\vspace{2cm}
		\textbf{How does a neural network make a decision?}\\
		By looking for patterns in input data it has learned to recognize based on training to solve a specific task, represented by a loss function, using training data.
	\end{frame}

	\begin{frame}[t]{Decision making: Summary}
		\vspace{2cm}
		\textbf{How does a neural network make a decision?}\\
		By looking for patterns in input data it has learned to recognize based on training to solve a \textcolor{red}{specific task}, represented by a \textcolor{red}{loss function}, using \textcolor{red}{training data}.
	\end{frame}

	\begin{frame}[t]{Decision making: Summary}
		\vspace{2cm}
		\textbf{How does a neural network make a decision?}\\
		By looking for patterns in input data it has learned to recognize based on training to solve a \textcolor{red}{\textit{specific} task}, represented by a \textcolor{red}{loss function}, using \textcolor{red}{training data}.
		\begin{itemize}
			\item[\textcolor{green}+] The model will get very good at this task.
			\item[\textcolor{red}-] The model will not take considerations beyond this task, e.g. emotions, justice, morality.
			\item[\textcolor{green}+] The model apply patterns from its training data that were sufficient to solve the task there.
			\item[\textcolor{red}-] There is no guarantee these patterns are sufficient in new data.
			\item[\textcolor{red}-] No guarantee these patterns are ones we want to use (e.g. bias).
		\end{itemize}
	\end{frame}

	\begin{frame}{Decision making: Group work}
		We are dealing with an automatic systems in a bank that decide which clients are allowed a loan.
		\begin{itemize}
			\item In the center of the system is a machine learning model that predicts the probability of a client defaulting. This model is a fully deterministic mathematical formula that takes some numbers in a give a number out. The model was trained on training data from the bank.
			\item Around the neural network is a software system which the user interacts with. After the user has input data, the system gives it to the neural network. If the neural network predicts a probability higher than 20\%, the loan is declined. The threshold of 20\% was implemented by a programmer, informed on the basis of a business analyst.
		\end{itemize}
		A client gets his loan declined. Who or what made the decision?
	\end{frame}

	\begin{frame}[t]{Decision making: Generalization} % Train/test
		\centering
		\textbf{There is no guarantee the patterns the model has learned are sufficient in new data.}
		\begin{itemize}
			\item "AI that is based on datasets cannot go beyond what is in the data." - Reasoning, Judging, Deciding: The Science of Thinking, Ch. 15
		\end{itemize}
		\vspace{0.5cm}
	\end{frame}

	\begin{frame}[t]{Decision making: Generalization} % Inference
		\centering
		\textbf{There is no guarantee the patterns the model has learned are sufficient in new data.}
		\begin{itemize}
			\item "AI that is based on datasets cannot go beyond what is in the data." - Reasoning, Judging, Deciding: The Science of Thinking, Ch. 15
			\item While machine learning models are trained on a specific dataset (commonly referred to as the training set), they are almost always evaluated on a different dataset (called the test set).
		\end{itemize}
	\end{frame}

	\begin{frame}[t]{Decision making: Generalization} % Inference
		\centering
		\textbf{There is no guarantee the patterns the model has learned are sufficient in new data.}
		\begin{itemize}
			\item "AI that is based on datasets cannot go beyond what is in the data." -Reasoning, Judging, Deciding: The Science of Thinking, Ch. 15
			\item While machine learning models are trained on a specific dataset (commonly referred to as the training set), they are almost always evaluated on a different dataset (called the test set).
		\end{itemize}
		\vspace{0.5cm}
		\begin{tikzpicture}
			\begin{axis}[
				xlabel={Apartment size (sq. m.)},
				ylabel={Apartment value (NOK)},
				width=8cm,
				height=6cm,
				ymax=6,
				ymin=-2,
				xmin=0,
				xmax=3,
				ticks=none,
				axis x line=bottom,
				axis y line=left
			]

			\addplot[blue!60, only marks] coordinates {
				(1, 2)
				(2, 4)
			};

			\end{axis}
		\end{tikzpicture}
	\end{frame}

	\begin{frame}[t]{Decision making: Generalization} % Interpolation
		\centering
		\textbf{There is no guarantee the patterns the model has learned are sufficient in new data.}
		\begin{itemize}
			\item "AI that is based on datasets cannot go beyond what is in the data." - Reasoning, Judging, Deciding: The Science of Thinking, Ch. 15
			\item While machine learning models are trained on a specific dataset (commonly referred to as the training set), they are almost always evaluated on a different dataset (called the test set).
		\end{itemize}
		\vspace{0.5cm}
		\begin{tikzpicture}
			\begin{axis}[
				xlabel={Apartment size (sq. m.)},
				ylabel={Apartment value (NOK)},
				width=8cm,
				height=6cm,
				ymax=6,
				ymin=-2,
				xmin=0,
				xmax=3,
				ticks=none,
				axis x line=bottom,
				axis y line=left
			]

			\addplot[blue!60, only marks] coordinates {
				(1, 2)
				(2, 4)
			};
			\addplot[green!60, only marks] coordinates {
				(1.5, 3)
			};

			\end{axis}
		\end{tikzpicture}
	\end{frame}

	\begin{frame}[t]{Decision making: Generalization} % Extrapolation
		\centering
		\textbf{There is no guarantee the patterns the model has learned are sufficient in new data.}
		\begin{itemize}
			\item "AI that is based on datasets cannot go beyond what is in the data." - Reasoning, Judging, Deciding: The Science of Thinking, Ch. 15
			\item While machine learning models are trained on a specific dataset (commonly referred to as the training set), they are almost always evaluated on a different dataset (called the test set).
		\end{itemize}
		\vspace{0.5cm}
		\begin{tikzpicture}
			\begin{axis}[
				xlabel={Apartment size (sq. m.)},
				ylabel={Apartment value (NOK)},
				width=8cm,
				height=6cm,
				ymax=6,
				ymin=-2,
				xmin=0,
				xmax=3,
				ticks=none,
				axis x line=bottom,
				axis y line=left
			]

			\addplot[blue!60, only marks] coordinates {
				(1, 2)
				(2, 4)
			};
			\addplot[green!60, only marks] coordinates {
				(1.5, 3)
			};
			\addplot[red!60, only marks] coordinates {
				(0.5, 1)
				(2.5, 5)
			};

			\end{axis}
		\end{tikzpicture}
	\end{frame}

	\begin{frame}[t]{Decision making: Biases} % Plot
		\textbf{No guarantee the patterns the models has learned are ones we want to use}
		\begin{itemize}
			\item The model can rely on variables we do not want to drive the predictions (age, gender, nationality) due to correlations in training data.
			\item This can occur even when the model is not explicitly trained to use these variables.
			\item Thus models perpetuate and potentially amplify societal biases occuring in its training data.
		\end{itemize}
	\end{frame}

	\setbeamertemplate{footline}[compas]

	\begin{frame}[t]{Decision making: Biases} % COMPAS
		\textbf{No guarantee the patterns the models has learned are ones we want to use}
		\begin{itemize}
			\item The model can rely on variables we do not want to drive the predictions (age, gender, nationality) due to correlations in training data.
			\item This can occur even when the model is not explicitly trained to use these variables.
			\item Thus models perpetuate and potentially amplify societal biases occuring in its training data.
		\end{itemize}
		\textbf{Bias in criminal risk assessment (Dressel \& Farid, 2018)}
		\begin{itemize}
			\item Comparison of the ability of COMPAS, a commercial risk assessment software, and non-expert humans to predict re-arrest.
		\end{itemize}
	\end{frame}

	\begin{frame}[t]{Decision making: Biases} % COMPAS: Plot
		\textbf{No guarantee the patterns the models has learned are ones we want to use}
		\begin{itemize}
			\item The model can rely on variables we do not want to drive the predictions (age, gender, nationality) due to correlations in training data.
			\item This can occur even when the model is not explicitly trained to use these variables.
			\item Thus models perpetuate and potentially amplify societal biases occuring in its training data.
		\end{itemize}
		\textbf{Bias in criminal risk assessment (Dressel \& Farid, 2018)}
		\begin{itemize}
			\item Comparison of the ability of COMPAS, a commercial risk assessment software, and non-expert humans to predict re-arrest.
		\end{itemize}
		\centering
		\begin{tikzpicture}
			\node[inner sep=0pt, draw=black] {
				\includegraphics[width=3cm]{data/compas.jpeg}
			};
		\end{tikzpicture}
	\end{frame}

	\begin{frame}[t]{Decision making: Biases} % COMPAS: Bias
		\textbf{No guarantee the patterns the models has learned are ones we want to use}
		\begin{itemize}
			\item The model can rely on variables we do not want to drive the predictions (age, gender, nationality) due to correlations in training data.
			\item This can occur even when the model is not explicitly trained to use these variables.
			\item Thus models perpetuate and potentially amplify societal biases occuring in its training data.
		\end{itemize}
		\textbf{Bias in criminal risk assessment (Dressel \& Farid, 2018)}
		\begin{itemize}
			\item Comparison of the ability of COMPAS, a commercial risk assessment software, and non-expert humans to predict re-arrest.
			\item Both COMPAS and humans were biased against black offenders, even when race was not used in the data.
		\end{itemize}
	\end{frame}

	\begin{frame}[t]{Decision making: Biases} % COMPAS: Accuracy
		\textbf{No guarantee the patterns the models has learned are ones we want to use}
		\begin{itemize}
			\item The model can rely on variables we do not want to drive the predictions (age, gender, nationality) due to correlations in training data.
			\item This can occur even when the model is not explicitly trained to use these variables.
			\item Thus models perpetuate and potentially amplify societal biases occuring in its training data.
		\end{itemize}
		\textbf{Bias in criminal risk assessment (Dressel \& Farid, 2018)}
		\begin{itemize}
			\item Comparison of the ability of COMPAS, a commercial risk assessment software, and non-expert humans to predict re-arrest.
			\item Both COMPAS and humans were biased against black offenders, even when race was not used in the data.
			\item "it is valuable to ask whether we would put these decisions in the hands of random people ..., [which] appear to be indistinguishable."
		\end{itemize}
	\end{frame}

	\setbeamertemplate{footline}[humanbias]

	\begin{frame}[t]{Decision making: Biases} % COMPAS: Accuracy
		\textbf{No guarantee the patterns the models has learned are ones we want to use}
		\begin{itemize}
			\item The model can rely on variables we do not want to drive the predictions (age, gender, nationality) due to correlations in training data.
			\item This can occur even when the model is not explicitly trained to use these variables.
			\item Thus models perpetuate and potentially amplify societal biases occuring in its training data.
		\end{itemize}
		\textbf{Bias in hiring (Bertrand \& Mullainathan, 2004)}
		\begin{itemize}
			\item Evaluation of bias in human decision making in help-wanted advertisements in the US.
			\item "Applicants" were given very African American or European-sounding names.
			\item European names received 50\% more callbacks for interviews.
			\item Applicants from neighbourshoods considered higher class received more callbacks.
			\item Employers listing themselves as an "Equal Opportunity Employer" were as biased as others.
		\end{itemize}
	\end{frame}

	\setbeamertemplate{footline}[default]

	\begin{frame}[t]{Decision making: Theory of mind}
		\textbf{Does AI consider humans as thinking and feeling beings?}
		\begin{itemize}
			\item "... This is an instance of AI programs lacking true Theory of Mind capability." - Reasoning, Judging, Deciding: The Science of Thinking, Ch. 15
			\item Theory of mind: The ability to "track others' unobservable mental states, such as their knowledge, intentions, beliefs, and desires." (Kosinski 2023)
		\end{itemize}
	\end{frame}

	\setbeamertemplate{footline}[pedestrian]

	\begin{frame}[t]{Decision making: Theory of mind}
		\textbf{Does AI consider humans as thinking and feeling beings?}
		\begin{itemize}
			\item "... This is an instance of AI programs lacking true Theory of Mind capability." - Reasoning, Judging, Deciding: The Science of Thinking, Ch. 15
			\item Theory of mind: The ability to "track others' unobservable mental states, such as their knowledge, intentions, beliefs, and desires." (Kosinski 2023)
		\end{itemize}
		\textbf{Pedestrian modelling in self-driving cars (Gulzar et al., 2021)}\\
		\centering
		\vspace{0.5cm}
		\begin{tikzpicture}
			\node[inner sep=0pt, draw=black] {
				\includegraphics[width=8cm]{data/pedestrian.png}
			};
		\end{tikzpicture}
	\end{frame}

	\setbeamertemplate{footline}[theoryofmind]

	\begin{frame}[t]{Decision making: Theory of mind}
		\textbf{Does AI consider humans as thinking and feeling beings?}
		\begin{itemize}
			\item "... This is an instance of AI programs lacking true Theory of Mind capability." - Reasoning, Judging, Deciding: The Science of Thinking, Ch. 15
			\item Theory of mind: The ability to "track others' unobservable mental states, such as their knowledge, intentions, beliefs, and desires." (Kosinski 2023)
		\end{itemize}
		\textbf{Theory of mind in ChatGPT (Kosinski, 2023)}\\
		\centering
		\vspace{0.5cm}
		\begin{tikzpicture}
			\node[inner sep=0pt, draw=black] {
				\includegraphics[width=9cm]{data/theory_of_mind.png}
			};
		\end{tikzpicture}
	\end{frame}

	\setbeamertemplate{footline}[default]

	\begin{frame}[t]{Decision making: Creativity}
		\textbf{Can AI create anything new?}
		\begin{itemize}
			\item "AI does not truly create" - Reasoning, Judging, Deciding: The Science of Thinking, Ch. 15
			\item "AI lacks true imagination" - Reasoning, Judging, Deciding: The Science of Thinking, Ch. 15
		\end{itemize}
	\end{frame}

	\begin{frame}[t]{Decision making: Creativity}
		\textbf{Can AI create anything new?}
		\begin{itemize}
			\item "AI does not truly create" - Reasoning, Judging, Deciding: The Science of Thinking, Ch. 15
			\item "AI lacks true imagination" - Reasoning, Judging, Deciding: The Science of Thinking, Ch. 15
		\end{itemize}
		\centering
		\vspace{0.5cm}
		\begin{tikzpicture}
			\node[label=below:{Imagen: A cute corgi lives in a house made out of sushi}, inner sep=0pt, draw=black] {
				\includegraphics[width=5cm]{data/corgi.jpeg}
			};
		\end{tikzpicture}
	\end{frame}

	\setbeamertemplate{footline}[sparks]

	\begin{frame}[t]{Decision making: Creativity}
		\textbf{Can AI create anything new?}
		\begin{itemize}
			\item "AI does not truly create" - Reasoning, Judging, Deciding: The Science of Thinking, Ch. 15
			\item "AI lacks true imagination" - Reasoning, Judging, Deciding: The Science of Thinking, Ch. 15
		\end{itemize}
		\textbf{GPT-4 displays creative mathematical thinking (Bubeck et al., 2023)}
		\begin{itemize}
			\item "The conversation reflects profound understanding of the undergraduate-level
			mathematical concepts discussed, as well as a significant extent of creativity"
		\end{itemize}
		\vspace{0.5cm}
		\centering
		\begin{tikzpicture}
			\node[inner sep=0pt, draw=black] {
				\includegraphics[width=8cm]{data/ksat}
			};
		\end{tikzpicture}
	\end{frame}

	\setbeamertemplate{footline}[creativity]

	\begin{frame}[t]{Decision making: Creativity}
		\textbf{Can AI create anything new?}
		\begin{itemize}
			\item "AI does not truly create" - Reasoning, Judging, Deciding: The Science of Thinking, Ch. 15
			\item "AI lacks true imagination" - Reasoning, Judging, Deciding: The Science of Thinking, Ch. 15
		\end{itemize}
		\textbf{ChatGPTs creative prowess impresses Twitter (Taecharungroj, 2023)}
		\begin{itemize}
			\item "One of the most prominent features of ChatGPT is its ability
			to generate creative writing. Twitter users have shared examples
			of poems, rap songs, and made-up stories that ChatGPT has written"
		\end{itemize}
		\vspace{0.3cm}
		\centering
		\begin{tikzpicture}
			\node[inner sep=0pt, draw=black] {
				\includegraphics[width=5cm]{data/creativity.png}
			};
		\end{tikzpicture}
	\end{frame}

	\setbeamertemplate{footline}[default]

	\begin{frame}{Decision making: Wisdom}
		\textbf{Are AIs wise?}
		\begin{itemize}
			\item "... the expertise in the domain of fundamental life pragmatics, such as life planning or life review. It requires a rich factual knowledge about life matters, rich procedural knowledge about life problems, knowledge of different life contexts and values or priorities, and knowledge about the unpredictability of life." - easoning, Judging, Deciding: The Science of Thinking, Ch. 15 (adopted from Birren and Svensson, attributed to Baltes and Smith)
			\item Current AI relies on correlations in data, not causal understanding.
			\item Lacks commonsense understanding.
			\item Unimodal (e.g. relies only on text), little opportunity to interact with the world.
			\item \textbf{Little introspection towards its own limits or uncertainties.}
		\end{itemize}
	\end{frame}

	\begin{frame}{Decision making: Summary}
		\textbf{How does AI make decisions?}
		\begin{itemize}
			\item Learns to solve a \textit{very} specific problem.
			\item Relies on correlations in training data.
		\end{itemize}
		\textbf{What can we expect from the decision made by AI?}
		\begin{itemize}
			\item Usually very good at the task it was trained for.
			\item Lacks moral judgement, empathy and sense of justice.
			\item Dangerous to rely on decisions based on input data that is out-of-distribution (extrapolation).
			\item Potentially biased (but so are humans).
			\item Uncertain whether they can imagine other actors with their own goals and desires.
			\item Uncertain whether they can create anything new.
			\item Lacks wisdom, a fundamental understanding of the world, and common sense.
		\end{itemize}
	\end{frame}

	\section{Decision support}

	\begin{frame}[t]{Decision support: Content personalization}
		\textbf{Helping users decide what to listen to}
		\vspace{0.3cm}
		\begin{center}
			\begin{tikzpicture}
				\node[] {
					\includegraphics[width=9cm]{data/spotify.png}
				};
			\end{tikzpicture}
		\end{center}
	\end{frame}

	\begin{frame}[t]{Decision support: Content personalization}
		\textbf{Helping users decide what to listen to}
		\vspace{0.3cm}
		\begin{center}
			\begin{tikzpicture}
				\node[] {
					\includegraphics[width=9cm]{data/spotify.png}
				};
			\end{tikzpicture}
		\end{center}
		\vspace{0.3cm}
		\begin{itemize}
			\item Recommends content to users based on their history.
			\item Has been around for a long time.
			\item Extremely intricate trade-offs between exploitation, showing users what they like, and exploration, showing users new content.
			\item \textbf{Based around recommendation, not clear cut decisions.}
			\item Can potentially lead to feedback loops?
		\end{itemize}
	\end{frame}

	\setbeamertemplate{footline}[vg]

	\begin{frame}[t]{Decision support: Fracture detection}
		\textbf{Helping doctors detect fractures in X-rays}\\
		\begin{itemize}
			\item Bærum sykehus is the first norwegian hospital to implement an AI powered decision support system into the clinic.
			\item Helps alleviate a 12.5\% year-on-year increase in the prevalence of fractures.
			\item 60\% to 70\% of all X-rays are normal, but still need to be reviewed by a radiologist.
		\end{itemize}
		\vspace{0.5cm}
		\centering
		\begin{tikzpicture}
			\node[inner sep=0pt, draw=black] {
				\includegraphics[width=4.5cm]{data/vg.png}
			};
		\end{tikzpicture}
	\end{frame}

	\setbeamertemplate{footline}[boneview]

	\begin{frame}[t]{Decision support: Fracture detection}
		\textbf{Helping doctors detect fractures in X-rays}\\
		\begin{itemize}
			\item Bærum sykehus is the first norwegian hospital to implement an AI powered decision support system into the clinic.
			\item Helps alleviate a 12.5\% year-on-year increase in the prevalence of fractures.
			\item 60\% to 70\% of all X-rays are normal, but still need to be reviewed by a radiologist.
		\end{itemize}
		\textbf{Assessing the efficacy of AI in fracture detection (Guermazi et al., 2022)}\\
		\vspace{0.5cm}
		\centering
		\begin{tikzpicture}
			\node[inner sep=0pt, draw=black] {
				\includegraphics[width=3.5cm]{data/boneview.png}
			};
		\end{tikzpicture}
	\end{frame}

	\begin{frame}[t]{Decision support: Fracture detection}
		\textbf{Helping doctors detect fractures in X-rays}\\
		\begin{itemize}
			\item Bærum sykehus is the first norwegian hospital to implement an AI powered decision support system into the clinic.
			\item Helps alleviate a 12.5\% year-on-year increase in the prevalence of fractures.
			\item 60\% to 70\% of all X-rays are normal, but still need to be reviewed by a radiologist.
		\end{itemize}
		\textbf{Assessing the efficacy of AI in fracture detection (Guermazi et al., 2022)}\\
		\begin{itemize}
			\item AI assistant on its own achieved an area under the receiver operator curve of 0.97.
			\item Radiologist in conjunction with the AI assistant achieved a 10.4\% increase in sensitivity (64.8\% to 75.2\%), and an increase in specificity (90.6\% vs 95.6\%).
			\item \textbf{Assistance from the AI reduced average reading time with 6.3 seconds}.
		\end{itemize}
	\end{frame}

	\setbeamertemplate{footline}[covid]

	\begin{frame}[t]{Decision support: COVID-19 severity}
		\textbf{Helping doctors decide the severity of COVID-19 cases (Wysocki et al., 2023)}\\
		\begin{itemize}
			\item 23 healthcare professionals tasked to assess the severity of COVID-19 in ten patients using the COVID-19 Risk in ONcology Evaluation Tool (CORONET) tool.
		\end{itemize}
		\vspace{0.5cm}
		\centering
		\begin{tikzpicture}
			\node[inner sep=0pt, draw=black] {
				\includegraphics[width=9cm]{data/covid.png}
			};
		\end{tikzpicture}
	\end{frame}

	\begin{frame}[t]{Decision support: COVID-19 severity}
		\textbf{Helping doctors decide the severity of COVID-19 cases (Wysocki et al., 2023)}\\
		\begin{itemize}
			\item 23 healthcare professionals tasked to assess the severity of COVID-19 in ten patients using the COVID-19 Risk in ONcology Evaluation Tool (CORONET) tool.
			\item Asked about their experience using the tool.
		\end{itemize}
		\vspace{0.5cm}
		\centering
		\begin{tikzpicture}
			\node[inner sep=0pt, draw=black] at (0, 0) {
				\includegraphics[width=9cm]{data/questionnaire.png}
			};
		\end{tikzpicture}
	\end{frame}

	\begin{frame}[t]{Decision support: COVID-19 severity} % Uncertainty
		\textbf{Helping doctors decide the severity of COVID-19 cases (Wysocki et al., 2023)}\\
		\begin{itemize}
			\item 23 healthcare professionals tasked to assess the severity of COVID-19 in ten patients using the COVID-19 Risk in ONcology Evaluation Tool (CORONET) tool.
			\item Questioned about their experience using the tool.
		\end{itemize}
		\vspace{0.5cm}
		\centering
		\begin{tikzpicture}
			\node[inner sep=0pt, draw=black] at (0, 0) {
				\includegraphics[width=9cm]{data/questionnaire.png}
			};
			\node[draw=red, minimum width=6.2cm, minimum height=0.5cm, thick] at (-1.23, -0.85) {};
		\end{tikzpicture}
	\end{frame}

	\begin{frame}[t]{Decision support: COVID-19 severity} % Uncertainty
		\textbf{Helping doctors decide the severity of COVID-19 cases (Wysocki et al., 2023)}\\
		\begin{itemize}
			\item 23 healthcare professionals tasked to assess the severity of COVID-19 in ten patients using the COVID-19 Risk in ONcology Evaluation Tool (CORONET) tool.
			\item Questioned about their experience using the tool.
		\end{itemize}
		\vspace{0.5cm}
		\centering
		\begin{tikzpicture}
			\node[inner sep=0pt, draw=black] at (0, 0) {
				\includegraphics[width=9cm]{data/questionnaire.png}
			};
			\node[draw=red, minimum width=6.3cm, minimum height=1cm, thick] at (-1.3, -0.1) {};
		\end{tikzpicture}
	\end{frame}

	\setbeamertemplate{footline}[default]

	\begin{frame}{Decision support: Black boxes} % Schematic
		\begin{tikzpicture}
			\node[draw=black, dashed] (in) at (-4, -0.75) {Laboratory report};

			\node[draw=black, fill=background] (n00) at (0, 0) {
				gram stain = gramneg
			};
			\node[draw=black, fill=background] (n01) at (0, -0.75) {
				morphology = rod
			};
			\node[draw=black, fill=background] (n02) at (0, -1.5) {
				aerobicity = anaerobic
			};

			\node[] (out) at (4, -0.75) {bacteroides};

			\draw[-Latex] (in.east) -- (n00.west);
			\draw[-Latex] (in.east) -- (n01.west);
			\draw[-Latex] (in.east) -- (n02.west);
			\draw[-Latex] (n00.east) -- (out.west);
			\draw[-Latex] (n01.east) -- (out.west);
			\draw[-Latex] (n02.east) -- (out.west);

			\draw[fill=background] (-1.85, -2.65) rectangle (1.85, -5.35);
			\node[anchor=north east] at (1.85, -2.65) {\small{Neural network}};

			\node[draw=black, dashed] (in) at (-4, -4) {Laboratory report};

			\node[draw=black, fill=nodefill, circle, inner sep=4pt] (n00) at (-1.5, -3) {};
			\node[draw=black, fill=nodefill, circle, inner sep=4pt] (n01) at (-1.5, -3.5) {};
			\node[draw=black, fill=nodefill, circle, inner sep=4pt] (n02) at (-1.5, -4) {};
			\node[draw=black, fill=nodefill, circle, inner sep=4pt] (n03) at (-1.5, -4.5) {};
			\node[draw=black, fill=nodefill, circle, inner sep=4pt] (n04) at (-1.5, -5) {};

			\node[draw=black, fill=nodefill, circle, inner sep=4pt] (n10) at (-0.75, -3.25) {};
			\node[draw=black, fill=nodefill, circle, inner sep=4pt] (n11) at (-0.75, -3.75) {};
			\node[draw=black, fill=nodefill, circle, inner sep=4pt] (n12) at (-0.75, -4.25) {};
			\node[draw=black, fill=nodefill, circle, inner sep=4pt] (n13) at (-0.75, -4.75) {};

			\node[draw=black, fill=nodefill, circle, inner sep=4pt] (n20) at (0, -3.5) {};
			\node[draw=black, fill=nodefill, circle, inner sep=4pt] (n21) at (0, -4) {};
			\node[draw=black, fill=nodefill, circle, inner sep=4pt] (n22) at (0, -4.5) {};

			\node[draw=black, fill=nodefill, circle, inner sep=4pt] (n30) at (0.75, -3.75) {};
			\node[draw=black, fill=nodefill, circle, inner sep=4pt] (n31) at (0.75, -4.25) {};

			\node[draw=black, fill=nodefill, circle, inner sep=4pt] (n40) at (1.5, -4) {};

			\node[] (out) at (4, -4) {bacteroides};

			\draw[-Latex] (in.east) -- (n00);
			\draw[-Latex] (in.east) -- (n01);
			\draw[-Latex] (in.east) -- (n02);
			\draw[-Latex] (in.east) -- (n03);
			\draw[-Latex] (in.east) -- (n04);

			\draw[] (n00) -- (n10);
			\draw[] (n00) -- (n11);
			\draw[] (n00) -- (n12);
			\draw[] (n00) -- (n13);
			\draw[] (n01) -- (n10);
			\draw[] (n01) -- (n11);
			\draw[] (n01) -- (n12);
			\draw[] (n01) -- (n13);
			\draw[] (n02) -- (n10);
			\draw[] (n02) -- (n11);
			\draw[] (n02) -- (n12);
			\draw[] (n02) -- (n13);
			\draw[] (n03) -- (n10);
			\draw[] (n03) -- (n11);
			\draw[] (n03) -- (n12);
			\draw[] (n03) -- (n13);
			\draw[] (n04) -- (n10);
			\draw[] (n04) -- (n11);
			\draw[] (n04) -- (n12);
			\draw[] (n04) -- (n13);

			\draw[] (n10) -- (n20);
			\draw[] (n10) -- (n21);
			\draw[] (n10) -- (n22);
			\draw[] (n11) -- (n20);
			\draw[] (n11) -- (n21);
			\draw[] (n11) -- (n22);
			\draw[] (n12) -- (n20);
			\draw[] (n12) -- (n21);
			\draw[] (n12) -- (n22);
			\draw[] (n13) -- (n20);
			\draw[] (n13) -- (n21);
			\draw[] (n13) -- (n22);

			\draw[] (n20) -- (n30);
			\draw[] (n20) -- (n31);
			\draw[] (n21) -- (n30);
			\draw[] (n21) -- (n31);
			\draw[] (n22) -- (n30);
			\draw[] (n22) -- (n31);

			\draw[] (n30) -- (n40);
			\draw[] (n31) -- (n40);

			\draw[-Latex] (n40) -- (out);

			\draw[densely dotted] (-5.3, -2.2) -- (5.3, -2.2);
			\node[anchor=south west] at (-5.3, -2.2) {Expert system};
			\node[anchor=north west] at (-5.3, -2.2) {Machine learning};

			\node[] at (-5.3, 0.75) {};
			\node[] at (5.3, -5.5) {};

		\end{tikzpicture}
	\end{frame}

	\setbeamertemplate{footline}[adverserial]

	\begin{frame}{Decision support: Black boxes} % Panda
		\centering
		\begin{tikzpicture}
			\node[inner sep=0pt, draw=black] {
				\includegraphics[width=8cm]{data/adverserial.png}
			};
		\end{tikzpicture}
	\end{frame}

	\setbeamertemplate{footline}[medical_adverserial]

	\begin{frame}{Decision support: Black boxes} % Medical
		\centering
		\begin{tikzpicture}
			\node[inner sep=0pt, draw=black] {
				\includegraphics[width=8cm]{data/medical_adverserial.png}
			};
		\end{tikzpicture}
	\end{frame}

	\setbeamertemplate{footline}[covid]

	\begin{frame}{Decision support: Explainability} % SHAP
		\centering
		\begin{tikzpicture}
			\node[inner sep=0pt, draw=black] {
				\includegraphics[width=10.5cm]{data/shap.png}
			};
		\end{tikzpicture}
	\end{frame}

	\setbeamertemplate{footline}[gradcam]

	\begin{frame}{Decision support: Explainability} % LRP
		\centering
		\begin{tikzpicture}
			\node[inner sep=0pt, draw=black] {
				\includegraphics[width=10cm]{data/gradcam.png}
			};
		\end{tikzpicture}
	\end{frame}

	\setbeamertemplate{footline}[default]

	\begin{frame}{Decision support: Explainability} % LRP
		\centering
		\begin{tikzpicture}
			\node[inner sep=0pt, draw=black] {
				\includegraphics[width=10.5cm]{data/lrp.png}
			};
		\end{tikzpicture}
	\end{frame}

	\begin{frame}{Decision support: Summary}
		\begin{itemize}
			\item AI already implemented in many domains for decision support, also those considered high stakes.
			\item Can help improve predictive performance, and reduce time.
			\item Lack of understanding of how the AI makes its decisions is a problem.
			\item Explainability is a hot topic in research, but still in its infancy.
		\end{itemize}
	\end{frame}

	\section{How are decisions made by AIs perceived?}

	\setbeamertemplate{footline}[publicattitudes]

	\begin{frame}[t]{Perception of AI: Skepticism}
		\textbf{In some studies, people show low acceptance for AI making high stake decisions}
		\begin{itemize}
			\item 58\% of Americans feel that computer programs will always reflect some level of human bias (Smith 2018).
			\item A majority of US Americans consider it unacceptable to use algorithmic decision making with real life consequences (Smith, 2018).
			\item Concerns that they (algorithms) may violate privacy, are unfair, and lack nuance (Smith 2018).
		\end{itemize}
		\vspace{0.5cm}
		\centering
		\begin{tikzpicture}
			\node[inner sep=0pt, draw=black] {
				\includegraphics[width=4cm]{data/smith.png}
			};
		\end{tikzpicture}
	\end{frame}

	\setbeamertemplate{footline}[moralaversion]

	\begin{frame}[t]{Perception of AI: Skepticism}
		\textbf{In some studies, people show low acceptance for AI making high stake decisions}
		\begin{itemize}
			\item 58\% of Americans feel that computer programs will always reflect some level of human bias (Smith 2018).
			\item A majority of US Americans consider it unacceptable to use algorithmic decision making with real life consequences (Smith, 2018).
			\item Concerns that they (algorithms) may violate privacy, are unfair, and lack nuance (Smith 2018).
			\item AI is seen as having less agency, and thus are less able to make moral decisions (Bigman \& Gray, 2018).
			\item Participants found it less permissible for AI to make decisions about life and death driving situations, parole (Bigman \& Gray, 2018).
			\item Participants found it less permissible for AI to make decisions about potentially life-saving, but risky, medical procedures, than a doctor (Bigman \& Gray).
			\item AI perceived to have less agency and experience, mediating the lower permissibility (Bigman \& Gray, 2018).
		\end{itemize}
	\end{frame}

	\setbeamertemplate{footline}[positivism]

	\begin{frame}[t]{Perception of AI: Positivism}
		\textbf{In other studies, people show high acceptance for AI making high stake decisions (Araujo et al., 2020)}\\
		A study among a representative sample in the Netherlands asked participants to rate usefulness, fairness, and risk of AI (vs. human) decision-making in the media, health sector, and justice system.
		\begin{itemize}
			\item For high-stake decisions, participants perceived decisions by AI (vs. human) to be more useful, fairer and less risky in health and justice contexts (no difference for low-stake decisions)
			\item Perceived usefulness and fairness increased with knowledge on AI, programming, and algorithms (self-reported).
		\end{itemize}
	\end{frame}

	\setbeamertemplate{footline}[moraloutrage]

	\begin{frame}[t]{Perception of AI: Man vs machine}
		\textbf{More moral outrage when humans discriminate than AI (Bigman et al., 2023)}\\
		Participants were asked to assess degree of discrimination, objectivity, prejudice and moral outrage after reading about a discriminatory hiring process. The discrimination was performed either by an AI or a human (HR specialist).
		\begin{itemize}
			\item When discrimination was performed by the AI, participants perceived the process as more objective, less discriminatory, and less prejudiced.
			\item More moral outrage when the discrimination was performed by a human.
			\item Less permissible that CVs are screened by an algorithm.
			\item Liability of the company was smaller when the biased screening procedure was performed by an AI.
		\end{itemize}
	\end{frame}

	\setbeamertemplate{footline}[aitrust]

	\begin{frame}[t]{Perception of AI: Trust in AI}
		\textbf{What predicts trust in AI (Kaplan et al., 2023)}\\
		A meta-anaylsis was performed across 65 studies that empirically investigated what leads people to trust in AI. Trust defined as "the reliance by an agent that actions prejudicial to their well-being will not be undertaken by influential others."
		\begin{itemize}
			\item In humans (interacting with the AI), competency, understanding and expertise were the most important factors for facilitating trust.
			\item In the AI itself, reliability was the most important factor, succeeded by performance.
			\item Also attributes such as personality, anthropomorphism, behaviour and reputation were significant predictors of trust.
			\item The context of the relationship between the human and the AI was also important, with the length of the relationship the most important predictor.
		\end{itemize}
	\end{frame}

	\setbeamertemplate{footline}[humanblackbox]

	\begin{frame}[t]{Perception of AI: perception of humans?}
		\textbf{Humans overrate their ability to understand eachother (Bonezzi et al., 2022)}\\
		Participants were tasked with evaluating how well they understood the decision process of an agent (human or an AI) performing one of three tasks: (1) evaluating risk for recidivism, (2) examining video interviews, (3) examining a Magnetic Resonance Image to diagnose a disease.
		\begin{itemize}
			\item When only the decision of the agent was made available, without explanation, respondents reported a higher degree of understanding.
			\item This difference was reduced when an explanation was provided alongside the decision.
			\item People project their own decision-making processes onto others.
			\item People overestimate their ability to understand the decision-making processes of other humans.
			\item Could also have a negative effect, e.g. by projecting ones own biases onto others.
			\item Are we unfair when asking AIs to explain themselves? Is the only thing that matters predictive proficiency?
		\end{itemize}
	\end{frame}

	\setbeamertemplate{footline}[default]

	\begin{frame}{Perception of AI: Summary}
		\begin{itemize}
			\item There is a tendency towards not trusting AI to make high-stake decisions, although this varies depending on the exact task at hand, the person doing the trusting, the algorithm being trusted, and the general context.
			\item Although we trust AIs less, we are also less inclined to blame them (or their owners/creators) when they make mistakes, at least morally.
			\item Reliability and performance, both metrics of efficacy, are the most important factors for trust in AI.
			\item Human-like attributes in the AI increase trust.
		\end{itemize}
	\end{frame}

	\begin{frame}{Decision making: Group work}
		What kind of decisions would you be comfortable with AI making on your behalf?
	\end{frame}

	\begin{frame}{Practice questions}
		\begin{itemize}
			\item Explain the differences between narrow and general intelligence.
			\item Explain how AI may lead to biased decisions, although their algorithms are objective mathematical constructs.
			\item Discuss the similarities and dissimilarities between human and artificial intelligence, in terms of their capacities and limitations.
			\item Describe why it is hard to interpret the decisions of modern AI, and what is being done to counteract this.
			\item How do people perceive AI decision making? Refer to two examples.
		\end{itemize}
	\end{frame}

\end{document}
