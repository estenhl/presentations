\section{Tree-based models}

\newcommand{\treedataplot}[1]{
    \def\height{5cm}
    \def\width{8cm}

    \ifnum#1>1
        \def\width{5cm}
    \fi

    \begin{tikzpicture}
        \begin{axis}[
            height=\height,
            width=\width,
            xlabel=$x_1$,
            ylabel=$x_2$,
            zlabel=$y$,
            view={75}{10},
            x dir=reverse
        ]
            \ifnum#1<3
                \addplot3 table [
                    only marks,
                    opacity=0.5,
                    col sep=comma,
                    x=x,
                    y=y,
                    z=value
                ] {data/treedata.csv};
            \fi

            \ifnum#1>0
                \ifnum#1<2
                    \addplot3[
                        surf,
                        opacity=0.25,
                        fill=red!80!blue,
                        draw=red!80!blue,
                        domain=0:0.6, y domain=0:0.5,
                        shader=flat
                    ] {0.8};

                    \addplot3[
                        surf,
                        opacity=0.25,
                        fill=red!90!blue,
                        draw=red!90!blue,
                        domain=0:0.6, y domain=0.5:1,
                        shader=flat
                    ] {0.9};
                    \addplot3[
                        surf,
                        opacity=0.25,
                        fill=red!50!blue,
                        draw=red!50!blue,
                        domain=0.6:1, y domain=0:0.5,
                        shader=flat
                    ] {0.5};
                    \addplot3[
                        surf,
                        opacity=0.25,
                        fill=red!10!blue,
                        draw=red!10!blue,
                        domain=0.6:1, y domain=0.5:1,
                        shader=flat
                    ] {0.1};
                \fi
            \fi

            \ifnum#1>2
                \addplot3 [
                    only marks,
                    scatter,
                    scatter src=explicit,
                    point meta min=0,
                    point meta max=1,
                    colormap name=blue red
                ] table [
                    col sep=comma,
                    x=x,
                    y=y,
                    z=value,
                    meta=value
                ] {data/treedata.csv};
            \fi
        \end{axis}
    \end{tikzpicture}
}

\newcommand{\polyplot}[1]{
    \begin{tikzpicture}
        \begin{axis}[
            height=5cm,
            width=8cm,
            xlabel=$x$,
            ylabel=$y$,
            xmin=-0.1,
            xmax=1,
            xtick={-0.1, 0.175, 0.45, 0.725, 1},
            xticklabels={0, 0.25, 0.5, 0.75, 1},
            ymajorticks=false,
            ymin=-10,
            ymax=2.5,
            xtick pos=bottom
        ]
            \ifnum#1<3
                \addplot[
                    only marks,
                    blue,
                    samples=100,
                    domain=-0.1:1,
                    opacity=0.5
                ] (x, 300*x^5 - 750 * x^4 + 630 * x^3 - 210 * x^2 + 20 * x + 1 + rand);
            \fi

            \ifnum#1=1
                \addplot[
                    very thick,
                    red
                ] coordinates {
                    (-0.1,-3)
                    (-0.05, -3)
                    (-0.05, 0)
                    (0, 0)
                    (0, 1.5)
                    (0.2, 1.5)
                    (0.2, -0.5)
                    (0.65, -0.5)
                    (0.65, -1)
                    (0.7, -1)
                    (0.7, -3)
                    (0.85, -3)
                    (0.85, -6)
                    (0.95, -6)
                    (0.95, -9)
                    (1, -9)
                };
            \fi
            \ifnum#1>1
                \addplot[
                    very thick,
                    red
                ] coordinates {
                    (-0.1,-3)
                    (-0.05, -3)
                    (-0.05, 0)
                    (0, 0)
                    (0, 1)
                    (0.05, 1)
                    (0.05, 1.5)
                    (0.15, 1.5)
                    (0.15, 1)
                    (0.2, 1)
                    (0.2, 0)
                    (0.4, 0)
                    (0.4, -0.5)
                    (0.65, -0.5)
                    (0.65, -1)
                    (0.7, -1)
                    (0.7, -2.5)
                    (0.75, -2.5)
                    (0.75, -4)
                    (0.85, -4)
                    (0.85, -6)
                    (0.95, -6)
                    (0.95, -9)
                    (1, -9)
                };
            \fi
            \ifnum#1=4
                \node[anchor=south, font=\scriptsize, inner sep=2pt] at (axis cs: 0.525, -0.5) {
                    $\hat{y}=-0.5$
                };
                \draw[red] (axis cs: 0.4, 2.5) -- (axis cs: 0.4, -10);
                \draw[red] (axis cs: 0.65, 2.5) -- (axis cs: 0.65, -10);

                \node[anchor=south, font=\scriptsize, inner sep=2pt] at (axis cs: 0.3, -0.0) {
                    $\hat{y}=0.0$
                };
                \draw[red] (axis cs: 0.2, 2.5) -- (axis cs: 0.2, -10);
                \draw[red] (axis cs: 0.4, 2.5) -- (axis cs: 0.4, -10);
            \fi
        \end{axis}
    \end{tikzpicture}
}

\newsavebox{\treedata}
\sbox{\treedata}{
    \treedataplot{0}
}

\newsavebox{\treestep}
\sbox{\treestep}{
    \treedataplot{1}
}

\newsavebox{\treesquare}
\sbox{\treesquare}{
    \treedataplot{2}
}
\newsavebox{\treecolours}
\sbox{\treecolours}{
    \treedataplot{3}
}

\newsavebox{\polydata}
\sbox{\polydata}{
    \polyplot{0}
}

\newsavebox{\polystep}
\sbox{\polystep}{
    \polyplot{1}
}

\newsavebox{\polyprecise}
\sbox{\polyprecise}{
    \polyplot{2}
}
\newsavebox{\polyconstant}
\sbox{\polyconstant}{
    \polyplot{3}
}
\newsavebox{\polyregions}
\sbox{\polyregions}{
    \polyplot{4}
}

\newsavebox{\emptyplot}
\sbox{\emptyplot}{
    \begin{tikzpicture}
        \begin{axis}[
            height=5cm,
            width=8cm,
            xlabel=$x$,
            ylabel=$y$,
            ymajorticks=false,
            xmajorticks=false,
            xmin=0,
            xmax=1,
            ymin=0,
            ymax=1
        ]
            \node[] at (axis cs: 0.5, 0.5) {
                \Huge{?}
            };
        \end{axis}
    \end{tikzpicture}
}

\begin{frame}{Tree-based models: Motivation}
    \begin{tikzpicture}
        \node[draw=black] at (-5.25, -3.5) {};
        \node[draw=black] at (5.25, 3.5) {};

        \visible<1-2>{
            \node[] at (0, 1) {
                \usebox{\treedata}
            };
        }
        \visible<2>{
            \node[font=\scriptsize\selectfont] at (0, -2.5) {
                $y=
                \begin{cases}
                    0.8&x_1\leq0.6\ \&\ x_2\leq0.5\\
                    0.9&x_1\leq0.6\ \&\ x_2>0.5\\
                    0.5&x_1>0.5\ \&\ x_2\leq0.5\\
                    0.1&x_1>0.5\ \&\ x_2>0.5\\
                \end{cases}
                $
            };
        }
        \visible<3-4>{
            \node[] at (0, 1) {
                \usebox{\polydata}
            };
        }
        \visible<4>{
            \node[] at (0, -2.5) {
                $y=?x^5+?x^4+?x^3+?x^2+?x+?$
            };
        }
        \visible<5>{
            \node[] at (0, 1) {
                \usebox{\emptyplot}
            };
        }
        \visible<6>{
            \node[] at (0, 1) {
                \usebox{\treestep}
            };
        }
        \visible<7>{
            \node[] at (0, 1) {
                \usebox{\polystep}
            };
        }
        \visible<8>{
            \node[] at (0, 1) {
                \usebox{\polyprecise}
            };
        }
        \visible<9>{
            \node[] at (0, 1) {
                \usebox{\polyconstant}
            };
            \node[] at (0, -2) {
                Piecewise constant function
            };
        }
        \visible<10-11>{
            \node[] at (0, 1) {
                \usebox{\polyregions}
            };
        }
        \visible<11>{
            \node[font=\footnotesize\selectfont] at (0, -2.5) {
                $\hat{y}=
                \begin{cases}
                    ...\\
                    0.0&x\geq0.27\ \& \ x<0.45\\
                    -0.5&x\geq0.45\ \& \ x<0.69\\
                    ...\\
                \end{cases}
                $
            };
        }
    \end{tikzpicture}
\end{frame}

\newcommand{\stratificationplot}[1]{
    \begin{tikzpicture}
        \begin{axis}[
            height=4cm,
            width=4cm,
            xlabel=\scriptsize{$x_2$},
            ylabel=\scriptsize{$x_1$},
            ticklabel style={font=\scriptsize\selectfont},
            tick style={draw=none},
            clip=false,
            xmin=-0.05,
            xmax=0.95,
            ymin=-0.05,
            ymax=0.95,
            point meta min=0,
            point meta max=1,
            colormap name=blue red,
            colorbar,
            colorbar style={
                title=\scriptsize{$y$}
            }
        ]
            \addplot[
                only marks,
                scatter,
                scatter src=explicit
            ] table [
                col sep=comma,
                x=y,
                y=x,
                meta=value
            ] {data/treedata.csv};

            \ifnum#1>0
                \draw[very thick] (axis cs: -0.05, 0.55) -- (axis cs: 0.95, 0.55);
            \fi
            \ifnum#1>1
                \draw[very thick] (axis cs: 0.45, 0.55) -- (axis cs: 0.45, 0.95);
            \fi
        \end{axis}
    \end{tikzpicture}
}

\newsavebox{\stratificationpoints}
\sbox{\stratificationpoints}{
    \stratificationplot{0}
}
\newsavebox{\stratificationfirst}
\sbox{\stratificationfirst}{
    \stratificationplot{1}
}

\newsavebox{\stratificationsecond}
\sbox{\stratificationsecond}{
    \stratificationplot{2}
}

\newcommand{\treeplot}[1]{
    \begin{tikzpicture}
        \node[] at (-1, 0.5) {};
        \node[] at (1.75, -3.5) {};

        \node[draw=black, font=\small\selectfont] (n0) at (0, 0) {
            $x_1 < 0.6$
        };

        \node[font=\small\selectfont, minimum height=1cm] (n10) at (-0.75, -1.5) {};
        \node[font=\small\selectfont, minimum height=1cm] (n11) at (0.75, -1.5) {};
        \node[font=\small\selectfont, minimum height=1cm] (n20) at (0, -3) {};
        \node[font=\small\selectfont, minimum height=1cm] (n21) at (1.5, -3) {};


        \draw[-stealth] (n0) -- (n10) node [midway, left] {Yes};
        \draw[-stealth] (n0) -- (n11) node [midway, right] {No};

        \ifnum#1>0
            \node[font=\small\selectfont] (n10) at (-0.75, -1.5) {
                0.85
            };
        \fi
        \ifnum#1>1
            \node[draw=black, font=\small\selectfont] (n11) at (0.75, -1.5) {
                $x_2 < 0.5$
            };

            \draw[-stealth] (n11) -- (n20) node [midway, left] {Yes};
            \draw[-stealth] (n11) -- (n21) node [midway, right] {No};
        \fi
        \ifnum#1>2
            \node[font=\small\selectfont] (n20) at (0, -3) {
                0.5
            };
            \node[font=\small\selectfont] (n21) at (1.5, -3) {
                0.1
            };
        \fi

    \end{tikzpicture}
}

\newsavebox{\treeroot}
\sbox{\treeroot}{
    \treeplot{0}
}
\newsavebox{\treeleft}
\sbox{\treeleft}{
    \treeplot{1}
}
\newsavebox{\treeright}
\sbox{\treeright}{
    \treeplot{2}
}
\newsavebox{\treefull}
\sbox{\treefull}{
    \treeplot{3}
}

\begin{frame}{Tree-based models: Decision trees}
    \begin{tikzpicture}
        \node[draw=black] at (-5.25, -3.5) {};
        \node[draw=black] at (5.25, 3.5) {};

        \visible<1>{
            \node[anchor=west] at (-5.25, 0) {
                \usebox{\treesquare}
            };
        }
        \visible<2>{
            \node[anchor=west] at (-5.25, 0) {
                \usebox{\treecolours}
            };
        }
        \visible<2-3,8-11,14-15>{
            \node[anchor=east] at (5.25, 0.3) {
                \usebox{\stratificationpoints}
            };
        }
        \visible<4>{
            \node[anchor=north] at (-3, 2.5) {
                \usebox{\treeroot}
            };
        }
        \visible<4-5,12-13>{
            \node[anchor=east] at (5.25, 0.3) {
                \usebox{\stratificationfirst}
            };
        }
        \visible<5>{
            \node[anchor=north] at (-3, 2.5) {
                \usebox{\treeleft}
            };
        }
        \visible<6>{
            \node[anchor=north] at (-3, 2.5) {
                \usebox{\treeright}
            };
        }
        \visible<6-7>{
            \node[anchor=east] at (5.25, 0.3) {
                \usebox{\stratificationsecond}
            };
        }
        \visible<7>{
            \node[anchor=north] at (-3, 2.5) {
                \usebox{\treefull}
            };
        }
        \visible<8-11>{
            \node[anchor=east] at (5.25, 0.3) {
                \usebox{\stratificationpoints}
            };
        }
        \visible<9-11,13-15>{
            \node[] (predictor) at (-3, 0.5) {
                $x_1$ or $x_2$?
            };
        }
        \visible<10-11,14-15>{
            \node[font=\footnotesize] (x1) at (-4, -0.5) {
                [0, 0.1, \ldots, 0.9]
            };
            \node[font=\footnotesize] (x2) at (-2, -0.5) {
                [0, 0.1, \ldots, 0.9]
            };
            \draw[-stealth] ($ (predictor.south) - (0.7, 0) $) -- (x1);
            \draw[-stealth] ($ (predictor.south) + (0.6, 0) $) -- (x2);
        }
        \visible<11>{
            \node[font=\footnotesize, inner sep=2pt] (l1) at ($ (x1.south) - (0.66, 0.5) $) {
                $\ell_0$
            };
            \draw[-stealth] ($ (l1) + (0, 0.55) $) -- (l1);
            \node[font=\footnotesize, inner sep=2pt] (l2) at ($ (x1.south) - (0.23, 0.5) $) {
                $\ell_1$
            };
            \draw[-stealth] ($ (l2) + (0, 0.55) $) -- (l2);
            \node[font=\footnotesize, inner sep=2pt] (l3) at ($ (x1.south) - (-0.66, 0.5) $) {
                $\ell_2$
            };
            \draw[-stealth] ($ (l3) + (0, 0.55) $) -- (l3);

            \node[font=\footnotesize, inner sep=2pt] (l4) at ($ (x2.south) - (0.66, 0.5) $) {
                $\ell_3$
            };
            \draw[-stealth] ($ (l4) + (0, 0.55) $) -- (l4);
            \node[font=\footnotesize, inner sep=2pt] (l5) at ($ (x2.south) - (0.23, 0.5) $) {
                $\ell_4$
            };
            \draw[-stealth] ($ (l5) + (0, 0.55) $) -- (l5);
            \node[font=\footnotesize, inner sep=2pt] (l6) at ($ (x2.south) - (-0.66, 0.5) $) {
                $\ell_5$
            };
            \draw[-stealth] ($ (l6) + (0, 0.55) $) -- (l6);
        }
        \visible<13>{
            \node[font=\footnotesize] (x1) at (-4, -0.5) {
                [0.6, 0.7, 0.8, 0.9]
            };
            \node[font=\footnotesize] (x2) at (-2, -0.5) {
                [0, 0.1, \ldots, 0.9]
            };
            \draw[-stealth] ($ (predictor.south) - (0.7, 0) $) -- (x1);
            \draw[-stealth] ($ (predictor.south) + (0.6, 0) $) -- (x2);
        }
        \visible<11,14-16>{
            \node[font=\footnotesize, inner sep=2pt] (l1) at ($ (x1.south) - (0.66, 0.5) $) {
                $\ell_0$
            };
            \draw[-stealth] ($ (l1) + (0, 0.55) $) -- (l1);
            \node[font=\footnotesize, inner sep=2pt] (l2) at ($ (x1.south) - (0.23, 0.5) $) {
                $\ell_1$
            };
            \draw[-stealth] ($ (l2) + (0, 0.55) $) -- (l2);
            \node[font=\footnotesize, inner sep=2pt] (l3) at ($ (x1.south) - (-0.66, 0.5) $) {
                $\ell_2$
            };
            \draw[-stealth] ($ (l3) + (0, 0.55) $) -- (l3);

            \node[font=\footnotesize, inner sep=2pt] (l4) at ($ (x2.south) - (0.66, 0.5) $) {
                $\ell_3$
            };
            \draw[-stealth] ($ (l4) + (0, 0.55) $) -- (l4);
            \node[font=\footnotesize, inner sep=2pt, text=red] (l5) at ($ (x2.south) - (0.23, 0.5) $) {
                $\ell_4$
            };
            \draw[-stealth, red] ($ (l5) + (0, 0.55) $) -- (l5);
            \node[font=\footnotesize, inner sep=2pt] (l6) at ($ (x2.south) - (-0.66, 0.5) $) {
                $\ell_5$
            };
            \draw[-stealth] ($ (l6) + (0, 0.55) $) -- (l6);
        }
        \visible<15-16>{
            \node[font=\footnotesize, inner sep=2pt, text=red] (l5) at (-3, 1) {
                $\ell$
            };
        }

    \end{tikzpicture}
\end{frame}
