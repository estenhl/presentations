\section{Extensions of linear models}


\newcommand{\datasetplot}[1]{
    \begin{tikzpicture}
        \begin{axis}[
            height=5cm,
            width=8cm,
            xmin=0,
            xmax=1,
            xtick pos=bottom,
            ytick pos=left,
            ymajorticks=false
        ]
            \ifnum#1=0
                \addplot[
                    only marks,
                    blue,
                    samples=100,
                    domain=0:1,
                    opacity=0.5
                ] (x, 20 + 5 * x + rand);
            \fi
            \ifnum#1=1
                \addplot[
                    only marks,
                    blue,
                    samples=100,
                    domain=0:1,
                    opacity=0.5
                ] (x, 20 + x^3 * 5 + rand);
            \fi
            \ifnum#1=2
                \addplot[
                    only marks,
                    blue,
                    samples=100,
                    domain=-1:1,
                    opacity=0.5
                ] coordinates {
                    (0.1, 0)
                    (0.15, 0)
                    (0.22, 0)
                    (0.4, 0)
                    (0.25, 0)
                    (0.55,0)
                    (0.43, 0)
                    (0.51, 0)
                };
                \addplot[
                    only marks,
                    blue,
                    samples=100,
                    domain=-1:1,
                    opacity=0.5
                ] coordinates {
                    (0.3, 1)
                    (0.47, 1)
                    (0.52, 1)
                    (0.65, 1)
                    (0.81, 1)
                    (0.84,1)
                    (0.91, 1)
                    (0.99, 1)
                };
            \fi
            \ifnum#1=3
                \addplot[
                    only marks,
                    blue,
                    samples=100,
                    domain=-1:1,
                    opacity=0.5
                ] coordinates {
                    (0.1, 0)
                    (0.15, 0)
                    (0.22, 0)
                    (0.4, 0)
                    (0.25, 0)
                    (0.55,0)
                    (0.43, 0)
                    (0.51, 0)
                };
                \addplot[
                    only marks,
                    blue,
                    samples=100,
                    domain=-1:1,
                    opacity=0.5
                ] coordinates {
                    (0.3, 1)
                    (0.47, 1)
                    (0.52, 1)
                    (0.65, 1)
                    (0.81, 1)
                    (0.84,1)
                    (0.91, 1)
                    (0.99, 1)
                };
                \addplot[
                    domain=0:1,
                    samples=100,
                    thick,
                    red
                ] {1 / (1 + exp(-(20 * (x - 0.5))))};
            \fi
            \ifnum#1=4
                \addplot[
                    only marks,
                    blue,
                    samples=100,
                    domain=0:1,
                    opacity=0.5
                ] (x, 20 + x^3 * 5 + rand);
                \addplot[
                    thick,
                    red,
                    samples=100,
                    domain=0:1
                ] (x, 20+x^3*5);
            \fi

            \ifnum#1=5
                \addplot[
                    only marks,
                    blue,
                    samples=100,
                    domain=0:1,
                    opacity=0.5
                ] (x, {x <= 0.2 ? 3*x : (x <= 0.8 ? 0.6 + 1*(x-0.2) : 1.2 + 0.1*(x-0.8))});
                %\addplot[thick, blue] (0.9*(1 - exp(-5*x)));
            \fi
        \end{axis}
    \end{tikzpicture}
}

\newsavebox{\linearbox}
\sbox{\linearbox}{
    \datasetplot{0}
}

\newsavebox{\exponentialbox}
\sbox{\exponentialbox}{
    \datasetplot{1}
}

\newsavebox{\logisticbox}
\sbox{\logisticbox}{
    \datasetplot{2}
}

\newsavebox{\logregbox}
\sbox{\logregbox}{
    \datasetplot{3}
}

\newsavebox{\expregbox}
\sbox{\expregbox}{
    \datasetplot{4}
}

\newsavebox{\complexbox}
\sbox{\complexbox}{
    \datasetplot{5}
}

\newcommand{\gammadistribution}{
    \begin{tikzpicture}
        \begin{axis}[
            domain=0:15,
            samples=100,
            xlabel={$y$},
            width=8cm,
            height=6cm,
            xmin=0,
            xmax=15,
            ymin=0,
            ymax=0.2,
            ymajorticks=false,
            ylabel={Probability density}
        ]
        \addplot[blue, thick] {x^(2-1)*exp(-x/2)/(2^2*gamma(2))};
        \end{axis}
    \end{tikzpicture}
}

% \begin{frame}{Extensions of linear models: Generalized linear models}
%     \begin{tikzpicture}
%         \node[draw=black] at (-5.25, -3.5) {};
%         \node[draw=black] at (5.25, 3.5) {};

%         \visible<1-3,5>{
%             \node[] at (0, -2.5) {
%                 $\hat{y}=$$\beta_0+\sum\limits_{i=0}^p \beta_ix_i$
%             };
%         }

%         \visible<2>{
%             \node[] at (0, 1) {
%                 \usebox{\linearbox}
%             };
%         }
%         \visible<3-4,12>{
%             \node[] at (0, 1) {
%                 \usebox{\exponentialbox}
%             };
%         }
%         \visible<4>{
%             \node[] at (0, -2.5) {
%                 $\hat{y}=$\textcolor{red}{$\beta_0+\sum\limits_{i=0}^p \beta_ix_i$}
%             };
%         }
%         \visible<5-7>{
%             \node[] at (0, 1) {
%                 \usebox{\logisticbox}
%             };
%         }
%         \visible<6>{
%             \node[] at (0, -2.5) {
%                 $\log\left(\dfrac{p(X)}{1 - p(X)}\right)=\beta_0+\sum\limits_{i=0}^p \beta_ix_i$
%             };
%         }
%         \visible<7-8>{
%             \node[] at (0, -2.5) {
%                 $p(X)=\dfrac{e^{\left(\beta_0+\sum\limits_{i=0}^p \beta_ix_i\right)}}{1 + e^{\left(\beta_0+\sum\limits_{i=0}^p \beta_ix_i\right)}}$
%             };
%         }
%         \visible<8-11>{
%             \node[] at (0, 1) {
%                 \usebox{\logregbox}
%             };
%         }
%         \visible<9>{
%             \node[] at (0, -2.5) {
%                 $p(X)=\dfrac{e^{\left({\color{red}\beta_0+\sum\limits_{i=0}^p \beta_ix_i} \right)}}{1 + e^{\left({\color{red}\beta_0+\sum\limits_{i=0}^p \beta_ix_i}\right)}}$
%             };
%         }
%         \visible<10>{
%             \node[] at (0, -2.5) {
%                 $p(X)=f(\beta_0+\sum\limits_{i=0}^p \beta_ix_i)$
%             };
%         }
%         \visible<11-12>{
%             \node[] at (0, -2.5) {
%                 $f(\hat{y})=\beta_0+\sum\limits_{i=0}^p \beta_ix_i$
%             };
%         }
%         \visible<13>{
%             \node[] at (0, 1) {
%                 \usebox{\expregbox}
%             };
%             \node[] at (0, -2.5) {
%                 $\log(\hat{y})=\beta_0+\sum\limits_{i=0}^p \beta_ix_i$
%             };
%         }
%         \visible<14>{
%             \node[] at (0, 0) {
%                 \gammadistribution
%             };
%         }
%         \visible<15>{
%             \node[align=flush left, text width=10.1cm] at (0, 0) {
%                 \textbf{Generalized linear models (GLMs):}\\
%                 Extends upon the regular linear model by associating the predictors to the response via a non-linear link function $f$.
%                 \begin{itemize}
%                     \item Requires us to specify $f$ (often determined by investigating the distribution of the response).
%                 \end{itemize}
%             };
%         }
%         \visible<16>{
%             \PythonInputNode{1}{(-5, 3)}{gammanode}{10cm}{8}{
%                 from sklearn.linear_model import GammaRegressor^^J
%                 ^^J
%                 model = GammaRegressor()^^J
%                 model.fit(X, y)^^J
%             }

%             \PythonInputNode{2}{($ (gammanode.south west) - (0, 0.2) $)}{glmnode}{10cm}{8}{
%                 import statsmodels.api as sm^^J
%                 ^^J
%                 link = sm.genmod.families.links.Log()^^J
%                 model = sm.GLM(y, X, family=sm.families.Gamma(link=link))^^J
%                 model.fit()
%             }

%             \RInputNode{($ (glmnode.south west) - (0, 0.7) $)}{rnode}{10cm}{8}{
%                 formula <- ...^^J
%                 result <- glm(formula, family=Gamma(link="log"), data=data)^^J
%             }
%         }
%     \end{tikzpicture}
% \end{frame}

\begin{frame}{Extensions of linear models: Generalized additive models}
    \begin{tikzpicture}
        \node[draw=black] at (-5.25, -3.5) {};
        \node[draw=black] at (5.25, 3.5) {};

        \visible<1>{
            \node[] at (0, 1) {
                \usebox{\complexbox}
            };
        }

    \end{tikzpicture}
\end{frame}
