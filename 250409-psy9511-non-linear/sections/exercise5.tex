\begin{frame}{Exercise 5}
\scriptsize\selectfont
%In the previous exercises you have gotten detailed instructions about how to solve a predictive task step-by-step. In this week's exercise it will be the opposite: You are given full freedom to do as you choose, given that you are able to solve the task at hand relatively well, and reflect on your choices. The idea is that this should resemble what would happen in a real-life project, be it in research or in the industry: Then you will have few a priori answers to what preprocessing steps are necessary or what model is the best, but you will have to make choices, revise them as you go, and reflect upon the process afterwards.

%The task is as follows:

\begin{enumerate}
    \item Download the Credit.csv dataset from ISL: \url{https://www.statlearning.com/s/Credit.csv}
    \item Preprocess the dataset as you see fit.
    \item Fit non-linear models (GLM, GAM, RF, XGBoost) to predict \textit{Balance} from the other variables in the dataset. \textbf{You are not allowed to use the \textit{Income} variable as a predictor, due to its high collinearity with \textit{Balance}.} Whether you fit one or more types of models is up to you, the same goes for the choice of hyperparameters and how you find them. However, you should aim for achieving good predictive performance, and being relatively certain the performance holds up in new data. In other words, using a train/validation/test-split or a (nested) cross-validation is probably a good idea. When you are satisfied, report model performance in a comprehensible manner.
    \item Reflect on the choices you made: Why did you choose the model(s) you did? What about the hyperparameters? How did you tune them?
    \item Reflect upon the performance of the model(s): How well did it/they perform? What would you expect the performance to be on new data? How can you argue this is the case? What other choices could have made you more or less certain about this?
\end{enumerate}

%It is clearly possible to get through this task with minimal effort, but I encourage you to spend some time on it and hone the skills you have learned so far in the course. Feel free to put questions and other reflections in between the code if you want specific feedback, and reach out by email if you want guidance. Good luck!
\end{frame}
